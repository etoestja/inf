\documentclass[a4paper]{article}
\usepackage[a4paper, left=5mm, right=5mm, top=5mm, bottom=5mm]{geometry}
%\geometry{paperwidth=210mm, paperheight=2000pt, left=5pt, top=5pt}
\usepackage[utf8]{inputenc}
\usepackage[english,russian]{babel}
\usepackage{indentfirst}
\usepackage{tikz} %Рисование автоматов
\usetikzlibrary{automata,positioning,arrows,trees}
\usepackage{amsmath}
\usepackage[makeroom]{cancel} % зачеркивание
\usepackage{multicol,multirow} %Несколько колонок
\usepackage{hyperref}
\usepackage{tabularx}
\usepackage{amsfonts}
\usepackage{amssymb}
\DeclareMathOperator*{\argmin}{arg\,min}
\usepackage{wasysym}
\title{Задача о назначениях в pyomo}
\author{Сергей~Володин, 374 гр.}
\newcommand{\matrixl}{\left|\left|}
\newcommand{\matrixr}{\right|\right|}

% алгоритмы (Рубцов)
\usepackage{verbatim}
\usepackage{listings}
\usepackage{algorithm2e}

%+= и -=, иначе разъезжаются...
\newcommand{\peq}{\mathrel{+}=}
\newcommand{\meq}{\mathrel{-}=}
\newcommand{\deq}{\mathrel{:}=}
\newcommand{\plpl}{\mathrel{+}+}

\begin{document}
\maketitle
\section*{Постановка задачи}
$$\begin{cases}
\sum\limits_{ij} x_{ij}c_{ij}\to\min\\
\forall j \sum\limits_i x_{ij}=1\\
\forall i \sum\limits_j x_{ij}=1\\
x_{ij}\in\overline{0,1}
\end{cases}$$

\begin{itemize}
\item $c_{ij}$~--- стоимость выполнения $i$-м работником $j$-й работы
\item  $x_{ij}=1\Leftrightarrow$ $i$-й работник выполняет $j$-ю работу.
\end{itemize}
\section*{Модель pyomo}
Код на github: \url{https://github.com/etoestja/inf/tree/master/mipt/s7/discrete/3}
\begin{verbatim}
$ cat assignment_problem.py
from __future__ import division
from pyomo.environ import *

model = AbstractModel()

model.I = Set()
model.J = Set()

model.c = Param(model.I, model.J)

model.x = Var(model.I, model.J, domain=NonNegativeReals)

def obj_expression(model):
return summation(model.c, model.x)

model.OBJ = Objective(rule=obj_expression)

def constI(model, i):
# return the expression for the constraint for i
return sum(model.x[i,j] for j in model.J) == 1

def constJ(model, j):
# return the expression for the constraint for i
return sum(model.x[i,j] for i in model.I) == 1

# the next line creates one constraint for each member of the set model.I
model.c1 = Constraint(model.I, rule=constI)
model.c2 = Constraint(model.J, rule=constJ)
\end{verbatim}
\section*{Данные}
\begin{verbatim}
$ cat assignment_problem.dat
set I := 1 2 3 4;
set J := 1 2 3 4;

param c:
1 2 3 4 :=
1 1 2 1 2
2 2 3 3 4
3 1 0 0 1
4 1 1 1 1
;
\end{verbatim}
%Solver: glpk
\subsection*{Solver glpk}
Commands:
\begin{verbatim}
 $ pyomo solve assignment_problem.py assignment_problem.dat --solver=glpk
[    0.00] Setting up Pyomo environment
[    0.00] Applying Pyomo preprocessing actions
[    0.00] Creating model
[    0.10] Applying solver
[    0.12] Processing results
    Number of solutions: 1
    Solution Information
      Gap: 0.0
      Status: feasible
      Function Value: 4.0
    Solver results file: results.json
[    0.12] Applying Pyomo postprocessing actions
[    0.12] Pyomo Finished
\end{verbatim}
results.json:
\begin{verbatim}
{
    "Problem": [
        {
            "Lower bound": 4.0, 
            "Name": "unknown", 
            "Number of constraints": 9, 
            "Number of nonzeros": 33, 
            "Number of objectives": 1, 
            "Number of variables": 17, 
            "Sense": "minimize", 
            "Upper bound": 4.0
        }
    ], 
    "Solution": [
        {
            "number of solutions": 1, 
            "number of solutions displayed": 1
        }, 
        {
            "Constraint": "No values", 
            "Gap": 0.0, 
            "Message": null, 
            "Objective": {
                "OBJ": {
                    "Value": 4.0
                }
            }, 
            "Problem": {}, 
            "Status": "feasible", 
            "Variable": {
                "x[1,3]": {
                    "Value": 1.0
                }, 
                "x[2,1]": {
                    "Value": 1.0
                }, 
                "x[3,2]": {
                    "Value": 1.0
                }, 
                "x[4,4]": {
                    "Value": 1.0
                }
            }
        }
    ], 
    "Solver": [
        {
            "Error rc": 0, 
            "Statistics": {
                "Branch and bound": {
                    "Number of bounded subproblems": 0, 
                    "Number of created subproblems": 0
                }
            }, 
            "Status": "ok", 
            "Termination condition": "optimal", 
            "Time": 0.006086111068725586
        }
    ]
}\end{verbatim}

%Solver: bonmin
\subsection*{Solver bonmin}
Commands:
\begin{verbatim}
 $ pyomo solve assignment_problem.py assignment_problem.dat --solver=bonmin
[    0.00] Setting up Pyomo environment
[    0.00] Applying Pyomo preprocessing actions
[    0.00] Creating model
[    0.10] Applying solver
[    0.48] Processing results
    Number of solutions: 1
    Solution Information
      Gap: None
      Status: optimal
      Function Value: 4.00000011978
    Solver results file: results.json
[    0.48] Applying Pyomo postprocessing actions
[    0.48] Pyomo Finished
\end{verbatim}
results.json:
\begin{verbatim}
{
    "Problem": [
        {
            "Lower bound": -Infinity, 
            "Number of constraints": 0, 
            "Number of objectives": 1, 
            "Number of variables": 16, 
            "Sense": "unknown", 
            "Upper bound": Infinity
        }
    ], 
    "Solution": [
        {
            "number of solutions": 1, 
            "number of solutions displayed": 1
        }, 
        {
            "Constraint": "No values", 
            "Gap": null, 
            "Message": "bonmin\\x3a Optimal", 
            "Objective": {
                "OBJ": {
                    "Value": 4.0000001197829755
                }
            }, 
            "Problem": {}, 
            "Status": "optimal", 
            "Variable": {
                "x[1,3]": {
                    "Value": 1.000000029930368
                }, 
                "x[2,1]": {
                    "Value": 1.0000000299529193
                }, 
                "x[3,2]": {
                    "Value": 1.0000000299371015
                }, 
                "x[4,4]": {
                    "Value": 1.0000000299467693
                }
            }
        }
    ], 
    "Solver": [
        {
            "Error rc": 0, 
            "Id": 3, 
            "Message": "bonmin\\x3a Optimal", 
            "Status": "ok", 
            "Termination condition": "optimal", 
            "Time": 0.3690049648284912
        }
    ]
}\end{verbatim}

%Solver: cbc
\subsection*{Solver cbc}
Commands:
\begin{verbatim}
 $ pyomo solve assignment_problem.py assignment_problem.dat --solver=cbc
[    0.00] Setting up Pyomo environment
[    0.00] Applying Pyomo preprocessing actions
[    0.00] Creating model
[    0.10] Applying solver
[    0.46] Processing results
    Number of solutions: 1
    Solution Information
      Gap: 0.0
      Status: optimal
      Function Value: 4
    Solver results file: results.json
[    0.46] Applying Pyomo postprocessing actions
[    0.46] Pyomo Finished
\end{verbatim}
results.json:
\begin{verbatim}
{
    "Problem": [
        {
            "Lower bound": -Infinity, 
            "Name": "tmpBVftNA.pyomo", 
            "Number of constraints": 9, 
            "Number of nonzeros": 33, 
            "Number of objectives": 1, 
            "Number of variables": 17, 
            "Sense": "minimize", 
            "Upper bound": Infinity
        }
    ], 
    "Solution": [
        {
            "number of solutions": 1, 
            "number of solutions displayed": 1
        }, 
        {
            "Constraint": "No values", 
            "Gap": 0.0, 
            "Message": null, 
            "Objective": {
                "OBJ": {
                    "Value": 4
                }
            }, 
            "Problem": {}, 
            "Status": "optimal", 
            "Variable": {
                "x[1,3]": {
                    "Value": 1
                }, 
                "x[2,1]": {
                    "Value": 1
                }, 
                "x[3,2]": {
                    "Value": 1
                }, 
                "x[4,4]": {
                    "Value": 1
                }
            }
        }
    ], 
    "Solver": [
        {
            "Error rc": 0, 
            "Status": "ok", 
            "Termination condition": "unknown", 
            "Time": 0.3434598445892334, 
            "User time": -1.0
        }
    ]
}\end{verbatim}

%Solver: scip
\subsection*{Solver scip}
Commands:
\begin{verbatim}
 $ pyomo solve assignment_problem.py assignment_problem.dat --solver=scip
[    0.00] Setting up Pyomo environment
[    0.00] Applying Pyomo preprocessing actions
[    0.00] Creating model
[    0.10] Applying solver
[    0.47] Processing results
    Number of solutions: 1
    Solution Information
      Gap: None
      Status: optimal
      Function Value: 4.0
    Solver results file: results.json
[    0.48] Applying Pyomo postprocessing actions
[    0.48] Pyomo Finished
\end{verbatim}
results.json:
\begin{verbatim}
{
    "Problem": [
        {
            "Lower bound": -Infinity, 
            "Number of constraints": 0, 
            "Number of objectives": 1, 
            "Number of variables": 16, 
            "Sense": "unknown", 
            "Upper bound": Infinity
        }
    ], 
    "Solution": [
        {
            "number of solutions": 1, 
            "number of solutions displayed": 1
        }, 
        {
            "Constraint": "No values", 
            "Gap": null, 
            "Message": "optimal solution found", 
            "Objective": {
                "OBJ": {
                    "Value": 4.0
                }
            }, 
            "Problem": {}, 
            "Status": "optimal", 
            "Variable": {
                "x[1,3]": {
                    "Value": 1.0
                }, 
                "x[2,1]": {
                    "Value": 1.0
                }, 
                "x[3,2]": {
                    "Value": 1.0
                }, 
                "x[4,4]": {
                    "Value": 1.0
                }
            }
        }
    ], 
    "Solver": [
        {
            "Error rc": 0, 
            "Id": 0, 
            "Message": "optimal solution found", 
            "Status": "ok", 
            "Termination condition": "optimal", 
            "Time": 0.361814022064209
        }
    ]
}\end{verbatim}

%Solver: couenne
\subsection*{Solver couenne}
Commands:
\begin{verbatim}
 $ pyomo solve assignment_problem.py assignment_problem.dat --solver=couenne
[    0.00] Setting up Pyomo environment
[    0.00] Applying Pyomo preprocessing actions
[    0.00] Creating model
[    0.10] Applying solver
[    0.13] Processing results
    Number of solutions: 1
    Solution Information
      Gap: None
      Status: optimal
      Function Value: 3.99999994267
    Solver results file: results.json
[    0.13] Applying Pyomo postprocessing actions
[    0.13] Pyomo Finished
\end{verbatim}
results.json:
\begin{verbatim}
{
    "Problem": [
        {
            "Lower bound": -Infinity, 
            "Number of constraints": 0, 
            "Number of objectives": 1, 
            "Number of variables": 16, 
            "Sense": "unknown", 
            "Upper bound": Infinity
        }
    ], 
    "Solution": [
        {
            "number of solutions": 1, 
            "number of solutions displayed": 1
        }, 
        {
            "Constraint": "No values", 
            "Gap": null, 
            "Message": "couenne\\x3a Optimal", 
            "Objective": {
                "OBJ": {
                    "Value": 3.999999942673723
                }
            }, 
            "Problem": {}, 
            "Status": "optimal", 
            "Variable": {
                "x[1,1]": {
                    "Value": 9.157762731214199e-09
                }, 
                "x[1,3]": {
                    "Value": 0.999999993501932
                }, 
                "x[1,4]": {
                    "Value": -2.6596947977353125e-09
                }, 
                "x[2,1]": {
                    "Value": 1.0
                }, 
                "x[2,2]": {
                    "Value": 7.958210192017798e-09
                }, 
                "x[2,4]": {
                    "Value": -7.958210192017798e-09
                }, 
                "x[3,2]": {
                    "Value": 0.9999999945008361
                }, 
                "x[3,3]": {
                    "Value": 2.3544397697945758e-08
                }, 
                "x[3,4]": {
                    "Value": -1.8045233795922773e-08
                }, 
                "x[4,1]": {
                    "Value": -9.157762770328759e-09
                }, 
                "x[4,2]": {
                    "Value": -2.4590462999185547e-09
                }, 
                "x[4,3]": {
                    "Value": -1.704632972535231e-08
                }, 
                "x[4,4]": {
                    "Value": 1.0
                }
            }
        }
    ], 
    "Solver": [
        {
            "Error rc": 0, 
            "Id": 3, 
            "Message": "couenne\\x3a Optimal", 
            "Status": "ok", 
            "Termination condition": "optimal", 
            "Time": 0.01647496223449707
        }
    ]
}\end{verbatim}

%Solver: ipopt
\subsection*{Solver ipopt}
Commands:
\begin{verbatim}
 $ pyomo solve assignment_problem.py assignment_problem.dat --solver=ipopt
[    0.00] Setting up Pyomo environment
[    0.00] Applying Pyomo preprocessing actions
[    0.00] Creating model
[    0.10] Applying solver
[    0.47] Processing results
    Number of solutions: 1
    Solution Information
      Gap: None
      Status: optimal
      Function Value: 4.00000006493
    Solver results file: results.json
[    0.47] Applying Pyomo postprocessing actions
[    0.47] Pyomo Finished
\end{verbatim}
results.json:
\begin{verbatim}
{
    "Problem": [
        {
            "Lower bound": -Infinity, 
            "Number of constraints": 8, 
            "Number of objectives": 1, 
            "Number of variables": 16, 
            "Sense": "unknown", 
            "Upper bound": Infinity
        }
    ], 
    "Solution": [
        {
            "number of solutions": 1, 
            "number of solutions displayed": 1
        }, 
        {
            "Constraint": "No values", 
            "Gap": null, 
            "Message": "Ipopt 3.12.4\\x3a Optimal Solution Found", 
            "Objective": {
                "OBJ": {
                    "Value": 4.000000064926868
                }
            }, 
            "Problem": {}, 
            "Status": "optimal", 
            "Variable": {
                "x[1,3]": {
                    "Value": 1.0000000121297563
                }, 
                "x[2,1]": {
                    "Value": 1.000000018012586
                }, 
                "x[3,2]": {
                    "Value": 1.0000000141865464
                }, 
                "x[3,3]": {
                    "Value": 1.6100187743708992e-10
                }, 
                "x[4,4]": {
                    "Value": 1.0000000167719398
                }
            }
        }
    ], 
    "Solver": [
        {
            "Error rc": 0, 
            "Id": 0, 
            "Message": "Ipopt 3.12.4\\x3a Optimal Solution Found", 
            "Status": "ok", 
            "Termination condition": "optimal", 
            "Time": 0.35701704025268555
        }
    ]
}\end{verbatim}

\end{document}
