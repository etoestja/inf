 \documentclass[12pt, leqno]{article}
\usepackage[T2A]{fontenc}
\usepackage[utf8]{inputenc}        % Кодировка входного документа;
                                    % при необходимости, вместо cp1251
                                    % можно указать cp866 (Alt-кодировка
                                    % DOS) или koi8-r.

\usepackage[english,russian]{babel} % Включение русификации, русских и
                                    % английских стилей и переносов
%%\usepackage{a4}
%%\usepackage{moreverb}
\usepackage{amsmath,amsfonts,amsthm,amssymb,amsbsy,amstext,amscd,amsxtra,multicol}
\usepackage{verbatim}
\usepackage{listings}
\usepackage{algorithm2e}

\usepackage{tabto} %Табуляция
\usepackage{tikz} %Рисование автоматов
\usetikzlibrary{automata,positioning,trees}
\usepackage{multicol} %Несколько колонок
\usepackage{graphicx}
\usepackage[colorlinks,urlcolor=blue]{hyperref}
\usepackage[stable]{footmisc}
\usepackage{indentfirst}  % абзацный отступ после заголовка
\usepackage{enumitem} %Item tricks


%% \voffset-5mm
%% \def\baselinestretch{1.44}
\renewcommand{\theequation}{\arabic{equation}}
\def\hm#1{#1\nobreak\discretionary{}{\hbox{$#1$}}{}}
\newtheorem{Lemma}{Лемма}
\theoremstyle{definiton}
\newtheorem{Remark}{Замечание}
%%\newtheorem{Def}{Определение}
\newtheorem{Claim}{Утверждение}
\newtheorem{Cor}{Следствие}
\newtheorem{Theorem}{Теорема}
\newtheorem*{known}{Теорема}
\def\proofname{Доказательство}
\theoremstyle{definition}
\newtheorem{Def}{Определение}
\theoremstyle{definition}
\newtheorem{Example}{Пример}

%% \newenvironment{Example} % имя окружения
%% {\par\noindent{\bf Пример.}} % команды для \begin
%% {\hfill$\scriptstyle\qed$} % команды для \end






%\date{22 июня 2011 г.}
\let\leq\leqslant
\let\geq\geqslant
\def\MT{\mathrm{MT}}
%Обозначения ``ажуром''
\def\BB{\mathbb B}
\def\CC{\mathbb C}
\def\RR{\mathbb R}
\def\SS{\mathbb S}
\def\ZZ{\mathbb Z}
\def\NN{\mathbb N}
\def\FF{\mathbb F}
%греческие буквы
\let\epsilon\varepsilon
\let\es\varnothing
\let\eps\varepsilon
\let\al\alpha
\let\sg\sigma
\let\ga\gamma
\let\ph\varphi
\let\om\omega
\let\ld\lambda
\let\Ld\Lambda
\let\vk\varkappa
\let\Om\Omega
\def\first{\mathrm{ FIRST} }
\def\follow{\mathrm{ FOLLOW} }
\let\yield\Rightarrow
\def\abstractname{}

\def\R{{\cal R}}
\def\A{{\cal A}}
\def\B{{\cal B}}
\def\C{{\cal C}}
\def\D{{\cal D}}
\let\w\omega

%классы сложности
\def\REG{{\mathsf{REG}}}
\def\CFL{{\mathsf{CFL}}}
\def\LL{{\mathrm{LL}}}
\newcounter{problem}
\newcounter{uproblem}
\newcounter{subproblem}
\def\pr{\medskip\noindent\stepcounter{problem}{\bf \theproblem .  }\setcounter{subproblem}{0} }
\def\prp{\medskip\noindent\stepcounter{problem}{\bf Задача \theproblem .  }\setcounter{subproblem}{0} }
\def\prstar{\medskip\noindent\stepcounter{problem}{\bf Задача $\bf{\theproblem^*}\negmedspace .$  }\setcounter{subproblem}{0} }
\def\prdag{\medskip\noindent\stepcounter{problem}{\bf Задача $\theproblem^\dagger$ .  }\setcounter{subproblem}{0} }
\def\upr{\medskip\noindent\stepcounter{uproblem}{\bf Упражнение \theuproblem .  }\setcounter{subproblem}{0} }
%\def\prp{\vspace{5pt}\stepcounter{problem}{\bf Задача \theproblem .  } }
%\def\prs{\vspace{5pt}\stepcounter{problem}{\bf \theproblem .*   }
\def\prsub{\medskip\noindent\stepcounter{subproblem}{\rm \thesubproblem .  } }
%прочее
\def\quotient{\backslash\negthickspace\sim}

\begin{document}
\centerline{\LARGE Задание 2}

\medskip

\begin{center}
	{\Large Арифметические операции и линейные рекуррентные последовательности }
\end{center}

\bigskip

{\bf Литература: }
\begin{enumerate}

\item Кормен Т., Лейзерсон Ч., Ривест Р., Штайн К. \\ {\it Алгоритмы. Построение и анализ. }\\  2-е изд. М.: Вильямс, 2005. Главы 3-4.
\item Кнут Д.Э. \\ {\it Искусство программирования (Том 2. Получисленные алгоритмы) }\\  3-е изд. М.: Вильямс, 2001. Подраздел 4.6.3 (с. 503-528).

\end{enumerate}

\section{Быстрое умножение матриц}

В этой части мы рассмотрим алгоритм Штрассена для умножени матриц. Для начала опишем его для матриц размера $2\times 2$

$$ 
	\begin{pmatrix}
		c_{11} & c_{12}\\
		c_{21} & c_{22}
	\end{pmatrix}
	= 
	\begin{pmatrix}
		a_{11} & a_{12}\\
		a_{21} & a_{22}
	\end{pmatrix}
	\cdot
	\begin{pmatrix}
		b_{11} & b_{12}\\
		b_{21} & b_{22}
	\end{pmatrix}_.	
$$

Как и в случае алгоритма Карацубы нам потребуются вспомогательные произведения:
\begin{align*}
	m_1 &= (a_{12}-a_{22})(b_{21}+b_{22}),\\
	m_2 &= (a_{11}+a_{22})(b_{11}+b_{22}),\\
	m_3 &= (a_{11}-a_{21})(b_{11}+b_{12}),\\
	m_4 &= (a_{11}+a_{12})b_{22},\\
	m_5 &= a_{11}(b_{12}-b_{22}),\\
	m_6 &= a_{22}(b_{21}-b_{11}),\\
	m_7 &= (a_{21}+a_{22})b_{11}.
\end{align*}

Элементы матрицы $C$ найдём по формулам:
\begin{align*}
	c_{11} &= m_1 + m_2 - m_4 + m6,\\
	c_{12} &= m_4+m_5,\\
	c_{21} &= m_6+m_7,\\
	c_{22} &= m_2-m_3+m_5-m_7.
\end{align*}

Алгоритм Штрассена определяется рекурсивно для матриц порядка $2^n\times 2^n$:
$$
 A\cdot B = 	\begin{pmatrix}
		A_{11} & A_{12}\\
		A_{21} & A_{22}
	\end{pmatrix} 
	\cdot
	\begin{pmatrix}
		B_{11} & B_{12}\\
		B_{21} & B_{22}
	\end{pmatrix}
	= 
	\begin{pmatrix}
		C_{11} & C_{12}\\
		C_{21} & C_{22}
	\end{pmatrix}
	=
	C.
$$

\upr Доказать, что алгоритмы Карацубы и Штрассена применимы для умножения элементов любого кольца и матриц над любым кольцом соответственно.

%\prp Оценить сложность алгоритма Штрассена в случае умножения чисел наивным методом и в случае умножения чисел методом Карацубы. 

\section{Линейные рекуррентные последовательности}
Последовательность $\{a_n\}^{\infty}_{n=1}$ называется линейно рекурентной последовательностью порядка $d$, если для всех $n > d $ для неё справедлива формула $$a_n = \sum_{i=1}^d c_ia_{n-i}. $$ 

Коэффициенты $c_i$ не зависят от $n$. Линейные рекурентные последовательности замечательны тем, что их члены можно находить возводя в степень матрицу:
$$
A = 
 \begin{pmatrix}
 	c_1 & c_{2} & \dots & c_{d-1} & c_d \\
	1	& 0		  & \dots & 0 	& 0 \\
	0 	& 1	  & \dots & 0 	& 0 \\
	\hdotsfor{5}\\
	0 	& 0		  & \dots & 1 & 0 \\
	
 \end{pmatrix}_{.}
$$

Взяв вектор $\vec{a} = (a_d, a_{d-1}, \ldots, a_2, a_1)$ и умножив его на матрицу $A^n$ мы получим $A^n\vec{a} = (a_{n+d}, a_{n+d-1}, \ldots, a_{n+2}, a_{n+1})$.

\upr Построить алгоритм, использующий для вычисления $a_n$ возведение в степень матриц и оценить его сложность. В качестве алгоритма умножения матриц использовать алгоритм Штрассена.
\medskip

Рассмотрим многочлен $p(\lambda) = \lambda^d - c_1\lambda^{d-1} - c_2\lambda^{d-2} - \ldots - c_d $ который назовём \emph{характерестичеким многочленом} последовательности $\{a_n\}_{n=1}^\infty$.

\upr Показать, что характеристический многочлен $p(\lambda)$ последовательности $\{a_n\}_{n=1}^\infty$ совпадает с характеристическим многочленом матрицы $A$.
\medskip

Пусть $\lambda_1,\lambda_2,\ldots,\lambda_d$ -- различные корни характеристического многочлена $p(\lambda)$.

\prstar Показать, что в этом случае справедлива формула $a_n = k_1\lambda_1^n + k_2\lambda_2^n + \ldots + k_d\lambda_d^n  $, где $k_i$ -- некоторые коэффициенты.

\upr Найти формулу для $a_n$ в случае произвольной кратности корней характеристического многочлена $p(\lambda)$.
\medskip

Также для вычисления $a_n$ можно использовать её производящую функцию. Напомним, что производящей функцией последовательности $\{a_n\}_{n=1}^\infty$ называется ряд $G(a_n, x) = a_1 + a_2x + a_3x^2+\ldots+ a_mx^{m+1}\ldots$.

\upr Показать, что производящая функция $G(a_n,x)$ линейно рекурентной последовательности $\{a_n\}_{n=1}^\infty$ порядка $d$ имеет вид $$ G(a_n, x) = \frac{a_1+a_2x+\ldots+a_dx^{d-1}}{1-x^d}_. $$

\upr Решите \textbf{задачу 4-5} из Кормена (второе издание).
\medskip

С линейными рекурентными последовательностями связана интересная открытая проблема -- проблема Сколема. Она состоит в проверке того, встречается ли в линейной рекурентной последовательности $0$. Доказано, что проблема Сколема разрешима только для частных случаев ЛРП, насколько я знаю для $d \leq 5$, возможно есть более свежие результаты.

\prstar Показать, что проблема Сколема разрешима для ЛПР порядка $d = 1,2,3$. 

\section{Домашнее задание}

Задачи из канонического задания \textbf{6, 7, 9}. Бонусные задачи -- задачи со сзвёздочкой из этого текста.




\end{document}