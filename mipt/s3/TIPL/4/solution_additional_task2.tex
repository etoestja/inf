\documentclass[a4paper]{article}
\usepackage[a4paper, left=5mm, right=5mm, top=5mm, bottom=5mm]{geometry}
\usepackage[utf8]{inputenc}
\usepackage[english,russian]{babel}
\usepackage{indentfirst}
\usepackage{tikz} %Рисование автоматов
\usetikzlibrary{automata,positioning,arrows}
\usepackage{amsmath}
\usepackage{enumerate}
\usepackage{amsfonts}
\usepackage{amssymb}
\usepackage{wasysym}
\title{Теория и реализация языков программирования.\\Задание 4: Замкнутость регулярных языков, теорема Майхилла-Нероуда и минимальные автоматы}
\date{задано 2013.09.25}
\author{Сергей~Володин, 272 гр.}
\newcommand{\matrixl}{\left|\left|}
\newcommand{\matrixr}{\right|\right|}
% названия автоматов
\def\A{{\cal A}}
\def\B{{\cal B}}
\def\C{{\cal C}}

%+= и -=, иначе разъезжаются...
\newcommand{\peq}{\mathrel{+}=}
\newcommand{\meq}{\mathrel{-}=}
\newcommand{\deq}{\mathrel{:}=}
\newcommand{\plpl}{\mathrel{+}+}

% регулярные языки
\def\REG{{\mathsf{REG}}}

\newcommand{\niton}{\not\owns}

\begin{document}
\maketitle
{\bf Сделано позже срока сдачи.}
\section*{Задача 2}
{\em{Идея обсуждалась вместе с Владом Гончаренко.}}
\begin{enumerate}[1.]
\item В одну сторону утверждение из условия очевидно: если $L(\A)=L(\B)$, то $\forall w\hookrightarrow w\in L(\A)\Leftrightarrow w\in L(\B)$, в том числе и для тех, о которых говорится в условии.
\item Докажем в другую сторону. $M=\{w\big ||w|\leqslant |Q^\A|\cdot |Q^\B|\}$. Если входные автоматы не полные, пополним их.
\begin{enumerate}[1.]
\item Утверждение: дан ДКА $\A$, $|Q|=n$, Состояние $q_i\in Q$ достижимо. Тогда кратчайший путь (слово) из $q_0$ в $q_i$ не длиннее $n$. Действительно, пусть иначе (кратчайший путь $w$ имеет большую длину). Значит (принцип Дирихле), автомат в какой-то вершине $q_1$ побывал дважды:  $w=xyz$, $(q_0,w)\equiv(q_0,xyz)\vdash^*(q_1,yz)\vdash^*(q_1,z)\vdash^*(q_i,\varepsilon)$, $|y|>0$. Удалив $y$, получим $w'=xz$, также попадем в $q_i$: $(q_0,w')\equiv^*(q_0,xz)\vdash^*(q_1,z)\vdash^*(q_i,\varepsilon)$, но путь стал короче~--- противоречие ($xyz$~--- самый короткий).
\item Рассмотрим автомат $\C$, имитирующий работу двух входный автоматов $\A$ и $\B$ (такой построен в задаче 4.4). В нем $|Q^\A| \cdot |Q^\B|$ состояний. Кратчайшие пути до достижимых состояний не длиннее $|Q^\A| \cdot |Q^\B|$ (п. 1), поэтому, перебрав все $w\in M$ (то есть, слова, которые не длиннее $|Q^\A| \cdot |Q^\B|$, в том числе и те, которые могут быть кратчайшими путями), автомат $\C$ побывает в каждом достижимом состоянии. Значит, пара $(q^\A_i,q^\B_j)$ из конечных состояний входных автоматов после прочтения слов $w\in M$ достигнет всех своих возможных значений. То есть, $$\forall q^i_j\mbox{~--- достижимое}\hookrightarrow\exists m\in M\colon q^0_0\overset{m}{\longrightarrow}q^i_j.$$
\item Рассмотрим произвольное $w\in\Sigma^*$. Пусть $q^0_0\overset{w}{\longrightarrow}q^i_j$ (здесь используется полнота автоматов). Значит, $q^i_j$~--- достижимое. Тогда (п.2) для него существует $m_0\in M\colon q^0_0\overset{m_0}{\longrightarrow}q^i_j$, иными словами, $q^\A_0\overset{m_0}{\longrightarrow}q^\A_i,q^\B_0\overset{m_0}{\longrightarrow}q^\B_j$.\newline
Из условия имеем $\forall m\in M\hookrightarrow m\in L(\A)\Leftrightarrow m\in L(\B)$. В том числе это выполнено и для $m_0: m_0\in L(\A)\Leftrightarrow m_0\in L(\B)$. Значит, $q^\A_i\in F^\A\Leftrightarrow q^\B_j\in F^\B$. А это означает, что $w\in L(\A)\Leftrightarrow w\in L(\B)\,\blacksquare$
\end{enumerate}
\end{enumerate}
\end{document}
