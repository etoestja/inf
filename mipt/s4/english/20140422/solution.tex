\documentclass[a4paper]{article}
\usepackage[a4paper, left=5mm, right=5mm, top=5mm, bottom=5mm]{geometry}
%\geometry{paperwidth=210mm, paperheight=2000pt, left=5pt, top=5pt}
\usepackage[utf8]{inputenc}
%\usepackage[english]{babel}
\usepackage{indentfirst}
\usepackage{amsmath}
\usepackage[makeroom]{cancel} % зачеркивание
\usepackage{multicol,multirow} %Несколько колонок
\usepackage{hyperref}
\usepackage{tabularx}
\usepackage{amsfonts}
\usepackage{amssymb}
\usepackage{wasysym}
\author{Sergey Volodin, group 272}
\title{Is technology good or bad}
%\date{2014.04.22}
\begin{document}
\maketitle
Technology, of cource, has some bad influence on human life. It may be growing gap between rich and poor or lots of useless information. But on the other hand, it has a lot of good points. This article is about those points.

We often hear that technological development in communication such as internet, skype etc made people less social in real life. But in many cases real interaction is impossible, because these two (or more) human beings live far away from each other, and this virtual communication is the only way for them to see each other or discuss something.

Also technology makes access to information and knowledge more available \cite{test1}, which results in speeding up scientific researches and educational progress.

Another example of benefit of technology are medical achievments. Generally speaking, people die less! For example, we have genetically modified plants which save children from becoming blind \cite{test1}.

Technology also makes the world smaller. It means that people can go almost everywhere they want on Earth (and in near space) not spending countless days or months, like it was in the middle ages. Technology even gives us ability to explore other planets and, possibly, travel there.

But the main advantage of technology are opportunities. Most of humanity can access any information in a click of a button, buy any thing or go to any place wanted. This makes life faster, and because of that, more interesting and full of events.

To sum up, technology gives us more options, from new medical threatments to ways of travel.
\begin{thebibliography}{widestlabel}
\bibitem{test1} Ray Kurzweil. No going back to nature
\end{thebibliography}
\end{document}
