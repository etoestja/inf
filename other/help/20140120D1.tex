\documentclass[a4paper]{article}
\usepackage[a4paper, left=50mm, right=50mm, top=5mm, bottom=5mm]{geometry}
%\geometry{paperwidth=210mm, paperheight=2000pt, left=5pt, top=5pt}
\usepackage[utf8]{inputenc}
\usepackage[english,russian]{babel}
\usepackage{indentfirst}
\usepackage{tikz} %Рисование автоматов
\usetikzlibrary{automata,positioning,arrows,trees}
\usepackage{amsmath}
\usepackage{enumerate}
\usepackage[makeroom]{cancel} % зачеркивание
\usepackage{multicol,multirow} %Несколько колонок
\usepackage{hyperref}
\usepackage{tabularx}
\usepackage{amsfonts}
\usepackage{amssymb}
\DeclareMathOperator*{\argmin}{arg\,min}
\usepackage{wasysym}
\title{Теория и реализация языков программирования.\\Задание 10: LL-анализ}
\date{задано 2013.11.13}
\author{Сергей~Володин, 272 гр.}
\newcommand{\matrixl}{\left|\left|}
\newcommand{\matrixr}{\right|\right|}


%+= и -=, иначе разъезжаются...
\newcommand{\peq}{\mathrel{+}=}
\newcommand{\meq}{\mathrel{-}=}
\newcommand{\deq}{\mathrel{:}=}
\newcommand{\plpl}{\mathrel{+}+}

% пустое слово  (rubtsov)
\def\eps{\varepsilon}

% регулярные языки  (rubtsov)
\def\REG{{\mathsf{REG}}}
\def\CFL{{\mathsf{CFL}}}
\def\eqdef{\overset{\mbox{\tiny def}}{=}}
\newcommand{\niton}{\not\owns}



\begin{document}
$(*)\Leftrightarrow 3^{\sin x}y'=y\cos x\Leftrightarrow 3^{\sin x}\frac{dy}{dx}=y\cos x\Leftrightarrow \frac{dy}{y}=\frac{\cos x}{3^{\sin x}}dx\Leftrightarrow \int\frac{dy}{y}=\int\frac{\cos x}{3^{\sin x}}dx\Leftrightarrow(*)$\newline
Рассмотрим последний интеграл: $\int\frac{\cos x}{3^{\sin x}}dx\equiv\int\frac{\cos x}{3^{\sin x}}dx=\int\frac{d\sin x}{3^{\sin x}}=|t=\sin x|=\int\frac{dt}{3^t}=-\frac{1}{3^t\ln 3}+C=-\frac{1}{3^{\sin x}\ln 3}+C$.\newline
Тогда $(*)\Leftrightarrow \ln|y|=-\frac{1}{3^{\sin x}\ln 3}+C\Leftrightarrow y=C e^{-\frac{1}{3^{\sin x}\ln 3}}$
\end{document}