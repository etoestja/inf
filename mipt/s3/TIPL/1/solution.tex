\documentclass[a4paper]{article}
\usepackage[a4paper, left=5mm, right=5mm, top=5mm, bottom=5mm]{geometry}
\usepackage[utf8]{inputenc}
\usepackage[english,russian]{babel}
\usepackage{graphicx}
\usepackage{indentfirst}
\usepackage{tikz} %Рисование автоматов
\usetikzlibrary{automata,positioning}
\usepackage{amsmath}
\usepackage{floatflt}
\usepackage{enumerate}
\usepackage{amsfonts}
\usepackage{amssymb}
\title{Теория и реализация языков программирования.\\Задание 1: регулярные языки и автоматы}
\date{задано 2013.09.04}
\author{С.~Е.~Володин, 272 гр.}
\newcommand{\matrixl}{\left|\left|}
\newcommand{\matrixr}{\right|\right|}
\newcommand{\Tx}{$T_\text{х}$}
\newcommand{\Tn}{$T_\text{н}$}

\begin{document}
\maketitle
\section*{Задача 1}
\begin{enumerate}
\item $\{a,aa\}\cdot\{b,bb\}\overset{\mathrm{def}}{\equiv}\{x\cdot y|x\in \{a,aa\},y\in \{b,bb\}\}=\{ab,abb,aab,aabb\}.$
\item $\{a,aa\}+\{b,bb\}\overset{\mathrm{def}}{\equiv}\{x|x\in \{a,aa\} \vee x\in \{b,bb\}\}=\{a,aa,b,bb\}$.
\item $\{a,aa\}\times\{b,bb\}\overset{\mathrm{def}}{\equiv}\{(x;y)|x\in \{a,aa\}, y\in \{b,bb\}\}=\{(a;b),(a;bb),(aa;b),(aa;bb)\}$.
\item Так как $(A|B)\supseteq A$, $X\overset{\mathrm{def}}{=}\{((aa|b)^*(a|bb)^*)^*\}\supseteq \{(b^*a^*)^*\}$. Также $b^*a^*\supseteq (a|b)$, поэтому $X\supseteq \{(a|b)^*\}$.
\item $Z\overset{\mathrm{def}}{=}\underbrace{\{ a^{3n} | \, n>0\}}_\text{X} \cap {\underbrace{\{ a^{5n+1} | n \geq 0\}}_\text{Y}}^*$. $Y\supseteq \{a\}$, $Y^*\supseteq\{a\}^*\supseteq \{a^{3n}|n>0\}=X$, поэтому $Z=X=\{ a^{3n} | \, n>0\}$.
\item $\varnothing\cap\{\varepsilon\}\equiv\{\}\cap\{\varepsilon\}=\varnothing$.
\end{enumerate}

\section*{Задача 2}
$L=\Sigma^*\setminus{\underbrace{{\{ (0^*110^*)^* \}}}_\mathrm{{L}_{-}}}$. Для слова $w$ из ${L}_{-}$ есть два варианта, в соответствии с количеством повторений $N$ в последней звездочке:
\begin{enumerate}[a.]
\item ($N=0$ раз) $w=\varepsilon$
\item ($N>0$) Докажем по индукции, что \emph{w~--- строки из четного количества <<$1$>>, отделенные друг от друга нулями, либо концом/началом слова, причем в слове хотя бы одна единица есть}.\newline
Для $N=1$ это верно: $w_1\in \{0^*110^*\}\Rightarrow w_1=\underbrace{0\dots 0}_{n_1}11\underbrace{0\dots 0}_{n_2}$, $n_1$ и $n_2 \geqslant 0$. Строка из двух единиц отделена нулями при $n_1,n_2>0$, либо концом/началом слова при $n_1=0$, либо $n_2=0$.\newline
Пусть верно для $N\leqslant n$. Докажем для $n+1$: $w_{n+1}=w_n\cdot w, w\in \{0^*110^*\}=w_n\underbrace{0\dots 0}_{n_1}11\underbrace{0\dots 0}_{n_2}$. Рассмотрим различные случаи: $w_n$ может заканчиваться на $0$, либо на $1$; $n_1=0$, либо $n_1>0$:
\begin{enumerate}[1.]
\item $(0,n_1=0)$ Добавленная строка из единиц отделена слева нулями из $w_n$.
\item $(0,n_1>0)$ Добавленная строка из единиц отделена слева нулями из $w$.
\item $(1,n_1=0)$ Получена строка более, чем из двух единиц, но она четной длины (т.к. строка единиц из $w_n$ имеет четную длину по предположению индукции, и $2$~--- четно).
\item $(1,n_1>0)$ Добавленная строка из единиц отделена слева нулями из $w$. Строка единиц из $w_n$ отделена теми же нулями.
\end{enumerate}
Очевидно, что под это определение не попадают слова не из $L_-$ (можно построением найти вхождения РВ: найдем все строки из единиц в слове. Рассмотрим их по-очереди, с первой. Если строка длины $2$, то единицы и все нули справа и слева~--- вхождение выражения. Если длина больше $2$, то нули слева от первой пары вместе с ней~--- вхождение, нули справа от последней пары вместе с ней~--- вхождение. Четное количество единиц между этими парами (если есть)~--- несколько вхождений).
\end{enumerate}
Таким образом, $L$~--- непустые слова, состоящие либо только из нулей, либо из строк единиц, отделенных друг от друга нулями или началом/концом слова, но длина хотя бы одной строки нечетна. Иными словами, непустое слово $w$:
\begin{enumerate}[1.]
\item либо состоит из нулей,
\item либо в нем присутствует строка из единиц нечетной длины, отделенная
\begin{enumerate}[a.]
 \item нулями
 \item началом слова слева и нулями справа
 \item началом слова слева и концом слова справа
 \item нулями слева и концом слова справа.
\end{enumerate}
\end{enumerate}
Тогда $L=\{(\underbrace{00^*}_{1}|\underbrace{{(0|1)}^*0{(11)}^*0{(0|1)}^*}_{2a}|\underbrace{{(11)}^*0{(0|1)}^*)}_{2b}|\underbrace{1{(11)}^*}_{2c}|\underbrace{{(0|1)}^*01{(11)}^*}_{2d}\}$
\section*{Задача 3}
\begin{enumerate}
\item{Конкатенация}\newline
\begin{tikzpicture}[level 1/.style={sibling distance=2.5cm}]
\node {$\mkern+10mu f\, \bullet^{N}\, l$}
child{
 node {$f_\alpha\, \alpha^{N_\alpha}\, l_\alpha$}
}
child{
 node {$f_\beta\, \beta^{N_\beta}\, l_\beta$}
};
\end{tikzpicture}
\newline
В результате будет порождено слово $c=ab$.\newline
Если $N_\alpha=F$, то $a$~--- его префикс, так как слово $a$ всегда непустое. Тогда $f=f_\alpha$. Иначе, если $N_\alpha=T$, либо $a$, либо $b$ (в случае $a=\varepsilon$)~--- префикс $c$. Аналогично, если $N_\beta=F$, то $b$~--- суффикс $c$, откуда $l=l_\beta$. Иначе $l=l_\alpha\cup l_\beta$.\newline
Всё выражение может быть пустым тогда и только тогда, когда $\alpha$ и $\beta$ могут быть пустыми. Результат в таблице ниже:\newline
$
\begin{array}{ccccc}
N_\alpha & N_\beta & f & l & N\\
F & F & f_\alpha & l_\alpha & F \\
F & T & f_\alpha & l_\alpha \cup l_\beta & F\\
T & F & f_\alpha \cup f_\beta & l_\beta & F\\
T & T & f_\alpha \cup f_\beta & l_\alpha \cup l_\beta & T\\
\end{array}
$
\item{Объединение}\newline
\begin{tikzpicture}[level 1/.style={sibling distance=2.5cm}]
\node {$\mkern+10mu f\, |^{N}\, l$}
child{
 node {$f_\alpha\, \alpha^{N_\alpha}\, l_\alpha$}
}
child{
 node {$f_\beta\, \beta^{N_\beta}\, l_\beta$}
};
\end{tikzpicture}
\newline
В результате будет порождено слово $c$.\newline
Во всех случаях $c$ может начинаться как с символов, порожденных первым выражением, так и с символов, порожденных вторым, и ни с каких других. Тогда $f=f_\alpha\cup f_\beta$, $l=l_\alpha\cup l_\beta$,
Всё выражение не может быть пустым тогда и только тогда, когда $\alpha$ и $\beta$ не могут быть пустыми. Результат в таблице ниже:
$
\newline
\begin{array}{ccccc}
N_\alpha & N_\beta & f & l & N\\
F & F & f_\alpha \cup f_\beta & l_\alpha \cup l_\beta & F \\
F & T & f_\alpha \cup f_\beta & l_\alpha \cup l_\beta & T\\
T & F & f_\alpha \cup f_\beta & l_\alpha \cup l_\beta & T\\
T & T & f_\alpha \cup f_\beta & l_\alpha \cup l_\beta & T\\
\end{array}
$
\end{enumerate}
\section*{Задача 4}
\begin{enumerate}[1.]
\item $\begin{array}{ccc}
& {\cal A} & {\cal B} \\
Q & \{q_0,q_1,q_2\} & \{q_0,q_1,q_2\}\\
\Sigma & \{0,1\} & \{0,1\}\\
q_0 & q_0 & q_0\\
\delta & \substack{\{((q_0,0),q_0),((q_0,1),q_1),((q_1,1),q_0),\\((q_1,0),q_2),((q_2,0),q_1),((q_2,1),q_2)\}} & \substack{\{((q_0,0),q_0),((q_0,1),q_1),\\((q_1,0),\{q_0,q_2\}),((q_2,0),q_1),((q_2,1),q_2)\}}\\
F & \{q_1\} & \{q_1\}\\
\end{array}$
\item ${\cal A}$~--- детерминированный, так как из каждого состояния есть только один переход с определенным символом.\newline
${\cal B}$~--- недетерминированный, так как из состояния $q_1$ есть два перехода по символу $0$: в $q_0$ и $q_2$.
\item $(q_0,01001)\vdash(q_0,1001)\vdash(q_1,001)\vdash(q_2,01)\vdash(q_1,1)\vdash(q_0,\varepsilon)$. Нет, не принимает, потому что $q_0\notin F$. То есть, $w\equiv 01001\notin L({\cal A})$.
\item Да: $(q_0,01001)\vdash(q_0,1001)\vdash(q_1,001)\vdash(q_0,01)\vdash(q_0,1)\vdash(q_1,\varepsilon)$ и $q_1\in F$.
\item ${\cal B}$ не примет пустое слово, так как $q_0\notin F$, но примет слово $100$: $(q_0,100)\vdash(q_1,00)\vdash(q_2,0)\vdash(q_1,\varepsilon)$\newline
${\cal A}$ не примет слово $0$: $(q_0,0)\vdash(q_0,\varepsilon)$ и $q_0\notin F$, но примет $10011$: $(q_0,10011)\vdash(q_1,0011)\vdash(q_2,011)\vdash(q_1,11)\vdash(q_0,1)\vdash(q_1,\varepsilon)$ и $q_1\in F$.
\end{enumerate}
\newpage
\section*{Задача 5}
\begin{enumerate}[1.]
\item
Докажем, что $L=T$:
\begin{enumerate}[1.]
\item $(L\subseteq T)$. Если $w\in L$, то $w$ получено из одного из слов $\varepsilon,b,bb$ применением правила $(2)$ $N(w)\geqslant 0$ раз. Действительно, $w\in L\Rightarrow
\left[
\begin{array}{lll}
(1) & w\in\{\varepsilon,b,bb\} & N(w)=0\\
(2) & \substack{w\in\{ax,bax,bbax\},\\\text{где } x\in L} & N(w)=1+N(x)\\
\end{array}
\right.
$. $N(w)<\infty$, так как в случае $(2)$ $|x|<|w|$.\newline
Таким образом, определена функция $N:L\longrightarrow {\mathbb N}\cup\{0\}$~--- количество применений правила $(2)$ для слова $w$. Заметим, что значение этой функции также равно количеству букв $a$ в слове $w$, так как правило $(2)$ добавляет одну букву $a$. Индукцией по $N$ докажем, что $L\subseteq T$:
\begin{enumerate}[a.]
\item ($N=0$) $w\in \{\varepsilon,b,bb\}$. В $w$ нет трех букв $b$ подряд, поэтому $w\in T$.
\item (доказано для $N=n-1$, докажем для $N=n$) $w\in \{ax,bax,bbax\}$, причем $N(x)=n-1$, так как в $w$ на одну букву $a$ больше, чем в $w$. Поэтому по предположению индукции в $w$ нет тех букв $b$ подряд. Заметим, что $x$ отделено буквой $a$ от $\varepsilon$, $b$ или $bb$, поэтому в $w$ нет трех букв $b$ подряд, отсюда $w\in T$.
\end{enumerate}
\item $(T\subseteq L)$. В слове $w\in T$ $M(w)$ букв $a$. Индукцией по $m$ докажем, что $w\in L:$
\begin{enumerate}[a.]
\item ($M=0$) Букв $a$ нет $\Rightarrow w$ состоит из букв $b$, причем не более, чем из двух. Тогда $w\in \{\varepsilon,b,bb\}\subset L$.
\item (доказано для $M=m-1$, докажем для $M=m$). Разобьем $w=x_1ax_2$, где в $x_1$ нет букв $a$ (можно сделать, так как случай, где в $w$ нет букв $a$ разобран выше). В слове $x_2$ будет $m-1$ букв $a$, и, по предположению индукции, $x_2\in L$. В $x_1$ только буквы $b$, поэтому $x_1\in\{\varepsilon,b,bb\}$. Таким образом, $w\in\{ax_2,bax_2,bbax_2\}$, и $x_2\in L$. По правилу $(2)$ получаем $w\in L~\blacksquare$
\end{enumerate}
\end{enumerate}
\item Докажем, что следующий автомат ${\cal A}$ распознает $T$:
\newline
\begin{tikzpicture}[shorten >=1pt,node distance=2cm,on grid,auto,initial text=]
%\draw[help lines] (0,0) grid (3,2);
\node[state, initial, accepting]  (q_0)                      {$q_0$};
\node[state, accepting] (q_1) [right = of q_0] {$q_1$};
\node[state, accepting]          (q_2) [right =of q_1] {$q_2$};
 \path[->] 
  (q_0)	edge	[loop below]	node	{$a$}	()
        edge	[bend left] 	node	{$b$}	(q_1)
  (q_1)	edge	[bend left]		node	{$a$} (q_0)
        edge	[bend left]		node	{$b$}	(q_2)
  (q_2)	edge	[bend left]		node	{$a$}	(q_0);
\end{tikzpicture}
\newline
Определим $N:T\longrightarrow {\mathbb N} \cup \{0\}:N(w)$~--- количество букв $a$ в слове $w\in T$. Индукцией по $N(w)$ докажем, что автомат \emph{принимает} $w$:
\begin{enumerate}[a.]
\item ($N=0$) слово состоит из букв $b$. Тогда $w\in T\Rightarrow\text{в }w\text{ не больше }2\text{ букв }b\text{ подряд}\Rightarrow w\in\{\varepsilon,b,bb\}$. Запишем цепочки конфигураций для этих слов:
\begin{enumerate}[1.]
\item ($w=\varepsilon$) $(q_0,\varepsilon)$. $q_0$~--- принимающее $\Rightarrow$ автомат принимает $w$.
\item ($w=b$) $(q_0,b)\vdash(q_1,\varepsilon)$. $q_1$~--- принимающее $\Rightarrow$ автомат принимает $w$.
\item ($w=bb$) $(q_0,bb)\vdash(q_1,b)\vdash(q_2,\varepsilon)$. $q_1$~--- принимающее $\Rightarrow$ автомат принимает $w$.
\end{enumerate}
\item (доказано для $N=m-1$, докажем для $N=m$) Разделим $w$ по последнему символу $a$ (это можно сделать, так как случай, где в $w$ нет символов $a$ разобран выше): $w=x_1ax_2$, в $x_2$ нет символов $a$. Тогда $N(x_1)=m-1$, откуда следует, что автомат принимает $x_1$. Поэтому после обработки $x_1$ он оказывается в одном из состояний. Тогда после обработки следующего символа, $a$, он окажется в состоянии $q_0$, так как $\forall q\in Q\hookrightarrow ((q,a),q_0)\in \delta$, где обозначения $Q,\delta$ стандартные.\newline
$x_2$ состоит из букв $b$, поэтому $x_2\in \{\varepsilon,b,bb\}$. Поскольку автомат находится в состоянии $q_0$, цепочка конфигураций после конфигурации $(q_0,x_2)$ будет такой же, как в п. 1.
\end{enumerate}
Этим доказано, что автомат принимает $T$, то есть, $L({\cal A})\supseteq T$.\newline
Теперь докажем, что $L({\cal A})\subseteq T\Leftrightarrow w\in L({\cal A})\Rightarrow w\in T\Leftrightarrow w\notin T\Rightarrow w\notin L({\cal A})$.\newline
$w\notin T\Rightarrow$ в $w$ есть больше двух букв $b$ подряд. Найдем первый символ $b$ в $w$, после которого идет больше двух: $w=x_1\underbrace{b\dots b}_{n}x_2$, $x_1$ не заканчивается на $b$, $x_1\in T$ (так как там не больше двух $b$ подряд), $n\geqslant 3$.\newline
$x_1\in T\Rightarrow$ после обработки $x_1$ автомат будет в одном из состояний. Также $x_1$ не заканчивается на $b\Rightarrow$ либо заканчивается на $a$, либо $x_1=\varepsilon$. В любом случае, это состояние будет $q_0:(q_0,b^nx_2)\vdash(q_1,b^{n-1}x_1)\vdash(q_2,b^{n-2}x_2)$. Но перехода из $q_2$ по $b$ нет, поэтому автомат останавливается (и не принимает слово), т.е. $w\notin {\cal A}~\blacksquare$
\end{enumerate}
\end{document}
