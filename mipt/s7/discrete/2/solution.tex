\documentclass[a4paper]{article}
\usepackage[a4paper, left=25mm, right=25mm, top=25mm, bottom=25mm]{geometry}
%\geometry{paperwidth=210mm, paperheight=2000pt, left=5pt, top=5pt}
\usepackage[utf8]{inputenc}
\usepackage[english,russian]{babel}
\usepackage{indentfirst}
\usepackage{tikz} %Рисование автоматов
\usetikzlibrary{automata,positioning,arrows,trees}
\usepackage{amsmath}
\usepackage[makeroom]{cancel} % зачеркивание
\usepackage{multicol,multirow} %Несколько колонок
\usepackage{hyperref}
\usepackage{tabularx}
\usepackage{amsfonts}
\usepackage{amssymb}
\DeclareMathOperator*{\argmin}{arg\,min}
\usepackage{wasysym}
\title{Дискретная оптимизация.\\Задание 2}
\date{задано 2016.09.14}
\author{Сергей~Володин, 374 гр.}
\newcommand{\matrixl}{\left|\left|}
\newcommand{\matrixr}{\right|\right|}

% алгоритмы (Рубцов)
\usepackage{verbatim}
\usepackage{listings}
\usepackage{algorithm2e}

%+= и -=, иначе разъезжаются...
\newcommand{\peq}{\mathrel{+}=}
\newcommand{\meq}{\mathrel{-}=}
\newcommand{\deq}{\mathrel{:}=}
\newcommand{\plpl}{\mathrel{+}+}

% пустое слово  (rubtsov)
\def\eps{\varepsilon}
\newcommand{\smalll}[1]{\overline{\overline{#1}}}
\newcommand{\smallo}{\bar{\bar{o}}}

\begin{document}
\maketitle

\subsection*{Вышки и частоты}
Условие: есть $n$ вышек с координатами $\{(x_i, y_i)\}$. Вышки, находящиеся на расстоянии $\rho_{ij}\leqslant D$ должны иметь разные частоты. Определить минимальное число частот.

Определим $\rho_{ij}=\sqrt{(x_i-x_j)^2+(y_i-y_j)^2}$.

Рассмотрим граф с матрицей смежности $A_{ij}=\begin{cases}
[\rho_{ij}\leqslant D],& i\neq j\\
0,& i=j
\end{cases}$
Задача сводится к задаче о раскраске вершин графа с минимальным числом цветов. Заметим, что граф можно раскрасить в $n$ цветов.
Рассмотрим переменные $X_{ij}=[i\mbox{-я вершина имеет }j\mbox{-й цвет}]$.
Обозначим $d_j(X)=\sum\limits_iX_{ij}$~--- количество вершин, имеющих $j$-й цвет.
Обозначим $c(X)=\sum\limits_{j}[d_j>0]$~--- количество цветов в раскраске.

Обозначим $s(X)=\sum\limits_j{a_jd_j}$, где $a_j=(n+1)^j$

Докажем, что задача минимизации
$$(1)\,c(X)\to\min\limits_X$$
эквивалентна задаче
$$(2)\,s(X)\to\min\limits_X$$
Пусть в обоих задачах нет неиспользуемых цветов с номером меньше $c$.

Пусть (2) дает решение $X$, не оптимальное по количеству цветов, а оптимальное решение~--- $X^*$. Тогда $c=c(X)>c(X^*)=c^*$, но $s(X)\leqslant s(X^*)$.
Рассмотрим $s(X)=\sum_j a_jd_j \geqslant a_c=(n+1)^c\geqslant (n+1)^{c^*}(n+1)>(n+1)^{c^*}n=a_{c^*}n\geqslant 
s(X^*)$~--- противоречие.

Ограничение: каждая вершина должна иметь только один цвет: $\forall i \sum\limits_j X_{ij}=1$

Ограничение: соседние вершины имеют разные цвета: $\sum\limits_{ijk}A_{ij}X_{ik}X_{jk}=0$

Задача оптимизации:
$$
\sum\limits_j (n+1)^j \sum_i X_{ij}\to\min\limits_{
\begin{cases}
\forall i \sum\limits_j X_{ij}=1\\
\sum\limits_{ijk}A_{ij}X_{ik}X_{jk}=0
\end{cases}}
$$
\begin{verbatim}
param N;
set Vertices;
param A {i in Vertices, j in Vertices};
var X{i in Vertices, j in Vertices} binary;
minimize S:
sum{i in Vertices, j in Vertices} (N + 1) ** (j) * X[i,j];
subject to const1 {i in Vertices}:
sum{j in Vertices} X[i,j] = 1;
subject to const2:
sum {i in Vertices, j in Vertices, k in Vertices} A[i,j] * X[i, k] * X[j, k] = 0;
data;
############ DATA STARTS HERE ############
param N := 5;
set Vertices := 1 2 3 4 5;
param A : 1 2 3 4 5 :=
1 0 1 1 0 1
2 1 0 0 1 0
3 1 0 0 0 0
4 0 1 0 0 1
5 1 0 0 1 0;

MINOS 5.51: ignoring integrality of 25 variables
MINOS 5.51: optimal solution found.
14 iterations, objective 90
Nonlin evals: constrs = 22, Jac = 21.
:    _varname      _var        :=
1    'X[1,1]'   0
2    'X[1,2]'   1
3    'X[1,3]'   0
4    'X[1,4]'   0
5    'X[1,5]'   0
6    'X[2,1]'   1
7    'X[2,2]'   0
8    'X[2,3]'   8.47636e-17
9    'X[2,4]'   0
10   'X[2,5]'   0
11   'X[3,1]'   1
12   'X[3,2]'   0
13   'X[3,3]'   0
14   'X[3,4]'   0
15   'X[3,5]'   0
16   'X[4,1]'   0
17   'X[4,2]'   1
18   'X[4,3]'   1.11022e-16
19   'X[4,4]'   0
20   'X[4,5]'   0
21   'X[5,1]'   1
22   'X[5,2]'   0
23   'X[5,3]'   0
24   'X[5,4]'   0
25   'X[5,5]'   0
;
== 7 ==========================
\end{verbatim}
\end{document}