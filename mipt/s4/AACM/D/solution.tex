\documentclass[a4paper]{article}
\usepackage[a4paper, left=5mm, right=5mm, top=5mm, bottom=5mm]{geometry}
%\geometry{paperwidth=210mm, paperheight=2000pt, left=5pt, top=5pt}
\usepackage[utf8]{inputenc}
\usepackage[english,russian]{babel}
\usepackage{indentfirst}
\usepackage{tikz} %Рисование автоматов
\usetikzlibrary{automata,positioning,arrows,trees,calc}
\usepackage{amsmath}
\usepackage[makeroom]{cancel} % зачеркивание
\usepackage{multicol,multirow} %Несколько колонок
\usepackage{hyperref}
\usepackage{tabularx}
\usepackage{amsfonts}
\usepackage{amssymb}
\DeclareMathOperator*{\argmin}{arg\,min}
\usepackage{listings}
\usepackage{wasysym}
\date{задано 2014.05.15}
\author{Сергей~Володин, 272 гр.}
\newcommand{\matrixl}{\left|\left|}
\newcommand{\matrixr}{\right|\right|}
% названия автоматов  (rubtsov)
\def\A{{\cal A}}
\def\B{{\cal B}}
\def\C{{\cal C}}

%классы сложности (rubtsov)
\def\PP{{\mathsf{P}}}
\def\NP{{\mathsf{NP}}}
\def\NPc{{\mathsf{NP}}\text{-}{\rm c}}
\def\coNP{{\rm co}\text{-}{\mathsf{NP}}}
\def\DTIME{{\mathsf{DTIME}}}
\def\NTIME{{\mathsf{NTIME}}}
\def\CLIQUE{{\mathsf{CLIQUE}}}
\def\HALT{{\rm{HALT}}}
\def\SAT{{\rm{SAT}}}
\def\3SAT{{\rm{3\text{-}SAT}}}
\def\2SAT{{\rm{2\text{-}SAT}}}
\def\UNSAT{{\rm{UNSAT}}}
\def\HP{{\rm{HAMPATH}}}
\def\UHP{{\rm{UHAMPATH}}}
\def\LL{{\mathrm{LL}}}
\def\poly{{\rm{poly}}}
\def\GC{{\mbox{ГЦ}}}
\def\GP{{\mbox{ГП}}}
\def\conv{{\mbox{conv}}}

\title{Алгоритмы и модели вычислений.\\Задание 13: Алгоритмы на графах II}

% алгоритмы (Рубцов)
\usepackage{verbatim}
\usepackage{listings}
\usepackage{algorithm2e}

%+= и -=, иначе разъезжаются...
\newcommand{\peq}{\mathrel{+}=}
\newcommand{\meq}{\mathrel{-}=}
\newcommand{\deq}{\mathrel{:}=}
\newcommand{\plpl}{\mathrel{+}+}

\newcommand{\todo}{{\em todo}}

% пустое слово  (rubtsov)
\def\eps{\varepsilon}

% регулярные языки  (rubtsov)
\def\REG{{\mathsf{REG}}}
\def\CFL{{\mathsf{CFL}}}
\def\eqdef{\overset{\mbox{\tiny def}}{=}}
\newcommand{\niton}{\not\owns}

%FIRST & FOLLOW (rubtsov)
\def\first{\mathrm{ FIRST} }
\def\follow{\mathrm{ FOLLOW} }

% LL LR (rubtsov)
\def\LL{{\mathrm{LL}}}
\def\LR{{\mathrm{LR}}}

\newcounter{rowItemCount}
\newcounter{subRowItemCount}
\newcommand\rowItem{
    \setcounter{subRowItemCount}{0}
    \arabic{rowItemCount}.\addtocounter{rowItemCount}{1}}
\newcommand\subRowItem{
    \addtocounter{subRowItemCount}{1}
    \addtocounter{rowItemCount}{-1}
    \arabic{rowItemCount}.\arabic{subRowItemCount}.\addtocounter{rowItemCount}{1}}

\newcommand{\smalll}[1]{\overline{\overline{#1}}}
\newcommand{\smallo}{\bar{\bar{o}}}
\newcommand{\ZZ}{\mathbb{Z}}
\newcommand{\Nz}{\mathbb{N}\cup\{0\}}
\newcommand{\NN}{\mathbb{N}}
\newcommand{\RR}{\mathbb{R}}
\begin{document}
\maketitle
\subsection*{(каноническое) Задача 48.0}
\begin{enumerate}
\item \href{https://bitbucket.org/etoestja/inf/raw/HEAD/mipt/s4/AACM/D/dfs.cpp}{Алгоритм}, \href{https://bitbucket.org/etoestja/inf/raw/HEAD/mipt/s4/AACM/D/test1}{вход}.
\item (обозначение: $v^{a/b}\Leftrightarrow d[v]=a,\,f[v]=b$)\newline
\begin{tikzpicture}[shorten >=1pt,node distance=2cm,on grid,auto,initial text=]
	  \node (v1) {$A^{1/14}$};
	  \node (v2) [right = of v1] {$B^{15/18}$};
	  \node (v3) [right = of v2] {$C^{16/17}$};
	  
	  \node (v4) [below = of v1] {$D^{2/5}$};
	  \node (v5) [right = of v4] {$E^{6/13}$};
	  \node (v6) [right = of v5] {$F^{7/12}$};
	  
	  \node (v7) [below = of v4] {$G^{3/4}$};
	  \node (v8) [right = of v7] {$H^{8/11}$};
	  \node (v9) [right = of v8] {$I^{9/10}$};
  	  \path[->]
			(v1)		edge node {$T$}	(v4)
			(v1)		edge node {$T$}	(v5)
			
			(v2)		edge node {$C$}	(v1)
			(v2)		edge[bend left] node {$T$}	(v3)
			(v2)		edge node {$C$}	(v5)
			(v2)		edge node[swap] {$C$}	(v6)
			
			(v3)		edge[bend left] node {$B$}	(v2)
			(v3)		edge node {$C$}	(v6)
			
			(v4)		edge node {$T$}	(v7)
			
			(v5)		edge node {$C$}	(v4)
			(v5)		edge node {$T$}	(v6)
			(v5)		edge node {$F$}	(v8)
			
			(v6)		edge node {$T$}	(v8)
			(v6)		edge node {$F$}	(v9)
			
			(v8)		edge node {$C$}	(v4)
			(v8)		edge node {$C$}	(v7)
			(v8)		edge node {$T$}	(v9)
			;
\end{tikzpicture}
\end{enumerate}
\subsection*{(каноническое) Задача 48}
\subsection*{(каноническое) Задача 49.1}
Зеленые~--- мосты, красные~--- точки раздела.\newline
\begin{tikzpicture}[shorten >=1pt,node distance=2cm,on grid,auto,initial text=]
	  \node (A) {$A$};
	  \node (B) [right = of A] {$B$};
	  \node (O) [below = of A] {$O$};
	  \node (N) [right = of O] {$N$};
	  \node (D) [right = of B, color=red] {$D$};
	  \node (E) [right = of D] {$E$};
	  \node (M) [below = of D] {$M$};
	  \node (L) [right = of M, color=red] {$L$};
	  \node (K) [below = of L] {$K$};
	  \node (J) [right = of K] {$J$};
	  \node (C) [above = of D] {$C$};
	  \node (I) [right = of E, color=red] {$I$};
	  \node (F) [above = of I] {$F$};
	  \node (H) [right = of I, color=red] {$H$};
	  \node (G) [above = of H] {$G$};
  	  \path[-]
				(A)		edge node {}	(B)
				(A)		edge node {}	(O)
				(B)		edge node {}	(D)
				(B)		edge node {}	(N)
				(C)		edge node {}	(D)
				(C)		edge node {}	(N)
				(D)		edge node {}	(M)
				(D)		edge node {}	(E)
				(D)		edge node {}	(L)
				(E)		edge node {}	(M)
				(E)		edge node {}	(L)
				(F)		edge node {}	(K)
				(F)		edge node {}	(I)
				(G)		edge [color=green] node {}	(H)
				(H)		edge [color=green] node {}	(I)
				(I)		edge node {}	(J)
				(J)		edge node {}	(L)
				(J)		edge node {}	(K)
				(K)		edge node {}	(L)
				(N)		edge node {}	(O)
			;
\end{tikzpicture}
\subsection*{(каноническое) Задача 50.1}
Пусть $r$~--- корень дерева поиска в глубину. Пусть $v_1,...,v_n$~--- его потомки, в порядке обхода.\begin{enumerate}
\item $n\geqslant 1$, так как если $n=0$, граф не связный (из $r$ нет ребер)
\item Пусть $r$~--- точка раздела. Пусть $n=1$. Тогда получаем $\forall v\in V\hookrightarrow$ существует путь $r\to v_1\rightsquigarrow v$, причем $r$ встречается в пути ровно один раз (путь в дереве), так как поиск в глубину находит достижимые из $r$ вершины, и только их. Значит, после удаления $r$ граф останется связным: для любой вершины $v\in V\setminus \{r\}$ существует путь $v_1\rightsquigarrow v$, не проходящий через $r$, откуда $\forall u,v\in V\setminus\{v\}\hookrightarrow u\rightsquigarrow v_1\rightsquigarrow v$. Противоречие. Значит, $n\geqslant 1$
\item Пусть $n>2$. Пусть вершины $u,v\colon r\to v_1\rightsquigarrow u$, $r\to v_2\rightsquigarrow v$ (из разных поддеревьев). Пусть $u\rightsquigarrow v$, путь не содержит $r$. Тогда $v$ была бы в первом поддереве (с корнем $v_1$)~--- противоречие. Значит, единственный путь $u\rightsquigarrow v$ проходит через $r$. Значит, $r$~--- точка раздела.
\end{enumerate}
\subsection*{(каноническое) Задача 50.2}
Граф $G$, совпадающий с $T$~--- деревом поиска в глубину:\newline
\begin{tikzpicture}[shorten >=1pt,node distance=2cm,on grid,auto,initial text=]
	  \node (A) {$A$};
	  \node (B) [below = of A] {$B$};
	  \node (C) [below = of B] {$C$};
  	  \path[-]
				(A)		edge node {}	(B)
				(B)		edge node {}	(C)
			;
\end{tikzpicture}
\begin{enumerate}
\item $B$ является точкой раздела
\item Но существует ребро $(B,A)$, которое является обратным ($A$ является предком $B$).
\end{enumerate}
% бляяяя
\end{document}
