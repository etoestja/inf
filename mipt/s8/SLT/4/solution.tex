\documentclass[a4paper]{article}
\usepackage[a4paper, left=5mm, right=5mm, top=5mm, bottom=5mm]{geometry}
%\geometry{paperwidth=210mm, paperheight=2000pt, left=5pt, top=5pt}
\usepackage[utf8]{inputenc}
\usepackage[english,russian]{babel}
\usepackage{indentfirst}
\usepackage{tikz}
\usepackage{cancel}
\usetikzlibrary{automata,positioning,arrows}
\usepackage{amsmath}
\usepackage{enumerate}
\usepackage{hyperref}
\usepackage{amsfonts}
\usepackage{amssymb}
\DeclareMathOperator*{\argmin}{arg\,min}
\DeclareMathOperator*{\argmax}{arg\,max}
\usepackage{wasysym}
\title{Статистическое обучение\\Задание 4}
\date{задано 2017.05.08}
\author{Сергей~Володин, 374 гр.}
\newcommand{\matrixl}{\left|\left|}
\newcommand{\matrixr}{\right|\right|}

\newcommand{\peq}{\mathrel{+}=}
\newcommand{\meq}{\mathrel{-}=}
\newcommand{\deq}{\mathrel{:}=}
\newcommand{\VC}{\mbox{VC}}
\newcommand{\plpl}{\mathrel{+}+}
\newcommand{\sign}{\mbox{sign}\,}
\newcommand{\F}{\mathcal{F}}
\newcommand{\R}{\mathbb{R}}
\newcommand{\N}{\mathcal{N}}
\newcommand{\conv}{\mbox{conv}\,}
\newcommand{\E}{\mathbb{E}}
\newcommand{\D}{\mathbb{D}}

% пустое слово
\def\eps{\varepsilon}

% регулярные языки
\def\eqdef{\overset{\mbox{\tiny def}}{=}}
\newcommand{\niton}{\not\owns}

\begin{document}
\maketitle
\section*{Meta}
Делал один. Список ссылок:
\begin{enumerate}
\item filled later
\end{enumerate}
\section*{Упражнение 1.1}
Рассмотрим
$$f\colon \R^d\to \R,\,f(\theta)=\|y-\theta\|_2^2+4\tau^2\|\theta\|_0$$

То есть, $f(\theta)=\sum\limits_{i=1}^d\left((y_i-\theta_i)^2+4\tau^2[\theta_i\neq0]\right)$

Рассмотрим одно слагаемое: $f_i=(y_i-\theta_i)^2+4\tau^2[\theta_i\neq 0]$. Есть два варианта\begin{enumerate}
	\item $\theta_i=0$. Тогда $f^*_i=y_i^2$
	\item $\theta_i\neq 0$ (при $y_i\neq 0$). Тогда $f_i=(y_i-\theta_i)^2+4\tau^2\to\min\limits_{\theta_i>0}$. Тогда $\theta_i=y_i$, и $f^*_i=4\tau^2$
\end{enumerate}
Поэтому, если $|y_i|<2\tau$, берём $\theta_i=0$. Иначе берем $\theta_i=y_i$. То есть,
$$
\theta_i^*=y_i[y_i>2\tau]
$$
\section*{Упражнение 1.2}
Рассмотрим $f(\theta)=\|y-\theta\|_2^2+4\tau\|\theta\|_1$. Пусть $\theta^*\in\argmin f(\theta)=\sum\limits_{i=1}^d\left((y_i-\theta_i)^2+4\tau |\theta_i|\right)$.
Рассмотрим одно слагаемое: $f_i(\theta_i)=(y_i-\theta_i)^2+4\tau|\theta_i|$
\begin{enumerate}
	\item Если $\theta_i^*=0$, то $f_i^*=y_i^2$.
	\item Иначе
$$\frac{d f_i}{d \theta_i}=2(-(y_i-\theta_i)+2\tau \sign\theta_i)=0$$
\item $\theta_i>0$. Тогда $-y_i+\theta_i+2\tau=0$ и $\theta_i=y_i-2\tau>0$. Здесь $f_i^*=4\tau(y_i-\tau)$
\item $\theta_i<0$. Тогда $-y_i+\theta_i-2\tau=0$ и $\theta_i=y_i+2\tau<0$. Здесь $f_i^*=4\tau (-y_i-\tau)$
\end{enumerate}

При $y_i<-2\tau$ получается $y_i^2-4\tau(-y_i-\tau)=(2\tau+y_i)^2>0$. Поэтому $\theta_i=y_i+2\tau$

При $y_i>2\tau$ получается $y_i^2-4\tau(y_i-\tau)=(y_i-2\tau) ^2\geqslant 0$. Поэтому $\theta_i=y_i-2\tau$

Иначе $\theta_i=0$

Это можно записать как:
$$
\theta_i^*=y_i\left(1-\frac{2\tau}{y_i}\right)_+
$$
\section*{Упражнение 2}
Имеем матрицу $X\colon n\times d$, параметр $\theta\in\R^d$, отклик $Y\in\mathbb{R}^n$. Функция потерь:

$$L(\theta)=\frac{1}{n}\|Y-X\theta\|_2^2+\tau\|\theta\|_2^2=\frac{1}{n}\sum\limits_{i=1}^n(y_i-\sum\limits_{s=1}^dx_{is}\theta_s)^2+\tau\sum\limits_{k=1}^d\theta_k^2$$

$L$ дифференцируема как композиция дифференцируемых функций. Найдём

$$\frac{\partial L(\theta)}{\partial\theta_j}=\frac{\partial}{\partial\theta_j}\left(\frac{1}{n}\sum\limits_{i=1}^n(y_i-\sum\limits_{s=1}^dx_{is}\theta_s)^2+\tau\sum\limits_{k=1}^d\theta_k^2\right)=2\theta_j\tau-\frac{1}{n}\sum\limits_{i=1}^n2(y_i-(X\theta)_i)x_{ij}=2\theta_j\tau-\frac{2}{n}(X^T(Y-X\theta))_j$$

Значит,
$$\nabla L(\theta)=2\theta\tau-\frac{2}{n}X^TY+\frac{2}{n}X^TX\theta$$

Решим
$$
\nabla L(\theta)=0\Leftrightarrow (n\tau E+X^TX)\theta=X^TY
$$
Поскольку $X^TX$~--- симметричная матрица, ее минимальное собственное число неотрицательное: $\lambda_{\min}(X^TX)\geqslant 0$. Значит, при $\tau>0$ матрица $n\tau E+X^TX$ обратима.

Получаем $$\hat{\theta}=(n\tau E+X^TX)^{-1}X^TY$$

Поскольку это единственный ноль градиента выпуклой функции, глобальный минимум находится в $\hat{\theta}$

Обозначим $Q=(X^TX+\beta I)^{-1}$, $\beta=\tau n$. Рассмотрим $\E \hat{\theta}=\E QX^TY=QX^T\E Y$. Поскольку $Y=X\theta^*+\eps$, $\E Y=X\theta^*$.

Тогда $\E\hat{\theta}=QX^TX\theta^*$.

Рассмотрим смещение $\Delta=\theta^*-\E\hat{\theta}=QQ^{-1}\theta^*QX^TX\theta^*=Q(X^TX+\beta I-X^TX)\theta^*=Q\beta\theta^*$.

$\|\Delta\|_2\leqslant \beta \|Q\|_2\|\theta^*\|_2$. Рассмотрим $\|Q\|_2=\|(X^TX+\beta I)^{-1}\|_2=\frac{1}{\|X^TX+\beta I\|_2}\leqslant \beta$.

Значит, $\|\Delta\|_2\leqslant \|\theta^*\|_2$
\section*{Упражнение 3}
{\em Идея оценки $\E e^{x_i^2}$ взята в \cite{massart}}

Рассмотрим $Z\sim\chi^2_k$ $\Leftrightarrow$ $Z\overset{d}{=}\sum\limits_{i=1}^k x_i^2$, где $\{x_i\}_{i=1}^k$~--- iid из $N(0,1)$.

Определим

$$Z_0=Z-k=\sum\limits_{i=1}^k(x_i^2-1)$$

\begin{enumerate}
	\item Рассмотрим одну из величин $x=x_i\sim N(0,1)$. Определим
$$
\psi(u)=\ln\E\exp(u(x^2-1))
$$

Найдём $\E \exp(ux^2)=\int\limits_{-\infty}^{+\infty}\frac{1}{\sqrt{2\pi}}e^{-\frac{x^2}{2}}e^{ux^2}dx=\int\limits_{-\infty}^{+\infty}\frac{1}{\sqrt{2\pi}}e^{-\frac{1}{2}(1-2u)x^2}dx=\sqrt{1-2u}$.

Тогда $\psi(u)=-u-\frac{1}{2}\ln(1-2u)$.

\item Определим $\varphi(u)=\frac{u^2}{1-2u}$. Докажем, что $\psi(u)\leqslant \varphi(u)$ при $u\in[0,\frac{1}{2}]$. Действительно, $\varphi(0)=\psi(0)=0$, и $\psi'(u)-\varphi'(u)=\frac{2u^2}{(1-2u)^2}\geqslant 0$ при $u\in[0,\frac{1}{2}]$.
\item Рассмотрим $\ln \E e^{uZ_0}=\ln\E \exp(u\sum\limits_{i=1}^k(x_i^2-1))=\ln\E\prod\limits_{i=1}^ke^{u(x_i^2-1)}\overset{\mbox{\tiny iid}}{=}\ln\prod\limits_{i=1}^k\E e^{u(x_i^2-1)}=\sum\limits_{i=1}^k\ln\E\exp(u(x_i^2-1))=k\psi(u)\overset{(2)}{\leqslant}n\varphi(u)=\frac{nu^2}{1-2u}$
\item Рассмотрим $P(Z\leqslant (1-\delta)\E Z)=P(\underbrace{Z-\E Z}_{Z_0}\leqslant \underbrace{-\delta\E Z}_t)\leqslant\inf\limits_{u>0}\frac{\E \exp(uZ_0)}{\exp(ut)}\leqslant\inf \frac{\exp(k\psi(u))}{\exp{ut}}$
\end{enumerate}
\section*{Упражнение 4}
Имеем $S_1(0)=\{x\in\R^n\big|\|x\|_2=1\}$, $\N_\eps\subseteq \R^n\colon\,\forall x\in S_1(0)\,\exists y\in\N_\eps\colon \|x-y\|\leqslant \eps$.
\begin{enumerate}
\item Рассмотрим $\|A\|_{op}=\sup\limits_{x\in S_1(0)}\|Ax\|$. Предположим, что в $\N_\eps$ есть точка вне сферы. Тогда можно перенести ее внутрь (на то же расстояние по нормали к сфере), и она будет покрывать больше точек. То есть, в минимальном покрытии можно взять все точки внутри сферы. Значит, $\|y\|\leqslant 1$. Поэтому $\sup\limits_{y\in\N_\eps}\|Ay\|\leqslant \sup\limits_{y\in \N_\eps}\|A\|_{op}\|y\|\leqslant \|A\|_{op}$ $\blacksquare$

Теперь, пусть норма $\|A\|_{op}$ достигается на $x^*$. То есть, $\|x^*\|=1$ ($x^*\in S_1(0)$) и $\|Ax^*\|=\|A\|$. Рассмотрим элемент $y\in\N_{\eps}$ такой что $\|y-x\|<\eps$. Рассмотрим $$\|Ay\|=\|A(y-x+x)\|=\|A(y-x)+Ax\|\geqslant \|A(y-x)+Ax\|\geqslant\|Ax\|-\underbrace{\|A(y-x)\|}_{\leqslant \|A\|\eps}\geqslant \|A\|(1-\eps)$$

То есть, нашли $y\in \N_\eps$ такой что $\|Ay\|\geqslant (1-\eps)\|A\|$. Значит, $\sup\limits_{y\in \N_\eps}\|Ay\|\geqslant (1-\eps)\|A\|$ $\blacksquare$
\end{enumerate}
\section*{Задача 1}
\section*{Задача 2}
\begin{thebibliography}{9}
\bibitem{massart}
Adaptive estimation of a quadratic functional by model selection
B. Laurent, P. Massart (2000)
\href{http://projecteuclid.org/download/pdf_1/euclid.aos/1015957395}{pdf}
\end{thebibliography}

\end{document}