\documentclass{article}
\usepackage[utf8]{inputenc}
\usepackage[russian]{babel}
\usepackage{graphicx}
\usepackage{indentfirst}
\usepackage{amsmath}
\usepackage{enumerate}
\usepackage{amsfonts}
\usepackage{amssymb}
\title{Задание от 2013.03.19}
\author{С.~Е.~Володин, 272 гр.}
\date{}
\begin{document}
\maketitle
\begin{enumerate} 
\item Задача 1
\begin{enumerate}
\item Докажем, что в таком массиве (в котором первый и последний элементы различны: $A[0]\neq A[N-1]$) существует искомая подстрока ($01$, либо $10$). Пусть иначе: $\nexists i:A[i]\neq A[i+1]\Leftrightarrow\forall i\hookrightarrow A[i]=A[i+1]$. Тогда, по индукции, $A[i]=A[N-1]$~--- противоречие.\newline
Рассмотрим элемент $A[m]$, $0<m<N-1$. Поскольку $A[0]\neq A[N-1]$, $A[0]\neq A[m] \vee A[m] \neq A[N-1]$. В первом случае по доказанному выше существует искомая подстрока в $(A[0],\dots,A[m])$, а во втором~--- в $(A[m],\dots,A[N-1])$. Выберем $m=\lfloor\frac{N}{2}\rfloor$. Тогда задача свелась к предыдущей с размерностью $\frac{N}{2}$.\newline
Реализация (двоичный поиск, рекурсия для простоты):\begin{verbatim}void find(l, r)
{
    if(l + 1 == r) ans = l;
    if(l + 1 >= r) return;

    int m = (l + r) / 2;
    if(A[l] != A[m]) find(l, m);
    else find(m, r);
}
\end{verbatim}
Время: $O(logN)$~--- из Master theorem, так как $T(N)=T(\frac{N}{2})+O(1)$

\item bla
\end{enumerate}
\item Задача 2.
\end{enumerate}

\end{document}
