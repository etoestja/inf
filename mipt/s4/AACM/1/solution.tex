\documentclass[a4paper]{article}
\usepackage[a4paper, left=5mm, right=5mm, top=5mm, bottom=5mm]{geometry}
%\geometry{paperwidth=210mm, paperheight=2000pt, left=5pt, top=5pt}
\usepackage[utf8]{inputenc}
\usepackage[english,russian]{babel}
\usepackage{indentfirst}
\usepackage{tikz} %Рисование автоматов
\usetikzlibrary{automata,positioning,arrows,trees}
\usepackage{amsmath}
\usepackage{enumerate}
\usepackage[makeroom]{cancel} % зачеркивание
\usepackage{multicol,multirow} %Несколько колонок
\usepackage{hyperref}
\usepackage{tabularx}
\usepackage{amsfonts}
\usepackage{amssymb}
\DeclareMathOperator*{\argmin}{arg\,min}
\usepackage{wasysym}
\title{Алгоритмы и модели вычислений.\\Задание 1: Алгоритмы и оценка сложности}
\date{задано 2014.02.13}
\author{Сергей~Володин, 272 гр.}
\newcommand{\matrixl}{\left|\left|}
\newcommand{\matrixr}{\right|\right|}
% названия автоматов  (rubtsov)
\def\A{{\cal A}}
\def\B{{\cal B}}
\def\C{{\cal C}}

%+= и -=, иначе разъезжаются...
\newcommand{\peq}{\mathrel{+}=}
\newcommand{\meq}{\mathrel{-}=}
\newcommand{\deq}{\mathrel{:}=}
\newcommand{\plpl}{\mathrel{+}+}

% пустое слово  (rubtsov)
\def\eps{\varepsilon}

% регулярные языки  (rubtsov)
\def\REG{{\mathsf{REG}}}
\def\CFL{{\mathsf{CFL}}}
\def\eqdef{\overset{\mbox{\tiny def}}{=}}
\newcommand{\niton}{\not\owns}

%FIRST & FOLLOW (rubtsov)
\def\first{\mathrm{ FIRST} }
\def\follow{\mathrm{ FOLLOW} }

% LL LR (rubtsov)
\def\LL{{\mathrm{LL}}}
\def\LR{{\mathrm{LR}}}

\newcounter{rowItemCount}
\newcounter{subRowItemCount}
\newcommand\rowItem{
    \setcounter{subRowItemCount}{0}
    \arabic{rowItemCount}.\addtocounter{rowItemCount}{1}}
\newcommand\subRowItem{
    \addtocounter{subRowItemCount}{1}
    \addtocounter{rowItemCount}{-1}
    \arabic{rowItemCount}.\arabic{subRowItemCount}.\addtocounter{rowItemCount}{1}}
    
\newcommand{\smalll}[1]{\overline{\overline{#1}}}
\newcommand{\smallo}{\bar{\bar{o}}}

\begin{document}
\maketitle
\subsection*{(каноническое) Задача 1}
$f(n) \eqdef \sum\limits_{i=1}^n\frac{1}{i}$, $g(n)=\log n$. Доказать: $f=\Theta(g)\Leftrightarrow \begin{cases}
f=O(g)\\
g=O(f)\\
\end{cases}\Leftrightarrow \begin{cases}
\exists C_1,n_1\colon\forall n\geqslant n_1\hookrightarrow f(n)\leqslant C_1 g(n) & (1)\\
\exists C_2,n_2\colon\forall n\geqslant n_2\hookrightarrow g(n)\leqslant C_2 f(n) & (2)\\
\end{cases}$
\begin{enumerate}
\item \label{c1_0} Докажем утверждение: пусть $f(n),g(n)\colon \exists n_0,C_1>0\colon \forall n\geqslant n_0\hookrightarrow \underbrace{f(n+1)-f(n)}_{\Delta_f(n)}\leqslant C_1\underbrace{g(n+1)-g(n)}_{\Delta_g(n)}$. Тогда $f=O(g)$. Действительно, выберем $C_2>0$ таким образом, что $f(n_0)\leqslant C_2g(n_0)$ (всегда можно сделать). Возьмем $C$ для определения $O$ как $C=\max(C_1,C_2)$. Докажем по индукции $\forall n\geqslant n_0\hookrightarrow f(n)\leqslant Cg(n)$:\begin{enumerate}
\item $f(n_0)\leqslant C_2g(n_0)\leqslant Cg(n_0)\blacksquare$
\item Пусть $f(n)\leqslant Cg(n)$. Докажем для $n+1$: по условию $\Delta_f(n)=f(n+1)-f(n)\leqslant C_1(g(n+1)-g(n))\leqslant C(g(n+1)-g(n))$. Перегруппируем, получим $f(n+1)-Cg(n+1)\leqslant f(n)-Cg(n)\overset{\mbox{предп.}}{\leqslant} 0$, т.е. $f(n+1)\leqslant Cg(n+1)\blacksquare$
\end{enumerate}
\item Докажем $(1)$. \begin{enumerate}
\item \label{c1_1_1} $\varangle \Delta_f(n)\eqdef f(n+1)-f(n)=\frac{1}{n+1}\leqslant \frac{1}{n}$. 
\item \label{c1_1_2} $\varangle \Delta_g(n)\eqdef g(n+1)-g(n)=\log(n+1)-\log n=\log\frac{n+1}{n}=\log(1+\frac{1}{n})=\frac{1}{n}+\smallo(\frac{1}{n})=\boxed{*}$, $n\to\infty$. Но по определению $\smallo$ $\exists n_1\colon\forall n\geqslant n_1\hookrightarrow \boxed{*}\geqslant \frac{1}{n}(1-\frac{1}{2})=\frac{1}{2}\frac{1}{n}$. Тогда $\frac{1}{n}\leqslant 2\boxed{*}=2(g(n+1)-g(n))$
\item Получаем $\Delta_f(n)=f(n+1)-f(n)\overset{\ref{c1_1_1}}{\leqslant} \frac{1}{n}\overset{\ref{c1_1_2}}{\leqslant} 2(g(n+1)-g(n))=2\Delta_g(n)$, и по $\ref{c1_0}$ получаем $f=O(g)$.
\end{enumerate}
\item Докажем $(2)$. \begin{enumerate}
\item \label{c1_2_1} $\varangle \Delta_f(n)=\frac{1}{n+1}$. Докажем, что это больше, чем $\frac{1}{2}\frac{1}{n}$: $\frac{1}{n+1}-\frac{1}{2}\frac{1}{n}=\frac{2n-n-1}{2n(n+1)}=\frac{n-1}{2n(n+1)}\geqslant 0$, $n\geqslant 1$. Итак, $\Delta_f(n)\geqslant \frac{1}{2}\frac{1}{n}$
\item \label{c1_2_2} $\ref{c1_1_2}\Rightarrow \Delta_g(n)=\frac{1}{n}+\smallo(\frac{1}{n})\leqslant \frac{1}{n}(1+\frac{1}{2})$ при $n\geqslant n_2>1$. Значит, $\frac{3}{2}\frac{1}{n}\geqslant\Delta_g(n)$
\item $\Delta_g(n)\overset{\ref{c1_2_2}}{\leqslant}\frac{3}{2}\frac{1}{n}\overset{\ref{c1_2_1}}{\leqslant}\frac{3}{2}\cdot 2\cdot \Delta_f(n)$ при $n\geqslant n_2$, и по $\ref{c1_0}$ получаем $g=O(f)$.
\end{enumerate}
\end{enumerate}
\subsection*{(каноническое) Задача 2}
$f(n)\eqdef C^{2n}_n\equiv\frac{(2n)!}{n!n!}$. Тогда $\ln f(n)=\ln(2n)!-2\ln n!\boxed{=}$. По формуле Стирлинга $\boxed{=}(2n)\ln(2n)-2n+O(\ln(2n))-n\ln n+n-O(\ln n)$. Поскольку $O(\ln(2n))-O(\ln n)\leqslant O(\ln(2n))=O(\ln 2+\ln n)=O(\ln n)$, получим $\boxed{=}2n(\ln 2+\ln n)-n-n\ln n+O(\ln n)=n\ln n-n(1-2\ln 2)+O(\ln n)$. Тогда $f(n)=e^{\cdots}=e^{n\ln n-n(1-2\ln 2)+O(\ln n)}=O(\frac{n^{n+1}}{e^{n}})$, так как $1-2\ln 2>1$.
{\bf ДОПИСАТЬ!}
\subsection*{(каноническое) Задача 3}
\begin{enumerate}
\item $T(n)=9T(\frac{n}{3})+f(n)$, $f(n)=\Theta(n^2\log n)$.
\begin{enumerate}
\item Докажем, что теорема неприменима. $a=9,b=3\Rightarrow \log_b a=\log_3 9=2$.
\begin{enumerate}
\item Если $\exists \eps>0\colon f(n)=O(n^{2-\eps})$, то $\exists C>0\exists n_0$, для $n\geqslant n_0$ получим $f(n)/n^{2-\varepsilon}\leqslant C>0$, то есть $n^2\log n/n^{2-\eps}\equiv n^\varepsilon\log n\leqslant C$, что неверно (функция неограничена сверху).
\item Если $f=\Theta(n^2)$, то $\exists n_0\exists C>0\colon f\leqslant Cn^2$ для $n\geqslant n_0$, и $\log n\leqslant C$, что неверно (функция неограничена сверху).
\item Если $\exists \eps>0\colon f=\Omega(n^{2-\eps})$, то $\exists n_0\colon\forall n\geqslant n_0\hookrightarrow f\geqslant Cn^{2+\eps}$, и $\log n\geqslant Cn^\eps$, откуда $\frac{\log^n}{n^\eps}\geqslant C>0$, что неверно, так как $\forall\eps>0\hookrightarrow\lim\limits_{n\to\infty}\frac{\log n}{n^\eps}=+0$
\end{enumerate}
\item Найдем ответ через дерево рекурсии. В корне ($i=0$) выполняется $n^2\log n$ операций, у каждой вершины 9 детей, на уровне $i+1$ $n_{i+1}=n_i/3$. У листьев (по индукции по высоте дерева) $1=n_h=\frac{n}{3^h}$, поэтому высота дерева {\em (не считая корня)} $h=\log_3n$. Найдем суммарное время:
$$T(n)=\Theta(n^2\log n+9(\frac{n}{3})^2\log\frac{n}{3}+9^2(\frac{n}{3^2})^2\log\frac{n}{3^2}+...+9^{h-1}(\frac{n}{3^{h-1}})^2\log\frac{n}{3^{h-1}})+9^hT(1)$$
Найдем сумму в аргументе $\Theta$: $\sum\limits_{i=0}^{h-1}9^i(\frac{n}{3^i})^2\log\frac{n}{3^i}=n^2\sum\limits_{i=0}^{h-1}(\log n-i\log 3)=n^2\log n(h-1)-n^2\frac{h-1}{2}\log 3=\newline=n^2\log n(\log_3n-1)-n^2\frac{\log_3n-1}{2}\log 3=n^2\log^2n-n^2\log n-n^2\log n+Cn^2=\Theta(n^2\log^2 n)$.\newline
Найдем $9^hT(1)=C9^{\log_3 n}=Cn^2$. Имеем $T(n)=\Theta(n^2\log^2n)+Cn^2=\boxed{\Theta(n^2\log^2n)}$
\end{enumerate}
\item $T(n)=16T(\frac{n}{4})+f(n)$, $f(n)=\Theta(n^2)$. $a=16,\,b=4$. Применим второй пункт Теоремы: $\Theta(n^{\log_b a})\equiv\Theta(n^2)$, поэтому $f(n)=\Theta(n^{\log_b a})$, и отсюда $T(n)=\boxed{\Theta(n^2\log n)}$.
\item $T(n)=4T(\frac{n}{2})+\Theta(\underbrace{\frac{n^2\sqrt{n}}{\log^2n}}_{g(n)})$. $a=4,\,b=2\Rightarrow\log_b a=2$. Возьмем $\eps=\frac{1}{4}$ и применим третий пункт Теоремы: $f(n)\overset{?}{=}\Omega(n^{2+\eps})$. Рассмотрим $\frac{f(n)}{n^{2+\eps}}=\frac{n^2\sqrt{n}}{n^2n^\eps\log^2 n}=\frac{n^{\frac{1}{2}-\eps}}{\log^2 n}=\frac{n^{1/4}}{\log^2 n}=(\frac{n^{1/8}}{\log_n})^2\overset{n\to\infty}{\longrightarrow}+\infty$, поэтому $\exists C>0\exists n_0>0\colon \forall n\geqslant n_0\hookrightarrow f(n)\geqslant C n^{2+\eps}$. Докажем, что $\exists 0<C<1\exists n_1\colon af(n/b)\leqslant Cf(n)$. $f=\Theta(g)\Rightarrow\exists n_2\colon\forall n\geqslant n_2\hookrightarrow C_1g(n)\leqslant f(n)\leqslant C_2 g(n)$. Тогда $af(\frac{n}{b})\leqslant 4C_2\frac{(\frac{n}{2})^\frac{5}{2}}{\log^2(\frac{n}{2})}=\frac{C_2}{\sqrt{2}C_1}\frac{\log^2n}{\log^2(\frac{n}{2})}C_1\frac{n^2\sqrt{n}}{\log^2 n}\leqslant \frac{C_2}{\sqrt{2}C_1}\frac{\log^2n}{\log^2(\frac{n}{2})}f(n)$. Значит, оценка верна, и по теореме получаем $T(n)=\boxed{\Theta(\frac{n^{5/2}}{\log^2n})}$
\end{enumerate}
Сравним первую и вторую функции: $\frac{n^2\log^2 n}{n^2\log n}=\log n\overset{n\to\infty}{\longrightarrow}+\infty$, поэтому первый алгоритм хуже. Сравним вторую и третью функции: $\frac{n^2\sqrt{n}}{\log^2n}\frac{1}{n^2\log n}=\frac{n^{1/2}}{\log^3 n}=(\frac{n^{1/6}}{\log n})^3\overset{n\to\infty}{\longrightarrow}+\infty$, поэтому третий алгоритм хуже.\newline
Ответ: $\boxed{\mbox{второй алгоритм}}$ имеет наименьшую асимптотическую стоимость.
\subsection*{(каноническое) Задача 4}
\begin{enumerate}
\item $T(n)=2T(\frac{n}{2})+\underbrace{n}_{f(n)}$. Воспользуемся пунктом $(2)$ Теоремы: $\log_b a=\log_2 2=1$, поэтому $f(n)\equiv n=\Theta(n^{\log_b a})\equiv\Theta(n)$.\newline
Ответ: $\boxed{T(n)=\Theta(n\log n)}$.
\item $T(n)=3T(\frac{n}{3})+\underbrace{n^2}_{f(n)}$. Воспользуемся пунктом $(3)$ Теоремы: $\log_b a=1$, $\lim\limits_{n\to\infty}\frac{f(n)}{n^{\log_b a+\eps}}=\lim\limits_{n\to\infty}n^{1-\eps}=+\infty$ например при $\eps=\frac{1}{2}$. Поэтому из определения предела для $\eps_{\mbox{\tiny lim}}=1\,\exists n_0>0\colon\forall n\geqslant n_0\hookrightarrow f(n)\geqslant \eps_{\mbox{\tiny lim}} n^{1+\eps}$, значит, $f(n)=\Omega(n^{1+\eps})$. Докажем условие регулярности: $af(\frac{n}{b})\equiv 2\frac{n^2}{2^2}=\frac{1}{2}n^2=\frac{1}{2}f(n)\leqslant\frac{1}{2}f(n)$, т.е. условие выполняется с $c=\frac{1}{2}<1$.\newline
Ответ: $\boxed{T(n)=\Theta(n^2)}$
\item $T(n)=4T(\frac{n}{2})+\underbrace{\frac{n}{\log n}}_{f(n)}$. Воспользуемся пунктом $(1)$ Теоремы: $\log_b a=\log_2 4=2$.\newline
Рассмотрим $\lim\limits_{n\to\infty}\frac{f(n)}{n^{\log_b a-\eps}}=\lim\limits_{n\to\infty}\frac{1}{n^{1-\eps}\log n}=0$ например при $\eps=\frac{1}{2}$. Из определения предела для $$\eps_{\mbox{\tiny lim}}=1\,\exists n_0\colon\forall n\geqslant n_0\hookrightarrow f(n)\leqslant \eps_{\mbox{\tiny lim}} n^{2-\eps},$$ откуда следует $f(n)=O(n^{2-\eps})$.\newline
Ответ: $\boxed{T(n)=\Theta(n^2)}$
\end{enumerate}
\subsection*{(каноническое) Задача 5}
\subsection*{Задача 1}
\begin{enumerate}
\item $T(n)=2T(\frac{n}{3})+f(n)$, $f(n)=\Theta(n^2)$. Дерево рекурсии:\newline
% code from 
\begin{tabular}{ll}
\begin{minipage}{0.5\textwidth}
\begin{tikzpicture}[scale=0.75,transform shape,level/.style={sibling distance = 5cm/#1, level distance = 1.5cm}]
\node [circle,draw] (z){$n^2$}
  child {node [circle,draw] (a) {$\frac{n^2}{3^2}$}
    child {node [circle,draw] (b) {$\frac{n^2}{3^4}$}
      child {node {$\vdots$}
        child {node [circle,draw] (d) {$T(1)$}}
        child {node [circle,draw] (e) {$T(1)$}}
      } 
      child {node {$\vdots$}}
    }
    child {node [circle,draw] (g) {$\frac{n^2}{3^4}$}
      child {node {$\vdots$}}
      child {node {$\vdots$}}
    }
  }
  child {node [circle,draw] (j) {$\frac{n^2}{3^2}$}
    child {node [circle,draw] (k) {$\frac{n^2}{3^4}$}
      child {node {$\vdots$}}
      child {node {$\vdots$}}
    }
  child {node [circle,draw] (l) {$\frac{n^2}{3^4}$}
    child {node {$\vdots$}}
    child {node (c){$\vdots$}
      child {node [circle,draw] (o) {$T(1)$}}
      child {node [circle,draw] (p) {$T(1)$}
          child [grow=right] {node (q) {$2^hT(1)$} edge from parent[draw=none]
          child [grow=up, level distance=0.7cm] {node (r) {$\vdots$} edge from parent[draw=none]
          child [grow=up, level distance=0.7cm] {node (r) {$2^k(\frac{n}{3^k})^2$} edge from parent[draw=none]
            child [grow=up, level distance=1cm] {node (r) {$\vdots$} edge from parent[draw=none]
              child [grow=up, level distance=0.6cm] {node (s) {$2^2(\frac{n}{3^2})^2$} edge from parent[draw=none]
                child [grow=up, level distance=1.5cm] {node (t) {$2(\frac{n}{3})^2$} edge from parent[draw=none]
                  child [grow=up, level distance=1.5cm] {node (u) {$n^2$} edge from parent[draw=none]}
                }
              }
              }
              }
            }
          }
        }
    }
  }
};
\end{tikzpicture}
\end{minipage}
&
\begin{minipage}{0.4\textwidth}
Рассмотрим рекуррентность. Последовательно подставляя $T(n)$ в правую часть, получим некоторую сумму $T(n)=\sum\limits_{i=0}^{h-1} C_i\cdot f(\frac{n}{3^i})+C_hT(1)$. Она конечна, так как аргумент $T(\cdot)$ в правой части меньше, чем в левой, причем в 3 раза. Прекращаем подставлять, когда аргумент станет равен 1. $C_i$~--- некоторые коэффициенты, найти которые можно при помощи дерева слева. Корень соответствует $i=0$ (база), та, каждый $i$-й уровень соответствует $i$-му слагаемому суммы. {\bf ДОПИСАТЬ ФОРМАЛЬНО}
При последней, $h$-й подстановке $\frac{n}{3^h}=1$, откуда $h=\log_3 n$.
\end{minipage}\\
\end{tabular}\newline
Таким образом, $T(n)=\overbrace{\sum\limits_{k=0}^{h-1}2^kf(\frac{n^2}{3^{2k}})}^{S}+2^hT(1)$.
\begin{enumerate}
\item Обозначим $g(n)=n^2$, по условию $f(n)=\Theta(g(n))$. Из определения $\Theta$ получаем $$\exists n_0>0,C_2>C_1>0\colon\forall n\geqslant n_0\hookrightarrow C_1g(n)\leqslant f(n)\leqslant C_2g(n)$$. Рассмотрим первую сумму $S$ при $n\geqslant n_0$:
\begin{equation}
n^2C_1\sum\limits_{k=0}^{h-1}\frac{2^k}{3^{2k}}\leqslant\overbrace{\sum\limits_{k=0}^{h-1}2^kf(\frac{n^2}{3^{2k}})}^S\leqslant n^2C_2\sum\limits_{k=0}^{h-1}\frac{2^k}{3^{2k}}
\end{equation}
Рассмотрим $S_1(n)\eqdef\sum\limits_{k=0}^{h-1}\frac{2^k}{3^{2k}}\underset{\mbox{\tiny прогр.}}{\overset{\mbox{\tiny геом.}}{=}}\frac{1-\frac{2^{h-1}}{9^{h-1}}}{1-\frac{2}{9}}\overset{n\to\infty}{\longrightarrow}\frac{1}{1-2/9}=\frac{9}{7}\eqdef l$. Здесь использовалось $h=\log_3 n\overset{n\to\infty}{\longrightarrow}+\infty$. Определение предела: $$\forall \eps>0\,\exists n_1(\eps)\colon\forall n\geqslant n_1\hookrightarrow S_1(n)\in U_\eps(l)$$
Фиксируем $\eps=\eps_0=l/2$, определим $n\eqdef \max{n_2(\eps_0),n_0}$. Тогда $\forall n\geqslant n_2\hookrightarrow 0<l-\eps\leqslant S_1(n)\leqslant l+\eps$.\newline
Снова рассмотрим (1):при $n\geqslant n_2$:
$n^2 C_1 (l-\eps)\leqslant n^2C_1\sum\limits_{k=0}^{h-1}\frac{2^k}{3^{2k}}\leqslant\underline{\sum\limits_{k=0}^{h-1}2^kf(\frac{n^2}{3^{2k}})}\leqslant n^2C_2\sum\limits_{k=0}^{h-1}\frac{2^k}{3^{2k}}\leqslant n^2 C_2 (l+\eps)$. Получаем $$S(n)=\Theta(n^2)$$
\item Рассмотрим $2^hT(1)=2^{\log_3 n}T(1)=n^{\log_3 2}\underbrace{T(1)}_{\mbox{\tiny const}}=\Theta(n^{\log_3 2})$
\item Получаем $T(n)=\Theta(n^2)+\Theta(n^{\log_3 2})=\Theta(n^2)$. Доказательство последнего равенства в конце работы (\ref{smallg2}) ($2>1>\log_3 2$, поэтому $n^{\log_3 2}=\smallo(n^2)$)
\end{enumerate}
Ответ: $\boxed{T(n)=\Theta(n^2)}$
\item $T(n)=4T(\frac{n}{3})+f(n)$, $f(n)=\Omega(n)$. Дерево рекурсии (все ветвления не показаны):\newline
\begin{tikzpicture}[scale=0.75,transform shape,level/.style={sibling distance = 5cm/#1, level distance = 1.5cm}]
\node [circle,draw] (z){$n$}
  child {node [circle,draw] (a) {$\frac{n}{3}$}
    child {node [circle,draw] (b) {$\frac{n}{3^2}$}
      child {node {$\vdots$}
        child {node [circle,draw] (d) {$T(1)$}}
        child {node [circle,draw] (e) {$T(1)$}}
      } 
      child {node {$\vdots$}}
    }
    child {node [circle,draw] (g) {$\frac{n}{3^2}$}
      child {node {$\vdots$}}
      child {node {$\vdots$}}
    }
  }
  child {node [circle,draw] (j) {$\frac{n}{3}$}
    child {node [circle,draw] (k) {$\frac{n}{3^2}$}
      child {node {$\vdots$}}
      child {node {$\vdots$}}
    }
  child {node [circle,draw] (l) {$\frac{n}{3^2}$}
    child {node {$\vdots$}}
    child {node (c){$\vdots$}
      child {node [circle,draw] (o) {$T(1)$}}
      child {node [circle,draw] (p) {$T(1)$}
          child [grow=right] {node (q) {$4^hT(1)$} edge from parent[draw=none]
          child [grow=up, level distance=0.7cm] {node (r) {$\vdots$} edge from parent[draw=none]
          child [grow=up, level distance=0.7cm] {node (r) {$4^k\frac{n}{3^k}$} edge from parent[draw=none]
            child [grow=up, level distance=1cm] {node (r) {$\vdots$} edge from parent[draw=none]
              child [grow=up, level distance=0.6cm] {node (s) {$4^2\frac{n}{3^2}$} edge from parent[draw=none]
                child [grow=up, level distance=1.5cm] {node (t) {$4\frac{n}{3}$} edge from parent[draw=none]
                  child [grow=up, level distance=1.5cm] {node (u) {$n$} edge from parent[draw=none]}
                }
              }
              }
              }
            }
          }
        }
    }
  }
};
\path (g) -- (b) node (ab) [midway] {$\cdots$};
\path (l) -- (k) node (ab) [midway] {$\cdots$};
\end{tikzpicture}\newline
Высота дерева $h=\log_3 n$, $T(n)=\sum\limits_{k=0}^{h-1}4^kf(\frac{n}{3^k})+4^hT(1)$.
Из определения $\Omega$ $\exists n_0\,\exists C>0\colon\forall n\geqslant n_0\hookrightarrow \sum\limits_{k=0}^{h-1}4^kf(\frac{n}{3^k})\geqslant Cn\sum\limits_{k=0}^{h-1}\frac{4^k}{3^k}\underset{\mbox{\tiny прогр.}}{\overset{\mbox{\tiny геом.}}{=}}Cn\frac{(4/3)^{h-1}-1}{4/3-1}=3Cn(\frac{3}{4}(\frac{4}{3})^{\log_3 n}-1)=3Cn(\frac{4}{3}\frac{n^{\log_3 4}}{n}-1)=4Cn^{\log_3 4}-3Cn$. Также $4^h=4^{\log_3 n}=n^{\log_3 4}$, поэтому $T(n)\geqslant 4Cn^{\log_3 4}-3Cn+n^{\log_3 4}T(1)$, откуда $T(n)=\Omega(n^{\log_3 4})$.\newline
Асимптотическую оценку сверху получить не удастся, так как $T(n)\geqslant f(n)$, и нет верхней оценки для $f(n)$.\newline
Ответ: $\boxed{T(n)=\Omega(n^{\log_3 4})}$
\item $T(n)=2T(\frac{n}{3})+f(n)$, $f(n)=O(n)$. Дерево рекурсии:\newline
\begin{tikzpicture}[scale=0.75,transform shape,level/.style={sibling distance = 5cm/#1, level distance = 1.5cm}]
\node [circle,draw] (z){$n$}
  child {node [circle,draw] (a) {$\frac{n}{3}$}
    child {node [circle,draw] (b) {$\frac{n}{3^2}$}
      child {node {$\vdots$}
        child {node [circle,draw] (d) {$T(1)$}}
        child {node [circle,draw] (e) {$T(1)$}}
      } 
      child {node {$\vdots$}}
    }
    child {node [circle,draw] (g) {$\frac{n}{3^2}$}
      child {node {$\vdots$}}
      child {node {$\vdots$}}
    }
  }
  child {node [circle,draw] (j) {$\frac{n}{3}$}
    child {node [circle,draw] (k) {$\frac{n}{3^2}$}
      child {node {$\vdots$}}
      child {node {$\vdots$}}
    }
  child {node [circle,draw] (l) {$\frac{n}{3^2}$}
    child {node {$\vdots$}}
    child {node (c){$\vdots$}
      child {node [circle,draw] (o) {$T(1)$}}
      child {node [circle,draw] (p) {$T(1)$}
          child [grow=right] {node (q) {$2^hT(1)$} edge from parent[draw=none]
          child [grow=up, level distance=0.7cm] {node (r) {$\vdots$} edge from parent[draw=none]
          child [grow=up, level distance=0.7cm] {node (r) {$2^k\frac{n}{3^k}$} edge from parent[draw=none]
            child [grow=up, level distance=1cm] {node (r) {$\vdots$} edge from parent[draw=none]
              child [grow=up, level distance=0.6cm] {node (s) {$2^2\frac{n}{3^2}$} edge from parent[draw=none]
                child [grow=up, level distance=1.5cm] {node (t) {$2\frac{n}{3}$} edge from parent[draw=none]
                  child [grow=up, level distance=1.5cm] {node (u) {$n$} edge from parent[draw=none]}
                }
              }
              }
              }
            }
          }
        }
    }
  }
};
\end{tikzpicture}\newline
Высота дерева $h=\log_3 n$. Получаем $T(n)=\sum\limits_{k=0}^{h-1}2^kf(\frac{n}{3^k})+2^hT(1)$. По определению $O$ $\exists n_0>0\,\exists C>0\colon\forall n\geqslant n_0\hookrightarrow \sum\limits_{k=0}^{h-1}2^kf(\frac{n}{3^k})\leqslant Cn\sum\limits_{k=0}^{h-1}(\frac{2}{3})^k\leqslant Cn\frac{1}{1-2/3}=3Cn=O(n)$.
Оценим $2^hT(1)=2^{\log_3 n}T(1)=n^{\log_3 2}T(1)=O(n^{\log_3 2})$.
Получаем $T(n)\leqslant O(n)+O(n^{\log_3 2})$. Но $\log_3 2<1$, поэтому $n^{\log_3 2}=\smallo(n)$, и по \ref{smallg2O} получаем $T(n)=O(n)$.\newline
С другой стороны, $T(n)\geqslant 2^hT(1)=\Omega(n^{\log_3 2})$.\newline
Ответ: $\boxed{T(n)=O(n),\,T(n)=\Omega(n^{\log_3 2})}$
\end{enumerate}
\subsection*{Задача 2}
\subsection*{Задача 3}
\subsection*{Вспомогательные утверждения}
\begin{enumerate}
\item \label{smallg2} Пусть $f_1=\Theta(g_1)$, $f_2=\Theta(g_2)$, $g_2=\smallo(g_1), g_2(n)>0$. Тогда $f_1+f_2=\Theta(g_1)$. Доказательство:\newline
Из определения $\Theta$ получаем $\exists n_0\,\exists C_i^j>0, (i,j)\in \overline{1,2}^2\colon\forall n\geqslant n_0 \left\{\begin{array}{lllll}
C_1^1g_1(n) & \leqslant & f_1(n) & \leqslant & C_2^1 g_1(n)\\
C_1^2g_2(n) & \leqslant & f_2(n) & \leqslant & C_2^2 g_2(n)\\
\end{array}\right.$ ($n_0$~--- максимальное из двух определений).
Тогда $$C_1^1\overset{n\to\infty}{\longleftarrow}C_1^1+C_1^2\cancelto{0}{\frac{g_2(n)}{g_1(n)}}=\frac{C_1^1 g_1(n)+C_1^2 g_2(n)}{g_1(n)}\leqslant\underline{\frac{f_1(n)+f_2(n)}{g_1(n)}}\leqslant \frac{C_2^1 g_1(n)+C_2^2 g_2(n)}{g_1(n)}=C_2^1+C_2^2\cancelto{0}{\frac{g_2(n)}{g_1(n)}}\overset{n\to\infty}{\longrightarrow}C_2^1$$
Здесь использовалось определение $\smallo$. Из определения предела для $\eps=\eps_0=\min(C_1^1,C_2^1)/2$ получаем при $n\geqslant n_0(\eps)$ $(C^1_1-\eps)g_1(n)\leqslant f_1(n)+f_2(n)\leqslant (C_2^1+\eps)g_1(n)$, а из этого следует $f_1+f_1=\smallo(g_1)\blacksquare$
\item \label{smallg2O} Пусть $f_1=O(g_1)$, $f_2=O(g_2)$, $g_2=\smallo(g_1), g_2(n)>0$. Тогда $f_1+f_2=O(g_1)$. Доказательство выше (нужно взять правую часть большого неравенства).
\end{enumerate}
\end{document}