\documentclass[a4paper]{article}
\usepackage[a4paper, left=5mm, right=5mm, top=5mm, bottom=5mm]{geometry}
%\geometry{paperwidth=210mm, paperheight=2000pt, left=5pt, top=5pt}
\usepackage[utf8]{inputenc}
\usepackage[english,russian]{babel}
\usepackage{indentfirst}
\usepackage{tikz}
\usepackage{cancel}
\usetikzlibrary{automata,positioning,arrows}
\usepackage{amsmath}
\usepackage{enumerate}
\usepackage{hyperref}
\usepackage{amsfonts}
\usepackage{amssymb}
\DeclareMathOperator*{\argmin}{arg\,min}
\usepackage{wasysym}
\title{Статистическое обучение\\Задание 2}
\date{задано 2017.02.12}
\author{Сергей~Володин, 374 гр.}
\newcommand{\matrixl}{\left|\left|}
\newcommand{\matrixr}{\right|\right|}

\newcommand{\peq}{\mathrel{+}=}
\newcommand{\meq}{\mathrel{-}=}
\newcommand{\deq}{\mathrel{:}=}
\newcommand{\plpl}{\mathrel{+}+}
\newcommand{\sign}{\mbox{sign}\,}
\newcommand{\F}{\mathcal{F}}
\newcommand{\R}{\mathbb{R}}
\newcommand{\E}{\mathbb{E}}
\newcommand{\D}{\mathbb{D}}

% пустое слово
\def\eps{\varepsilon}

% регулярные языки
\def\eqdef{\overset{\mbox{\tiny def}}{=}}
\newcommand{\niton}{\not\owns}

\begin{document}
\maketitle
\section*{Meta}
Делал один. Список ссылок:
\begin{enumerate}
\item blah
\end{enumerate}
\section*{Упражнение 1.1}
Класс функций $S=\{f\colon [0,1]\to\{0,1\}\big| \#\{x|f(x)=1\}<+\infty\}$

$R(S)=\frac{1}{n}\mathbb{E}_{x_i}\mathbb{E}_{\eps_i}\sup\limits_{f\in S}\big|\sum\limits_{i=1}^n\eps_if(x_i)\big|$

Фиксируем $x_i$ и $\eps_i$. Необходимо максимизировать $|\sum \eps_i q_i|$, где $\eps_i\in\{-1,1\}$, $q_i\in\{0,1\}$.

Тогда $\sup|\sum \eps_i q_i|=\max\{\#+1,\#-1\}=\max\{k,n-k\}$, где $k$~--- число $\eps_i=+1$

Тогда $ $
\section*{Упражнение 1.2}

\section*{Упражнение 2.1}
$n\leqslant d,\,\mathcal{F}=\{x\to \sign(\theta^Tx)\big| \theta\in\R^d\}$

Рассмотрим $R_n(\F)=\frac{1}{n}\mathbb{E}_{\eps_i}\sup\limits_{f\in\F}|\sum\limits_{i=1}^n \eps_i f(x_i)|$

Заметим, что $\boxed{R_n(\F)\leqslant 1}$.

Фиксируем $\{x_i\}$ и $\{\eps_i\}$. Выберем $\theta$: $\underbrace{\matrixl\begin{array}{c}
x_1^T\\
...\\
x_n^T
\end{array}\matrixr}_X\theta=\matrixl\begin{array}{c}
\eps_1\\
...\\
\eps_n
\end{array}
\matrixr$. Это можно сделать по теореме Кронекера-Капелли: матрица $X\colon n\times d$ имеет ранг $n\leqslant d$, значит, при приписывании столбца $(\eps_1,...,\eps_n)$ ранг также будет равен $n$.

Мы нашли $f\in\F\colon $ $|\sum\limits_{i=1}^n \eps_i f(x_i)|=|\sum\limits_{i=1}^n \eps_i \sign(\theta^Tx_i)|=\sum\limits_{i=1}^n \eps_i^2=n$. Значит, $\boxed{R_n(\F)\geqslant 1}$ $\blacksquare$
\section*{Упражнение 2.2}
\section*{Упражнение 3}
$$
\begin{array}{ccccc}
G(F)&=&\frac{1}{n}\E_X&\E_{g_i\sim N(0,1)}&\sup\limits_{f\in \F}|\sum\limits_{i=1}^n g_if(x_i)|\\
R(F)&=&\frac{1}{n}\E_X&\E_{\eps}&\sup\limits_{f\in \F}|\sum\limits_{i=1}^n \eps_if(x_i)|
\end{array}$$

Фиксируем $\{x_i\}$. Рассмотрим $\varphi(\vec{t})=\sup\limits_{f\in\F}|\sum\limits_{i=1}^n f(x_i)\eps_i t_i|$. Тогда $\varphi(\alpha\vec{t}_1+(1-\alpha)\vec{t}_2)=\sup\limits_{f\in\F}|\alpha\sum\limits_{i=1}^n f(x_i)\eps_i t_{1i}+(1-\alpha)\sum\limits_{i=1}^n f(x_i)\eps_i t_{2i}|\boxed{\leqslant}$. $|x|$~--- выпуклая, поэтому $\boxed{\leqslant}\sup f\in F\sup\limits_{f\in\F}\alpha|\sum\limits_{i=1}^n f(x_i)\eps_i t_{1i}|+(1-\alpha)|\sum\limits_{i=1}^n f(x_i)\eps_i t_{2i}|\leqslant \alpha\varphi(\vec{t}_1)+(1-\alpha)\varphi(\vec{t}_2)$.

Пусть $\{g_i\}_{i=1}^n\colon g_i\sim N(0,1)$ и $\{g_i\}$~--- i.i.d.

Обозначим $c=\mathbb{E}|g_i|=2\int\limits_{x=0}^\infty x e^{-x^2/2}dx >0$. Тогда $\E g_i/c=1$.

Рассмотрим $\varphi(\vec{1})=\varphi(\E\vec{g}/c)=\sup\limits_{f\in F}|\sum\limits_{i=1}^n \eps_i f(x_i) \underbrace{\frac{\E g_i}{c}}_1|\leqslant \E_{g}\frac{1}{c}\sup\limits_{f\in \F}|\sum\limits_{i=1}^n \eps_if(x_i)|g_i||$.

Рассмотрим $R_n(\F)=\frac{1}{n}\E_x\E_\eps\sup\limits_{f\in\F}|\sum\limits_{i=1}^n \eps_if(x_i)|=\frac{1}{n}\E_x\E_\eps \varphi(\vec{1})\leqslant \frac{1}{nc}\E_x\E_\eps\E_g \sup\limits_{f\in \F}|\sum\limits_{i=1}^n \eps_if(x_i)|g_i||\overset{d}{=}$. Поскольку $\eps |g|\overset{d}{=}g$,

$\overset{d}{=}\frac{1}{nc}\E_x\E_g\sup\limits_{f\in \F}|\sum\limits_{i=1}^n f(x_i)g_i|=\frac{1}{c}G(\F)$.

Значит, $R(\F)\leqslant CG(\F)$, где $C=1/c$.
\section*{Упражнение 4.1}
Все функции $X\to \{-1,1\}$, то есть, $\F=\{-1,1\}^X$
\section*{Упражнение 4.2}
\section*{Упражнение 4.3}
\section*{Упражнение 4.4}
\section*{Упражнение 4.5}
\section*{Упражнение 5.1}
Contraction: http://www.cs.nyu.edu/~mohri/mls/lecture_4.pdf
\section*{Упражнение 5.2}
\section*{Задача 1}
\section*{Задача 2}
2017.03.16
\section*{Задача 3.1}

\section*{Задача 3.2}
\end{document}