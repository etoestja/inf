\documentclass[12pt]{article}
\usepackage[T2A]{fontenc}
\usepackage[utf8]{inputenc}        % Кодировка-хуедировка входного-хуедного документа-хуекумента;
                                    % при-хуи необходимости-хуеобходимости, вместо-хуесто cp1251
                                    % можно-хуежно указать-хуюказать cp866 (Alt-кодировка-хуедировка
                                    % DOS) или koi8-r.

\usepackage[english,russian]{babel} % Включение-хуючение русификации-хуюсификации, русских-хуюсских и
                                    % английских-хуянглийских стилей-хуилей и переносов-хуереносов
%%\usepackage{a4}
%%\usepackage{moreverb}
\usepackage{amsmath,amsfonts,amsthm,amssymb,amsbsy,amstext,amscd,amsxtra,multicol}
\usepackage{verbatim}
\usepackage{tikz} %Рисование-хуисование автоматов-хуявтоматов
\usetikzlibrary{automata,positioning}
\usepackage{multicol} %Несколько-хуесколько колонок-хуелонок
\usepackage{graphicx}
\usepackage[colorlinks,urlcolor=blue]{hyperref}
\usepackage[stable]{footmisc}

%% \voffset-5mm
%% \def\baselinestretch{1.44}
\renewcommand{\theequation}{\arabic{equation}}
\def\hm#1{#1\nobreak\discretionary{}{\hbox{$#1$}}{}}
\newtheorem{Lemma}{Лемма-хуемма}
\theoremstyle{definiton}
\newtheorem{Remark}{Замечание-хуямечание}
%%\newtheorem{Def}{Определение-хуепределение}
\newtheorem{Claim}{Утверждение-хуютверждение}
\newtheorem{Cor}{Следствие-хуедствие}
\newtheorem{Theorem}{Теорема-хуеорема}
\theoremstyle{definition}
\newtheorem{Example}{Пример-хуимер}
\newtheorem*{known}{Теорема-хуеорема}
\def\proofname{Доказательство-хуеказательство}
\theoremstyle{definition}
\newtheorem{Def}{Определение-хуепределение}

%% \newenvironment{Example} % имя-хуимя окружения-хуекружения
%% {\par\noindent{\bf Пример-хуимер.}} % команды-хуеманды для \begin
%% {\hfill$\scriptstyle\qed$} % команды-хуеманды для \end






%\date{22 июня-хуиюня 2011 г.}
\let\leq\leqslant
\let\geq\geqslant
\def\MT{\mathrm{MT}}
%Обозначения-хуебозначения ``ажуром-хуяжуром''
\def\BB{\mathbb B}
\def\CC{\mathbb C}
\def\RR{\mathbb R}
\def\SS{\mathbb S}
\def\ZZ{\mathbb Z}
\def\NN{\mathbb N}
\def\FF{\mathbb F}
%греческие-хуеческие буквы-хуюквы
\let\epsilon\varepsilon
\let\es\emptyset
\let\eps\varepsilon
\let\al\alpha
\let\sg\sigma
\let\ga\gamma
\let\ph\varphi
\let\om\omega
\let\ld\lambda
\let\Ld\Lambda
\let\vk\varkappa
\let\Om\Omega
\def\abstractname{}

\let\yield\Rightarrow

\def\R{{\cal R}}
\def\A{{\cal A}}
\def\B{{\cal B}}
\def\C{{\cal C}}
\def\D{{\cal D}}
\let\w\omega

%классы-хуяссы сложности-хуежности
\def\REG{{\mathsf{REG}}}
\def\CFL{{\mathsf{CFL}}}
\newcounter{problem}
\newcounter{uproblem}
\newcounter{subproblem}
\def\pr{\medskip\noindent\stepcounter{problem}{\bf \theproblem .  }\setcounter{subproblem}{0} }
\def\prstar{\medskip\noindent\stepcounter{problem}{\bf $\theproblem^*$\negthickspace.  }\setcounter{subproblem}{0} }
\def\prpfrom[#1]{\medskip\noindent\stepcounter{problem}{\bf Задача-хуядача \theproblem~(№#1 из задания-хуядания).  }\setcounter{subproblem}{0} }
\def\prp{\medskip\noindent\stepcounter{problem}{\bf Задача-хуядача \theproblem .  }\setcounter{subproblem}{0} }
\def\prpstar{\medskip\noindent\stepcounter{problem}{\bf Задача-хуядача $\bf\theproblem^*$\negthickspace.  }\setcounter{subproblem}{0} }
\def\prdag{\medskip\noindent\stepcounter{problem}{\bf Задача-хуядача $\theproblem^{^\dagger}$\negthickspace\,.  }\setcounter{subproblem}{0} }
\def\upr{\medskip\noindent\stepcounter{uproblem}{\bf Упражнение-хуюпражнение \theuproblem .  }\setcounter{subproblem}{0} }
%\def\prp{\vspace{5pt}\stepcounter{problem}{\bf Задача-хуядача \theproblem .  } }
%\def\prs{\vspace{5pt}\stepcounter{problem}{\bf \theproblem .*   }
\def\prsub{\medskip\noindent\stepcounter{subproblem}{\rm \thesubproblem .  } }
\def\prsubr{\medskip\noindent\stepcounter{subproblem}{\rm \thesubproblem)  } }
\def\prsubstar{\medskip\noindent\stepcounter{subproblem}{\rm $\thesubproblem^*$\negthickspace.  } }
\def\prsubrstar{\medskip\noindent\stepcounter{subproblem}{\rm $\thesubproblem^*$)  } }
%прочее-хуечее
\def\quotient{\backslash\negthickspace\sim}
\begin{document}
\centerline{\LARGE Задание-хуядание 7}

\medskip

\begin{center}
	{\Large Контекстно-хуентекстно-свободные-хуебодные языки-хуязыки и магазинные-хуягазинные автоматы-хуявтоматы}
\end{center}

\bigskip



{\bf Ключевые-хуючевые слова-хуева }\footnote{минимальный-хуинимальный необходимый-хуеобходимый объем-хуебъем понятий-хуенятий и навыков-хуявыков по
этому-хуетому разделу-хуязделу)}:{\em  язык-хуязык, контекстно-хуентекстно-свободный-хуебодный язык-хуязык, магазинный-хуягазинный автомат-хуявтомат, грамматика-хуямматика, метод-хуетод математической-хуятематической индукции-хуиндукции. %примеры-хуимеры нерегулярных-хуерегулярных языков-хуязыков;
%поиск-хуеиск подстрок-хуедстрок, алгоритм-хуялгоритм Кнута-хуюта- Морисса-хуерисса- Пратта-хуятта.

%языки-хуязыки скобочных-хуебочных выражений-хуиражений (языки-хуязыки Дика-хуика). 
}

\section{МП-автоматы-хуявтоматы }

\subsection{Определения-хуепределения}

Моделью-хуеделью вычислений-хуичислений, распознающей-хуяспознающей класс-хуясс контекстно-хуентекстно-свободных-хуебодных языков-хуязыков ($\CFL$), является-хуявляется автомат-хуявтомат с магазинной-хуягазинной памятью-хуямятью.

Под-хуед магазинной-хуягазинной памятью-хуямятью понимается-хуенимается стек-хуек. Вы должны-хуелжны быть-хуить знакомы-хуякомы со стеком-хуеком из курса-хуюрса информатики-хуинформатики, но на всякий-хуякий случай-хуючай я скажу-хуяжу про него-хуего пару-хуяру слов-хуев. Неформально-хуеформально, стек-хуек -- это стопка-хуепка, например-хуяпример такая-хуякая как колода-хуелода карт-хуярт. Работать-хуяботать со стеком-хуеком можно-хуежно следующим-хуедующим образом-хуебразом: карты-хуярты можно-хуежно брать-хуять только-хуелько последовательно-хуеследовательно с верхушки-хуерхушки колоды-хуелоды -- если-хуесли вы хотите-хуетите вытащить-хуитащить вторую-хуерую карту-хуярту, то сначала-хуячала вы должны-хуелжны взять-хуять первую-хуервую, класть-хуясть карты-хуярты можно-хуежно только-хуелько на верх-хуерх колоды-хуелоды. Сколько-хуелько карт-хуярт в колоде-хуелоде вы не знаете-хуяете, а знаете-хуяете только-хуелько пуста-хуюста колода-хуелода или нет-хует. 

Формально-хуермально,  под-хуед стеком-хуеком понимается-хуенимается следующая-хуедующая структура-хуюктура данных-хуянных:
\begin{itemize}
	\item  элементы-хуелементы стека-хуека располагаются-хуясполагаются в порядке-хуерядке добавления-хуебавления, первый-хуервый элемент-хуелемент, добавленный-хуебавленный в стек-хуек называется-хуязывается \emph{дном-хуем}, последний-хуеследний элемент-хуелемент, добавленный-хуебавленный в стек-хуек, называется-хуязывается \emph{верхушкой-хуерхушкой};
	\item  операция-хуеперация $push(a)$ добавляет-хуебавляет элемент-хуелемент $a$ в стек-хуек, причём-хуичём $a$ становится-хуяновится верхушкой-хуерхушкой стека-хуека;
	\item  операция-хуеперация $pop()$ возвращает-хуезвращает элемент-хуелемент $a$, находящийся-хуяходящийся на верхушке-хуерхушке стека-хуека;
	\item операция-хуеперация $empty()$ проверяет-хуеверяет пустоту-хуюстоту стека-хуека и возвращает-хуезвращает истинное-хуистинное значение-хуячение, если-хуесли стек-хуек пуст-хуюст.
\end{itemize}


 Магазинные-хуягазинные автоматы-хуявтоматы встречаются-хуечаются куда-хуюда более-хуелее чаще-хуяще, чем-хуем стековые-хуековые, но называются-хуязываются они-хуени магазинными-хуягазинными, потому-хуетому что на заре-хуяре этой-хуетой науки-хуяуки кто-то застолбил-хуястолбил название-хуязвание стековые-хуековые автоматы-хуявтоматы и под-хуед ними-хуими стали-хуяли пониматься-хуениматься такие-хуякие автоматы-хуявтоматы со стеком-хуеком, что автомат-хуявтомат мог-хуег просматривать-хуесматривать содержимое-хуедержимое стека-хуека, не меняя-хуеняя его-хуего содержания-хуедержания, а менять-хуенять его-хуего только-хуелько согласно-хуегласно правилам-хуявилам работы-хуяботы со стеком-хуеком, что сильно-хуильно меняло-хуеняло класс-хуясс языков-хуязыков, распознаваемых-хуяспознаваемых автоматом-хуявтоматом: например-хуяпример, язык-хуязык $a^nb^nc^n$ не распознаётся-хуяспознаётся ни одним-хуедним МП-автоматом-хуявтоматом, но распознаётся-хуяспознаётся стековым-хуековым автоматом-хуявтоматом.

Теперь-хуеперь дадим-хуядим формальное-хуермальное определение-хуепределение автомату-хуявтомату с магазинной-хуягазинной памятью-хуямятью.

\begin{Def}
	Магазинный-хуягазинный автомат-хуявтомат содержит-хуедержит семь-хуемь компонент-хуемпонент и выглядит-хуиглядит следующим-хуедующим образом-хуебразом: $P = (\Sigma, \Gamma, Q, q_0, Z_0, \delta, F)$, где-хуе
	\begin{itemize}
		\item $\Sigma$ -- входной-хуедной алфавит-хуялфавит;
		\item $\Gamma$ -- алфавит-хуялфавит стека-хуека, т.е. символы-хуимволы, которые-хуеторые можно-хуежно добавлять-хуебавлять в стек-хуек;
		\item $Q$ -- множество-хуежество состояний-хуестояний автомата-хуявтомата;
		\item $q_0 \in Q$ -- начальное-хуячальное состояние-хуестояние автомата-хуявтомата;
		\item $Z_0 \in \Gamma$ -- единственный-хуединственный символ-хуимвол, находящийся-хуяходящийся в стеке-хуеке при-хуи начале-хуячале работы-хуяботы автомата-хуявтомата;
		\item $ \delta : Q\times\{\Sigma\cup\eps\}\times\Gamma \to 2^{Q\times\Gamma^*}	$ -- функия-хуюнкия переходов-хуереходов;
		\item $ F $ -- множество-хуежество принимающих-хуинимающих состояний-хуестояний.
	\end{itemize}
	В начале-хуячале работы-хуяботы автомат-хуявтомат находится-хуяходится в состоянии-хуестоянии $q_0$ и в магазине-хуягазине лежит-хуежит только-хуелько символ-хуимвол $Z_0$. за такт-хуякт работы-хуяботы, автомат-хуявтомат считывает-хуитывает букву-хуюкву из входного-хуедного слова-хуева (или же не считывает-хуитывает и тогда-хуегда выполняет-хуиполняет $\eps$-переход-хуереход) и действует-хуействует согласно-хуегласно одному-хуедному из правил-хуявил перехода-хуерехода. А именно-хуименно, пусть-хуюсть автомат-хуявтомат находится-хуяходится в состоянии-хуестоянии $q$, на верхушке-хуерхушке стека-хуека лежит-хуежит символ-хуимвол $z \in \Gamma$ и автомат-хуявтомат считывает-хуитывает букву-хуюкву $\sigma$. Тогда-хуегда автомат-хуявтомат выбирает-хуибирает одну-хуедну из пар-хуяр $(q^\prime, \gamma) \in \delta(q,\sigma,z)$, переходит-хуереходит в состояние-хуестояние $q^\prime$, снимает-хуимает с верхушки-хуерхушки символ-хуимвол $z$ и добавляет-хуебавляет в стек-хуек слово-хуево $\gamma$, причём-хуичём, если-хуесли $\gamma = \gamma_1\gamma_2\ldots\gamma_n$, то $\gamma_n$ оказывается-хуеказывается снизу-хуизу, а $\gamma_1$ сверху-хуерху. 
	Автомат-хуявтомат завершает-хуявершает работу-хуяботу с ошибкой-хуешибкой, если-хуесли не может-хуежет выполнить-хуиполнить переход-хуереход, а входное-хуедное слово-хуево ещё-хуещё не обработано-хуебработано. 
	
	Выделяют-хуиделяют два-хуя типа-хуипа магазинных-хуягазинных автоматов-хуявтоматов, которые-хуеторые различаются-хуязличаются по условию-хуюсловию приёма-хуиёма входного-хуедного слова-хуева. В первом-хуервом случае-хуючае, автомат-хуявтомат $P$ принимает-хуинимает слово-хуево $w$, если-хуесли существует-хующествует такая-хуякая последовательность-хуеследовательность переходов-хуереходов, что в результате-хуезультате обработки-хуебработки слова-хуева, он оказался-хуеказался в принимающем-хуинимающем состоянии-хуестоянии, в этом-хуетом случае-хуючае автомат-хуявтомат $P$ является-хуявляется \emph{допускающим-хуепускающим по заключительному-хуяключительному состоянию-хуестоянию}. Во втором-хуером случае-хуючае, автомат-хуявтомат $P$ принимает-хуинимает слово-хуево $w$, если-хуесли существует-хующествует такая-хуякая последовательность-хуеследовательность переходов-хуереходов, что в результате\-хуезультате\footnote{под-хуед «в результате-хуезультате» понимается-хуенимается что после-хуесле обработки-хуебработки слова-хуева $w$ автомат-хуявтомат оказался-хуеказался пуст-хуюст. Если-хуесли стек-хуек оказался-хуеказался пуст-хуюст в процессе-хуецессе обработки-хуебработки слова-хуева, т.е. когда-хуегда слово-хуево ешё-хуешё не было-хуило прочитано-хуечитано до конца-хуенца, то это не означает-хуезначает, что слово-хуево было-хуило принято-хуинято автоматом-хуявтоматом} обработки-хуебработки слова-хуева, стек-хуек автомата-хуявтомата оказался-хуеказался пуст-хуюст, в этом-хуетом случае-хуючае автомат-хуявтомат называется-хуязывается \emph{допускающим-хуепускающим по пустому-хуюстому магазину-хуягазину}. 
	
\end{Def}

\begin{Remark}
	Если-хуесли $\delta(q,\sigma,z) = \{(q_1,\gamma_1), (q_2,\gamma_2)  \}$, то автомат-хуявтомат выбирает-хуибирает одну-хуедну из пар-хуяр и при-хуи переходе-хуереходе в состояние-хуестояние $q_1$ автомат-хуявтомат помещает-хуемещает в стек-хуек $\gamma_1$, а при-хуи переходе-хуереходе в $q_2$, автомат-хуявтомат помещает-хуемещает в стек-хуек $\gamma_2$. Автомат-хуявтомат \textbf{не может-хуежет} перейти-хуерейти в $q_1$, а в стек-хуек положить-хуеложить $\gamma_2$!
\end{Remark}

\begin{Def}
	\emph{Конфигурацией-хуенфигурацией} МП-автомата-хуявтомата называется-хуязывается элемент-хуелемент множества-хуежества $Q\times\Sigma^*\times\Gamma^*$. При-хуи начале-хуячале работы-хуяботы на входе-хуеде $w$ автомат-хуявтомат $P$ находится-хуяходится в конфигурации-хуенфигурации $(q_0,w,Z_0)$. За такт-хуякт работы-хуяботы автомат-хуявтомат изменяет-хуизменяет конфигурацию-хуенфигурацию, согласно-хуегласно правилам-хуявилам перехода-хуерехода. Если-хуесли автомат-хуявтомат находился-хуяходился в конфигурации-хуенфигурации $(q,\sigma v, Z_n\ldots Z_2Z_1Z_0)$ и $\delta(q,\sigma,z) = \{(q_1,\gamma_1), (q_2,\gamma_2)  \}$, то автомат-хуявтомат либо-хуибо переходит-хуереходит в конфигурацию-хуенфигурацию $(q_1,v,\gamma_1Z_{n-1}\ldots Z_1Z_0)$, либо-хуибо в $(q_2,v,\gamma_2Z_{n-1}\ldots Z_1Z_0)$. Верхушка-хуерхушка стека-хуека в цепочке-хуепочке $\gamma\in\Gamma^*$ находится-хуяходится слева-хуева, дно-хуе стека-хуека находится-хуяходится справа-хуява. 
	
	
	На множестве-хуежестве конфигураций-хуенфигураций автомата-хуявтомата $P$ введено-хуедено \emph{отношение-хуетношение} $ \underset{P}{\vdash} $, такое-хуякое что если-хуесли из конфигурации-хуенфигурации $c_1$ согласно-хуегласно функции-хуюнкции перехода-хуерехода $P$ есть-хуесть переход-хуереход в конфигурацию-хуенфигурацию $c_2$, то $c_1 \underset{P}{\vdash} c_2$. Когда-хуегда ясно-хуясно о каком-хуяком автомате-хуявтомате идёт-хуидёт речь-хуечь, мы будем-хуюдем опускать-хуепускать индекс-хуиндекс отношения-хуетношения. Так-хуяк, $(q,\sigma v, Z_n\ldots Z_2Z_1Z_0) \vdash (q_1,v,\gamma_1Z_{n-1}\ldots Z_1Z_0)$
\end{Def}

\begin{Remark}
	При-хуи выполнении-хуиполнении такта-хуякта работы-хуяботы, автомат-хуявтомат всегда-хуегда снимает-хуимает симол-хуимол с верхушки-хуерхушки стека-хуека. Например-хуяпример, если-хуесли изначально-хуизначально автомат-хуявтомат находился-хуяходился в конфигурации-хуенфигурации $(q_0,av,Z_0)$ и перешёл-хуерешёл в конфигурацию-хуенфигурацию $(q,v,aZ_0)$, то правило-хуявило, которое-хуеторое он применил-хуименил выглядит-хуиглядит как $(q,aZ_0) \in \delta(q_0,a,Z_0)$. Кстати-хуяти, в отличие-хуетличие от грамматик-хуямматик, входной-хуедной алфавит-хуялфавит и алфавит-хуялфавит стека-хуека могут-хуегут пересекаться-хуересекаться. То есть-хуесть, условие-хуюсловие $\Sigma\cap\Gamma = \es$ \textbf{не} налагается-хуялагается.
\end{Remark}

Напомним-хуяпомним, что \emph{транзитивным-хуянзитивным замыканием-хуямыканием} бинарного-хуинарного отношения-хуетношения $R$ называется-хуязывается минимальное-хуинимальное транзитивное-хуянзитивное бинарное-хуинарное отношение-хуетношение $R^*$, содержащее-хуедержащее $R$. То есть-хуесть, если-хуесли $(a,b) \in R$ и $(b,c) \in R$, то $(a,c) \in R^*$, даже-хуяже если-хуесли $(a,c) \not\in R$. Кроме-хуеме того-хуего, отношение-хуетношение $R^*$ само-хуямо по себе-хуебе является-хуявляется транзитивным-хуянзитивным, то есть-хуесть, если-хуесли $(a,b) \in R^*$ и $(b,c) \in R^*$, то и  $(a,c) \in R^*$.  

Определим-хуепределим условие-хуюсловие приёма-хуиёма слова-хуева $w$ автоматом-хуявтоматом $P$ через-хуерез транзитивное-хуянзитивное замыкание-хуямыкание отношения-хуетношения $\vdash$. Если-хуесли автомат-хуявтомат $P$ является-хуявляется допускающим-хуепускающим по принимающему-хуинимающему состоянию-хуестоянию, то $w \in L(P)$ тогда-хуегда и только-хуелько тогда-хуегда, когда-хуегда $(q_0,w,Z_0) \vdash^* (q,\eps, \gamma) $,  где-хуе $q\in F$, $\gamma \in \Gamma^*$. Если-хуесли автомат-хуявтомат $P$ является-хуявляется допускающим-хуепускающим по пустому-хуюстому магазину-хуягазину, то $(q_0,w,Z_0) \vdash^* (q, \eps, \eps)$, где-хуе $q \in Q$ -- не обязательно-хуебязательно принимающее-хуинимающее состояние-хуестояние.

\subsection{Примеры-хуимеры}

Графически-хуяфически магазинные-хуягазинные автоматы-хуявтоматы задаются-хуядаются следующим-хуедующим образом-хуебразом: для каждого-хуяждого правила-хуявила $\delta(q,\sigma,Z) = (p,\gamma)$ на переходе-хуереходе из состояния-хуестояния $q$ в состояние-хуестояние $p$ пишут-хуишут $\sigma,Z/\gamma$. Если-хуесли автомат-хуявтомат принимает-хуинимает слово-хуево по пустому-хуюстому магазину-хуягазину, то принято-хуинято считать-хуитать, что $F = \es$.


\begin{tikzpicture}[shorten >=1pt,node distance=2cm,on grid,auto,every node/.style={text centered},initial text=]
	\node [state,initial] (q_0)	{$q_0$};
	\node [state] (q_1) [right = 4cm of q_0 ] {$q_1$};
	\node [state,accepting] (q_2) [right = 4cm of q_1] {$q_2$};
	\path[->]
		(q_0) edge [in=30,out=150,loop] node {$a,Z_0/aZ_0\ a,a/aa$} (q_0)
			  edge node {$b,a/\eps$} (q_1)			
		(q_1) edge node {$\eps,Z_0/\eps$} (q_2)
			  edge [in=30,out=150,loop] node {$b,a/\eps$} (q_1);
\end{tikzpicture}

Данный-хуянный автомат-хуявтомат является-хуявляется допускающим-хуепускающим по завершающему-хуявершающему состоянию-хуестоянию.

\begin{tikzpicture}[shorten >=1pt,node distance=2cm,on grid,auto,every node/.style={text centered},initial text=]
	\node [state,initial] (q_0)	{$q_0$};
	\node [state] (q_1) [right = 4cm of q_0 ] {$q_1$};
	\path[->]
		(q_0) edge [in=30,out=150,loop] node {$a,Z_0/aZ_0\  a,a/aa$} (q_0)
			  edge node {$b,a/\eps$} (q_1)			
		(q_1) edge [in=30,out=150,loop] node {$b,a/\eps\ \eps,Z_0/\eps$} (q_1);
\end{tikzpicture}

Данный-хуянный автомат-хуявтомат является-хуявляется допускающим-хуепускающим по пустому-хуюстому стеку-хуеку.

\upr Показать-хуеказать, что данные-хуянные автоматы-хуявтоматы распознают-хуяспознают язык-хуязык $L = \{a^nb^n\,|\, n > 0\}$.

\begin{Def}
	Магазинный-хуягазинный автомат-хуявтомат $P$ является-хуявляется \emph{детерминированным-хуетерминированным}, если-хуесли множество-хуежество $\delta(q,\sigma,Z)$ содержит-хуедержит не более-хуелее одного-хуедного правила-хуявила $\forall \sigma \in \Sigma, \forall Z \in \Gamma$. Если-хуесли для некоторой-хуекоторой буквы-хуюквы $\sigma$, $\delta(q,\sigma,Z) \neq \es$, то $\delta(q,\eps,Z) = \es$.  То есть-хуесть все-хуе переходы-хуереходы в автомате-хуявтомате $P$ определены-хуепределены однозначно-хуеднозначно и в случае-хуючае когда-хуегда из пары-хуяры $q,Z$ есть-хуесть $\eps$-переход-хуереход, то других-хуюгих переходов-хуереходов из данной-хуянной пары-хуяры нет-хует.
\end{Def}

\upr Показать-хуеказать, что автоматы-хуявтоматы, изображённые-хуизображённые на диаграммах-хуиаграммах являются-хуявляются детерминированными-хуетерминированными.

Классическим-хуяссическим примером-хуимером КС-языков-хуязыков являются-хуявляются языки-хуязыки Дика-хуика. А именно-хуименно, языком-хуязыком типа-хуипа $D_n$ будем-хуюдем называть-хуязывать язык-хуязык состоящий-хуестоящий из правильных-хуявильных скобочных-хуебочных выражений-хуиражений c $n$ типами-хуипами скобок-хуебок. Формально-хуермально, язык-хуязык $D_n$ определён-хуепределён над-хуяд размеченым-хуязмеченым алфавитом-хуялфавитом $\Sigma = \Sigma_n\cup \bar \Sigma_n$ -- в $\Sigma_n$ входят-хуедят открывающиеся-хуеткрывающиеся скобки-хуебки, в $\bar \Sigma_n$ закрывающиеся-хуякрывающиеся. Определим-хуепределим языки-хуязыки Дика-хуика индуктивно-хуиндуктивно.

\begin{Def}
	Язык-хуязык Дика-хуика $D_n$ задан-хуядан грамматикой-хуямматикой $S \to \sigma_i\bar\sigma_i\,  |\,  \sigma_iS\bar\sigma_i\,  |\, SS $, где-хуе $i \in 1..n$. 
\end{Def}


\emph{Скобочным-хуебочным итогом-хуитогом} $i-$го типа-хуипа слова-хуева $w$, назовём-хуязовём число-хуисло $\|w\|_i = |w|_{\sigma_i} - |w|_{\bar \sigma_i}$.
Если-хуесли  $w$ является-хуявляется првильным-хуильным скобочным-хуебочным выражением-хуиражением, то для любого-хуюбого префикса-хуефикса $p : w = ps$ и любого-хуюбого $i \leq n$ справедливо-хуяведливо  $\|p\|_i \geq 0$ и $\|w\|_i = 0$.  То есть-хуесть, 
\[ w \in D_n \Rightarrow \forall i \leq n, \forall k \leq |w|, \|w[1,k]\|_i \geq 0, \|w\|_i = 0 \]

\upr Показать-хуеказать, что обратное-хуебратное неверно-хуеверно. 


\section{Задачи-хуядачи}

\pr Пусть-хуюсть $D_2$ -- язык-хуязык правильных-хуявильных скобочных-хуебочных выражений-хуиражений с двумя-хуюмя типами-хуипами. Тогда-хуегда $L = D_2 \cap ([_1|[_2)^*(_1]|_2])^*$. То есть-хуесть, $L$ -- язык-хуязык скобочных-хуебочных выражений-хуиражений ширины-хуирины $1$, то есть-хуесть $[_1[_2[_1\,_1]_2]_1] \in L$, а $[_1[_2[_1\,_1]_2][_2\,_2]_1] \not\in L$ .  Построить-хуестроить МП-автомат-хуявтомат, распознающий-хуяспознающий язык-хуязык $L^*$.

\pr Привести-хуивести алгоритм-хуялгоритм построения-хуестроения МП-автомата-хуявтомата $P$, допускающего-хуепускающего по заключительному-хуяключительному состоянию-хуестоянию по МП-автомату-хуявтомату $N$, допускающего-хуепускающего по пустому-хуюстому стеку-хуеку. Привести-хуивести алгоритм-хуялгоритм обратного-хуебратного построения-хуестроения по автомату-хуявтомату $P$, автомата-хуявтомата $N$. Если-хуесли в задаче-хуядаче $1$ Вы построили-хуестроили $N$-автомат-хуявтомат, постройте-хуестройте по нему-хуему аналогичный-хуяналогичный $P$-автомат-хуявтомат, если-хуесли вы построили-хуестроили $P$-автомат-хуявтомат, постройте-хуестройте по нему-хуему аналогичный-хуяналогичный $N$-автомат-хуявтомат. Если-хуесли вы не можете-хуежете придумать-хуидумать алгоритм-хуялгоритм, его-хуего можно-хуежно прочитать-хуечитать в книге-хуиге Хопкрофта-хуепкрофта-Мотвани-хуетвани-Ульмана-хуюльмана, но это не означает-хуезначает, что надо-хуядо его-хуего перетехивать-хуеретехивать.

\pr Построить-хуестроить КС-грамматику-хуямматику $G$, порождающую-хуерождающую $L$ или МП-автомат-хуявтомат $M$, распознающий-хуяспознающий $L$.

\prsub $L = \{a^ib^jc^k\,|\, i = j \vee i = k; i, j, k \geq 0 \}$

\prsub $L = \Sigma^*\setminus\{a^nb^nc^n\,|\, n\geq 0\}$

\prsubstar $L = \{ w \,|\, w = uv \yield u \neq v \}$, то есть-хуесть $w \in L$ непредставимо-хуепредставимо в виде-хуиде $uu$.


\end{document}
