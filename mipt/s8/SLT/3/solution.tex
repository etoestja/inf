\documentclass[a4paper]{article}
\usepackage[a4paper, left=5mm, right=5mm, top=5mm, bottom=5mm]{geometry}
%\geometry{paperwidth=210mm, paperheight=2000pt, left=5pt, top=5pt}
\usepackage[utf8]{inputenc}
\usepackage[english,russian]{babel}
\usepackage{indentfirst}
\usepackage{tikz}
\usepackage{cancel}
\usetikzlibrary{automata,positioning,arrows}
\usepackage{amsmath}
\usepackage{enumerate}
\usepackage{hyperref}
\usepackage{amsfonts}
\usepackage{amssymb}
\DeclareMathOperator*{\argmin}{arg\,min}
\DeclareMathOperator*{\argmax}{arg\,max}
\usepackage{wasysym}
\title{Статистическое обучение\\Задание 3}
\date{задано 2017.03.27}
\author{Сергей~Володин, 374 гр.}
\newcommand{\matrixl}{\left|\left|}
\newcommand{\matrixr}{\right|\right|}

\newcommand{\peq}{\mathrel{+}=}
\newcommand{\meq}{\mathrel{-}=}
\newcommand{\deq}{\mathrel{:}=}
\newcommand{\VC}{\mbox{VC}}
\newcommand{\plpl}{\mathrel{+}+}
\newcommand{\sign}{\mbox{sign}\,}
\newcommand{\F}{\mathcal{F}}
\newcommand{\R}{\mathbb{R}}
\newcommand{\conv}{\mbox{conv}\,}
\newcommand{\E}{\mathbb{E}}
\newcommand{\D}{\mathbb{D}}

% пустое слово
\def\eps{\varepsilon}

% регулярные языки
\def\eqdef{\overset{\mbox{\tiny def}}{=}}
\newcommand{\niton}{\not\owns}

\begin{document}
\maketitle
\section*{Meta}
Делал один. Список ссылок:
\begin{enumerate}
\item \url{http://web.eecs.umich.edu/~cscott/past_courses/eecs598w14/notes/05_vc_theory.pdf} (упражнение 3)
\item \url{http://www-math.mit.edu/~rigollet/courses/Notes/L4.pdf} (упражнение 5)
\end{enumerate}
\section*{Упражнение 1}
Имеем задачу классификации без шума.
\begin{enumerate}
\item Рассмотрим $\E_{X^{n}}LOO(X^{n},n)=\frac{1}{n}\sum\limits_{i=1}^n\E_{X^n\setminus \{x_i\}}\E_{x_i}[\hat{f}_{X^n\setminus \{x_i\}}(x_i)\neq f^*(x_i)]=\frac{1}{n}\sum\limits_{i=1}^n \E_{X^{n-1}}L(\hat{f})=\E_{X^{n-1}}L(\hat{f})$
\item Пусть $\hat{f}$~--- SVM. Рассмотрим $\E_{X^n}L(\hat{f})=\E_{X^{n+1}}LOO(X^{n+1},n+1)=\frac{1}{n+1}\sum\limits_{i=1}^{n+1}[f_{X^{n+1}\setminus \{x_i\}}(x_i)\neq f^*(x_i)]$
\item В сумме объект $x_i$ может быть либо опорным, либо не опорным. Если $x_i$~--- не опорный объект, то для него $f_{X^{n+1}\setminus \{x_i\}}(x_i)=f_{X^{n+1}}(x_i)$. Поскольку $x_i$ содержался в разделимой выборке, то классификатор выдаст для $x_i$ правильный ответ. Значит, это слагаемое нулевое.
\item Тогда $\E_{X^n}L(\hat{f})=\frac{1}{n+1}\sum\limits_{x_i \mbox{\tiny опорный}}[f_{X^{n+1}\setminus \{x_i\}}(x_i)\neq f^*(x_i)]\leqslant\boxed{ \frac{\E M}{n+1}}$, где $M$~--- число опорных векторов в выборке из $n+1$ элемента.
\end{enumerate}
\section*{Упражнение 2.1}
\begin{enumerate}
\item Рассмотрим $\alpha$-сеть $Z$ для $A$: $|Z|=N(A,\alpha)$, $\forall a\in A\,\exists z\in Z\colon \rho(z,a)\leqslant \alpha$, где $\rho(f,g)=\sqrt{\frac{1}{n}\sum\limits_{i=1}^n[f_i\neq g_i]}$. Обозначим $z_a=\argmin\limits_{z\in z}\rho(z, a)$. $\rho(z_a,a)\leqslant\alpha$. То есть, $\sum |a^i-z_a^i|\leqslant \alpha^2n$
\item $R_n(A)=\frac{1}{n}\E_\eps\sup\limits_{a\in A}|(\eps,a)|$. Обозначим $a_\eps=\argmax\limits_{a\in A}|(\eps,a)|$. Тогда $R_n(A)=\frac{1}{n}\E|(\eps,a_\eps)|=\frac{1}{n}\E|(\eps,a_\eps-z_{a_\eps}+z_{a_\eps})|\leqslant\frac{1}{n}\E|(\eps,a_\eps-z_{a_\eps})|+\frac{1}{n}\E|(\eps,z_{a_\eps})|$
\item $\frac{1}{n}\E |(\eps,z_{a_\eps})|\leqslant \frac{1}{n}\E\sup\limits_{z\in Z}|(\eps, z)|=R_n(Z)\leqslant \frac{1}{\sqrt{n}}\sup\limits_{z\in Z}\|z\|^2\sqrt{2\log 2|Z|}\leqslant\sqrt{2\log2 N(A,\alpha)/n}$
\item Рассмотрим вектор $a_\eps-z_{a_\eps}$. В нем не более $\alpha^2n$ единиц. Значит, $\frac{1}{n}\E |(\eps,a_\eps-z_{a_\eps})|\leqslant\frac{1}{n}\sup\limits_{\eps}|(\eps, a_\eps-z_{a_\eps})|\leqslant \alpha^2\leqslant \alpha$ ($\alpha\in[0,1]$).
\item В итоге: $R_n(A)\leqslant \sqrt{\frac{2\log 2N(A,\alpha)}{n}}+\alpha$ для всех $\alpha$. Значит, $R_n(A)\leqslant\inf\limits_{\alpha\in[0,1]}(\alpha+\sqrt{\frac{2\log2N(A,\alpha)}{n}})$ $\blacksquare$
\end{enumerate}
\section*{Упражнение 2.2}
\begin{enumerate}
\item Оценка для $N$ в случае $VC(F)=d<+\infty$: $N(\alpha, F(x))\leqslant \left(\frac{4e}{\alpha^2}\right) ^{\frac{d}{1-1/e}}$
\item В нашем случае $d\leqslant n$.
\item Значит, $N(A,\alpha)\leqslant\left(\frac{4e}{\alpha^2}\right)^{\frac{n}{1-1/e}}$
\item Тогда оценка примет вид: $R_n(A)\leqslant \inf\limits_{\alpha\in[0,1]}(\alpha+\sqrt{\frac{2}{n}}\sqrt{\ln2+\frac{n}{1-1/e}\ln(4e)-\frac{2n}{1-1/e}\ln\alpha})=\inf\limits_{\alpha\in[0,1]}(\alpha+\sqrt{a+b\ln\alpha})\leqslant \frac{1}{2}+\sqrt{\frac{2\ln 2}{n}+\frac{2}{1-1/e}(1+2\ln 2)}$, $a=2/n\ln2+2\ln(4e)/(1-1/e)$, $b=-4/(1-1/e)$
\item Интеграл Дадли: $R_n(A)\leqslant\frac{12}{\sqrt{n}}\int\limits_0^1\sqrt{\ln 2N(A,\alpha)}d\alpha\leqslant\frac{12}{\sqrt{n}}\int\limits_0^1\sqrt{\ln2+\frac{n\ln 4e}{1-1/e}-\frac{2n}{1-1/e}\ln\alpha}d\alpha\approx 12\sqrt{1/(1-1/e)}\int\limits_0^1\sqrt{1-\frac{2}{\ln 4e}\ln\alpha}d\alpha$
\end{enumerate}
\section*{Упражнение 3}
\begin{enumerate}
\item Имеем распределение $x\sim F(x)$.
\item Определим семейство классификаторов $\F=\{[x\in G]\big|G=(-\infty,t),t\in\R\}$. $Y=\{0,1\}$, $X=\R$. %Определим распределение на $X\times Y$: $P_{X\times Y}(Y=1)=0$, $P_{X\times Y}(X<x,Y=0)=F(x)$
\item Фиксируем $x_0$. Пусть $f\in F$, $f(x)=[x\in (-\infty, x_0)]$. % Тогда $L(f)=\E [Y\neq f(X)]=P(f(X)=1\big|Y=0)P(Y=0)+P(f(X)=0\big|Y=1)\cancel{P(Y=1)}=P(f(X)=1)=F(x)$. Также, $L_n(f)=\frac{1}{n}\sum\limits_{i=1}^n[f(x_i)\neq y_i]=\frac{1}{n}\sum\limits_{i=1}^n[f(x_i)=1]=\frac{1}{n}\sum\limits_{i=1}^n[x_i<x_0]=F_n(x_0)$
Тогда $$Pf=\E f(x)=P(x< x_0)=F(x)$$ $$P_nf=\frac{1}{n}\sum\limits_{i=1}^nf(x_i)=\frac{1}{n}\sum\limits_{i=1}^n[x_i<x]=F_n(x)$$
\item Значит, $\sup\limits_{x\in \R}|F(x)-F_n(x)|=\sup\limits_{f\in\F}|Pf-P_nf|$
\item Найдем $\VC(\F)$. Выберем $n$ точек $\{x_i\}\subset\R$, $x_i<x_{i+1}$. Тогда матрица $f_i(x_j)$ имеет вид:

$$\matrixl
\begin{array}{cccc}
0 & 0 & ... & 0\\
1 & 0 & ... & 0\\
1 & 1 & ... & 0\\
1 & 1 & ... & 1\\
\end{array}
\matrixr$$

В ней $n+1$ строк. Значит, $S_\F(n)=n+1$. Значит, $\VC(\F)=1<\infty$.
\item Определим $\xi_i=\sup\limits_{f\in\F}|Pf-P_nf|$. Тогда $\E \xi_i\leqslant 2R(\F)\leqslant 2\sqrt{\frac{2\VC(\F)\ln(n+1)}{n}}\leqslant 2\sqrt{\frac{2\ln(n+1)}{n}}$ (задание 2). Также, $\xi_i$ удовлетворяет условию ограниченных разностей, поэтому $P(\xi_i-\E\xi_i\geqslant t)\leqslant e^{-nt^2/2}$.
\item Тогда $P(\xi_i\geqslant t)\leqslant \exp(-\frac{1}{2}n(t-\E\xi_i)^2)=\exp(-nt^2/2)\exp(-n(\E\xi_i)^2/2)\exp(n\E\xi_it)$
\end{enumerate}
\section*{Упражнение 4}
%2017.03.21
\section*{Упражнение 5.1}

\begin{enumerate}
\item Рассмотрим $P(x)=\E_{Y|X}([f(x)\neq Y]-[f^*(x)\neq Y])$. Тогда $L(f)-L(f^*)=\E_X P(x)$. Рассмотрим $Q(x)=\E_{Y|X}(|\eta(x)|[f(x)\neq f^*(x)])$
\item Рассмотрим $\eta(x)=\E Y=P(Y=1)-P(Y=0)$
\item Если $f(x)=f^*(x)$, то $P(x)=Q(x)$.
\item Если $f(x)=+1$ и $f^*(x)=-1$, то $\eta(x)<0$, $Q(x)=-\eta(x)=P(Y=-1)-P(Y=1)$ а $P(x)=\E_Y[Y=-1]-\E_Y[Y=+1]=P(Y=-1)-P(Y=+1)=Q(x)$.
\item Аналогично, если $f(x)=-1$, $f^*(x)=+1$, то $Q(x)=P(x)$
\item Значит, $\E_XP(x)=\E_XQ(x)$ $\blacksquare$
\end{enumerate}
\section*{Упражнение 5.2}
$\alpha\in\R$. Условия:
$$
\begin{array}{llrl}
(1) & \exists B\,\forall g &\hookrightarrow & \E g^2\leqslant B(Eg)^{\alpha}\\
(2) & \exists \beta\,\forall t & \hookrightarrow & P(|\eta(X)|\leqslant t)\leqslant \beta t^{\alpha/(1-\alpha)}
\end{array}
$$
\begin{enumerate}
\item Рассмотрим $\E g^2=\E([f\neq y]-[f^*\neq y])^2=\E[f\neq f^*]$
\item Пусть (2). Рассмотрим $\E g=\E |\eta|[f\neq f^*]\geqslant \E |\eta|[\eta >t][f\neq f^*]\geqslant t\E [\eta>t][f\neq f^*]\geqslant t(\underbrace{P(f\neq f^*)}_{=\E g^2}-P(|\eta|<t))\geqslant t(\E g^2-\beta t^{\alpha/(1-\alpha)})$. Выберем $t=c(\E g^2)^{(1-\alpha)/\alpha}$. Тогда $\E g\geqslant c(\E g^2)^{1/\alpha}-\beta c^{(1-\alpha)/\alpha}(\E g^2)^{1/\alpha}=c(1-\beta c^{(1-2\alpha)/\alpha})(\E g^2)^{1/\alpha}$. Выберем $c\colon \beta c^{(1-2\alpha)/\alpha}<1$. Обозначим $B=(c(1-\beta c^{(1-2\alpha)/\alpha}))^{-\alpha}>0$. Тогда $\E g^2\leqslant B(\E g)^\alpha$ $\Rightarrow (1)$ $\blacksquare$
\item (1) $\Rightarrow$ (2)
\end{enumerate}
\section*{Упражнение 6}
%Conservative: http://www.cs.tau.ac.il/~mansour/ml-course-10/scribe4.pdf
%\section*{Задача 1}
%\section*{Задача 2}
%\section*{Задача 3}
%\section*{Задача 4*}
\end{document}