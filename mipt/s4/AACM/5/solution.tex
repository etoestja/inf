\documentclass[a4paper]{article}
\usepackage[a4paper, left=5mm, right=5mm, top=5mm, bottom=5mm]{geometry}
%\geometry{paperwidth=210mm, paperheight=2000pt, left=5pt, top=5pt}
\usepackage[utf8]{inputenc}
\usepackage[english,russian]{babel}
\usepackage{indentfirst}
\usepackage{tikz} %Рисование автоматов
\usetikzlibrary{automata,positioning,arrows,trees}
\usepackage{amsmath}
\usepackage[makeroom]{cancel} % зачеркивание
\usepackage{multicol,multirow} %Несколько колонок
\usepackage{hyperref}
\usepackage{tabularx}
\usepackage{amsfonts}
\usepackage{amssymb}
\DeclareMathOperator*{\argmin}{arg\,min}
\usepackage{listings}
\usepackage{wasysym}
\date{задано 2014.03.13}
\author{Сергей~Володин, 272 гр.}
\newcommand{\matrixl}{\left|\left|}
\newcommand{\matrixr}{\right|\right|}
% названия автоматов  (rubtsov)
\def\A{{\cal A}}
\def\B{{\cal B}}
\def\C{{\cal C}}

%классы сложности (rubtsov)
\def\PP{{\mathsf{P}}}
\def\NP{{\mathsf{NP}}}
\def\NPc{{\mathsf{NP}}\text{-}{\rm c}}
\def\coNP{{\rm co}\text{-}{\mathsf{NP}}}
\def\DTIME{{\mathsf{DTIME}}}
\def\NTIME{{\mathsf{NTIME}}}
\def\HALT{{\rm{HALT}}}
\def\SAT{{\rm{SAT}}}
\def\3SAT{{\rm{3\text{-}SAT}}}
\def\2SAT{{\rm{2\text{-}SAT}}}
\def\UNSAT{{\rm{UNSAT}}}
\def\HP{{\rm{HAMPATH}}}
\def\UHP{{\rm{UHAMPATH}}}
\def\LL{{\mathrm{LL}}}
\def\poly{{\rm{poly}}}
\def\GC{{\mbox{ГЦ}}}
\def\GP{{\mbox{ГП}}}

\title{Алгоритмы и модели вычислений.\\Задание 5: сложность вычислений: классы $\PP$, $\NP$, $\coNP$ II}

% алгоритмы (Рубцов)
\usepackage{verbatim}
\usepackage{listings}
\usepackage{algorithm2e}

%+= и -=, иначе разъезжаются...
\newcommand{\peq}{\mathrel{+}=}
\newcommand{\meq}{\mathrel{-}=}
\newcommand{\deq}{\mathrel{:}=}
\newcommand{\plpl}{\mathrel{+}+}

% пустое слово  (rubtsov)
\def\eps{\varepsilon}

% регулярные языки  (rubtsov)
\def\REG{{\mathsf{REG}}}
\def\CFL{{\mathsf{CFL}}}
\def\eqdef{\overset{\mbox{\tiny def}}{=}}
\newcommand{\niton}{\not\owns}

%FIRST & FOLLOW (rubtsov)
\def\first{\mathrm{ FIRST} }
\def\follow{\mathrm{ FOLLOW} }

% LL LR (rubtsov)
\def\LL{{\mathrm{LL}}}
\def\LR{{\mathrm{LR}}}

\newcounter{rowItemCount}
\newcounter{subRowItemCount}
\newcommand\rowItem{
    \setcounter{subRowItemCount}{0}
    \arabic{rowItemCount}.\addtocounter{rowItemCount}{1}}
\newcommand\subRowItem{
    \addtocounter{subRowItemCount}{1}
    \addtocounter{rowItemCount}{-1}
    \arabic{rowItemCount}.\arabic{subRowItemCount}.\addtocounter{rowItemCount}{1}}
    
\newcommand{\smalll}[1]{\overline{\overline{#1}}}
\newcommand{\smallo}{\bar{\bar{o}}}

\begin{document}
\maketitle
\subsection*{Задача 1}
\begin{enumerate}
\item Докажем, что $\HP\leqslant_m^p\UHP$.\newline
$\HP=\{(G,s,t)\big|G\mbox{~--- ориентированный граф}, \mbox{в }G\mbox{ существует гамильтонов путь из }s\mbox{ в }t\}$,\newline
$\UHP=\{(G,s,t)\big|G\mbox{~--- неориентированный граф}, \mbox{в }G\mbox{ существует гамильтонов путь из }s\mbox{ в }t\}$.\newline
Пусть $G$~--- ориентированный граф, $s$ и $t$~--- его вершины. $x=(G,s,t)$. Определим $f(x)=(G',s',t')$. Для каждой вершины $v\in V(G)$, кроме $s$ и $t$, добавим в $V(G')$ три вершины $v_i,v_m,v_o$. Для $s$ и $t$ добавим $s_o$ и $t_i$. Соединим $v_i\leftrightarrow v_m$ и $v_m\leftrightarrow v_o$ (стрелкой $\leftrightarrow$ обозначено неориентированное ребро). Для каждого $(u,v)\in E(G)$ добавим $(u_o,v_i)\in E(G')$. $G'$~--- получившийся граф, $s'=s_o$, $t'=t_i$.\begin{enumerate}
\item Пусть $x=(G,s,t)\in\HP$. Тогда существует путь $s\to v_1\to v_2\to...\to v_n\to t$. По построению, тогда существует путь $s_o\leftrightarrow v_{1i}\leftrightarrow v_{1m}\leftrightarrow v_{1o}\leftrightarrow...\leftrightarrow v_{ni}\leftrightarrow v_{nm}\leftrightarrow v_{no}\leftrightarrow t_i$, который является гамильтоновым путем в $G'$ {\em todo}, поэтому $f(x)\in \UHP$
\item Пусть $f(x)=(G',s_o,t_i)\in\UHP$. Из вершины с индексом $\cdot_o$ по построению есть ребра только в вершины с индексом $\cdot_i$. Из вершины $v_i$ есть ребро только в $v_m$, из вершины $v_m$~--- только в $v_o$. Поэтому гамильтонов путь имеет вид $s_o\leftrightarrow v_{1i}\leftrightarrow v_{1m}\leftrightarrow v_{1o}\leftrightarrow...\leftrightarrow v_{ni}\leftrightarrow v_{nm}\leftrightarrow v_{no}\leftrightarrow t_i$, значит, в исходном графе $G$ есть путь $s\to v_1\to v_2\to...\to v_n\to t$, и он гамильтонов {\em todo}, поэтому $x\in \HP$
\item $f$~--- вычислима за полиномиальное время (линейное по количеству ребер и вершин время)
\end{enumerate}
\item Поскольку $\HP\in\NPc$, $\HP\leqslant\UHP$, $\UHP\in\NP$, то (см. решение 4-го задания, вспомогательные утверждения, 2) $\UHP\in\NPc\,\blacksquare$
\end{enumerate}
\subsection*{Задача 2}
См. (каноническое) 21
\subsection*{Задача 3}
\begin{enumerate}
\item $\C\supset \NP\cup\coNP$.\begin{enumerate}
\item Пусть $L\in\NP$. Тогда (семинар) $L\leqslant_m^p\SAT\Leftrightarrow \exists f\colon\Sigma^*\to\Sigma^*\colon\forall x(x\in L\Leftrightarrow f(x)\in\SAT)$, $f$~--- вычислима за полиномиальное время. Определим $M_{\SAT}$: вычисляем за полиномиальное время (определение сводимости) $f(x)$ ($x$~--- вход), спрашиваем оракула $f(x)\overset{?}{\in}\SAT$ за $O(1)$. Ответ~--- ответ оракула (корректно из определения сводимости).  Время работы полиномиально: $T(|x|)=\poly(|x|)+O(1)=\poly(|x|)$.
\item Пусть $L\in\coNP$. Тогда $\overline{L}\leqslant_m^p\SAT\Leftrightarrow \exists f\colon\Sigma^*\to\Sigma^*\colon\forall x(x\in \overline{L}\Leftrightarrow f(x)\in\SAT)\Leftrightarrow\forall x(x\in L\Leftrightarrow f(x)\notin\SAT)$, $f$~--- вычислима за полиномиальное время. Определим $M_{\SAT}$: вычисляем за полиномиальное время (определение сводимости) $f(x)$ ($x$~--- вход), спрашиваем оракула $f(x)\overset{?}{\in}\SAT$ за $O(1)$. Ответ~--- \underline{противоположный} ответу оракула (корректно из определения сводимости). Время работы полиномиально: $T(|x|)=\poly(|x|)+O(1)=\poly(|x|)$.
\end{enumerate}
\item $\C\subset\NP\cup\coNP$. Пусть $L\in \C$. Тогда существует МТ $M_\SAT$, вычисляющая $x\overset{?}{\in}L$ за полиномиальное время, и делающая не более одного обращения к оракулу $t\overset{?}{\in}\SAT$
\end{enumerate}
\subsection*{(каноническое) Задача 21}
$\GC=\{G\mbox{~--- ориентированный граф}\big|\mbox{в }G\mbox{ существует гамильтонов цикл}\}$.\newline
$\GP=\{(G,s,t)\mbox{~--- ориентированный граф, две его вершины}\big|\mbox{в }G\mbox{ существует гамильтонов путь из }s\mbox{ в }t\}$.\begin{enumerate}
\item Докажем, что $\GP\leqslant_m^p\GC$. Пусть $x=(G,s,t)$~--- граф и две его вершины. Определим граф $f(x)$: возьмем $G$, удалим все ребра между $s$ и $t$, все ребра в $s$, все ребра из $t$. Добавим одно $t\to s$.\begin{enumerate}
\item Пусть $x\in\GP$, то есть, в $G$ есть гамильтонов путь из $s$ в $t$. Тогда в этом пути нет ребер из $t$ в $s$ (иначе через $t$ или $s$ путь пройдет дважды). Значит, путь будет гамильтоновым и в $f(x)$. Но в $f(x)$ есть ребро $t\to s$, получаем гамильтонов цикл, составленный из пути и одного ребра. Значит, $f(x)\in\GC$
\item Пусть $f(x)\in\GC$, то есть, в $f(x)$ есть гамильтонов цикл. В этот цикл входят вершины $s$ и $t$, так как в него входят все вершины графа. Но из $t$ нет других ребер, кроме как в $s$ (по построению), значит, в цикл входит ребро $t\to s$. Рассмотрим весь путь без этого ребра. Он гамильтонов, так как является гамильтоновым циклом без одного ребра. Этот путь будет также путем в $G$, так как не содержит ребра $t\to s$, а в $G$ ребер больше (кроме $t\to s$). Также этот путь будет гамильтоновым, так как множества вершин $G$ и $f(x)$ совпадают. Значит, $x\in\GP$
\item Сводимость $f$ в явном виде. $A[i][j]$~--- матрица графа $G$, $B[i][j]$~--- матрица графа $f(x)$. Алгоритм: {\em todo}. Получаем, что $f$~--- вычислима за полиномиальное время.
\end{enumerate}
\item {\em (Идея обсужалась с Игорем Силиным)}. Докажем, что $\GC\leqslant_m^p\GP$. Пусть $x=G$~--- граф. Фиксируем некоторую его вершину $v$. <<Разделим>> ее на две вершины $s$ и $t$, из $s$ добавим все ребра из $v$, в $t$ направим все ребра в $v$. Получим граф $G'$. Определим $f(x)=(G',s,t)$.\begin{enumerate}
\item Пусть $x\in\GC$. Тогда в $x=G$ существует гамильтонов цикл. Он содержит все вершины, в том числе и вершину $v$: $v\to v_1\to...\to v_n\to v$. Тогда в графе $G'$ образа $f(x)$ будет путь $s\to v_1\to...\to v_n\to t$, и он будет гамильтоновым {\em todo}, т.е. $f(x)\in\GP$
\item Пусть $f(x)\in\GP$. Тогда существует гамильтонов путь $s\to...\to t$. Значит, в $G$ есть цикл $v\to...\to v$, и он гамильтонов {\em todo}. Получаем $x\in\GC$.
\item Сводимость $f$ в явном виде. $A[i][j]$~--- матрица графа $x=G$, $B[i][j]$~--- матрица графа из $f(x)$. Алгоритм: {\em todo}. Получаем, что $f$~--- вычислима за полиномиальное время.
\end{enumerate}
\end{enumerate}
\subsection*{(каноническое) Задача 23}
\begin{enumerate}
\item $\Psi(x_1,x_2)\eqdef (\urcorner x_1\vee x_2)$. Базовое множество ($n=2$) $\{x_1,x_2,\urcorner x_1,\urcorner x_2\}$.\newline
Семейство подмножеств $A_\Psi=\big\{\{x_1,\urcorner x_1\},\,\{x_2,\urcorner x_2\},\,\{\urcorner x_1,x_2\}\big\}$.\newline
$\varangle A\eqdef\{x_1,\,x_2\}$. Получаем $A\cap\{x_1,\urcorner x_1\}\ni x_1$, $A\cap \{x_2,\urcorner x_2\}\ni x_2$, $A\cap \{\urcorner x_1, x_2\}\ni x_2$.\newline
Значит, $A$~--- протыкающее множество для $A_\Psi$, и $|A|=2$.
\item $\chi(x_1,x_2,x_3)\eqdef (x_1\vee x_2\vee x_3)\wedge(\urcorner x_1\vee \urcorner x_2)\wedge(x_1\vee \urcorner x_2)\wedge (\urcorner x_1\vee x_2\vee x_3)\wedge\urcorner x_3$. Семейство подмножеств ($n=3$) $A_\chi=\big\{\{x_1,\urcorner x_1\},\{x_2,\urcorner x_2\},\{x_3,\urcorner x_3\},\{x_1,x_2,x_3\},\{\urcorner x_1,\urcorner x_2\},\{x_1,\urcorner x_2\},\{\urcorner x_1,x_2,x_3\},\{\urcorner x_3\}\big\}$. Пусть $A$~--- протыкающее множество. Тогда $A\cap \{\urcorner x_3\}\neq \varnothing\Rightarrow A\ni \urcorner x_3$. Также $A\cap\{x_1,\urcorner x_1\}\neq\varnothing$, поэтому $A$ содержит $x_1$ или $\urcorner x_1$. Аналогично $x_2\in A$ или $\urcorner x_2\in A$. Получаем, что $A$ содержит не менее трех элементов. Предположим, что их ровно 3. Рассмотрим все возможные 4 случая (или$\times$или раньше по тексту):\begin{enumerate}
\item $A=\{x_1,x_2,\urcorner x_3\}$. Тогда $A\cap \{\urcorner x_1,\urcorner x_2\}=\varnothing$~--- противоречие.
\item $A=\{x_1,\urcorner x_2,\urcorner x_3\}$. Тогда $A\cap \{\urcorner x_1,x_2,x_3\}=\varnothing$~--- противоречие.
\item $A=\{\urcorner x_1,x_2,\urcorner x_3\}$. Тогда $A\cap \{x_1,\urcorner x_2\}=\varnothing$~--- противоречие.
\item $A=\{\urcorner x_1,\urcorner x_2,\urcorner x_3\}$. Тогда $A\cap\{x_1,x_2,x_3\}=\varnothing$~--- противоречие.
\end{enumerate}
Получаем, что $A$ содержит более трех элементов $\blacksquare$
\end{enumerate}
\end{document}