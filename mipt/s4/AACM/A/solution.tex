\documentclass[a4paper]{article}
\usepackage[a4paper, left=5mm, right=5mm, top=5mm, bottom=5mm]{geometry}
%\geometry{paperwidth=210mm, paperheight=2000pt, left=5pt, top=5pt}
\usepackage[utf8]{inputenc}
\usepackage[english,russian]{babel}
\usepackage{indentfirst}
\usepackage{tikz} %Рисование автоматов
\usetikzlibrary{automata,positioning,arrows,trees,calc}
\usepackage{amsmath}
\usepackage[makeroom]{cancel} % зачеркивание
\usepackage{multicol,multirow} %Несколько колонок
\usepackage{hyperref}
\usepackage{tabularx}
\usepackage{amsfonts}
\usepackage{amssymb}
\DeclareMathOperator*{\argmin}{arg\,min}
\usepackage{listings}
\usepackage{wasysym}
\date{задано 2014.04.17}
\author{Сергей~Володин, 272 гр.}
\newcommand{\matrixl}{\left|\left|}
\newcommand{\matrixr}{\right|\right|}
% названия автоматов  (rubtsov)
\def\A{{\cal A}}
\def\B{{\cal B}}
\def\C{{\cal C}}

%классы сложности (rubtsov)
\def\PP{{\mathsf{P}}}
\def\NP{{\mathsf{NP}}}
\def\NPc{{\mathsf{NP}}\text{-}{\rm c}}
\def\coNP{{\rm co}\text{-}{\mathsf{NP}}}
\def\DTIME{{\mathsf{DTIME}}}
\def\NTIME{{\mathsf{NTIME}}}
\def\CLIQUE{{\mathsf{CLIQUE}}}
\def\HALT{{\rm{HALT}}}
\def\SAT{{\rm{SAT}}}
\def\3SAT{{\rm{3\text{-}SAT}}}
\def\2SAT{{\rm{2\text{-}SAT}}}
\def\UNSAT{{\rm{UNSAT}}}
\def\HP{{\rm{HAMPATH}}}
\def\UHP{{\rm{UHAMPATH}}}
\def\LL{{\mathrm{LL}}}
\def\poly{{\rm{poly}}}
\def\GC{{\mbox{ГЦ}}}
\def\GP{{\mbox{ГП}}}
\def\conv{{\mbox{conv}}}

\title{Алгоритмы и модели вычислений.\\Задание 10: теория чисел}

% алгоритмы (Рубцов)
\usepackage{verbatim}
\usepackage{listings}
\usepackage{algorithm2e}

%+= и -=, иначе разъезжаются...
\newcommand{\peq}{\mathrel{+}=}
\newcommand{\meq}{\mathrel{-}=}
\newcommand{\deq}{\mathrel{:}=}
\newcommand{\plpl}{\mathrel{+}+}

\newcommand{\todo}{{\em todo}}

% пустое слово  (rubtsov)
\def\eps{\varepsilon}

% регулярные языки  (rubtsov)
\def\REG{{\mathsf{REG}}}
\def\CFL{{\mathsf{CFL}}}
\def\eqdef{\overset{\mbox{\tiny def}}{=}}
\newcommand{\niton}{\not\owns}

%FIRST & FOLLOW (rubtsov)
\def\first{\mathrm{ FIRST} }
\def\follow{\mathrm{ FOLLOW} }

% LL LR (rubtsov)
\def\LL{{\mathrm{LL}}}
\def\LR{{\mathrm{LR}}}

\newcounter{rowItemCount}
\newcounter{subRowItemCount}
\newcommand\rowItem{
    \setcounter{subRowItemCount}{0}
    \arabic{rowItemCount}.\addtocounter{rowItemCount}{1}}
\newcommand\subRowItem{
    \addtocounter{subRowItemCount}{1}
    \addtocounter{rowItemCount}{-1}
    \arabic{rowItemCount}.\arabic{subRowItemCount}.\addtocounter{rowItemCount}{1}}

\newcommand{\smalll}[1]{\overline{\overline{#1}}}
\newcommand{\smallo}{\bar{\bar{o}}}
\newcommand{\ZZ}{\mathbb{Z}}
\newcommand{\Nz}{\mathbb{N}\cup\{0\}}
\newcommand{\NN}{\mathbb{N}}
\newcommand{\RR}{\mathbb{R}}
\begin{document}
\maketitle
\subsection*{(каноническое) Задача 41}
{\bf Модель вычислений: RAM, трудоемкость $C$~--- суммарное количество арифметических операций, присваиваний, сравнений.}\newline
Мое решение задания 2 $\Rightarrow$ \begin{enumerate}
\item \label{41GnRec} $g_n = 2g_{n-1} + g_{n-2}$, $g_0=1,\,g_1=3$
\item \label{41Gn2} $g_n=\frac{1}{2}\left[(1+\sqrt{2})^{n+1}+(1-\sqrt{2})^{n+1}\right]$
\end{enumerate}
\begin{enumerate}
\item \begin{enumerate}
\item Алгоритм:
\lstset{
    language=C,
    basicstyle=\ttfamily\small,
    breaklines=true,
    prebreak=\raisebox{0ex}[0ex][0ex]{\ensuremath{\hookleftarrow}},
    frame=lines,
    showtabs=false,
    showspaces=false,
    showstringspaces=false,
    keywordstyle=\color{red}\bfseries,
    stringstyle=\color{green!50!black},
    commentstyle=\color{gray}\itshape,
    numbers=left,
    captionpos=t,
    escapeinside={\%*}{*)}
}
\begin{lstlisting}
number g(number n, number p)
{
  number a = 1;
  number b = 3;

  if(n == 0) return(a);
  if(n == 1) return(b);

  number i, bt;
  for(i = 2; i <= n; i++)
  {
    bt = 2 * b + a;
    a = b;
    b = bt;
  }

  return(b);
}
\end{lstlisting}
\item Корректность\begin{enumerate}
\item $g_0=1=g(0)$ (строка 6)
\item $g_1=3=g(1)$ (строка 7)
\item $n\geqslant 2$:\begin{enumerate}
\item $P_i=\left[\mbox{после }i\mbox{-й итерации цикла }a=g_{i-1},\,b=g_i\right]$. $i$-я итерация цикла~--- при таком значении переменной $i$.
\item $P_1$ (до цикла) верно: (строки 3, 4): $a=g_{1-1}=1$, $b=g_1=3$.
\item Пусть $P_k$. Тогда $a\equiv a_{\mbox{\tiny old}}=g_{k-1}$, $b\equiv b_{\mbox{\tiny old}}=g_k$ после $k$-й итерации. После следующей ($k+1$) итерации $a=b_{\mbox{\tiny old}}=g_k$, $b=2b_{\mbox{\tiny old}}+a_{\mbox{\tiny old}}=2g_k+g_{k-1}\overset{\ref{41GnRec}}{=}g_{k+1}$ $\blacksquare\,\forall k\geqslant 2\hookrightarrow P(k)$
\item В конце (после $n$-й итерации) $P(n)\Rightarrow b=g_n$ $\blacksquare\,\forall n\geqslant 2\hookrightarrow g(n)=g_n$ (строка 17)
\end{enumerate}
\end{enumerate}
\item Время работы. При $n\in\{0,1\}$ $C(0)=3,\,C(1)=4$. На каждой итерации цикла трудоемкость константная $c=8$, поэтому общее количество арифметических операций $$C(n)=\begin{cases}
3, & n=0\\
4, & n=1\\
5+8(n-1), & n\geqslant 2\\
\end{cases}$$
\item Вычисление по модулю: вычислим $g(n)$, вычислим $g(n) \mod p$. Добавляется одна единица трудоемкости.
\item Асимптотика $C(n)=O(n)$
\item Трудоемкость вычисления $A=g_{10000}\mod 19$: $C(10000)=5+8(9999)=79997$
\end{enumerate}
\item \begin{enumerate}
\item Фиксируем $p\in\NN$. Рассмотрим функцию $f\colon \NN\to \overline{0,p-1}^2$: $f(k)=(g_{k-1}\mod p,g_k\mod p)$. Область определения $|D_f|=\infty$, множество значений $|E_f|=p^2$, откуда $|E_f|<|D_f|$, получаем, что $f$~--- не инъективна, то есть, $\exists \NN\ni i\neq j\in\NN \colon f(i)=f(j)\Leftrightarrow (g_{i-1}\mod p,g_i\mod p)=(g_{j-1}\mod p,g_j\mod p)$
\item Фиксируем эти $i\neq j\colon f(i)=f(j)$. $P(t)\eqdef\left[f(i+t)=f(j+t)\right]$. $P(0)$ выполнено. Пусть $P(t)$. Тогда $f(i+t)=f(j+t)\Leftrightarrow \begin{cases}
g_{i+t-1}=g_{j+t-1}\mod p\\
g_{i+t}=g_{j+t}\mod p\\
\end{cases}$. Тогда $g_{i+t+1}\overset{\ref{41GnRec}}{=}2g_{i+t}+g_{i+t-1}\overset{P(t)}=2g_{j+t}+g_{j+t-1}\overset{\ref{41GnRec}}{=}g_{j+t+1}$, откуда $f(i+t+1)=f(j+t+1)$ (второе условие из $P(t)$). Значит, $P(t+1)$. По индукции $\forall t\in\NN\cup\{0\}\hookrightarrow P(t)\Rightarrow\forall t\in\NN\cup\{-1,0\}\hookrightarrow g_{i+t}=g_{j+t}$
\item То есть, последовательности $\{g_n\}_{n=i-1}^\infty=\{g_n\}_{n=j-1}^\infty$, откуда следует, что $\{g_n\mod p\}_{n=0}^\infty$~--- периодическая с периодом $|i-j|$, начиная с $\min(i-1,j-1)$.
% махание руками
Используя рекуррентность <<в обратную сторону>> получаем, что она периодическая с начала (с $n=0$).
\item Оценим период $|i-j|$. $|E_f|=p^2$, откуда $|i-j|\leqslant p^2$. Пусть иначе: $|i-j|\geqslant p^2+1$. Без ограничения общности, $i<j$. Тогда $f(i),f(i+1),...,f(j-1)$~--- все различны. Их количество $j-i\geqslant p^2+1$, и они из $E_f$~--- противоречие, $|E_f|=p^2$.
\item Для $p=19$: $|i-j|\leqslant 19^2=361$.
\item Алгоритм: вычисляем период: храним $f(1)$, сравниваем $f(i)$ с $f(1)$. Вычисляем $g_i$ через рекуррентность (см. выше). Ищем место от начала периода для $n$ и выдаем ответ. Сложность $O(p^2)$ (величина периода). Для $A$: $p^2=361$, откуда $C\leqslant 2\times(\underbrace{5+8(361-1)}_{\mbox{\tiny период}})=5770$ ($2$~--- т.к. в два прохода, сначала поиск периода, потом вычисление $A$).
\end{enumerate}
\item $p=23$. $x=5$: $x^2=25\equiv 2\mod p$. $\ref{41Gn2}\Rightarrow g_n=|t\eqdef\sqrt{2}|=\frac{1}{2}\left[(1+t)^{n+1}+(1-t)^{n+1}\right]=\frac{1}{2}\left[\sum\limits_{k=0}^{n+1}\binom{n+1}{k}t^k-\sum\limits_{k=0}^{n+1}\binom{n+1}{k}(-t)^k\right]=\sum\limits_{k=0}^{n+1}\binom{n+1}{k}\frac{1+(-1)^k}{2}t^k\boxed{=}$. $1+(-1)^k=\begin{cases}
2,&k=2l\\
0,&k=2l+1
\end{cases}$, поэтому $\boxed{=}\sum\limits_{l=0}^{\lfloor\frac{n+1}{2}\rfloor}\binom{n+1}{2l}t^{2l}\equiv\sum\limits_{l=0}^{\lfloor\frac{n+1}{2}\rfloor}\binom{n+1}{2l}2^l$. Поэтому $g_n\mod p=\left[\sum\limits_{l=0}^{\lfloor\frac{n+1}{2}\rfloor}\binom{n+1}{2l}2^l\right]\mod p=|x^2\equiv 2\mod p|=\sum\limits_{l=0}^{\lfloor\frac{n+1}{2}\rfloor}\left[\binom{n+1}{2l}x^{2l}\right]\mod p$. Раскрывая обратно по той же формуле, получаем $$g_n\mod p=\frac{(1+x)^{n+1}+(1-x)^{n+1}}{2}\mod p.$$
Для конкретной задачи $$g_n\mod 23=\frac{6^{n+1}+19^{n+1}}{2}\mod 23$$
Возводим в степень Быстрым возведением в степень. Количество операций $O(\log n)$.
Алгоритм:
\lstset{
    language=C,
    basicstyle=\ttfamily\small,
    breaklines=true,
    prebreak=\raisebox{0ex}[0ex][0ex]{\ensuremath{\hookleftarrow}},
    frame=lines,
    showtabs=false,
    showspaces=false,
    showstringspaces=false,
    keywordstyle=\color{red}\bfseries,
    stringstyle=\color{green!50!black},
    commentstyle=\color{gray}\itshape,
    numbers=left,
    captionpos=t,
    escapeinside={\%*}{*)}
}
\begin{lstlisting}
number n = 10000;
number p = 23;
number x = 5;
number pow1(number a, number n)
{
  if(n == 0) return(1 % p);
  else if(n % 2 == 0)
  {
    number m = pow1(a, n / 2);
    return((m * m) % p);
  }
  else return((a * pow1(a, n - 1)) % p);
}

number g(n)
{
  return((pow1(6, n + 1) + pow1(19, n + 1)) / 2);
}

print(g(n));
\end{lstlisting}
Ответ: $g_{10000}\mod 23=10$.
\item Вернемся к формуле $g_n\mod p=\left[\sum\limits_{l=0}^{\lfloor\frac{n+1}{2}\rfloor}\binom{n+1}{2l}2^l\right]\mod p$, посчитаем по ней {\bf ???}
\end{enumerate}

\subsection*{(каноническое) Задача 42}
{\em (Кормен)}
\begin{enumerate}
\item $N\in\NN,a\colon (a,N)=1, a^{N-1}\neq 1\mod N$. $\ZZ_N^*=\{a\big|a<N,\,(a,N)=1\}$. Рассмотрим множество $G=\{x\big|x^{N-1}=1\}$. Пусть $x\in G$. Тогда $x\cdot x^{N-2}=1\mod N$, то есть, существует обратный элемент к $x$, откуда $x\in \ZZ_N^*$. Значит, $G\subseteq \ZZ_N^*$. Пусть $x_1,\,x_2\in G\Rightarrow x_1^{N-1}=1\mod p$, $x_2^{N-1}=1\mod p$. Тогда $x_1x_2=1\mod p$, и $x_1x_2\in G$. Получаем, что $G$~--- замкнута, значит, $G$~--- подгруппа $\ZZ_N^*$. Теорема Лагранжа $\Rightarrow$ $|\ZZ_N^*|=k|G|$. По условию, $a\in\ZZ_N^*\setminus G$, откуда $|G|<|\ZZ_N^*|$. Значит, $k\geqslant 2$, и $|G|\leqslant \frac{|\ZZ_N^*|}{2}$. Но $|\ZZ_N^*|=\varphi(N)\leqslant N-1$, откуда $|G|\leqslant\frac{N-1}{2}$. Рассмотрим $\overline{G}=\overline{1,N-1}\setminus G$. Очевидно, $1\notin \overline{G}$, так как $1^{N-1}=1\mod p$. Тогда $\overline{G}\subseteq\overline{2,N-1}$, причем $|\overline{G}|=|\overline{1,N-1}|-|G|\geqslant N-1-\frac{N-1}{2}=\frac{N-1}{2}\,\blacksquare$
\item $\mbox{НОД}(a,b)$~--- полиномиален по $|a|,\,|b|$ (лекции), вычисление $a^{N-1}\mod N$~--- также (быстрое возведение в степень, см. мое решение задачи 12).
\item $P(a=i\in\overline{2,N-1})=\frac{1}{N-1}=p$. Пусть $N$~--- составное. Тогда $\exists a<N\colon (a,N)=1$.\begin{enumerate}
\item С вероятностью $\frac{1}{N-1}$ алгоритм выдаст правильный ответ (угадан делитель)
\item В противном случае с вероятностью $\geqslant \frac{1}{2}$ (см. первый пункт. По условию, хотя бы одно такое $a$ существует, значит, таких $a$ не меньше половины) будет выбрано $a\colon a^{N-1}\neq 1\mod p$, и будет выдан правильный ответ.
\end{enumerate}
Поэтому $P\geqslant \frac{1}{N-1}+(1-\frac{1}{N-1})\frac{1}{2}=1-\frac{1}{2(N-1)}$. $N\geqslant 2\Rightarrow N-1\geqslant 1\Rightarrow P\geqslant\frac{1}{2}\,\blacksquare$
\end{enumerate}
\subsection*{(каноническое) Задача 43}
Открытый ключ $(e,n)=(11, 899)$. $M\in\ZZ_n$~--- сообщение.\newline
Чтобы вычислить цифровую подпись сообщения $M$, необходимо вычислить $S(M)=M^d\mod n$,\newline
где $d\colon ed=1\mod \varphi(n)$. $n=\underbrace{29}_p\cdot \underbrace{31}_q$, $p,\,q$~--- простые, поэтому $\varphi(n)=(p-1)(q-1)=28\cdot 30=840$.\newline
Подбором найдем $d=611$, проверим: $ed=6721\equiv 1\mod 840$ $\Rightarrow$ Ответ: в степень $d=611$.
\subsection*{(каноническое) Задача 44}
\begin{enumerate}
\item $8\varphi(n)\geqslant\sqrt{n}$ при $n\notin\{2,6\}$. Получим, что $n\leqslant 8^2=64$. Перебором найдем $n\in\{15,16,20,24,30\}$.
\item Перебором найдем\newline
$\begin{array}{|c|l|}
\hline
\mbox{Порядок} & \mbox{Элементы}\\\hline
1 & 1\\\hline
11 & 2,3,4,6,8,9,12,13,16,18\\\hline
22 & 5,7,10,11,14,15,17,19,20,21\\\hline
\end{array}$
\item Перебором найдем множество первообразных корней $\{5,\,7,\,10,\,11,\,14,\,15,\,17,\,19,\,20,\,21\}$
\end{enumerate}
\subsection*{(каноническое) Задача 45}
$\ZZ_9=\{0,1,2,3,4,5,6,7,8\}$, $\ZZ_9^*=\{x\big|x<9,\,(x,9)=1\}=\{1,2,4,5,7,8\}$. $h_{a,b}(k)\eqdef\left[(ak+b)\mod 8\right]\mod 5$.\newline $H\eqdef\{h_{a,b}\big|a\in\ZZ_9^*,\,b\in\ZZ_9\}$.\begin{enumerate}
\item $n_2=5$, так как $\forall a,b,k\hookrightarrow h_{a,b}(k)\in\overline{0,4}$.
\item $|H|\leqslant|\ZZ_9\times\ZZ_9^*|=9\cdot 6=54$.
\item $h_{a_1,b_1}(k)=h_{a_2,b_2}(k)\Leftrightarrow ((a_1k+b_2)\mod 8)\mod 5=((a_2k+b_2)\mod 8)\mod 5\Leftrightarrow(\left[(a_1-a_2)k+(b_1-b_2)\right]\mod 8)\mod 5=0$.
\item Функции $h_{a_1,b_1}$, $h_{a_2,b_2}$ совпадают $\Leftrightarrow$ $\forall k\in U\hookrightarrow h_{a_1,b_1}(k)=h_{a_2,b_2}(k)$.
\item $((ak+b)\mod 8)\mod 5=0\Leftrightarrow\exists l_1\colon 5\big| (ak+b)\mod 8\Leftrightarrow \left[\begin{array}{l}
ak+b\equiv 5\mod 8\\
ak+b\equiv 0\mod 8
\end{array}
\right.$ {\bf ???}
\item Перебором найдем $|H|=45$ (количество различных хеш-функций).
\item $n_1=8$ ($k=k_0+8l\Rightarrow h_{a,b}(k)=h_{a,b}(k_0)$)
\item Нет, $H$~--- не универсальное (например, для $k=0$, $l=2$ количество функций $h\in H\colon h(k)=h(l)$ равно $13>\frac{|H|}{n_2}=\frac{45}{5}=9$).
\end{enumerate}

\end{document}