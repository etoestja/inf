\documentclass[a4paper]{article}
\usepackage[a4paper, left=5mm, right=5mm, top=5mm, bottom=5mm]{geometry}
%\geometry{paperwidth=210mm, paperheight=2000pt, left=5pt, top=5pt}
\usepackage[utf8]{inputenc}
\usepackage[english,russian]{babel}
\usepackage{indentfirst}
\usepackage{tikz} %Рисование автоматов
\usetikzlibrary{automata,positioning,arrows}
\usepackage{amsmath}
\usepackage{enumerate}
\usepackage{amsfonts}
\usepackage{amssymb}
\usepackage{wasysym}
\title{Теория и реализация языков программирования.\\Задание 6: Грамматики}
\date{задано 2013.10.09}
\author{Сергей~Володин, 272 гр.}
\newcommand{\matrixl}{\left|\left|}
\newcommand{\matrixr}{\right|\right|}
% названия автоматов
\def\A{{\cal A}}
\def\B{{\cal B}}
\def\C{{\cal C}}

%+= и -=, иначе разъезжаются...
\newcommand{\peq}{\mathrel{+}=}
\newcommand{\meq}{\mathrel{-}=}
\newcommand{\deq}{\mathrel{:}=}
\newcommand{\plpl}{\mathrel{+}+}

% регулярные языки
\def\REG{{\mathsf{REG}}}
\def\eqdef{\overset{\mbox{\tiny def}}{=}}
\newcommand{\niton}{\not\owns}

\begin{document}
\maketitle
\section*{Задача 1}
\section*{Задача 2}
$\Sigma\eqdef\{a,b\}$, $\Sigma^*\supset L\eqdef\{w\in\Sigma^*\big|w=w^R\}$~--- язык палиндромов из $a,b$.
\begin{enumerate}
\def\ansba{$\Gamma\eqdef(\{S\},\Sigma,P,S)$, $P\eqdef\big\{\underbrace{S\longrightarrow\varepsilon}_{(1)},\,\underbrace{S\longrightarrow aSa}_{(2)},\,\underbrace{S\longrightarrow bSb}_{(3)},\,\underbrace{S\longrightarrow a}_{(4)},\,\underbrace{S\longrightarrow b}_{(5)}\big\}$}
\def\ansbac{\Gamma\eqdef(\{S\},\Sigma,P,S)$, $P\eqdef\big\{S\longrightarrow\varepsilon,\,S\longrightarrow aSa,\,S\longrightarrow bSb,\,S\longrightarrow a,\,S\longrightarrow b\big\}}
\item Определим КС-грамматику \ansba.\newline
%\begin{enumerate}[a.]
%\item $S\longrightarrow\varepsilon$,
%\item $S\longrightarrow aSa$,
%\item $S\longrightarrow bSb\big\}$
%\end{enumerate}
Докажем, что $L(\Gamma)=L$:
\begin{enumerate}
\item $L(\Gamma)\subseteq L$. Пусть $w\in L(\Gamma)\Rightarrow S\Longrightarrow^* w$. $|w|=n$. Рассмотрим последовательность $\{w_k\}^I_{k=0}\subset (N\cup\Sigma)^*$ слов в выводе. $w_0=S,\,w_I=w$. Индукцией по $k$ докажем $P(k)=\big[w_k=w_k^R,\forall i\colon w_k[i]=S\hookrightarrow i=\frac{|w_k|+1}{2}\big]$.\begin{enumerate}[1.]
\item $k=0\Rightarrow w_k\equiv w_0=S$. Поэтому $\exists ! i=1\colon w_0[i]=S$. Но $1\equiv\frac{1+1}{2}$ и $w_0^R=S^R=S=w_0\Rightarrow P(0)\,\blacksquare$
\item Пусть $P(n),\,n<I$. Докажем, $P(n+1)$. $P(n)\Rightarrow w_n=w_n^R,\,\forall i\colon w_n[i]=S\hookrightarrow i=\frac{|w_n|+1}{2}$.
\\[1pt]
Предположим, что $\nexists i\colon w_n[i]=S\Rightarrow w\in \Sigma^*$. Тогда ни одно правило не может быть применено, так как в левой части каждого правила $S\in N$. Но $n<I$ (это не конец вывода) $\Rightarrow$ противоречие.
\\[1pt]
Значит, $\exists i\colon w_n[i]=S$. Но $P(n)\Rightarrow \forall i\colon w_n[i]=S\hookrightarrow i=\frac{|w_n|+1}{2}$. Поэтому $\exists!i=\frac{|w_n|+1}{2}\colon w_n[i]=S$. Значит, $w_n=xSy$, $|x|=|y|$, $x,y\in\Sigma^*$. $w_n^R=y^RSx^R$. $S$ в $w_n$ входит один раз $\Rightarrow$ $x=y^R$.\newline
Рассмотрим правила $(1)$---$(4)$:\begin{enumerate}[(1).]
%\item $w_n[1]...w_n[\frac{n-1}{2}]Sw_n[\frac{n+3}{2}]...w_n[n]\overset{(1)}{\Longrightarrow}w_n[1]...w_n[\frac{n-1}{2}]w_n[\frac{n+3}{2}]...w_n[n]\equiv w_{n+1}$.\newline
%Но $w_n^R=w_n\Rightarrow$ $w_n[1]=w_n{n},...,w_n[\frac{n-1}{2}]=w_n[\frac{n+3}{2}]$. Поэтому $w_{n+1}^R=w_{n+1}]$.\newline
%Также $S$ не встречается в $w_{n+1}\Rightarrow P(n+1)\,\blacksquare$
\item $w_n=xSy\overset{(1)}{\Longrightarrow} x\varepsilon y\equiv xy=xx^R=w_{n+1}$~--- палиндром: $(xx^R)^R={x^R}^Rx^R=xx^R$. $w_{n+1}=xy\in\Sigma^*\Rightarrow\forall i\hookrightarrow w_{n+1}[i]\neq S\Rightarrow P(n+1)\,\blacksquare$
\item $w_n=xSx^R\overset{(2)}{\Longrightarrow}xaSax^R=w_{n+1}$. $w_{n+1}^R={x^R}^Ra^RS^Ra^Rx^R=xaSax^R\equiv w_{n+1}$. $a\neq S\Rightarrow \exists!i\colon w_{n+1}=S$,\newline$i=\frac{|w_n|+1}{2}+1=\frac{|w_n|+3}{2}\equiv\frac{|w_{n+1}|+1}{2}\Rightarrow P(n+1)\,\blacksquare$.
\item $w_n=xSx^R\overset{(3)}{\Longrightarrow}xbSbx^R=w_{n+1}$. $w_{n+1}^R={x^R}^Rb^RS^Rb^Rx^R=xbSbx^R\equiv w_{n+1}$. $b\neq S\Rightarrow \exists!i\colon w_{n+1}=S$,\newline$i=\frac{|w_n|+1}{2}+1=\frac{|w_n|+3}{2}\equiv\frac{|w_{n+1}|+1}{2}\Rightarrow P(n+1)\,\blacksquare$.
\item $w_n=xSx^R\overset{(4)}{\Longrightarrow}xax^R=w_{n+1}$. $w_{n+1}^R={x^R}^Ra^Rx^R=xax^R\equiv w_{n+1}$. $w_{n+1}=xax^R\in\Sigma^*\Rightarrow \forall i\hookrightarrow w_{n+1}[i]\neq S\Rightarrow P(n+1)\,\blacksquare$
\item $w_n=xSx^R\overset{(5)}{\Longrightarrow}xbx^R=w_{n+1}$. $w_{n+1}^R={x^R}^Rb^Rx^R=xbx^R\equiv w_{n+1}$. $w_{n+1}=xbx^R\in\Sigma^*\Rightarrow \forall i\hookrightarrow w_{n+1}[i]\neq S\Rightarrow P(n+1)\,\blacksquare$
\end{enumerate}
Итак, доказано $\forall k\in\overline{0,I}\hookrightarrow P(k)\Rightarrow P(I)\Rightarrow w\equiv w_I\overset{P(I)}{=}w_I^R\equiv w^R\Rightarrow w\in L\,\blacksquare$ 
\end{enumerate}
\item $L\subseteq L(\Gamma)$. Пусть $w\in L\Rightarrow w^R=w$. $|w|=n$. Рассмотрим $n\mod 2$:\begin{enumerate}[0.]
\item $n\mod 2=0\Rightarrow w=xy,\,|x|=|y|$. $w=w^R\Rightarrow xy=y^Rx^R$. Поскольку $|x|=|y|$, $y=x^R\Rightarrow\boxed{w=xx^R}$.
\item $n\mod 2=1\Rightarrow w=x\sigma y,\, |x|=|y|,\,\sigma\in\Sigma$. $w=w^R\Rightarrow x\sigma y=y^R\sigma^Rx^R=y^R\sigma x^R$. Так как $|x|=|y|$, $y=x^R\Rightarrow\boxed{w=x\sigma x^R}$.
\end{enumerate}
Значит, $L=\{xx^R,\,xax^R,\,xbx^R\big| x\in\Sigma^*\}$.
\\[5pt]
Построим вывод $S\Longrightarrow^*xSx^R$:\begin{enumerate}[a.]
\item Пусть $x=\varepsilon$. $S\overset{(1)}{\Longrightarrow}\varepsilon=\varepsilon\varepsilon^R=w\Rightarrow w\in L(\Gamma)\,\blacksquare$. 
\item Иначе $x=x_1...x_m,\,\forall i\in\overline{1,m}\hookrightarrow x_i\in\Sigma$. Рассмотрим символы $x_m,...,x_1$. Применим правило $(2)$, если $x_i=a$ и $(3)$ иначе. Примененное правило обозначим за $R(i)$ Получим $S\overset{(R(m))}{\Longrightarrow}x_mSx_m\Longrightarrow...\overset{(R(1))}{\Longrightarrow}x_1...x_mSx_m...x_1$.
\end{enumerate}
Теперь покажем, как получить $w$:\begin{enumerate}[1.]
\item $w=xx^R$. Было получено $S\Longrightarrow^*xSx^R$. Тогда $S\Longrightarrow^*xSx^R\overset{(1)}{\Longrightarrow}xx^R\,\blacksquare$
\item $w=xax^R$. Было получено $S\Longrightarrow^*xSx^R$. Тогда $S\Longrightarrow^*xSx^R\overset{(4)}{\Longrightarrow}xax^R\,\blacksquare$
\item $w=xbx^R$. Было получено $S\Longrightarrow^*xSx^R$. Тогда $S\Longrightarrow^*xSx^R\overset{(5)}{\Longrightarrow}xbx^R\,\blacksquare$
\end{enumerate}
Получаем $w\in L(\Gamma)$.\newline
$\boxed{\mbox{Ответ: }\ansbac}$
\end{enumerate}
\item Определим грамматику $\overline{\Gamma}$
\end{enumerate}
\section*{Задача 3}
$\Sigma\eqdef\{a,b\}$. $\Sigma^*\supset L^=\eqdef\{w\in\Sigma^*\big||w|_a=|w|_b\}$. КС-грамматика $\Gamma=\{N,\Sigma,P,S\}$, $P\eqdef\big\{\underbrace{S\longrightarrow SS}_{(1)},\,\underbrace{S\longrightarrow aSb}_{(2)},\,\underbrace{S\longrightarrow bSa}_{(3)}\big\}$
%\begin{enumerate}[a.]
%\item $S\longrightarrow SS$,
%\item $S\longrightarrow aSb$,
%\item $S\longrightarrow bSa\big\}$
%\end{enumerate}
\section*{Задача 4}
$\Sigma\eqdef\{a,b\}$, $\Sigma^*\supset L\eqdef\{w\in\Sigma^*\big||w|_b\leqslant|w|_a\}$. Определим грамматику $\Gamma$.
\section*{Задача 5}
\end{document}
