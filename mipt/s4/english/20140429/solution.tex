\documentclass[a4paper]{article}
\usepackage[a4paper]{geometry}
%\geometry{paperwidth=210mm, paperheight=2000pt, left=5pt, top=5pt}
\usepackage[utf8]{inputenc}
%\usepackage[english]{babel}
\usepackage{indentfirst}
\usepackage{amsmath}
\usepackage[makeroom]{cancel} % зачеркивание
\usepackage{multicol,multirow} %Несколько колонок
\usepackage{hyperref}
\usepackage{tabularx}
\usepackage{amsfonts}
\usepackage{amssymb}
\usepackage{wasysym}
\author{Sergey Volodin, group 272}
\title{An overview of Globalization, its positive and negative aspects\\(abstract)}
%\date{2014.04.22}
\begin{document}
\maketitle
Nowadays many products are being made in very different places of the Earth. It is caused by globalization.
\\[5pt]
Globalization makes world interconnected: items produced in one area appear in different ones. 
\\[1pt]
Globalization has several qualities. Firstly, transportation and communication become more available and widely-used.

Secondly, people are moving from developing countries to developed ones. In addition, there are a lot of electronic money transfer and investing in developing countries.

Thirdly, knowledge spreads fast.

Fourthly, there are non-governmental organizations which are national or worldwide. Countries which have access to the rest of the world, form a market. Also there is outsourcing, which is employing foreign workers instead of domestic ones. In addition, borders are becoming less important.

There are pros and cons of globalization. On the one hand, standarts of living in developing countries are increasing, prices are stable because of competition, developing countries are able to get state-of-the art technologies without doing research into it, governments are able to work together, there is access to other cultures.

On the other hand, outsourcing may leave some people without a job, cultures meld, there is a threat of an disease, absence of rules for international interactions.
\end{document}
