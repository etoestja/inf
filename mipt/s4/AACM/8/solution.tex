\documentclass[a4paper]{article}
\usepackage[a4paper, left=5mm, right=5mm, top=5mm, bottom=5mm]{geometry}
%\geometry{paperwidth=210mm, paperheight=2000pt, left=5pt, top=5pt}
\usepackage[utf8]{inputenc}
\usepackage[english,russian]{babel}
\usepackage{indentfirst}
\usepackage{tikz} %Рисование автоматов
\usetikzlibrary{automata,positioning,arrows,trees,calc}
\usepackage{amsmath}
\usepackage[makeroom]{cancel} % зачеркивание
\usepackage{multicol,multirow} %Несколько колонок
\usepackage{hyperref}
\usepackage{tabularx}
\usepackage{amsfonts}
\usepackage{amssymb}
\DeclareMathOperator*{\argmin}{arg\,min}
\usepackage{listings}
\usepackage{wasysym}
\date{задано 2014.03.27}
\author{Сергей~Володин, 272 гр.}
\newcommand{\matrixl}{\left|\left|}
\newcommand{\matrixr}{\right|\right|}
% названия автоматов  (rubtsov)
\def\A{{\cal A}}
\def\B{{\cal B}}
\def\C{{\cal C}}

%классы сложности (rubtsov)
\def\PP{{\mathsf{P}}}
\def\NP{{\mathsf{NP}}}
\def\NPc{{\mathsf{NP}}\text{-}{\rm c}}
\def\coNP{{\rm co}\text{-}{\mathsf{NP}}}
\def\DTIME{{\mathsf{DTIME}}}
\def\NTIME{{\mathsf{NTIME}}}
\def\CLIQUE{{\mathsf{CLIQUE}}}
\def\HALT{{\rm{HALT}}}
\def\SAT{{\rm{SAT}}}
\def\3SAT{{\rm{3\text{-}SAT}}}
\def\2SAT{{\rm{2\text{-}SAT}}}
\def\UNSAT{{\rm{UNSAT}}}
\def\HP{{\rm{HAMPATH}}}
\def\UHP{{\rm{UHAMPATH}}}
\def\LL{{\mathrm{LL}}}
\def\poly{{\rm{poly}}}
\def\GC{{\mbox{ГЦ}}}
\def\GP{{\mbox{ГП}}}
\def\conv{{\mbox{conv}}}

\title{Алгоритмы и модели вычислений.\\Задание 8: линейное программирование}

% алгоритмы (Рубцов)
\usepackage{verbatim}
\usepackage{listings}
\usepackage{algorithm2e}

%+= и -=, иначе разъезжаются...
\newcommand{\peq}{\mathrel{+}=}
\newcommand{\meq}{\mathrel{-}=}
\newcommand{\deq}{\mathrel{:}=}
\newcommand{\plpl}{\mathrel{+}+}

\newcommand{\todo}{{\em todo}}

% пустое слово  (rubtsov)
\def\eps{\varepsilon}

% регулярные языки  (rubtsov)
\def\REG{{\mathsf{REG}}}
\def\CFL{{\mathsf{CFL}}}
\def\eqdef{\overset{\mbox{\tiny def}}{=}}
\newcommand{\niton}{\not\owns}

%FIRST & FOLLOW (rubtsov)
\def\first{\mathrm{ FIRST} }
\def\follow{\mathrm{ FOLLOW} }

% LL LR (rubtsov)
\def\LL{{\mathrm{LL}}}
\def\LR{{\mathrm{LR}}}

\newcounter{rowItemCount}
\newcounter{subRowItemCount}
\newcommand\rowItem{
    \setcounter{subRowItemCount}{0}
    \arabic{rowItemCount}.\addtocounter{rowItemCount}{1}}
\newcommand\subRowItem{
    \addtocounter{subRowItemCount}{1}
    \addtocounter{rowItemCount}{-1}
    \arabic{rowItemCount}.\arabic{subRowItemCount}.\addtocounter{rowItemCount}{1}}
    
\newcommand{\smalll}[1]{\overline{\overline{#1}}}
\newcommand{\smallo}{\bar{\bar{o}}}
\newcommand{\ZZ}{\mathbb{Z}}
\newcommand{\Nz}{\mathbb{N}\cup\{0\}}
\newcommand{\NN}{\mathbb{N}}
\newcommand{\RR}{\mathbb{R}}
\begin{document}
\maketitle
\subsection*{(каноническое) Задача 32}
$n\in\NN$, $\{(x_i,y_i)\}_{i=1}^n\subset\RR^2$. Задача: найти $(a_0,c_0)=\argmin\limits_{(a,c)\in\RR^2}\max\limits_{i\in\overline{1,n}}|ax_i+y_i+c|$.\newline
Сведем к задаче ЛП: переменные $(a,c,M)$, неравенства: $\left\{\begin{array}{lcr}
ax_i+y_i+c& \leqslant &M\\
ax_i+y_i+c&\geqslant &-M\\
\end{array}\right.,\,i\in\overline{1,n}$, $M\to\min$\newline
Выпишем конкретную задачу ($n=7$, точки даны):\newline
$M\to\min$,
$\left\{\begin{array}{lcr}
a+3+c & \leqslant & M\\
-a-3-c & \leqslant & M\\

2a+5+c & \leqslant & M\\
-2a-5-c & \leqslant & M\\

3a+7+c & \leqslant & M\\
-3a-7-c & \leqslant & M\\

5a+11+c & \leqslant & M\\
-5a-11-c & \leqslant & M\\

7a+14+c & \leqslant & M\\
-7a-14-c & \leqslant & M\\

8a+15+c & \leqslant & M\\
-8a-15-c & \leqslant & M\\

10a+19+c & \leqslant & M\\
-10a-19-c & \leqslant & M\\
\end{array}\right.
$. Каноническая форма: $T\eqdef -M$, $a=a_+-a_-$, $c=c_+-c_-$, новые переменные $t_1,...,t_{14}$:\newline
$\left\{\begin{array}{lll}
z=T\\
1a_+-1a_-+3+c_+-c_-+T+t_{1}=0\\
-1a_++1a_--3-c_++c_-+T+t_{2}=0\\

2a_+-2a_-+5+c_+-c_-+T+t_{3}=0\\
-2a_++2a_--5-c_++c_-+T+t_{4}=0\\

3a_+-3a_-+7+c_+-c_-+T+t_{5}=0\\
-3a_++3a_--7-c_++c_-+T+t_{6}=0\\

5a_+-5a_-+11+c_+-c_-+T+t_{7}=0\\
-5a_++5a_--11-c_++c_-+T+t_{8}=0\\

7a_+-7a_-+14+c_+-c_-+T+t_{9}=0\\
-7a_++7a_--14-c_++c_-+T+t_{10}=0\\

8a_+-8a_-+15+c_+-c_-+T+t_{11}=0\\
-8a_++8a_--15-c_++c_-+T+t_{12}=0\\

10a_+-10a_-+19+c_+-c_-+T+t_{13}=0\\
-10a_++10a_--19-c_++c_-+T+t_{14}=0\\

a_+,a_-,c_+,c_-,T,t_1,...,t_{14}\geqslant 0
\end{array}\right.$
\newpage
\subsection*{(каноническое) Задача 33}
$P_\eps\eqdef\left\{(x_1,x_2,x_3)\in\RR^3\left| \left\{\begin{array}{lclllc}
(*_1) & 0        & \leqslant & x_1 & \leqslant & 1\\
(*_2) & \eps x_1 & \leqslant & x_2 & \leqslant & 1-\eps x_1\\
(*_3) & \eps x_2 & \leqslant & x_3 & \leqslant & 1-\eps x_2
\end{array}
\right.\right.\right\}$.\newline
Путь:\begin{enumerate}
\item $\vec{x}_1=(0,0,0)\in P_\eps$:\begin{enumerate}
\item[$(*_1)$] $0\leqslant 0\leqslant 1$
\item[$(*_2)$] $0\leqslant 0\leqslant 1-0$
\item[$(*_3)$] $0\leqslant 0\leqslant 1-0$
\end{enumerate}
\item $\vec{x}_2=(1,\eps,\eps^2)\in P_\eps$:\begin{enumerate}
\item[$(*_1)$] $0\leqslant 1\leqslant 1$
\item[$(*_2)$] $\eps\leqslant \eps\leqslant 1-\eps$ ($\eps<\frac{1}{2}$)
\item[$(*_3)$] $\eps^2\leqslant \eps^2\leqslant 1-\eps^2$ ($\eps^2<\frac{1}{4}<\frac{1}{2}$)
\end{enumerate}
Высота больше: $\eps^2>0$

\item $\vec{x}_3=(1,1-\eps,\eps-\eps^2)\in P_\eps$:\begin{enumerate}
\item[$(*_1)$] $0\leqslant 1\leqslant 1$
\item[$(*_2)$] $\eps\leqslant 1-\eps\leqslant 1-\eps$ ($\eps<\frac{1}{2}$)
\item[$(*_3)$] $\eps-\eps^2\leqslant \eps-\eps^2\leqslant 1-\eps+\eps^2$ ($2\eps^2-2\eps+1>0,\,D=4-8<0$)
\end{enumerate}
Высота больше: $\eps-\eps^2>\eps^2$ ($\eps<\frac{1}{2}$)

\item $\vec{x}_4=(0,1,\eps)\in P_\eps$:\begin{enumerate}
\item[$(*_1)$] $0 \leqslant 0 \leqslant 1$
\item[$(*_2)$] $0 \leqslant 1 \leqslant 1$
\item[$(*_3)$] $\eps \leqslant \eps \leqslant 1-\eps$ ($\eps<\frac{1}{2}$)
\end{enumerate}
Высота больше: $\eps>\eps-\eps^2$ ($\eps>0$)

\item $\vec{x}_5=(0,1,1-\eps)\in P_\eps$:\begin{enumerate}
\item[$(*_1)$] $0\leqslant 0 \leqslant 1$
\item[$(*_2)$] $0 \leqslant 1 \leqslant 1$
\item[$(*_3)$] $\eps \leqslant 1-\eps \leqslant 1-\eps$ ($\eps<\frac{1}{2}$)
\end{enumerate}
Высота больше: $1-\eps>\eps$ ($\eps<\frac{1}{2}$)

\item $\vec{x}_6=(1,1-\eps,1-\eps+\eps^2)\in P_\eps$:\begin{enumerate}
\item[$(*_1)$] $0\leqslant 1 \leqslant 1$
\item[$(*_2)$] $\eps \leqslant 1-\eps \leqslant 1-\eps$ ($\eps<\frac{1}{2} $)
\item[$(*_3)$] $\eps-\eps^2 \leqslant 1-\eps+\eps^2 \leqslant 1-\eps+\eps^2$ ($2\eps^2-2\eps+1>0,\,D=4-8<0$)
\end{enumerate}
Высота больше: $1-\eps+\eps^2>1-\eps$ ($\eps>0$)

\item $\vec{x}_7=(1,\eps,1-\eps^2)\in P_\eps$:\begin{enumerate}
\item[$(*_1)$] $0\leqslant 1 \leqslant 1$ ($ $)
\item[$(*_2)$] $\eps \leqslant \eps \leqslant 1-\eps$ ($\eps>\frac{1}{2} $)
\item[$(*_3)$] $\eps^2 \leqslant 1-\eps^2 \leqslant 1-\eps^2$ ($\eps^2<\frac{1}{4}<\frac{1}{2} $)
\end{enumerate}
Высота больше: $1-\eps^2>1-\eps+\eps^2$ ($\eps<\frac{1}{2}$)

\item $\vec{x}_8=(0,0,1)\in P_\eps$:\begin{enumerate}
\item[$(*_1)$] $0\leqslant 0 \leqslant 1$
\item[$(*_2)$] $0\leqslant 0 \leqslant 1$
\item[$(*_3)$] $0\leqslant 1 \leqslant 1$
\end{enumerate}
Высота больше: $1>1-\eps^2$ ($\eps>0$)

\end{enumerate}
\subsection*{(каноническое) Задача 34}
$A=\begin{Vmatrix}
a_{ij}
\end{Vmatrix}_{i,j=1}^{m,n}$. $P_1\eqdef\big[\exists p\in\RR^m\colon A^Tp<0\big]$. $P_2\eqdef\big[\exists y\in\RR^n\colon y\geqslant 0,\,y\neq 0,\,Ay=0\big]$. Доказать: $\urcorner P_1\Leftrightarrow P_2$\begin{enumerate}
\item $e_i\eqdef \begin{Vmatrix}
0 & ... & \underbrace{1}_{i} & ... & 0
\end{Vmatrix}\in\RR^n$ $\Rightarrow$ $e\eqdef(e_1,...,e_n)$~--- стандартный базис в $\RR^n$. Скалярное произведение $(\cdot,\cdot)$~--- тоже стандартное, т.е. матрица Грама в $e$ единичная, т.е. $(\begin{Vmatrix}
x_1\\
...\\
x_n
\end{Vmatrix},
\begin{Vmatrix}
y_1\\
...\\
y_n
\end{Vmatrix})=x_1y_1+...+x_ny_n$
\item Пусть $P_2$. \begin{enumerate}
\item Тогда $\exists y\colon Ay=0$, $y\geqslant 0$, $y\neq 0$. Обозначим столбцы матрицы $A=\begin{Vmatrix}
\underline{b_1} & ... & \underline{b_n}
\end{Vmatrix}$. $y\in\RR^n\Rightarrow y=\begin{Vmatrix}
y_1 & ... & y_n
\end{Vmatrix}^T$ Тогда $Ay=0\Leftrightarrow\begin{Vmatrix}
\underline{b_1} & ... & \underline{b_n}
\end{Vmatrix}\cdot
\begin{Vmatrix}
y_1\\
...\\
y_n
\end{Vmatrix}
=0\Leftrightarrow \sum\limits_{i=1}^n \underline{b_i}y_i\overset{(*)}{=}0$. Условие $y\neq 0\Rightarrow\exists i\in\overline{1,n}\colon y_i\neq 0$. Без ограничения общности это $y_1$. Тогда в $(*)$ перенесем всё, кроме $y_1\underline{b_1}$ в правую часть, и поделим на $y_1\neq 0$: $\underline{b_1}=-\frac{y_2}{y_1}\underline{b_2}-...-\frac{y_n}{y_1}\underline{b_n}$
\item Рассмотрим $A^Tp=\begin{Vmatrix}
\underline{b_1}^T\\
...\\
\underline{b_n}^T\\
\end{Vmatrix}\cdot\begin{Vmatrix}
p_1\\
...\\
p_m
\end{Vmatrix}=\begin{Vmatrix}
(\underline{b_1},p)\\
...\\
(\underline{b_n},p)\\
\end{Vmatrix}$
\item Предположим, что $P_1$, т.е. $\exists p\colon\forall i\in\overline{1,n}\hookrightarrow (\underline{b_i},p)<0$.\newline
Рассмотрим $(\underline{b_1},p)=(-\frac{y_2}{y_1}\underline{b_2}-...-\frac{y_n}{y_1}\underline{b_n},p)=-\frac{y_2}{y_1}(\underline{b_2},p)-...-\frac{y_n}{y_1}(\underline{b_n},p)$. Поскольку $(\underline{b_i},p)<0$, $\frac{y_i}{y_1}\geqslant 0$, то $(b_1,p)\geqslant 0$~--- противоречие.
\item Значит, $\urcorner P_1$.
\end{enumerate}
\end{enumerate}
\subsection*{(каноническое) Задача 35}
\subsection*{(каноническое) Задача 36}
{\em (Тарасов, лекция 2014.04.01)}\newline
Фиксируем $k\in\NN$, $\{t_i\}_{i=1}^k\subset\RR$. Определим $\vec{r}\colon \RR\to \RR^4: \vec{r}(t)\eqdef \begin{Vmatrix}
t^4 & t^3 & t^2 & t
\end{Vmatrix}^T
$. Рассмотрим точки $\vec{x}_i=\vec{r}(t_i)$. Рассмотрим $G\eqdef\conv(\{\vec{x}_i\}_{i=1}^k)$~--- выпуклую оболочку этих точек. Фиксируем $i_1\neq i_2\in\overline{1,k}$. Докажем, что $\vec{x}_{i_1},\vec{x}_{i_2}$~--- вершины $G$, соединенные ребром $\overset{\mbox{def}}{\Leftrightarrow}$ $\exists$ гиперплоскость $\pi\colon$ ($\vec{x}_{i_1}$, $\vec{x}_{i_2}\in\pi$) и (многогранник $G$ лежит по одну сторону от $\pi$).\begin{enumerate}
\item Определим многочлен $P(t)\eqdef (t-t_{i_1})^2\cdot(t-t_{i_2})^2\equiv t^4+a_3t^3+a_2t^2+a_1t+a_0$
\item Определим гиперплоскость $\pi$. $\RR^4\ni\vec{x}\equiv \begin{Vmatrix}
x_1 & x_2 & x_3 & x_4\end{Vmatrix}^T\in\pi\Leftrightarrow F(\vec{x})\equiv x_1+a_3x_2+a_2x_3+a_1x_4+a_0=0$.
\item Тогда $F(\vec{r}(t))=P(t)$: $F(\vec{r}(t))=F(t^4,t^3,t^2,t)=t^4+a_3t^3+a_2t^2+a_1t+a_0$
\item $t_{i_1}$ и $t_{i_2}$~--- корни $P(t)$, откуда $P(t_{i_1})=P(t_{i_2})=0$, значит, $F(\vec{x}_{i_1})=F(\vec{x}_{i_2})=0$, значит, $\vec{x}_{i_1},\vec{x}_{i_2}\in\pi$
\item Фиксируем $t\in\RR$. Тогда $F(\vec{r}(t))=P(t)\geqslant 0$. Значит, все точки $\{\vec{x}_i\}_{i=1}^k$ лежат по одну сторону от $\pi$. Значит, $G$ лежит по одну сторону 
\item Пусть $t\colon \vec{r}(t)\in\pi\Leftrightarrow F(\vec{r}(t))=0\Leftrightarrow P(t)=0\Leftrightarrow$ $t\in\{t_{i_1},t_{i_2}\}$
\end{enumerate}
$\blacksquare$
\subsection*{(каноническое) Задача 37}
\end{document}