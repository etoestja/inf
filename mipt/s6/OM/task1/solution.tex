\documentclass[a4paper]{article}
\usepackage[a4paper, left=5mm, right=5mm, top=5mm, bottom=5mm]{geometry}
\usepackage[utf8]{inputenc}
\usepackage[english,russian]{babel}
\usepackage{graphicx}
\usepackage{cancel}
\usepackage{indentfirst}
\usepackage{tikz} %Рисование автоматов
\usetikzlibrary{automata,positioning}
\usepackage{amsmath}
\usepackage{floatflt}
\usepackage{enumerate}
\usepackage{amsfonts}
\usepackage{amssymb}
\newcommand{\matrixr}{\right|\right|}
\newcommand{\matrixl}{\left|\left|}
\def\eqdef{\overset{\mbox{\tiny def}}{=}}
\title{Методы оптимизации. Сдача, задача 3}
%\date{задано 2016.04.12}
\author{Сергей~Володин, 374 гр.}

\begin{document}
\maketitle
\section*{Задача 3}
Пусть $\{a_i\}_{i=1}^m\subset \mathbb{R}^n$. $\rho > 0$. Функция $f\colon \mathbb{R}^n\to\mathbb{R}$:
$$f(y)=\left(\sum\limits|(a_i,y)|^\rho\right)^{\frac{1}{\rho}}$$

Задача (1):
$$\sup\limits_{||y||_2\leqslant 1} f(y)$$

Требуется построить двойственную задачу.

Перепишем (1):
$$\min\limits_{||y||\leqslant 1} -f(y)$$

Функция Лагранжа:
$$L(y,\lambda)=-f(y)+\lambda(||y||-1)$$

Двойственная задача:

$$\inf\limits_{y\in\mathbb{R}^n} L(y,\lambda)\to\max\limits_{\lambda \geqslant 0}$$

Перепишем и получим двойственную задачу:

$$-\inf\limits_{y\in\mathbb{R}^n} L(y,\lambda)=\sup\limits_{y\in\mathbb{R}^n}-L(y,\lambda)=\sup\limits_{y\in\mathbb{R}^n}(f(y)-\lambda ||y||+\lambda)=\underbrace{\sup\limits_{y\in\mathbb{R}^n}(f(y)-\lambda||y||_2)+\lambda}_{g(\lambda)}\to\min\limits_{\lambda\geqslant 0}$$

Осталось найти $g(\lambda)$. Обозначим $$v(\lambda)=\sup\limits_{y\in\mathbb{R}^n}\underbrace{f(y)-\lambda||y||_2}_{M(y,\lambda)}.$$
Тогда $g(\lambda)=\lambda+v(\lambda)$. Найдем $v$.

Пусть $\alpha\in\mathbb{R}$. Заметим, что $f(\alpha y)=\left(\sum\limits_{i=1}^m|(a_i,\alpha y)|^\rho\right)^{\frac{1}{\rho}}=\left(|\alpha|^\rho\sum\limits_{i=1}^m|(a_i,y)|^\rho\right)^{\frac{1}{\rho}}=|\alpha|f(y)$. Также $||\alpha y||_2=|\alpha|\cdot||y||_2$, откуда
$M(\alpha y,\lambda)=|\alpha|(f(y)-\lambda ||y||_2)$.

Фиксируем $\lambda\geqslant 0$. Пусть $\exists y\in\mathbb{R}^n\colon\, f(y)-\lambda ||y||_2>0$. Тогда возьмем $0<\alpha_k\to\infty$ и получим $M(\alpha_k y,\lambda)=\alpha_k \underbrace{M(y,\lambda)}_{>0}\to+\infty$. То есть, $v(\lambda)=+\infty$

Пусть верно обратное, то есть, $\forall y\in\mathbb{R}^n\hookrightarrow f(y)-\lambda||y||_2\leqslant 0$. Но это значит, что $v(\lambda)\leqslant 0$. Но $v(\lambda)\geqslant M(0,\lambda)=0$, значит, $v(\lambda)=0$.

Перепишем условие $\forall y\in\mathbb{R}^n f(y)-\lambda ||y||_2\leqslant 0\Leftrightarrow \lambda\geqslant \sup\limits_{y\in\mathbb{R}^n}\frac{f(y)}{||y||_2}=\lambda^*$.

Получаем, что $g(\lambda)=\begin{cases}
\lambda & \lambda\geqslant \lambda^*\\
+\infty & \lambda\in[0,\lambda^*]
\end{cases}$

Заметим, что решением двойственной задачи $\inf\limits_{\lambda\geqslant 0}g(\lambda)$ является число $\lambda^*$, так как $\inf g(\mathbb{R}_+)=\inf[\lambda^*,+\infty]=\lambda^*$.

Также
 $$\lambda^*=\sup\limits_{y\in\mathbb{R}^n}\frac{f(y)}{||y||}=\sup\limits_{||y||=1, t\in\mathbb{R}_+}\frac{f(ty)}{||ty||}=\sup\limits_{||y||=1}f(y)=\sup\limits_{||y||=1,t\in[0,1]}tf(y)=\sup\limits_{||y||\leqslant 1}f(y)$$

То есть, число $\lambda^*$ является решением исходной задачи (1).

Вопрос: этого достаточно, или нужно найти $\lambda^*$ (т.е. решить исходную задачу)?

Поиск $\lambda^*$

Рассмотрим $\lambda^*=\sup\limits_{y\in\mathbb{R}^n}\frac{f(y)}{||y||}$. Найдем $$\frac{\partial }{\partial y_j}\frac{f(y)}{||y||_2}=\frac{1}{||y||^2}(f_j||y||-f\frac{y_j}{||y||})\mbox{, где }f_j=\frac{\partial f}{\partial y_j}$$. Приравняем нулю, получим

$$f_j||y||^2=y_jf$$

Найдем

$$f_j=\frac{1}{\rho}\left(\sum\limits_{i=1}^m|(a_i,y)|^\rho\right)^{\frac{1}{\rho}-1}\sum\limits_{i=1}^m\rho |(a_i,y)|^{\rho-1}\frac{\partial |(a_i,y)|}{\partial y_j}=f^{1-\rho}\sum\limits_{i=1}^m|(a_i,y)|^{\rho-1}\frac{\partial |(a_i,y)|}{\partial y_j}$$

Подставим, получим

$$f^{1-\rho}\sum\limits_{i=1}^m|(a_i,y)|^{\rho-1}\frac{\partial |(a_i,y)|}{\partial y_j}||y||^2=y_jf$$

$$\sum\limits_{i=1}^m|(a_i,y)|^{\rho-1}\frac{\partial |(a_i,y)|}{\partial y_j}||y||^2=y_jf^\rho$$

То есть, $\lambda^*=f(y)$, где $||y||=1$ и

$$y_j\sum\limits_{i=1}^m|(a_i,y)|^\rho=\sum\limits_{i=1}^m|(a_i,y)|^{\rho-1}\frac{\partial |(a_i,y)|}{\partial y_j}$$

Заметим, что в этой точке также выполнено $f^2=f_1^2+...+f^2_n$

%Пусть $m=n$. $a_i$~--- столбцы единичной матрицы. Докажем, что $y=(\frac{1}{\sqrt{n})$
\section*{Случай $m=2$}
Пусть $m=2$. Найдем $\lambda^*$.
\begin{enumerate}
\item Пусть $X=Lin\{a_i\}_{i=1}^m$. Пусть $y\in\mathbb{R}^n$, причем $y=y_\parallel+y_\bot$, $y_\parallel\in X$, $y_\bot\in X^\bot$. Тогда $f(y)=f(y_\parallel)$. Действительно, $f(y)$ зависит только от скалярных произведений с $a_i$, а $(y_\bot,a_i)=0$ $\blacksquare$

Значит, искать точку максимума можно только среди векторов $y\in X$

\item Заметим, что решение задачи не меняется при замене $a_i\to-a_i$. Действительно, $f$ зависит только от модулей скалярных произведений с $a_i$. Поэтому, без ограничения общности, угол между $a_1$ и $a_2$~--- острый: $(a_1,a_2)>0$.

\item Заметим, что как скалярное произведение, так и норма $||\cdot||_2$ не зависят от выбора ортогонального базиса. Выберем базис таким образом: $e_1\upuparrows a_1$, $e_2\colon $ $a_2$ лежит в плоскости $e_1, e_2$ в первом ортанте. Остальные $n-2$ векторов базиса выбираем произвольно.

Параметризуем $$y=\matrixl \cos\varphi \, \sin\varphi \matrixr^T$$ (ранее доказано, что максимум можно искать по сфере $||y||_2=1$)

Представим $$\vec{a_1}=a_1 \cdot \matrixl 1\, 0\matrixr^T$$ $a_1$~--- вторая норма $\vec{a_1}$
Представим $$\vec{a_2}=a_2 \cdot \matrixl \cos\theta\, \sin\theta\matrixr^T$$

Тогда получим, что $(a_1,y)=a_1\cos\varphi$, $(a_2,y)=a_2(\cos\varphi\cos\theta+\sin\varphi\sin\theta)=a_2(\cos(\varphi-\theta))$.

\item Задача свелась к задаче БМ $$f^\rho(\varphi)=a_1^\rho\cos^\rho\varphi+a_2^\rho\cos^\rho(\varphi-\theta)\to\max\limits_{\varphi\in[0,\frac{\pi}{2}]}$$

\item Обозначим $$\alpha=\frac{a_1^\rho}{a_2^\rho}$$

\item Задача эквивалентна

$$z(\varphi)=\alpha\cos^\rho\varphi+\cos^\rho(\varphi-\theta)\to\max\limits_{\varphi\in[0,\frac{\pi}{2}]}$$

\item Сделаем замену $t=\cos\theta$, получим выражение

$$z(t)=\alpha t^\rho + (t\cos\theta+\sqrt{1-t^2}\sin\theta)^\rho\to\max\limits_{t\in[0,1]}$$

$$z(t)'/\rho=\alpha t^{\rho-1}+(t\cos\theta+\sqrt{1-t^2}\sin\theta)^{\rho-1}(\cos\theta-\sin\theta\frac{t}{\sqrt{1-t^2}})=0$$

Задача свелась к поиску нуля выражения выше, получаем 

\item Рассмотрим случай $\rho=\frac{1}{3}$.

$$z'(\varphi)/\rho=\alpha t^{-2/3}+(t\cos\theta+\sqrt{1-t^2}\sin\theta)^{-2/3}(\cos\theta-\frac{t}{\sqrt{1-t^2}}\sin\theta)=0$$

$$\alpha t^{-2/3}=(t\cos\theta+\sqrt{1-t^2}\sin\theta)^{-2/3}(\frac{t}{\sqrt{1-t^2}}\sin\theta-\cos\theta)$$

$$\alpha^3t^{-2}=(\frac{t\sin\theta}{\sqrt{1-t^2}}-\cos\theta)^3(t\cos\theta+\sqrt{1-t^2}\sin\theta)^{-2}$$

$$\alpha^3(t\cos\theta+\sqrt{1-t^2}\sin\theta)^{2}=(\frac{t\sin\theta}{\sqrt{1-t^2}}-\cos\theta)^3t^{2}$$

$$t^2\alpha^3(\cos\theta+\frac{\sqrt{1-t^2}}{t}\sin\theta)^{2}=(\frac{t\sin\theta}{\sqrt{1-t^2}}-\cos\theta)^3t^{2}$$

Обозначим $$\gamma=\frac{t}{\sqrt{1-t^2}}$$

$$\alpha^3(\cos\theta+\frac{\sin\theta}{\gamma})^{2}=(\gamma\sin\theta-\cos\theta)^3$$

$$\alpha^3(\gamma\cos\theta+\sin\theta)^{2}=\gamma^2(\gamma\sin\theta-\cos\theta)^3$$

Это уравнение пятой степени относительно $\gamma$.
\end{enumerate}
\end{document}
