\documentclass[a4paper]{article}
\usepackage[a4paper, left=5mm, right=5mm, top=5mm, bottom=5mm]{geometry}
%\geometry{paperwidth=210mm, paperheight=2000pt, left=5pt, top=5pt}
\usepackage[utf8]{inputenc}
\usepackage[english,russian]{babel}
\usepackage{indentfirst}
\usepackage{tikz} %Рисование автоматов
\usetikzlibrary{automata,positioning,arrows,trees,calc}
\usepackage{amsmath}
\usepackage[makeroom]{cancel} % зачеркивание
\usepackage{multicol,multirow} %Несколько колонок
\usepackage{hyperref}
\usepackage{tabularx}
\usepackage{amsfonts}
\usepackage{amssymb}
\DeclareMathOperator*{\argmin}{arg\,min}
\usepackage{listings}
\usepackage{wasysym}
\date{\today}
\author{Сергей~Володин, 272 гр.}
\newcommand{\matrixl}{\left|\left|}
\newcommand{\matrixr}{\right|\right|}
% названия автоматов  (rubtsov)
\def\A{{\cal A}}
\def\B{{\cal B}}
\def\C{{\cal C}}

%классы сложности (rubtsov)
\def\PP{{\mathsf{P}}}
\def\NP{{\mathsf{NP}}}
\def\NPc{{\mathsf{NP}}\text{-}{\rm c}}
\def\coNP{{\rm co}\text{-}{\mathsf{NP}}}
\def\DTIME{{\mathsf{DTIME}}}
\def\NTIME{{\mathsf{NTIME}}}
\def\CLIQUE{{\mathsf{CLIQUE}}}
\def\HALT{{\rm{HALT}}}
\def\SAT{{\rm{SAT}}}
\def\3SAT{{\rm{3\text{-}SAT}}}
\def\2SAT{{\rm{2\text{-}SAT}}}
\def\UNSAT{{\rm{UNSAT}}}
\def\HP{{\rm{HAMPATH}}}
\def\UHP{{\rm{UHAMPATH}}}
\def\LL{{\mathrm{LL}}}
\def\poly{{\rm{poly}}}
\def\GC{{\mbox{ГЦ}}}
\def\GP{{\mbox{ГП}}}
\def\conv{{\mbox{conv}}}

\title{Алгоритмы и модели вычислений.\\Поиск подматрицы}

% алгоритмы (Рубцов)
\usepackage{verbatim}
\usepackage{listings}
\usepackage{algorithm2e}

%+= и -=, иначе разъезжаются...
\newcommand{\peq}{\mathrel{+}=}
\newcommand{\meq}{\mathrel{-}=}
\newcommand{\deq}{\mathrel{:}=}
\newcommand{\plpl}{\mathrel{+}+}

\newcommand{\todo}{{\em todo}}

% пустое слово  (rubtsov)
\def\eps{\varepsilon}

% регулярные языки  (rubtsov)
\def\REG{{\mathsf{REG}}}
\def\CFL{{\mathsf{CFL}}}
\def\eqdef{\overset{\mbox{\tiny def}}{=}}
\newcommand{\niton}{\not\owns}

%FIRST & FOLLOW (rubtsov)
\def\first{\mathrm{ FIRST} }
\def\follow{\mathrm{ FOLLOW} }

% LL LR (rubtsov)
\def\LL{{\mathrm{LL}}}
\def\LR{{\mathrm{LR}}}

\newcounter{rowItemCount}
\newcounter{subRowItemCount}
\newcommand\rowItem{
    \setcounter{subRowItemCount}{0}
    \arabic{rowItemCount}.\addtocounter{rowItemCount}{1}}
\newcommand\subRowItem{
    \addtocounter{subRowItemCount}{1}
    \addtocounter{rowItemCount}{-1}
    \arabic{rowItemCount}.\arabic{subRowItemCount}.\addtocounter{rowItemCount}{1}}

\newcommand{\smalll}[1]{\overline{\overline{#1}}}
\newcommand{\smallo}{\bar{\bar{o}}}
\newcommand{\ZZ}{\mathbb{Z}}
\newcommand{\Nz}{\mathbb{N}\cup\{0\}}
\newcommand{\NN}{\mathbb{N}}
\newcommand{\RR}{\mathbb{R}}
\begin{document}
\maketitle
\section{Определения}
\begin{enumerate}
\item Даны матрицы $P\colon m\times n$, $T\colon M\times N$, состоящие из $1$ и $2$. $M>m$, $N>n$.
\item Нумеруем строки и столбцы с нуля.
\item Определение: $P$ входит в $T$ с позиции $(i,j)$ $\Leftrightarrow \forall (a,b)\in\overline{0,m-1}\times\overline{0,n-1}\hookrightarrow p^a_b=t^{i+a}_{j+b}$
\item Задача: найти пары $(i,j)$, такие что $P$ входит в $T$ с $(i,j)$.
\end{enumerate}
\section{Сведение к поиску подстроки}
\begin{enumerate}
\item Определим $r(i)=\lfloor\frac{i}{N}\rfloor$~--- номер строки, $c(i)=i\mod N$~--- номер столбца (см. далее).
\item Запишем матрицу $T$ в видем массива (строки $A$) $t_i=t^{r(i)}_{c(i)}$. $|t|=MN$.
\item Пусть $p^0,...,p^{m-1}$~--- строки матрицы $P$, записанные как массив чисел (строки). Обозначим $p=p^00^{N-n}p^10^{N-n}...p^{m-1}$. $|p|=Nm+n$. Тогда $p_i=p^{r(i)}_{c(i)}$
\item Пусть $p$ входит в $t$ с позиции $i$, причем $c(i)\leqslant N-n$, $r(i)\leqslant M-m$. Докажем, что это равносильно $P$ входит в $T$ с позиции $(r(i),c(i))$:\begin{enumerate}
\item $p$ входит в $t$ с позиции $i$ $\boxed{\Leftrightarrow}$ $\forall k\in\overline{0,Nm+n-1}\hookrightarrow \left[\begin{array}{lcl}
p_k&=&t_{k+i}\\
p_k&=&0\\
t_{k+i}&=&0
\end{array}\right.$. В $t$ нет символов $0$, поэтому $\boxed{\Leftrightarrow}\forall k\in\overline{0,Nm+n-1}\hookrightarrow\left[\begin{array}{lclr}
p_k&=&t_{k+i}&(1)\\
p_k&=&0&(2)\\
\end{array}\right.$. При $c(k)\in\overline{0,n-1}$ $p_k\neq 0$, поэтому должно выполняться $(1)$, для остальных должно выполняться $(2)$, так как $t$ не содержит $0$. Поэтому $\boxed{\Leftrightarrow}\forall k\in\overline{0,Nm+n-1}\colon c(k)\in\overline{0,n-1}\hookrightarrow p_k=t_{k+i}\Leftrightarrow \forall k\in\overline{0,Nm+n-1}\colon c(k)\in\overline{0,n-1}\hookrightarrow p^{r(k)}_{c(k)}=t^{r(k+i)}_{c(k+i)}$
%$p_k=0\Leftrightarrow k\mod N\in\overline{n,N-1}\Leftrightarrow c(k)\in\overline{n,N-1}$, поэтому для таких $k$ условие выпол
\end{enumerate}
\end{enumerate}
\section{Время работы}
\begin{enumerate}
\item $|t|=MN$, $|p|=Nm+n$, поэтому алгоритм поиска подстроки работает за $O(MN\log MN)$ (доказано в решении задания).
\item Заметим, что <<очевидная>> версия (перебор) работает за $O(MNmn)$~--- $O(MN)$ позиций и $O(mn)$ на проверку вхождения.
\end{enumerate}
\end{document}