\documentclass[a4paper]{article}
\usepackage[a4paper, left=5mm, right=5mm, top=5mm, bottom=5mm]{geometry}
%\geometry{paperwidth=210mm, paperheight=2000pt, left=5pt, top=5pt}
\usepackage[utf8]{inputenc}
\usepackage[english,russian]{babel}
\usepackage{indentfirst}
\usepackage{tikz} %Рисование автоматов
\usetikzlibrary{automata,positioning,arrows}
\usepackage{amsmath}
\usepackage{enumerate}
\usepackage{hyperref}
\usepackage{amsfonts}
\usepackage{amssymb}
\DeclareMathOperator*{\argmin}{arg\,min}
\usepackage{wasysym}
\title{Теория и реализация языков программирования.\\Задание 8: контекстно-свободные языки и магазинные автоматы II}
\date{задано 2013.10.23}
\author{Сергей~Володин, 272 гр.}
\newcommand{\matrixl}{\left|\left|}
\newcommand{\matrixr}{\right|\right|}
% названия автоматов
\def\A{{\cal A}}
\def\B{{\cal B}}
\def\C{{\cal C}}

%+= и -=, иначе разъезжаются...
\newcommand{\peq}{\mathrel{+}=}
\newcommand{\meq}{\mathrel{-}=}
\newcommand{\deq}{\mathrel{:}=}
\newcommand{\plpl}{\mathrel{+}+}

% пустое слово
\def\eps{\varepsilon}

% регулярные языки
\def\REG{{\mathsf{REG}}}
\def\eqdef{\overset{\mbox{\tiny def}}{=}}
\newcommand{\niton}{\not\owns}

\begin{document}
\maketitle
\section*{Упражнение 1}
\section*{Упражнение 2}
\section*{Задача 1}
\begin{enumerate}
\item Определим МП-автомат $\A=(\Sigma,\Gamma,Q,q_0,Z,\delta,F)$, допускающий по принимающему состоянию.\newline
\begin{tabular}{cc}
\begin{minipage}{0.4\textwidth}
\begin{enumerate}
\item $\Sigma\eqdef\{a,b,c,f\}$
\item $\Gamma\eqdef\{F,Z\}$
\item $Q\eqdef\{q_0,q_1,q_2,q_3\}$
\item $\delta$ изображена справа
\item $F\eqdef\{q_0,q_3\}$
\end{enumerate}
\end{minipage}
&
\begin{minipage}{0.6\textwidth}

\begin{tikzpicture}[shorten >=1pt,node distance=2cm,on grid,auto,every node/.style={text centered},initial text=]
	\node [state,initial,accepting] (q_0)	{$q_0$};
	\node [state] (q_1) [right = 2.5cm of q_0 ] {$q_1$};
	\node [state] (q_2) [right = 2.5cm of q_1 ] {$q_2$};
	\node [state,accepting] (q_3) [right = 2.5cm of q_2 ] {$q_3$};
	\path[->]
		(q_0) edge [in=225,out=-45,loop] node {$f,Z/F$} (q_0)
			  edge [in=135,out=45,loop] node[swap] {$f,F/F$} (q_0)
			  edge node {$\substack{a,F/aF\\a,Z/aZ}$} (q_1)
			  edge [out=-25.5,in=205.5] node[swap] {$c,Z/Z$} (q_3)
		(q_1) edge [in=135,out=45,loop] node[swap] {$a,a/aa$} (q_1)
			  edge node {$b,a/\varepsilon$} (q_2)
		(q_2) edge [in=135,out=45,loop] node[swap] {$b,a/\varepsilon$} (q_2)
			  edge node {$\substack{\varepsilon,Z/Z\\\varepsilon,F/F}$} (q_3)
		(q_3) edge [in=135,out=45,loop] node[swap] {$c,Z/Z$} (q_3)
		;
\end{tikzpicture}
\end{minipage}
\end{tabular}
\item Определим грамматику {\bf{ОНА НЕОДНОЗНАЧНАЯ, ПОФИКСИТЬ!}} $G=(\Sigma,T,P,S)$:\begin{enumerate}[1.]
\item $T\eqdef\{S,A,B,X,C,F\}$
\item P:\begin{enumerate}
\item $S\longrightarrow XC|FX|X$
\item $X\longrightarrow aXb|\varepsilon$
\item $C\longrightarrow cC|c$
\item $F\longrightarrow fF|f$
\end{enumerate}
\end{enumerate}
\end{enumerate}
\section*{Задача 2}
\section*{Задача 3}
\section*{Задача 4}
\section*{Задача 5}
\end{document}