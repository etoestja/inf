\documentclass[a4paper]{article}
\usepackage[a4paper,left=10mm, right=10mm]{geometry}
%\geometry{paperwidth=210mm, paperheight=2000pt, left=5pt, top=5pt}
\usepackage[utf8]{inputenc}
%\usepackage[english]{babel}
\usepackage{indentfirst}
\usepackage{amsmath}
\usepackage[makeroom]{cancel} % зачеркивание
\usepackage{multicol,multirow} %Несколько колонок
\usepackage{hyperref}
\usepackage{tabularx}
\usepackage{amsfonts}
\usepackage{amssymb}
\usepackage{wasysym}
\author{Sergey Volodin, group 272}
\title{An overview of Globalization, its positive and negative aspects\\(abstract)}
%\date{2014.04.22}
\begin{document}
\maketitle
{\Large
Products made in one region (or regions) appear in another, which is a result of globalization.
\\[5pt]
Globalization has several qualities. Transportation and communication become more available and widely used, there are a lot of electronic money transfer and investing in developing countries, knowledge spreads fast, non-governmental organizations appear, there is outsourcing, and borders are becoming less important.
\\[5pt]
There are pros and cons of globalization. On the one hand, standarts of living in developing countries are increasing, prices are stable because of competition, developing countries are able to get state-of-the art technologies without doing research into it, governments are able to work together, there is access to other cultures.
\\[5pt]
On the other hand, outsourcing may leave some people without a job, cultures meld, there is a threat of an disease and absence of rules for international interactions.}
\end{document}
