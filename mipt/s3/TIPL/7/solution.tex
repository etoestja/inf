\documentclass[a4paper]{article}
\usepackage[a4paper, left=5mm, right=5mm, top=5mm, bottom=5mm]{geometry}
%\geometry{paperwidth=210mm, paperheight=2000pt, left=5pt, top=5pt}
\usepackage[utf8]{inputenc}
\usepackage[english,russian]{babel}
\usepackage{indentfirst}
\usepackage{tikz} %Рисование автоматов
\usetikzlibrary{automata,positioning,arrows}
\usepackage{amsmath}
\usepackage{enumerate}
\usepackage{hyperref}
\usepackage{amsfonts}
\usepackage{amssymb}
\DeclareMathOperator*{\argmin}{arg\,min}
\usepackage{wasysym}
\title{Теория и реализация языков программирования.\\Задание 7: контекстно-свободные языки и магазинные автоматы}
\date{задано 2013.10.16}
\author{Сергей~Володин, 272 гр.}
\newcommand{\matrixl}{\left|\left|}
\newcommand{\matrixr}{\right|\right|}
% названия автоматов
\def\A{{\cal A}}
\def\B{{\cal B}}
\def\C{{\cal C}}

%+= и -=, иначе разъезжаются...
\newcommand{\peq}{\mathrel{+}=}
\newcommand{\meq}{\mathrel{-}=}
\newcommand{\deq}{\mathrel{:}=}
\newcommand{\plpl}{\mathrel{+}+}

% пустое слово
\def\eps{\varepsilon}

% регулярные языки
\def\REG{{\mathsf{REG}}}
\def\eqdef{\overset{\mbox{\tiny def}}{=}}
\newcommand{\niton}{\not\owns}

\begin{document}
\maketitle
\section*{Упражнение 1}
\section*{Упражнение 2}
\section*{Упражнение 3}
\begin{enumerate}[1.]
\item Грамматика $\Gamma=(\{S\},\Sigma_n\cup\overline{\Sigma}_n,P,S)$. $P=\big\{S\longrightarrow \sigma_i\overline{\sigma}_i|\sigma_iS\overline{\sigma}_i|SS\big\}$. $D_n=L(\Gamma)$.
\item Исходное утверждение: $\forall w\left(\underbrace{w\in D_n}_{A}\Rightarrow\underbrace{\forall i\leqslant n\,\forall k\leqslant |w|\hookrightarrow ||w[1,k]||_i\geqslant 0,\,||w||_i=0}_{B}\right)$
\item Отрицание обратного утверждения: $\exists w\colon \left(B\,\wedge\,\urcorner A\right)$. Пусть $w=\varepsilon$.\begin{enumerate}[a.]
\item Тогда $k\leqslant |w|\Rightarrow k=0$, поэтому $\forall i\leqslant n\hookrightarrow ||w[1,k]||_i\equiv|\varepsilon|_{\sigma_i}-|\varepsilon|_{\overline{\sigma_i}}=0$ и $\forall i\leqslant n\hookrightarrow||w||_i=0$. Получаем $B$.
\item Но $w=\varepsilon$ не порождается грамматикой $\Gamma$: первые два правила добавляют нетерминалов, поэтому не могут быть применены, и применение третьего правила не уменьшает количества нетерминалов. Получаем $\urcorner A\,\blacksquare$
\end{enumerate}
\item Другой пример: $\Sigma_2=\big\{\{,[ \big\},\,\overline{\Sigma}_2=\big\{\},]\big\}$ $w=\{[\}]$.\newline
Тогда $||w||_1=||w||_2=0$, и $||w[1,i]||_j\geqslant 0\,\forall i\in \overline{1,2}\,\forall i\in\overline{0,2}$, но $w\notin D_2$ (не ПСВ).
\end{enumerate}
\section*{Задача 1}
\begin{enumerate}
\item Определим МП-автомат $\A=(\Sigma,\Gamma,Q,q_0,Z,\delta,F)$, допускающий по пустому стеку.\newline
\begin{tabular}{cc}
\begin{minipage}{0.46\textwidth}
\begin{enumerate}
\item $n\eqdef2$
\item $\Sigma_n\eqdef\{[_1,...,[_n\}\equiv\{[_1,[_2\},\,\overline{\Sigma}_n\eqdef\{]_1,...,]_n\}\equiv\{]_1,]_2\}$.
\item $\Sigma\eqdef\Sigma_n\cup\overline{\Sigma}_n\equiv\{[_1,]_1,[_2,]_2\}$
\item $\Gamma\eqdef\{Z\}\Sigma_n\equiv\{Z,[_1,[_2\}$.
\item $Q\eqdef\{q_0,q_1\}$
\item $\delta$ изображена справа
\item $F\eqdef\varnothing$ ($N$-автомат)
\end{enumerate}
\end{minipage}
&
\begin{minipage}{0.46\textwidth}

\begin{tikzpicture}[shorten >=1pt,node distance=2cm,on grid,auto,every node/.style={text centered},initial text=]
	\node [state,initial] (q_0)	{$q_0$};
	\node [state] (q_1) [right = 4cm of q_0 ] {$q_1$};
	\path[->]
		(q_0) edge [out=40,in=140,loop] node[swap] {$\substack{ [_1,Z/[_1Z \\ [_1,[_1/[_1[_1 \\ [_1,[_2/[_1[_2 \\}$} (q_0)
			  edge [out=-40,in=-140,loop] node {$\substack{ [_2,Z/[_2Z \\ [_2,[_1/[_2[_1 \\ [_2,[_2/[_2[_2 \\}$} (q_0)
			  edge [out=20,in=160] node {$\substack{ ]_1,[_1/\varepsilon \\ ]_2,[_2/\varepsilon}$} (q_1)
			  edge node [anchor=0,above=-0.15,swap] {$\varepsilon,Z/Z$} (q_1)
		(q_1) edge [in=40,out=140,loop] node {$\substack{ ]_1,[_1/\varepsilon \\ ]_2,[_2/\varepsilon \\}$} (q_0)
			  edge [in=-40,out=-140,loop] node [swap] {$\varepsilon,Z/\varepsilon$} (q_1)
			  edge [in=-20,out=-160] node {$\substack{ [_1,Z/[_1Z \\ [_2,Z/[_2Z}$} (q_0);
\end{tikzpicture}
\end{minipage}
\end{tabular}
\def\lb{{\big([_1\big|[_2\big)}}
\def\rb{{\big(]_1\big|]_2\big)}}
\item Определим морфизм $P\colon P\colon (\Sigma_n\cup\overline{\Sigma}_n)^*\longrightarrow (\Sigma_n\cup\overline{\Sigma}_n)^*$: $P([_i)=]_i,\,P(]_i)=[_i$~--- пары для скобок. Доопределим до морфизма: $P(w_1...w_l)=P(w_1)...P(w_l)$.
\item \label{wLdiv} $L=D_2\cap\lb^*\rb^*$. $w\in L\Rightarrow w=w_1w_2,\,w_1=\lb^{n_1},\,w_2=\rb^{n_2}$. $w\in D_2\Rightarrow 0=||w||_i=||w_1||_i+||w_2||_i=|w_1|_{[_i}+|w_2|_{[_i}-|w_1|_{]_i}-|w_2|_{]_i}$. $w_1$ не содержит $]_i$, $w_2$ не содержит $[_i$, поэтому $0=|w_1|_{[_i}-|w_2|_{]_i}$. Сложим равенства, получим $0=|w_1|_{[_1}+|w_1|_{[_2}-|w_2|_{]_1}-|w_2|_{]_2}\Rightarrow |w_1|=|w_2|\Rightarrow n_1=n_2$.
\item \label{wPalyndrom} $w\in L,\,|w_1|=s,\,w_1=[_{i_1}...[_{i_s},\,w_2=]_{j_1}...]_{j_s}$. Докажем, что $P(w_2)=w_1^R$:\newline
$Q(k)\eqdef\big[P(w_2)[1,k]=w_1^R[1,k]\big]$.\begin{enumerate}[a.]
\item Очевидно, $Q(0)$, так как $P(w_2)[1,0]\equiv\varepsilon\equiv w_1^R[1,0]$.
\item Пусть $Q(k)$. Тогда $w_1=p[_{i_{s-k+1}}...[_{i_s},w_2=]_{i_s}...]_{i_{s-k+1}}q$. То есть, $k$ скобок от центра парные друг к другу. Обозначим их за $t=[_{i_{s-k+1}}...[_{i_s}]_{i_s}...]_{i_{s-k+1}}\Rightarrow ||t||_i=0$, $t$~--- ПСВ. Предположим $\urcorner Q(k+1)\overset{Q(k)}{\Rightarrow}P(w_2)[k+1]\neq w_1^R[k+1]$. Без ограничения общности $p=p_0[_1$, $q=]_2q_0$. Тогда $w=p_0[_1t]_2q_0$. Но $t$~--- ПСВ, поэтому пара для $[_1$~--- в $q_0$, пара для $]_2$~--- в $p_0$: $w=...[_2...[_1t]_2...]_1...$~--- не ПСВ $\Rightarrow w\notin D_2$~--- противоречие. Значит, $Q(k+1)$.
\end{enumerate}
%\item Определим $||\cdot||\colon \Sigma^*\rightarrow {\mathbb Z}$: $||w||=||w||_1+||w||_2$.
%\item Пусть $w\in L,|w|=n$. Докажем $\forall i\in\overline{1,n-1}\hookrightarrow ||w[1,i]||<0$. %Действительно, $||w[1,i]||=||w||-||w[i+1,n]||=0-||w[i+1,n]||$
%Действительно, $m$
\item \label{wOne} Пусть $w\in L$. Докажем, что $(q_0,w,Z)\vdash^*(q_1,\varepsilon,Z)$ и $(q_1,w,Z)\vdash^*(q_1,\varepsilon,Z)$. $\ref{wLdiv}\Rightarrow w=w_1w_2$, $\ref{wPalyndrom}\Rightarrow P(w_1)^R=w_2$.\begin{enumerate}[a.]
\item \label{wq0} Докажем $Q(k)\eqdef \big[(q_0,w_1[1,k],Z)\vdash^*(q_0,\varepsilon,(w_1[1,k])^RZ)\big]$:\begin{enumerate}[a.]
\item $k=0\Rightarrow w_1[1,k]=\varepsilon\Rightarrow (w_1[1,k])^R=\varepsilon$. Получаем $(q_0,w_1[1,k],Z)\equiv(q_0,(w_1[1,k])^R,Z)\Rightarrow Q(0)$
\item Пусть $Q(k)\Rightarrow (q_0,w_1[1,k],Z)\vdash^*(q_0,\varepsilon,(w_1[1,k])^RZ)$. Рассмотрим $w_1[k+1]=[_{i_{k+1}}$. По определению $\delta$ имеем $\forall \gamma (q_0,[_{i_{k+1}},\gamma)\vdash(q_0,\varepsilon,[_{i_{k+1}}\gamma)$. Тогда $(q_0,w[1,k+1],Z)\equiv(q_0,w_1[1,k][_{i_{k+1}},Z)\overset{Q(k)}{\vdash^*}(q_0,[_{i_{k+1}},(w_1[1,k])^RZ)\overset{\mbox{\tiny def }\delta}{\vdash}(q_0,\varepsilon,w_1[k+1](w_1[1,k])^RZ)\equiv(q_0,\varepsilon,(w_1[1,k+1])^RZ)\Rightarrow Q(k+1)$.
\end{enumerate}
\item \label{wq1} Докажем $Q(k)\eqdef\big[\forall \gamma\in\Gamma^+\hookrightarrow(q_1,w_2[1,k],P(w_2)[1,k]\gamma)\vdash^*(q_1,\varepsilon,\gamma)\big]$:\begin{enumerate}[a.]
\item $k=0\Rightarrow w_2[1,k]\equiv\varepsilon\equiv P(w_2)[1,k]\Rightarrow Q(0)$
\item Пусть $Q(k)\Rightarrow\forall\gamma\hookrightarrow(q_1,w_2[1,k],P(w_2)[1,k]\gamma)\vdash^*(q_1,\varepsilon,\gamma)$. $\varangle w_2[k+1]=]_{i_{k+1}}$.\newline
Из определения $\delta$ получаем $\forall\gamma_1\hookrightarrow (q_1,]_{i_{k+1}},[_{i_{k+1}}\gamma_1)\vdash(q_1,\varepsilon,\gamma_1)$.\newline
Значит, $(q_1,w_2[1,k+1],P(w_2)[1,k+1]\gamma)\equiv(q_1,w_2[1,k]]_{i_{k+1}},P(w_2)[1,k][_{i_{k+1}}\gamma)\overset{Q(k)}{\vdash^*}(q_1,]_{i_{k+1}},[_{i_{k+1}}\gamma)\overset{\mbox{\tiny def }\delta}{\vdash}(q_1,\varepsilon,\gamma)\Rightarrow Q(k+1)$.
\end{enumerate}
\item \label{pr1} Рассмотрим $w_2=]_iw^0_2$. Но $\ref{wPalyndrom}\Rightarrow w_2=P(w_1)^R\Rightarrow w_1=P(w^0_2)^R[_i$ Из определения $\delta$ получаем $\forall\gamma(q_0,]_i,[_i\gamma)\vdash(q_1,\varepsilon,\gamma)$. Тогда $\underline{(q_0,w,Z)}\overset{\ref{wq0}}{\vdash^*}(q_0,w_2,(w_1)^RZ)\equiv(q_0,]_iw^0_2,[_iP(w^0_2)Z)\overset{\mbox{\tiny def }\delta}{\vdash}(q_1,w^0_2,P(w^0_2)Z)\overset{\ref{wq1}}{\vdash^*}\underline{(q_1,\varepsilon,Z)}$.
\item $w_1=[_iw^0_1$. Из определения $\delta$ получаем $(q_1,[_i,Z)\vdash(q_1,\varepsilon,[_iZ)$. Тогда $(q_1,w,Z)\equiv(q_1,[_iw^0_1w_2,Z)\overset{\mbox{\tiny def }\delta}{\vdash}(q_0,w^0_1w_2,[_iZ)$. Но эта конфигурация может быть получена иначе: $(q_0,[_i,Z)\vdash(q_0,[_i,[_iZ)$. Значит, дальнейшие конфигурации также могут совпадать. Имеем $\ref{pr1}\Rightarrow\underline{(q_1,w,Z)\vdash^*(q_1,\varepsilon,Z)}$.
\end{enumerate}
\item Пусть $w\in L^*\setminus\{\varepsilon\}\Rightarrow w=w_1...w_k$, $\forall i\in\overline{1,k}\hookrightarrow w_i\in L$. Определим $f\colon L^*\longrightarrow {\mathbb N}\cup\{0\}$: $f(w)\ni k$ (многозначная функция). Если $w=\varepsilon$, определим $f(w)\eqdef 0$.
\item $P(k)\eqdef\big[\forall w\in L^*\colon f(w)\ni k\hookrightarrow (q_0,w,Z)\vdash^*(q_1,\varepsilon,Z)\big]$\begin{enumerate}
\item Пусть $k=0$. Тогда $w=\varepsilon$. $(q_0,w,Z)\equiv(q_0,\varepsilon,Z)\vdash(q_1,\varepsilon,Z)\Rightarrow P(0)$.
\item Пусть $k=1, w\in L^*\colon f(w)\ni 1\Rightarrow w\equiv w_1\in L$. $\ref{wOne}\Rightarrow (q_0,w,Z)\vdash^*(q_1,\varepsilon,Z)\Rightarrow P(1)\,\blacksquare$
\item Пусть $P(k)$. $w\in L^*\colon f(w)\ni k+1\Rightarrow w=w_1...w_{k+1},\,\forall i\in\overline{1,k+1}\hookrightarrow w_i\in L$. $\varangle w_0\eqdef w_1...w_k\in L^*$. $f(w_0)\ni k\overset{P(k)}{\Rightarrow} (q_0,w_0,Z)\vdash^*(q_1,\varepsilon,Z)$. Тогда $(q_0,w,Z)\equiv(q_0,w_0w_{k+1},Z)\vdash^*(q_1,\varepsilon w_{k+1},Z)\overset{\ref{wOne}}{\vdash^*}(q_1,\varepsilon,Z)\Rightarrow P(k+1)\,\blacksquare$
\end{enumerate}
Получаем $\forall w\in L^*\hookrightarrow (q_0,w,Z)\vdash^*(q_1,\varepsilon,Z)\overset{\mbox{\tiny def }\delta}{\vdash}(q_1,\varepsilon,\varepsilon)\Rightarrow \forall w\in L^*\hookrightarrow w\in L(\A)\Rightarrow \boxed{L^*\subseteq L(\A)}$.
\item \label{bottom} $\varangle\delta$. Заметим, что каждый переход, кроме $q_1\overset{\varepsilon
,Z/\varepsilon}{\longrightarrow}q_1$ сохраняет количество $Z$ в стеке, и, более того, оставляет $Z$ на дне стека.
\item \label{normAndAutomata} Пусть $(q_a,w,\phi)\vdash^*(q_b,\varepsilon,\gamma)$. Тогда $||\gamma||_i-||\phi||_i=||w||_i$. Докажем по индукции:\newline
$Q(k)\eqdef\big[\forall w\colon |w|=k\,\forall q_a\,\forall q_b\,\forall\phi\,\forall\gamma\colon (q_a,w,\phi)\vdash^*(q_b,\varepsilon,\gamma)\hookrightarrow ||\gamma||_i-||\phi||_i=||w||_i\big]$.\begin{enumerate}[a.]
\item $k=0\Rightarrow w=\varepsilon$. Поскольку все $\varepsilon$-переходы $q_0\overset{\varepsilon,Z/Z}{\longrightarrow}q_1$ и $q_1\overset{\varepsilon,Z/\varepsilon}{\longrightarrow}q_1$ не изменяют $||\cdot||_i$ для символов стека, получаем $||w||_i\equiv 0\equiv ||\phi||_i-||\delta||_i\Rightarrow Q(0)$.
\item Пусть $Q(k)$. $\varangle w\colon |w|=k+1,\,(q_a,w,\phi)\vdash^*(q_c,\varepsilon,\gamma)$. $w=w_0\sigma$, $\sigma\in\Sigma$. $\varangle (q_a,w,\phi)\equiv(q_a,w_0\sigma,\phi)\vdash^*(q_b,\sigma,\psi)\vdash(q_c,\varepsilon,\gamma)$. $Q(k)\Rightarrow ||\psi||_i-||\phi||_i=||w_0||_i$. $\varangle$ последний переход. Из определения $\delta$ следует, что $||\gamma||_i-||\psi||_i=||\sigma||_i$: если $\sigma_i=[_i$, то в стек будет добавлена $\sigma_i$, иначе она будет удалена. Поэтому $||w||_i=||w_0||_i+||\sigma||_i=||\psi||_i-||\phi||_i+||\gamma||_i-||\psi||_i\equiv||\gamma||_i-||\phi||_i\Rightarrow Q(k+1)$.
\end{enumerate}
\item Пусть $w\in L(\A)\Rightarrow(q_0,w,Z)\vdash^*(q,\varepsilon,\varepsilon)$.
\begin{enumerate}[a.]
\item Если $w=\varepsilon$, то $w\in L^*$
\item Пусть иначе. Изначально $Z$ в стеке, в конце его нет. Значит (\ref{bottom}), был переход $q_1\overset{\varepsilon
,Z/\varepsilon}{\longrightarrow}q_1$. Но $Z$ был на дне стека, поэтому после стек пуст. Значит, это последняя конфигурация. Имеем $(q_0,w,Z)\vdash^*(q_1,\varepsilon,Z)\vdash(q_1,\varepsilon,\varepsilon)$. Рассмотрим $(q_0,w,Z)\vdash^*(q_1,\varepsilon,Z)$. Пусть $\{c_i\}_{i=0}^I$~--- эта цепочка конфигураций, $c_i=(q_{k_i},w_i,\gamma_i)$\begin{enumerate}
\item $\varangle \delta$. Заметим, что автомат реализует алгоритм проверки на ПСВ: если была прочитана скобка $[_i$, то она положена в стек. Скобки вынимаются из стека тогда и только тогда, когда прочитана парная скобка. Значит, $w$~--- ПСВ.
\item Рассмотрим все конфигурации $c_{i_j}\colon \gamma_{i_j}=Z\Rightarrow c_i\equiv (q_{k_{i_j}},w_{i_j},Z)$. Рассмотрим первую пару $c_{i_1}\vdash^*c_{i_{2}}$. Было прочитано слово $x_1$. $\ref{normAndAutomata}\Rightarrow ||x_1||_i=||Z||_i-||Z||_i=0$. Получаем, что  $x_1$~--- подстрока ПСВ со скобочным итогом, равным нулю. Значит, $x_1$~--- ПСВ. Пусть $x_1=ab$, в $a$ только открывающие скобки, в $b$ первая закрывающая. Пусть в $b$ есть открывающие скобки, а именно, $b=cd$, в $d$ первая открывающая скобка. После прочтения $a$ автомат находится в $q_0$ (\ref{wq0}). Далее после прочтения $c$ автомат в $q_1$ (\ref{wq1}). Стек не пуст, так как иначе эта пара конфигураций не первая. Но из $q_1$ нет переходов по открывающим скобкам с непустым стеком~--- противоречие. Получаем, что в $b$ нет открывающих скобок $\Rightarrow$ $x_1\in L$. Далее рассуждение можно продолжить, так как следующий после $x_1$ символ в $w$~--- открывающая скобка (иначе скобочный итог отрицательный), по ней автомат переходит в $q_0$.\newline
Получаем, что $w=x_1...x_m$, $\forall q\in\overline{1,m}\hookrightarrow x_q\in L$. Поэтому $w\in L^*$.
\end{enumerate}
Получаем $\boxed{L(\A)\subseteq L^*}$.
%Переход $q_1\overset{\varepsilon,Z/\varepsilon}{\longrightarrow}q_1$ не использовался, поэтому рассмотрим автомат $\A'$ без него. Заметим, что $\A'$~--- детерминированный.
\end{enumerate}
\end{enumerate}
\section*{Задача 2}
\begin{enumerate}
\item Пусть $N=(\Sigma,\Gamma_N,Q_N,\delta_N,Z_0,q_0,F_N)$~--- МП-автомат, допускающий по пустому стеку. Построим МП-автомат $P=(\Sigma,\Gamma,Q,\delta,Z_0,q_s,F)$, допускающий по заключительному состоянию $\colon L(N)=L(P)$.\begin{enumerate}[a.]
\item $\Gamma=\Gamma_N\cup \{X\}$
\item $Q=Q_N\cup\{q_s,q_f\}$
\item $F=\{q_f\}$
\item $\delta$~--- $\delta_N$ с добавленными переходами: $q_s\overset{\varepsilon,Z_0/Z_0X}{\longrightarrow}q_0$, $q_i\overset{\varepsilon,X/X}{\longrightarrow}q_f,\,q_i\in Q_N$.
\end{enumerate}
\begin{enumerate}[1.]
\item Пусть $w\in L(N)$. Тогда $(q_0,w,Z_0)\vdash^*(q,\varepsilon,\varepsilon)$~--- в $N$. Для $P$: $(q_s,w,Z_0)\vdash(q_0,w,Z_0X)\overset{(*)}{\vdash^*}(q,\varepsilon,X)\vdash(q_f,\varepsilon,X)$. Переходы, отмеченные $(*)$ возможны, так как в $P$ сохранены переходы из $N$. Добавление $X$ на дно стека не изменит работу автомата, т.к. $X$ не будет удаляться из стека (в противном случае получим удаление символа из пустого стека в $N$). Но $q_f\in F\Rightarrow w\in L(P)$
\item Пусть $w\in L(P)$. Принимающее состояние одно, поэтому цепочка конфигураций имеет вид $(q_s,w,Z_0)\vdash^*(q_f,\varepsilon,\gamma)$. Из $q_s$ переход один, поэтому цепочка имеет вид $(q_s,w,Z_0)\vdash(q_0,w,Z_0X)\vdash^*(q_f,\varepsilon,\gamma)$. Переходы в $q_f$ только при $X$ на верхушке стека. Также $X$ всегда остается на дне стека, т.к. переходы из исходного автомата не удаляют $X$. Поэтому $\gamma=X$. Имеем $(q_0,w,Z_0X)\vdash^*(q,\varepsilon,X)\vdash(q_f,\varepsilon,X)$. Удалим $X$, получим цепочку конфигураций в $N$: $(q_0,w,Z_0)\vdash^*(q,\varepsilon,\varepsilon)\Rightarrow w\in L(N)$.
\end{enumerate}
\item Пусть $P=(\Sigma,\Gamma_P,Q_P,\delta_P,Z_0,q_0,F_P)$~--- МП-автомат, допускающий по принимающему состоянию. Построим МП-автомат $N=(\Sigma,\Gamma,Q,\delta,Z_0,q_s,F)$, принимающий по пустому стеку $\colon L(N)=L(P)$.\begin{enumerate}[a.]
\item $F=\varnothing$
\item $\Gamma=\Gamma_P\cup\{X\}$
\item $Q=Q_P\cup\{q_s,q_f\}$
\item $\delta$~--- $\delta_P$ с добавленными переходами: $q_s\overset{\varepsilon,Z_0/Z_0X}{\longrightarrow}q_0$; $q_i\overset{\varepsilon,\gamma/\gamma}{\longrightarrow}q_f,\,\gamma\in\Gamma,\,q_i\in F$; а также $q_f\overset{\varepsilon,\gamma/\varepsilon}{\longrightarrow}q_f,\,\gamma\in \Gamma$.
\end{enumerate}
\begin{enumerate}
\item Пусть $w\in L(P)$. Тогда в $P$ $(q_0,w,Z_0)\vdash^*(q,\varepsilon,\mu),\,q\in F$. Тогда в $N$ имеем $(q_s,w,Z_0)\vdash(q_0,w,Z_0X)\overset{(*)}{\vdash^*}(q,\varepsilon,\mu X)\vdash(q_f,\varepsilon,\kappa)\overset{(*_2)}{\vdash^*}(q_f,\varepsilon,\varepsilon)$. Корректность переходов $(*)$ доказывается также, как в предыдущем пункте, переходы $(*_2)$ возможны, т.к. в $\delta$ есть переходы $q_f\overset{\varepsilon,\gamma/\varepsilon}{\longrightarrow}q_f$. Получаем $(q_s,w,Z)\vdash^*(q_f,\varepsilon,\varepsilon)\Rightarrow w\in L(N)$.
\item Пусть $w\in L(N)$. После $q_s$ в стеке на дне всегда $X$. В конце его нет, и в изначальном автомате $P$ нет удалений $X$ (т.к. $X\notin \Gamma_P$). Значит, был переход $q_f\longrightarrow q_f$. Но из $q_f$ нет переходов в другие состояния, поэтому $q_f$~--- последнее состояние: $(q_s,w,Z_0)\vdash(q_0,w,Z_0X)\vdash^*(q_f,\varepsilon,\varepsilon)$. Найдем первую конфигурацию с конца, состояние которой~--- не $q_f$: $(q_s,w,Z_0)\vdash(q_0,w,Z_0X)\vdash^*(q,w,\mu X)\vdash^*(q_f,\varepsilon,\varepsilon)$. Переходы в $q_f$ есть только из $q_k\in F$, поэтому $q\in F$. Отсюда получаем цепочку конфигураций в $P$: $(q_0,w,Z_0)\vdash^*(q,w,\mu)$, так как наличие одного символа на дне стека не изменяет работу автомата в данном случае. $q\in F\Rightarrow w\in L(P)$.
\end{enumerate}
\item МП-автомат, эквивалентный автомату из задачи 1, принимающий по допускающему состоянию:
\begin{center}
\begin{tikzpicture}[shorten >=1pt,node distance=2cm,on grid,auto,every node/.style={text centered},initial text=]
	\node [state,initial] (q_s)	{$q_s$};
	\node [state] (q_0) [right = 3cm of q_s]	{$q_0$};
	\node [state] (q_1) [right = 4cm of q_0 ] {$q_1$};
	\node [state,accepting] (q_f) [shift={(5 cm,-4cm)}] {$q_f$};
	\path[->]
		(q_s) edge node {$\varepsilon,Z/ZX$} (q_0)
		(q_0) edge [out=40,in=140,loop] node[swap] {$\substack{ [_1,Z/[_1Z \\ [_1,[_1/[_1[_1 \\ [_1,[_2/[_1[_2 \\}$} (q_0)
			  edge [out=-40,in=-140,loop] node {$\substack{ [_2,Z/[_2Z \\ [_2,[_1/[_2[_1 \\ [_2,[_2/[_2[_2 \\}$} (q_0)
			  edge [out=20,in=160] node {$\substack{ ]_1,[_1/\varepsilon \\ ]_2,[_2/\varepsilon}$} (q_1)
			  edge node [anchor=0,above=-0.15,swap] {$\varepsilon,Z/Z$} (q_1)
			  edge [out=-30,in=95] node [anchor=0,above=-0.8,swap] {$\varepsilon,X/X$} (q_f)
		(q_1) edge [in=40,out=140,loop] node {$\substack{ ]_1,[_1/\varepsilon \\ ]_2,[_2/\varepsilon \\}$} (q_0)
			  edge [in=-40,out=-140,loop] node [swap] {$\varepsilon,Z/\varepsilon$} (q_1)
			  edge [in=-20,out=-160] node {$\substack{ [_1,Z/[_1Z \\ [_2,Z/[_2Z}$} (q_0)
			  edge [out=210,in=85] node [anchor=0,above=-0.4,swap] {$\varepsilon,X/X$} (q_f);
\end{tikzpicture}
\end{center}
\end{enumerate}
\section*{Задача 3}
\begin{enumerate}
\item $\Sigma\eqdef\{a,b,c\}$. КС-грамматика $\Gamma\equiv(N,\Sigma,P,S)$.\begin{enumerate}
\item $N\eqdef\{S,A,B,A_1,A_2,B_1\}$
\item $P=\{S\longrightarrow A|B,\,A\longrightarrow A_1A_2,\,A_1\longrightarrow aA_1b|\varepsilon,\,A_2\longrightarrow cA_2|\varepsilon,\,B\longrightarrow aBc|B_1,\,B_1\longrightarrow bB_1|\varepsilon\}$
\end{enumerate}
\begin{enumerate}[1.]
\item $L\eqdef\{a^ib^jc^k\big|i=j\vee i=k,\,i,j,k\geqslant 0\}$
\item \label{aibi} Докажем, что $A_1$ порождает $a^ib^i$:\begin{enumerate}
\item Фиксируем $i$. Применим $A_1\longrightarrow aA_1b$ $i$ раз, получим $a^iA_1b^i$. Применим $A_1\longrightarrow\varepsilon$. Получим $a^ib^i$
\item Пусть $A_0\longrightarrow^*w\in \Sigma^*$. Заметим, что к $A_1$ могут быть применены только правила $A_1\longrightarrow aA_1b$ и $A_1\longrightarrow\varepsilon$. Оба не добавляют нетерминалов, первое не уменьшает количество $A_1$, второе уменьшает его на $1$. Значит (КС-грамматика, правила применяются к нетермиранам), в выводе $i$ применений первого, одно применение второго. Получаем $w=a^ib^i$.
\end{enumerate}
\item \label {same} Аналогично (используя количество нетерминалов в правилах) докажем, что $A_2$ порождает $c^j$, $B_1$ порождает $b^k$, $B$ порождает $a^ib^jc^i$.
\item Пусть $w\in L$. Построим вывод $w$ в $\Gamma$:\begin{enumerate}
\item Если $w=\varepsilon$, то вывод следующий: $S\overset{S\rightarrow B}{\longrightarrow} B\overset{B\rightarrow B_1}{\longrightarrow} B_1\overset{B_1\rightarrow\varepsilon}{\longrightarrow} \varepsilon$
\item Пусть $w\neq\varepsilon, w=a^ib^ic^k$. $S\overset{S\rightarrow A_1A_2}{\longrightarrow}A_1A_2$.\begin{enumerate}
\item $\ref{aibi}\Longrightarrow A_1\rightarrow^* a^ib^i$
\item $\ref{same}\Longrightarrow A_2\rightarrow^* c^j$
\end{enumerate}
Получаем $S\rightarrow^* a^ib^ic^j$
\item Пусть $w\neq\varepsilon, w=a^ib^ja^i$. $\ref{same}\Rightarrow S\rightarrow^* a^ib^jc^i$.
\end{enumerate}
Получаем $\boxed{L\subseteq L(\Gamma)}$
\item Пусть $S\longrightarrow^* w$. Из $S$ могут быть получены только $A$ и $B$. Рассмотрим эти случаи:\begin{enumerate}
\item Первое правило $S\longrightarrow A$. Из $A$ могут быть получены только $A_1A_2$, \ref{aibi} $\Rightarrow$ из $A_1$ получено $a_ib_i$, \ref{same} $\Rightarrow$ из $A_2$~--- $c_j$. Получаем, что $w=a^ib^ic^j\in L$.
\item Первое правило $S\longrightarrow B$. \ref{same} $\Rightarrow$ из $B$ может быть получено только $w\equiv a_ib^jc^i\in L$
\end{enumerate}
Получаем $\boxed{L(\Gamma)\subseteq L}$
\end{enumerate}
\end{enumerate}
\end{document}
% ГРЕБАНЫЙ ACM!!11