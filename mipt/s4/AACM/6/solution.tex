\documentclass[a4paper]{article}
\usepackage[a4paper, left=5mm, right=5mm, top=5mm, bottom=5mm]{geometry}
%\geometry{paperwidth=210mm, paperheight=2000pt, left=5pt, top=5pt}
\usepackage[utf8]{inputenc}
\usepackage[english,russian]{babel}
\usepackage{indentfirst}
\usepackage{tikz} %Рисование автоматов
\usetikzlibrary{automata,positioning,arrows,trees}
\usepackage{amsmath}
\usepackage[makeroom]{cancel} % зачеркивание
\usepackage{multicol,multirow} %Несколько колонок
\usepackage{hyperref}
\usepackage{tabularx}
\usepackage{amsfonts}
\usepackage{amssymb}
\DeclareMathOperator*{\argmin}{arg\,min}
\usepackage{listings}
\usepackage{wasysym}
\date{задано 2014.03.20}
\author{Сергей~Володин, 272 гр.}
\newcommand{\matrixl}{\left|\left|}
\newcommand{\matrixr}{\right|\right|}
% названия автоматов  (rubtsov)
\def\A{{\cal A}}
\def\B{{\cal B}}
\def\C{{\cal C}}

%классы сложности (rubtsov)
\def\PP{{\mathsf{P}}}
\def\NP{{\mathsf{NP}}}
\def\NPc{{\mathsf{NP}}\text{-}{\rm c}}
\def\coNP{{\rm co}\text{-}{\mathsf{NP}}}
\def\DTIME{{\mathsf{DTIME}}}
\def\NTIME{{\mathsf{NTIME}}}
\def\HALT{{\rm{HALT}}}
\def\SAT{{\rm{SAT}}}
\def\3SAT{{\rm{3\text{-}SAT}}}
\def\2SAT{{\rm{2\text{-}SAT}}}
\def\UNSAT{{\rm{UNSAT}}}
\def\HP{{\rm{HAMPATH}}}
\def\UHP{{\rm{UHAMPATH}}}
\def\LL{{\mathrm{LL}}}
\def\poly{{\rm{poly}}}
\def\GC{{\mbox{ГЦ}}}
\def\GP{{\mbox{ГП}}}

\title{Алгоритмы и модели вычислений.\\Задание 6: всякая хуйня}

% алгоритмы (Рубцов)
\usepackage{verbatim}
\usepackage{listings}
\usepackage{algorithm2e}

%+= и -=, иначе разъезжаются...
\newcommand{\peq}{\mathrel{+}=}
\newcommand{\meq}{\mathrel{-}=}
\newcommand{\deq}{\mathrel{:}=}
\newcommand{\plpl}{\mathrel{+}+}

\newcommand{\todo}{{\em todo}}

% пустое слово  (rubtsov)
\def\eps{\varepsilon}

% регулярные языки  (rubtsov)
\def\REG{{\mathsf{REG}}}
\def\CFL{{\mathsf{CFL}}}
\def\eqdef{\overset{\mbox{\tiny def}}{=}}
\newcommand{\niton}{\not\owns}

%FIRST & FOLLOW (rubtsov)
\def\first{\mathrm{ FIRST} }
\def\follow{\mathrm{ FOLLOW} }

% LL LR (rubtsov)
\def\LL{{\mathrm{LL}}}
\def\LR{{\mathrm{LR}}}

\newcounter{rowItemCount}
\newcounter{subRowItemCount}
\newcommand\rowItem{
    \setcounter{subRowItemCount}{0}
    \arabic{rowItemCount}.\addtocounter{rowItemCount}{1}}
\newcommand\subRowItem{
    \addtocounter{subRowItemCount}{1}
    \addtocounter{rowItemCount}{-1}
    \arabic{rowItemCount}.\arabic{subRowItemCount}.\addtocounter{rowItemCount}{1}}
    
\newcommand{\smalll}[1]{\overline{\overline{#1}}}
\newcommand{\smallo}{\bar{\bar{o}}}

\begin{document}
\maketitle
\subsection*{(каноническое) Задача 24}
\subsection*{(каноническое) Задача 25}
\subsection*{(каноническое) Задача 26}
\subsection*{(каноническое) Задача 27}
Пусть $f\colon \GC\subset\Sigma^*\to\{0,1\}$, $f(x)=1\Leftrightarrow x\in\GC$, и $T_f(x)=\poly(|x|)$.
\begin{enumerate}
\item Фиксируем граф $G$, его описание $x\in\Sigma^*$. Построим алгоритм поиска гамильтонова пути (если он существует), использующий $f$. Обозначим за $h(G, v)$~--- граф, полученный из $G$ удалением вершины $v$ и направлением $(u,v), (v, w)$ в $(v, w)$. Фиксируем некоторую вершину $v$ графа $G$. Рассмотрим граф $h(G,v)$. Он также гамильтонов \todo. Будем пребирать все вершины $u$ графа $h(G,v)$ и рассматривать $h(h(G,v),u)$. Один из них будет гамильтоновым \todo. Значит, в некотором гамильтоновом пути в $G$ вершины $u$ и $v$ стояли рядом \todo. Продолжим этот процесс, пока не останутся две вершины $v_1$ и $v_2$. Они стоят рядом. Полученная последовательность $(v,u,u',...,u^{(l)},v_1,v_2,v)$~--- искомый гамильтонов цикл.
\item Псевдокод
\lstset{
    language=C,
    basicstyle=\ttfamily\small,
    breaklines=true,
    prebreak=\raisebox{0ex}[0ex][0ex]{\ensuremath{\hookleftarrow}},
    frame=lines,
    showtabs=false,
    showspaces=false,
    showstringspaces=false,
    keywordstyle=\color{red}\bfseries,
    stringstyle=\color{green!50!black},
    commentstyle=\color{gray}\itshape,
    numbers=left,
    captionpos=t,
    escapeinside={\%*}{*)}
}
\begin{lstlisting}
path(x)
{
    if(!f(x)) return(empty); // no path
    else
    {
    }
}
\end{lstlisting}
\item Время работы. Всего в графе $n$ вершин, для каждой перебираем не более, чем $n$, откуда сложность $T(x)=O(n^2)$. Поскольку $|x|=\Omega(n)$, то $T(x)=O(|x|^2)=\poly(|x|)$. Более точно, используя псевдокод: \todo.
\end{enumerate}
\end{document}