\documentclass{article}
\usepackage[utf8]{inputenc}
\usepackage[russian]{babel}
\usepackage{graphicx}
\usepackage{indentfirst}
\usepackage{amsmath}
\usepackage{amsfonts}
\title{Задание от 2013.02.15}
\author{С.~Е.~Володин, 272 гр.}
\date{}
\begin{document}
\maketitle
\section{Задача 1}
Строка $\tilde{S}$ в указанном массиве будет расположена перед всеми суффиксами, которые ей равны, т.к. её индекс~--- $1$~--- меньше. Поэтому условие эквивалентно следующему: посчитать количество $k$ суффиксов строки $\tilde{S}$, которые меньше самой строки. Ответом будет $k+1$. Нумерация элементов строки начинается с $0$: $\tilde{S}[0]=s_1$.

Посчитаем z-функцию \begin{math}z(x):[0,n-1]\rightarrow\mathbb{N}\cup\{0,\infty\}\end{math} для префикса $\tilde{S}$ длины $n$. Если при расчете сравниваются символы с индексом, большим $2n$, то $z(x)\stackrel{\mathrm{def}}{=}\infty$, т.к. в таком случае префикс элемента $x$ целиком совпадает со строкой $\tilde{S}$, т.е.  $z(x)$ больше любого конечного числа. {\em Время и память: $O(n)$}

По определению z-функции $\forall x\in[0,n-1]\hookrightarrow\tilde{S}[x+z(x)]\neq\tilde{S}[z(x)]$, т.к. в противном случае z-функция посчитана неверно (можно увеличить на $1$). Так как суффиксы бесконечны, то результат сравнения $\tilde{S}_{x+1}<\tilde{S}_1$ совпадает с результатом сравнения $\tilde{S}[x+z(x)]<\tilde{S}[z(x)]$, т.к., по определению $z(x)$, все символы с номерами $(x,\dots,x+z(x)-1)$ и $(0,\dots,z(x)-1)$ совпадают, а следующие~--- различны.

Поэтому $k=|\{x\in[0,n-1]:z(x)\neq\infty\wedge\tilde{S}[x+z(x)]<\tilde{S}[z(x)]\}|$. Для того, чтобы посчитать это значение, достаточно цикла из $n$ итераций. {\em Время: $O(n)$, память: $O(1)$}.
\end{document}
