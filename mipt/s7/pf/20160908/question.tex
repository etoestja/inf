\documentclass[a4paper]{article}
\usepackage[a4paper, left=25mm, right=25mm, top=25mm, bottom=25mm]{geometry}
%\geometry{paperwidth=210mm, paperheight=2000pt, left=5pt, top=5pt}
\usepackage[utf8]{inputenc}
\usepackage[english,russian]{babel}
\usepackage{indentfirst}
\usepackage{tikz} %Рисование автоматов
\usetikzlibrary{automata,positioning,arrows,trees}
\usepackage{amsmath}
\usepackage{enumerate}
\usepackage[makeroom]{cancel} % зачеркивание
\usepackage{multicol,multirow} %Несколько колонок
\usepackage{hyperref}
\usepackage{tabularx}
\usepackage{amsfonts}
\usepackage{amssymb}
\DeclareMathOperator*{\argmin}{arg\,min}
\usepackage{wasysym}
\date{\today}

\begin{document}
%{\bf Прошу указать, где подразумеваемое рассуждение в статье расходится с данным (получаются разные результаты)}
	
Пусть $f\colon \mathbb{R}^n\to\mathbb{R}^m$, $f_i(x)=x^TA_ix-2b_i^Tx$, $A_i=A_i^T$. Пусть $c\in\mathbb{R}^m$.

Обозначим $c\cdot  A=\sum\limits_{i=1}^n c_iA_i$, $c\cdot b=\sum\limits_{i=1}^n c_ib_i$, $F_c(x)=c^Tf(x)$

	{\bf Хотим найти:}
$$\min_{||x||^2=1}F_c(x)$$

Функция Лагранжа:
$L(x,\lambda)=x^T(c\cdot A)x-2(c\cdot b)^T x-\lambda(||x||^2-1)$.

Находим $L_x=2(c\cdot A)x-2c\cdot b-2\lambda x=0$, $L_\lambda=||x||^2-1=0$,

Получаем систему $\begin{cases}
||x||=1\\
(c\cdot A-\lambda)x=c\cdot b
\end{cases}$
Это совпадает с (2.3).

Далее переходим в базис из собственных векторов $\{x_i\}$ симметричной матрицы $c\cdot A$

$$S=||x_1...x_n||,\,S^TS=E$$

$$x=Sy,\,\Lambda=S^T(c\cdot A)S,\,c\cdot b=S\alpha$$

Получаем
$$\begin{cases}
||y||=1 &(1)\\
(\Lambda-\lambda)y=\alpha &(2)
\end{cases}
$$

Выражаем из (2) $$y_k=\frac{\alpha_k}{\lambda_k-\lambda}$$

Подставляем в (1)~--- это совпадает с (2.7)

$$\sum\limits_{i=1}^n (\frac{\alpha_k}{\lambda_k-\lambda})^2=1$$

Это соотношение определяет $\lambda$, по $\lambda$ находим $y_k$, затем находим $x$ в старом базисе.

\end{document}