\documentclass[a4paper]{article}
\usepackage[a4paper, left=5mm, right=5mm, top=5mm, bottom=5mm]{geometry}
%\geometry{paperwidth=210mm, paperheight=2000pt, left=5pt, top=5pt}
\usepackage[utf8]{inputenc}
\usepackage[english,russian]{babel}
\usepackage{indentfirst}
\usepackage{tikz} %Рисование автоматов
\usetikzlibrary{automata,positioning,arrows,trees}
\usepackage{amsmath}
\usepackage[makeroom]{cancel} % зачеркивание
\usepackage{multicol,multirow} %Несколько колонок
\usepackage{hyperref}
\usepackage{tabularx}
\usepackage{amsfonts}
\usepackage{amssymb}
\DeclareMathOperator*{\argmin}{arg\,min}
\usepackage{listings}
\usepackage{wasysym}
\date{задано 2014.02.27}
\author{Сергей~Володин, 272 гр.}
\newcommand{\matrixl}{\left|\left|}
\newcommand{\matrixr}{\right|\right|}
% названия автоматов  (rubtsov)
\def\A{{\cal A}}
\def\B{{\cal B}}
\def\C{{\cal C}}

%классы сложности (rubtsov)
\def\PP{{\mathsf{P}}}
\def\NP{{\mathsf{NP}}}
\def\NPc{{\mathsf{NP}}\text{-}{\rm c}}
\def\coNP{{\rm co}\text{-}{\mathsf{NP}}}
\def\DTIME{{\mathsf{DTIME}}}
\def\NTIME{{\mathsf{NTIME}}}
\def\HALT{{\rm{HALT}}}
\def\SAT{{\rm{SAT}}}
\def\3SAT{{\rm{3\text{-}SAT}}}
\def\2SAT{{\rm{2\text{-}SAT}}}
\def\UNSAT{{\rm{UNSAT}}}
\def\LL{{\mathrm{LL}}}
\def\poly{{\rm{poly}}}

\title{Алгоритмы и модели вычислений.\\Задание 3: Сложность вычислений, классы $\PP$, $\NP$}

% алгоритмы (Рубцов)
\usepackage{verbatim}
\usepackage{listings}
\usepackage{algorithm2e}

%+= и -=, иначе разъезжаются...
\newcommand{\peq}{\mathrel{+}=}
\newcommand{\meq}{\mathrel{-}=}
\newcommand{\deq}{\mathrel{:}=}
\newcommand{\plpl}{\mathrel{+}+}

% пустое слово  (rubtsov)
\def\eps{\varepsilon}

% регулярные языки  (rubtsov)
\def\REG{{\mathsf{REG}}}
\def\CFL{{\mathsf{CFL}}}
\def\eqdef{\overset{\mbox{\tiny def}}{=}}
\newcommand{\niton}{\not\owns}

%FIRST & FOLLOW (rubtsov)
\def\first{\mathrm{ FIRST} }
\def\follow{\mathrm{ FOLLOW} }

% LL LR (rubtsov)
\def\LL{{\mathrm{LL}}}
\def\LR{{\mathrm{LR}}}

\newcounter{rowItemCount}
\newcounter{subRowItemCount}
\newcommand\rowItem{
    \setcounter{subRowItemCount}{0}
    \arabic{rowItemCount}.\addtocounter{rowItemCount}{1}}
\newcommand\subRowItem{
    \addtocounter{subRowItemCount}{1}
    \addtocounter{rowItemCount}{-1}
    \arabic{rowItemCount}.\arabic{subRowItemCount}.\addtocounter{rowItemCount}{1}}
    
\newcommand{\smalll}[1]{\overline{\overline{#1}}}
\newcommand{\smallo}{\bar{\bar{o}}}

\begin{document}
\maketitle
\subsection*{Задача 2}
$f(n)=\mbox{poly}(n)$~--- время работы машины $M$ из условия на входе $x$ длины $n$. За каждый такт машина читает не более одного символа, поэтому количество прочитанных символов $|y_r|\leqslant f(n)$. Причем машина не могла читать их не подряд, так как за один такт головка смещается на $\leqslant\pm 1$ ячеек.\begin{enumerate}
\item Если $x\in L$, то, по условию, $\exists y\colon M(x,y)=1$. Возьмем $y'=y[1...f(n)]$, тогда $|y'|=O(\mbox{poly}(|x|))$. Тогда $M(x, y')\equiv M(x,y)$, так как машина <<не заметит>> изменение длины слова (к суффиксу она не обращалась).
\item Если $\exists y'\in\Sigma^{f(|x|)}\colon M(x,y')=1$, то возьмем $y=y'$, и по условию, $x\in L$.
\end{enumerate}
Получаем $x\in L\Leftrightarrow \exists y'\in\Sigma^{f(|x|)}\colon M(x,y')=1$. МТ полиномиальна по $|x|$, значит, полиномиальна по $|x\#y|$. Получаем $L\in\NP$, в качестве полиномиального по $|x|$ сертификата берем $y'(x)=y(x)[1...f(|x|)]$, где $f(n)$~--- полином из условия полиномиальности МТ по $|x|$.

\subsection*{(каноническое) Задача 8}
Модифицируем доказательство из Кормена. Вход~--- массив $A$, $|A|=n$. Количество целиком заполненных групп из $9$-ти элементов, медиана которых меньше $x$ не меньше, чем $\lceil \frac{1}{2}\lceil \frac{n}{9}\rceil\rceil-2$. Две группы не учитываются, так как в одной из них само $x$, а другая может не быть заполнена целиком. В каждой такой группе $5$ элементов, не превышающих свою медиану, поэтому $A_{<x}$~--- количество элементов $A$, меньших $x$ будет $A_{<x}\geqslant 5\lceil \frac{1}{2}\lceil \frac{n}{9}\rceil\rceil-2\geqslant\frac{5}{18}n-10$. Аналогично $A_{>x}\geqslant \frac{5}{18}n-10$. Тогда размеры групп $A_{\leqslant x},A_{\geqslant x}\leqslant n-\frac{5}{18}n+10=\frac{13}{18}n+10$. Рекуррентность:
$$
T(n)=T(\lceil\frac{n}{9}\rceil)+T(\frac{13}{18}n+10)+O(n).
$$
%Пусть последнее слагаемое $\leqslant cn$.
%Рассмотрим функцию $f\colon f(n)=\Theta(n)$, откуда $c_1n\leqslant f(n)\leqslant c_2n$. Подставим $f$ в рекуррентность:
%$c_1 n\leqslant f(n)\leqslant f(\lceil\frac{n}{9}\rceil)+f(\frac{13}{18}n+10)+cn\leqslant c_2(\frac{n}{9}+1+\frac{13}{18}n+10)+cn$, откуда найдем соотношение на константы $n(c_1-\frac{15}{18}c_2-c)\leqslant 11c_2$. Возьмем $c_1=c,\,c_2=2c$, тогда $c_1-\frac{15}{18}c_2-c=-\frac{15}{9}c<0$, и неравенства $c_2>0,\,c_2>c_1$ выполнены.
Фиксируем $n$. Пусть $T(n')\leqslant cn$ для правой части (доказательство по дереву рекурсии, от листьев к корню). Пусть функция $O(n)\leqslant an$. Тогда $T(n)\leqslant c(\frac{n}{9}+1+\frac{13}{18}n+10)+an=n(\frac{5}{6}c+a)+11c$. Докажем, что эта величина также меньше $cn$ (шаг индукции) при некоторых $c$: $n(\frac{5}{6}c+a-c)+11c\leqslant 0\Leftrightarrow n(a-\frac{c}{6})+11c\leqslant 0$. Возьмем $c=7a$, откуда получим требуемое (при достаточно больших $n$ неравенство выполнено).
\subsection*{(каноническое) Задача 11}
$\mbox{M}^{\mathbb{Z}, S}_{p\times q}$~--- множество матриц $\matrixl a_{ij} \matrixr$ размера $p\times q$ с целыми коэффициентами, такими, что $|a_{ij}|\leqslant S$. $S=10000, m=2014$. Язык $\{0,1\}^*\supset L_{n}=\{\mbox{bin}(m,n,A,b)\big|m\in\mathbb{N},\,(A,b)\in\mbox{M}^{\mathbb{Z}, S}_{m\times n}\times \mbox{M}^{\mathbb{Z}, S}_{m\times 1},\,Ax=b\mbox{~--- несовместна}\}$~--- двоичные записи несовместных систем линейных уравнений с целыми коэффициентами (функция bin кодирует матрицу в двоичной записи).
% {\em (имеются в виду %$L'\subset(\overline{-L,L}^{m\cdot n\cdot m\cdot 1})^*$) двоичные записи}
\begin{enumerate}
\item Рассмотрим $w^i_j=\big(\matrixl\begin{array}{cccc}
i & 0 & ... & 0\\
\end{array}\matrixr,\,\matrixl
\begin{array}{c}
j\\
\end{array}\matrixr\big)$. При $i=0,\,j\in\{1,2\}$ система несовместна, поэтому $\mbox{bin}(w^0_1),\,\mbox{bin}(w^0_2)\in L_{2014}$. При $i=1,\,j\in\{1,2\}$ система совместна, поэтому $\mbox{bin}(w^1_1),\,\mbox{bin}(w^1_2)\notin L_{2014}$
\item \begin{enumerate}
\item Опишем алгоритм и докажем его корректность. Рассмотрим расширенную матрицу $C={\matrixl A\big| b \matrixr}^{\Square}$. Модуль ее элементов не превосходит $L$. Будем применять к ней последовательно элементарные операции над строками $S_i$, получая матрицу $C'_i={\matrixl A'_i\big| b'_i \matrixr}^{\Square}$. Поскольку $Ax=b\Leftrightarrow A'_ix=b'_i$ (системы эквивалентны), исходная система совместна $\Leftrightarrow$ полученная после операций система совместна. Применим метод Гаусса (прямой ход) к матрице $C$ (ненулевые элементы берем не из последнего столбца), состоящий из элементарных операций над строками. Пусть в $i$-й строке найден столбец $j$ с ненулевым элементом $a_{ij}\neq 0$. Перед методом Гаусса переставим строки так, чтобы $j'(i)=i$ (ненулевые элементы на главной диагонали)~--- элементарная операция над столбцами (т.е. переобозначим неизвестные). После прямого хода метода Гаусса получим матрицу
$$C'=
\begin{Vmatrix}
1 & & \mbox{\bf{*}} &&& \multirow{5}{*}{\mbox{\bf{*}}} & b'_1\\
 & \ddots & & & && \vdots\\
\mbox{\bf{0}} & & 1& & & & b'_r\\
0 & ... & 0 & 0 & 0 & 0 & b'_{r+1}\\
\multicolumn{7}{c}{...}\\
0 & ... & 0 & 0 & 0 & 0 & b'_{n}
\end{Vmatrix}
$$
Единицы получились именно на диагонали, так как столбцы были переставлены. $r$-я строка является последней ненулевой (в противном случае можно продолжить метод Гаусса)
\item Докажем, что система несовместна $\Leftrightarrow$ $\exists i\in\overline{r+1,n}\colon b'_{i}\neq 0$\begin{enumerate}
\item $\boxed{\Leftarrow}$ Имеем уравнение $0^T x=1$
\item $\boxed{\Rightarrow}$ (от противного) Пусть система несовместна, и все $b_i$ отличны от нуля. Выполним метод Гаусса до конца, убрав <<$*$>> выше единиц на диагонали. Левее столбца $b'$ не могла получится строка из нулей (по алгоритму вычитаем $i$-ю строку из всех строк выше, поэтому $i$-я единица на диагонали останется). Поэтому выше нет строк вида $\begin{Vmatrix}
0 & ... & 0 & 1
\end{Vmatrix}$. Но их нет и ниже $r$-й строки, поэтому их нет вовсе. Метод Гаусса привел матрицу к упрощенному виду, и по Предложению 1 (Беклемишев, стр. 151) система совместна~--- противоречие.
\end{enumerate}
\item Рассмотрим метод Гаусса. Пусть $\{C_k\}_{k=0}^{r}$~--- преобразованные матрицы, $C_i$~--- матрица после $i$ шагов алгоритма (рассмотрены первые $i$ строк). $C_0\equiv C$. Обозначим элементы матрицы $A_k=\begin{Vmatrix}a_k^{ij}\end{Vmatrix}$. Пусть алгоритм выполнил $k-1$ шагов. Рассмотрим изменение элементов матрицы на $k$-м шаге.\newline
$\matrixl
\begin{array}{ccccc}
... & a_{kk} & ... & a_{kj} & ...\\
\multicolumn{5}{c}{...}\\
... & a_{ik} & ... & a_{ij} & ...\\
\end{array}
\matrixr\sim
\matrixl
\begin{array}{ccccc}
... & 1 & ... & \frac{a_{kj}}{a_{kk}} & ...\\
\multicolumn{5}{c}{...}\\
... & a_{ik} & ... & a_{ij} & ...\\
\end{array}
\matrixr\sim
\matrixl
\begin{array}{ccccc}
... & 1 & ... & \frac{a_{kj}}{a_{kk}} & ...\\
\multicolumn{5}{c}{...}\\
... & 0 & ... & a_{ij}-\frac{a_{kj}}{a_{kk}}a_{ik} & ...\\
\end{array}
\matrixr
$

\begin{enumerate}
\item $k$-я строка делится на $a_{k-1}^{kk}$, поэтому $a_k^{kj}=\frac{a_{k-1}^{kj}}{a_{k-1}^{kk}}$
\item $k$-я строка вычитается из всех $k<i$-х ниже\begin{enumerate}
\item В $k$-м столбце нули ниже главной диагонали: $a_k^{ik}=0, i>k$.
\item В $k<j$-м столбце $k<i$-й строки $a_k^{ij}=a_{k-1}^{ij}-a_{k-1}^{ik}\frac{a_{k-1}^{kj}}{a_{k-1}^{kk}}$.\newline
<<Вынесем за скобки>> индекс $k-1$ (в этой формуле он один для всех $a_{k-1}$): $a_k^{ij}=\big(\frac{a^{ij}a^{kk}-a^{kj}a^{ik}}{a_{kk}}\big)_{k-1}$\newline
Пусть дана матрица $A:m\times n$. Определим $\Delta^{i_1,...,i_t}_{j_1,...,j_t}$~--- определитель подматрицы, полученной из $A$ вычеркиванием всех строк кроме $i_1,...,i_t$ и всех столбцов кроме $j_1,...,j_t$.\newline
С этим обозначением $a_k^{ij}=\big(\frac{\Delta^{ki}_{kj}}{\Delta^k_k}\big)_{k-1}$ \label{matrixrec}
\end{enumerate}
\end{enumerate}
Получаем $A_k=\matrixl
\begin{array}{ccccc}
... & ... & ... & ... & ...\\
... & 1 & \multicolumn{3}{c}{(\frac{a^{kj}}{a^{kk}})_{k-1}}\\
... & 0 & &   &\\
... & 0 & & (\frac{\Delta^{ki}_{kj}}{\Delta^k_k})_{k-1} &\\
... & 0 & &   &\\
\end{array}
\matrixr$, где <<$...$>> означают, что элементы не меняются.
\item (Задача 11.3)
\begin{enumerate}
\item {\em (Я проверил для $k<=3$, т.е. утвверждение не доказано)}. Получим по индукции формулу $a_k^{ij}=\frac{\Delta^{12...ki}_{12...kj}}{\Delta^{12...k}_{12...k}}$ ??? % как блять это сделать?
\item Из формулы выше следует, что получающиеся при промежуточных вычислениях числители и знаменатили элементов матрицы ограничены сверху $\max(|\Delta^{12...ki}_{12...kj}|,\,|\Delta^{12...k}_{12...k}|)$. По формуле полного разложения для числителя\newline $\Delta^{12...ki}_{12...kj}=\sum\limits_{t_1,...,t_{k+1}}(-1)^{\mbox{sign}(t_1,...,t_{k+1})}a_{xx}\cdot...\cdot a_{xx}$ (индексы опущены), что по модулю\newline $|\Delta^{12...ki}_{12...kj}|\leqslant [\max({m,n})]!\max\limits_{A,b}|a_{ij}|^{\max(m,n)}\boxed{\leqslant}$. Обозначим $M=\max(m,n)$, получим $\boxed{\leqslant}M^Mh^M=(Mh)^M$. Аналогично для знаменателя.\newline
Итак, числители и знаменатели элементов матрицы, получающихся при промежуточных вычислениях, ограничены сверху $(Mh)^M$, где $M=\max(m,n)$.
\end{enumerate}
\item Для оценки времени работы приведен псевдокод:
\lstset{
    language=C,
    basicstyle=\ttfamily\small,
    breaklines=true,
    prebreak=\raisebox{0ex}[0ex][0ex]{\ensuremath{\hookleftarrow}},
    frame=lines,
    showtabs=false,
    showspaces=false,
    showstringspaces=false,
    keywordstyle=\color{red}\bfseries,
    stringstyle=\color{green!50!black},
    commentstyle=\color{gray}\itshape,
    numbers=left,
    captionpos=t,
    escapeinside={\%*}{*)}
}

%
% a11  a1n
% am1  amn
%

\begin{lstlisting}
//M[i][j] - matrix A|b
for(i = 1; i <= m; i++) // rows i=1...m
{
  for(j = 1; j <= n; j++) // find j: aij != 0
  {
    if(M[i][j] != 0) // found
    {
      C = M[i][j];

      // dividing i-th row by non-zero element
      for(k = 1; i <= n + 1; i++)
        M[i][k] /= C;

      for(k = i + 1; k <= m; k++) // subtracting from row k down
      {
        C = M[k][j];
        for(l = 1; l <= n + 1; l++) // column l
          M[k][l] -= M[i][l] * C;
      }
      
      break;
    }
  }
}
\end{lstlisting}
\item Храним в МТ рациональные числа как числитель и знаменатель. Оценим их сверху. Вернемся к формуле \ref{matrixrec}, запишем ее в виде $a_k^{ij}=\frac{\frac{a_1}{a_2}\frac{b_1}{b_2}-\frac{c_1}{c_2}\frac{d_1}{d_2}}{\frac{a_1}{a_2}}=\frac{a_1b_1c_2d_2-c_1d_1a_2b_2}{b_2c_2d_2a_1}$. Если числители и знаменатели на $k-1$ шаге ограничены $L$, то на $k+1$-м они будут ограничены $2L^4$. Рассуждая по индукции, на последнем шаге получим, что они ограничены $2(...2(2(2L^4)^4)...)^4$, где возведение в четвертую степень происходит количество раз, равное рангу матрицы (количество шагов алгоритма). Но он не превосходит $n=2014$. Поэтому максимальный модуль числа фиксирован. Получаем, что арифметические операции выполняются за $O(1)$.
\item Оценим время работы как $T(A,b,m)\leqslant m\times n\times (O(1) + n \times O(1) + m \times (O(1) + n \times O(1)))\overset{n=2014}{=}O(m^2)$. $\mbox{bin}(\cdot)$~--- двоичная запись числа. Длина входа $I(A,b,m)=(mn+m)\min\limits_{A,b}|\mbox{bin}(a_{ij})|=\Omega(m)\geqslant cm$, поэтому $T(A,b,m)\leqslant c_1m^2\leqslant cI^2(A,b,m)=O(I^2)$.
%\item Оценим стоимость одного умножения как $O(\log^2X)$, сложения и вычитания как $O(\log X)$, откуда 
\end{enumerate}
\newpage
\subsection*{(каноническое) Задача 12}
\begin{enumerate}
\item Используем быстрое возведение в степень по модулю $d$. Умножаем числа не более, чем по $2|d|$ бит. Остаток от деления считается за квадрат длины битовой записи. Псевдокод:
\lstset{
    language=C,
    basicstyle=\ttfamily\small,
    breaklines=true,
    prebreak=\raisebox{0ex}[0ex][0ex]{\ensuremath{\hookleftarrow}},
    frame=lines,
    showtabs=false,
    showspaces=false,
    showstringspaces=false,
    keywordstyle=\color{red}\bfseries,
    stringstyle=\color{green!50!black},
    commentstyle=\color{gray}\itshape,
    numbers=left,
    captionpos=t,
    escapeinside={\%*}{*)}
}
\begin{lstlisting}
number power(a, b, d)
{
   if(b == 0) return(1);
   if(b % 2 == 0)
   {
      number x = power(a, b / 2, d);
      return((x * x) % d);
   }
   else
   {
      number x = power(a, (b - 1) / 2, d);
      x = (x * x) % d;
      return((a * x) % d);
   }
}
ans = (power(a, b, d) == (c % d));
\end{lstlisting}
На каждом шаге второй аргумент уменьшается как минимум вдвое, поэтому высота дерева рекурсии $h\leqslant \log_2 b$. На каждом шаге производятся операции над числами битовой длины не более $2\log d$, на листе дерева рекурсии ($b=0$) выполняется $O(1)$ операций. Последний шаг (сравнение) выполняется за $O(\log d)$ операций. Сложность арифметических операций не более, чем квадратичная по длине битовой записи.\newline
Получаем $T(a, b, c, d)\leqslant \log_2 b\cdot O(\log^2 d)+O(1)=O(\log^2 d\log b)$. Длина входа $I(a,b,c,d)=\log a+\log b+\log c+\log d$, поэтому $T=O(I^3)$.
\item Слова, соответсвующие $(1,1,1,2),\,(1,2,1,2)\in L$, $(1,1,2,2),\,(1,2,2,2)\notin L$
\end{enumerate}
\end{enumerate}
\subsection*{(каноническое) Задача 13}

\lstset{
    language=C,
    basicstyle=\ttfamily\small,
    breaklines=true,
    prebreak=\raisebox{0ex}[0ex][0ex]{\ensuremath{\hookleftarrow}},
    frame=lines,
    showtabs=false,
    showspaces=false,
    showstringspaces=false,
    keywordstyle=\color{red}\bfseries,
    stringstyle=\color{green!50!black},
    commentstyle=\color{gray}\itshape,
    numbers=left,
    captionpos=t,
    escapeinside={\%*}{*)}
}
Бинпоиском ищем корень 2014 степени. $L=1$, $R$~--- вход. Шагов $\log_{2} R$=$\log_2 2^t=t$, возводим числа $<= 2^t$ в 2014 степень за $\log^{2014} 2^t=t^{2014}$. Псевдокод:\newline
\begin{lstlisting}
number L = 1;
number R = X = input();

number M, B = 2014;
while(R - L > 1)
{
  M = (R + L) / 2;
  if(power(M, B) < X)
    R = M;
  else L = M;
}
ans = 0;
if(power(L, B) == X)
	ans = 1;
else if(power(R, B) == X)
	ans = 1;
\end{lstlisting}
Поддерживается свойство: ответ всегда лежит в $[L,R]$. На каждой итерации цикла $|R-L|$ уменьшается вдвое, откуда цикл совершает $O(\log X)$ итераций. На каждой производится возведение в степень $B=2014$ за $O(\log^{2014} X)$. Последние сравнения занимают $O(\log^{2014}X)$, поэтому $T(I)=O(\log X)\cdot O(\log^{2014}X)+O(\log^{2014}X)=O(\log^{2015} X)$, где длина входа $I$~--- длина битовой записи числа $X$, т.е. $I=\Theta(\log X)$, откуда $T=O(I^{2015})$.
\subsection*{(каноническое) Задача 14}
Определение замкнутости $\PP$ относительно $\cdot^*$:
$$
\forall L\in\PP\hookrightarrow L^*\in\PP.
$$
Пусть $L\in\PP$. Тогда $\exists$ МТ $M$: время ее работы $T(|w|)=f(|w|)=\mbox{poly}(|w|)$. 
\subsection*{(каноническое) Задача 15}
\begin{enumerate}
\item $DA$
\begin{enumerate}
\item $DA, L(\cdot)=\varnothing$. Обходом графа в ширину ищем пути из принимающего состояния. Время $T=O(|V|+|E|)$, где $|V|$ и $|E|$~--- количества вершин и ребер соответственно. Длина входа $I$~--- описание графа. $I=\Theta(|V|^2)$ (матрица смежности). $|E|\leqslant |V|^2$, поэтому $T=O(|V|^2)=O(I)$.
\item $DA, |L(\cdot)|=\infty$. Ищем циклы в графе обходом в ширину.
\item $DA, w\in L(\cdot)$. Переходим по графу за $O(|w|)$. Если перешли в принимающее состояние автомата~--- МТ переходит в $q\in Acc$. МТ останавливается в любом случае, так как для каждого символа слова совершается один переход в автомате за ограниченное сверху время.
\item $DA, w\notin L(\cdot)$. Решаем предыдущую разрешимую задачу и выдаем противоположный ответ.
\end{enumerate}
\item $NA$
\begin{enumerate}
\item Работает тот же алгоритм, что и для $DA$.
\item Работает тот же алгоритм, что и для $DA$.
\item Храним не одно состояние автомата, а множество состояний, в котором он может оказаться при прочтении префикса слова. Поддерживаем это свойство для каждого нового символа. В конце, если среди множества есть принимающие состояния автомата, МТ переходит в принимающее состояние.
\item Предыдущая задача, противоположный ответ.
\end{enumerate}
\item $R$. Строим НКА за линейное по размеру $R$ время. Далее аналогично.
%\item $G$ Построим ориентированный граф, вершинами которого будут нетерминалы. Есть ребро $A\to B\Leftrightarrow$ есть правило $A\to ...B...$. Если есть правило $$
\item $\A, \B$~--- ДКА. Построим минимальные ДКА за полиномиальное по $|A|+|B|$ время: на каждом шаге алгоритма количество состояний уменьшается, поэтому количество шагов не превосходит $|\A|$. На каждом шаге выполняется полиномиальное число действий (от количества состояний). Проверим изоморфность двух минимальных ДКА за $|A|+|B|$. Длина входа $|A|^2+|B|^2$ (графы входных автоматов заданы матрицами смежности).
\end{enumerate}
\end{document}