\documentclass[a4paper]{article}
\usepackage[a4paper, left=5mm, right=5mm, top=5mm, bottom=5mm]{geometry}
%\geometry{paperwidth=210mm, paperheight=2000pt, left=5pt, top=5pt}
\usepackage[utf8]{inputenc}
\usepackage[english,russian]{babel}
\usepackage{indentfirst}
\usepackage{tikz} %Рисование автоматов
\usetikzlibrary{automata,positioning,arrows}
\usepackage{amsmath}
\usepackage{enumerate}
\usepackage[makeroom]{cancel} % зачеркивание
\usepackage{multicol} %Несколько колонок
\usepackage{hyperref}
\usepackage{amsfonts}
\usepackage{amssymb}
\DeclareMathOperator*{\argmin}{arg\,min}
\usepackage{wasysym}
\title{Теория и реализация языков программирования.\\Задание 9: преобразование контекстно-свободных языков}
\date{задано 2013.10.30}
\author{Сергей~Володин, 272 гр.}
\newcommand{\matrixl}{\left|\left|}
\newcommand{\matrixr}{\right|\right|}
% названия автоматов
\def\A{{\cal A}}
\def\B{{\cal B}}
\def\C{{\cal C}}

%+= и -=, иначе разъезжаются...
\newcommand{\peq}{\mathrel{+}=}
\newcommand{\meq}{\mathrel{-}=}
\newcommand{\deq}{\mathrel{:}=}
\newcommand{\plpl}{\mathrel{+}+}

% пустое слово
\def\eps{\varepsilon}

% регулярные языки
\def\REG{{\mathsf{REG}}}
\def\eqdef{\overset{\mbox{\tiny def}}{=}}
\newcommand{\niton}{\not\owns}

\begin{document}
\maketitle
\section*{Упражнение 1}
\section*{Упражнение 2}
\section*{Упражнение 3}
\section*{Упражнение 4}
\section*{Задача 1}
$L\eqdef\{xcy\big|x,y\in \{a,b\}^*,x\neq y^R\}\subset\Sigma\eqdef\{a,b,c\}$.
\begin{enumerate}[1.]
\item Определим МП-автомат $\A\eqdef(\Sigma,\Gamma,Q,q_0,Z,\delta,F)$, допускающий по принимающему состоянию:\newline
\begin{tabular}{ll}
\begin{minipage}{0.3\textwidth}
\begin{enumerate}[1.]
\item $\Gamma\eqdef\{a,b,A,B,Z\}$
\item $Q\eqdef\{q_0,q_1,q_2,q_3,q_4\}$
\item $\delta$ изображена справа
\item $F\eqdef\{q_1,q_2,q_4\}$
\end{enumerate}
\end{minipage}
&
\begin{minipage}{0.6\textwidth}
\begin{tikzpicture}[shorten >=1pt,node distance=2cm,on grid,auto,every node/.style={text centered},initial text=]
	\node [state,initial] (q_0)	{$q_0$};
	\node [state,accepting] (q_1) [right = 3.2cm of q_0 ] {$q_1$};
	\node [state] (q_3) [right = 2.5cm of q_1 ] {$q_3$};
	\node [state, accepting] (q_2) [below = 2cm of q_3 ] {$q_2$};
	\node [state, accepting] (q_4) [right = 2.5cm of q_3 ] {$q_4$};
	\path[->]
		(q_0) edge [out=40,in=140,loop] node [swap] {$\substack{a,Z/AZ\\a,A,aA\\a,a/aa\\a,B/aB\\a,b/ab}$} (q_0)
			  edge [out=-40,in=-140,loop] node {$\substack{b,Z/BZ\\B,A/bA\\b,a/ba\\b,B/bB\\b,b/bb}$} (q_0)
			  edge node [swap] {$\substack{c,A/A\\c,a/a\\c,B/B\\c,b/b}$} (q_1)
			  edge [out=20,in=160] node {$c,Z/Z$} (q_3)
		(q_1) edge [out=-80,in=-160,loop] node [below=0.07,swap] {$\substack{a,a/\varepsilon\\b,b/\varepsilon}$} (q_1)
			  edge [out=-70,in=110] node [below=-0.01,swap] {$\substack{a,B/b\\a,b/b\\b,A/a\\b,a/a}$} (q_2)
			  edge node [swap] {$\substack{a,A/\varepsilon\\b,B/\varepsilon}$} (q_3)
		(q_2) edge [out=-50,in=50,loop] node [swap] {$\substack{a,a/a\\b,b/b\\a,b/b\\b,a/a}$} (q_2)
		(q_3) edge node {$\substack{a,Z/Z\\b,Z/Z}$} (q_4)
		(q_4) edge [out=40,in=140,loop] node [swap] {$\substack{a,Z/Z\\b,Z/Z}$} (q_4)
		;
\end{tikzpicture}
\end{minipage}\\
\end{tabular}
\item $\A$~--- детерминированный, так как из каждой конфигурации $(q,w,\gamma)$ переход определен однозначно, и $\varepsilon$-переходов нет.
\item Определим $U\colon \{a,b\}\rightarrow \{A,B\}\colon U(a)=A,\,U(b)=B$. Определим $U_r\colon \{a,b\}^*\rightarrow \{a,b,A,B\}^*$:\newline
$U_r(w)=\begin{cases}
\varepsilon, & w=\varepsilon\\
w_1...w_{n-1}U(w_n), & w=w_1...w_n,\,\forall i\in\overline{1,n}\hookrightarrow w_i\in\{a,b\}
\end{cases}$~--- заменяет последний символ на заглавный.
\item Докажем, что $L\subseteq L(\A)$:\begin{enumerate}
\item \label{1.1adding} Пусть $w\in \{a,b\}^*$. Докажем, что $(q_0,w,Z)\vdash^*(q_0,\varepsilon,U_r(w^R)Z)$ индукцией по $|w|$:\newline
$P(n)\eqdef\big[\forall w\in\{a,b\}^*\colon |w|=n\hookrightarrow (q_0,w,Z)\vdash^*(q_0,\varepsilon,U_r(w^R)Z)\big]$\begin{enumerate}
\item $n=0\Rightarrow |w|=0\Rightarrow w=\varepsilon$. Тогда $U_r(w^R)\equiv\varepsilon$, и $(q_0,w,Z)\equiv(q_0,\varepsilon,Z)\equiv(q_0,\varepsilon,U_r(w^R)Z)\Rightarrow P(0)$.
\item $n=1\Rightarrow w=\sigma\in\Sigma$. Рассмотрим переходы из $(q_0,\sigma,Z)$. В стек будет добавлен $U_r(\sigma)$ $\Rightarrow$ $(q_0,w,Z)\equiv(q_0,\sigma,Z)\vdash(q_0,\varepsilon,U_r(\sigma)Z)\equiv(q_0,\varepsilon,U_r(w^R)Z)\Rightarrow P(1)$.
\item Фиксируем $n\geqslant 1$, пусть $\underline{P(n)}$. Пусть $w\in\{a,b\}^*,|w|=n+1$. Тогда $w=w_0\sigma,|w_0|=n>0$.\newline
$P(n)\Rightarrow (q_0,w_0,Z)\vdash^*(q_0,\varepsilon,U_r(w_0^R)Z)$. Тогда $(q_0,w,Z)\equiv(q_0,w_0\sigma,Z)\vdash^*(q_0,\sigma,U_r(w_0^R)Z).$\newline
$\varangle$ переходы из $(q_0,\sigma,U(w_0^R)Z)$. На верхушке стека $\gamma\in\{a,b,A,B\}$~--- первый символ $U_r(w_0^R)$,\newline
входной символ $\sigma\in\{a,b\}$. Во всех случаях он будет добавлен в стек (см. определение $\delta$), значит, $(q_0,\sigma,U_r(w_0^R)Z)\vdash(q_0,\varepsilon,\sigma U_r(w_0^R)Z)\overset{|w_0|>0}{=}(q_0,\varepsilon,U_r(w^R)Z)\Rightarrow \underline{P(n+1)}$.
\end{enumerate}
\item \label{1.1c} Из определения $\delta$ имеем $(q_0,cw,\gamma)\vdash^*(q_1,w,\gamma)$, $|\gamma|>0,\gamma\neq Z$.
\item \label{1.1removing} Докажем $(q_1,x,xZ)\vdash^*(q_1,\varepsilon,Z)$ индукцией по $|x|$: $P(n)\eqdef\big[\forall x\in\{a,b\}^*\colon |x|=n\hookrightarrow (q_1,x,xZ)\vdash^*(q_1,\varepsilon,Z)\big]$\begin{enumerate}
\item $n=0\Rightarrow |x|=0\Rightarrow x=\varepsilon$. Тогда $(q_1,x,xZ)\equiv (q_1,\varepsilon,Z)\Rightarrow P(0)$
\item Фиксируем $n\geqslant 0$. Пусть $\underline{P(n)}$. Пусть $x\in\{a,b\}^*\colon |x|=n+1\Rightarrow x=x_0\sigma,|x_0|=n\overset{P(n)}{\Rightarrow} (q_1,x_0,x_0Z)\vdash^*(q_1,\varepsilon,Z)$. Тогда $(q_1,x,xZ)\equiv(q_1,x_0\sigma,x_0\sigma Z)\vdash^*(q_1,\sigma,\sigma Z)$. Входной символ совпадает с символом на верхушке стека, из определения $\delta$ получаем, что символ будет удален из стека: $(q_1,\sigma,\sigma Z)\vdash(q_1,\varepsilon,Z)\Rightarrow P(n)$.
\end{enumerate}
\item \label{1.1q1q2} Пусть $\sigma_1,\sigma_2\in\{a,b\}$, $\sigma_1\neq\sigma_2$. Тогда $(q_1,\sigma_1,U_r(\sigma_2)\gamma)\vdash(q_2,\varepsilon,\sigma_2\gamma)$ и $(q_1,\sigma_1,\sigma_2\gamma)\vdash(q_2,\varepsilon,\sigma_2\gamma)$~--- из определения $\delta$.
\item \label{1.1q2eat} Пусть $x\in\{a,b\}^*$, $\gamma\in\{a,b\}$. Тогда $(q_2,x,\gamma\kappa)\vdash^*(q_2,\varepsilon,\gamma\kappa)$~--- доказывается очевидно по индукции (переходы из $q_2$ в $q_2$ определены для всех символов $a,b$ на входе и в стеке и не изменяют стек).
\item \label{1.1q1q3} Пусть $\sigma\in\{a,b\}$. Тогда $(q_1,\sigma,U_r(\sigma)\gamma)\vdash(q_3,\varepsilon,\gamma)$~--- из определения $\delta$.
\item \label{1.1q3q4} Пусть $\sigma\in\{a,b\}$. Тогда $(q_3,\sigma,Z)\vdash(q_4,\varepsilon,Z)$~--- из определения $\delta$
\item \label{1.1q4eat} Пусть $x\in\{a,b\}^*$. Тогда $(q_4,x,Z)\vdash^*(q_4,\varepsilon,Z)$~--- доказывается очевидно по индукции (из $q_4$ есть переходы в $q_4$ по $a$ и $b$ при $Z$ на верхушке стека)
\item \label{q1q3jump} Из определения $\delta$ имеем $(q_0,c,Z)\vdash(q_3,\varepsilon,Z)$.
\item Пусть $\underline{w\in L}\Rightarrow w=xcy,x\neq y^R;x,y\in\{a,b\}^*$. $x\neq y^R\Leftrightarrow x^R\neq y$. Выделим максимальную по длине общую часть $\tau$ длины $i$ у слов $x^R$ и $y$: $x^R=\tau x_1,y=\tau y_1$, $x_1\neq y_1$. Тогда $x=x_1^R\tau^R, w=xcy=x_1^R\tau^Rc\tau y_1$.\begin{enumerate} \item Пусть $|x_1|>0$. $(q_0,w,Z)\equiv(q_0,x_1^R\tau ^Rc\tau y_1,Z)\underset{|x_1|>0}{\overset{\ref{1.1adding}}{\vdash^*}}(q_0,c\tau y_1,U_r(\tau x_1)Z)\overset{\ref{1.1c}}{\vdash}(q_1,\tau y_1,U_r(\tau x_1)Z)\overset{|x_1|>0}{\equiv}\newline
\equiv(q_1,\tau y_1,\tau U_r(x_1)Z)\overset{\ref{1.1removing}}{\vdash^*}(q_1,y_1,U_r(x_1)Z)$.\begin{enumerate}
\item Пусть $|y_1|>0$, $x_1[1]\neq y_1[1]$. Обозначим $y_1=y^1...y^l,\,\forall i\in\overline{1,l}\hookrightarrow y^i\in \{a,b\}^*$
Тогда $(q_1,y_1,U_r(x_1)Z)\equiv(q_1,y^1...y^l,U_r(x_1)Z)\overset{\ref{1.1q1q2}}{\vdash}(q_2,y^2...y^l,U_r(x_1)Z)\overset{\ref{1.1q2eat}}{\vdash^*}(q_2,\varepsilon,U_r(x_1)Z)$. $q_2\in F\Rightarrow \underline{w\in L(\A)}$.
\item Пусть $|y_1|=0$. Тогда $w=x_1^R\tau^Rc\tau y_1\equiv x_1^R\tau^Rc\tau\Rightarrow (q_0,w,Z)\equiv(q_0,x_1^R\tau^Rc\tau,Z)\underset{|x_1|>0}{\overset{\ref{1.1adding}}{\vdash^*}}(q_0,c\tau,\tau U_r(x_1)Z)\underset{|x_1|>0}{\overset{\ref{1.1c}}{\vdash}}(q_1,\tau,\tau U_r(x_1)Z)\overset{\ref{1.1q2eat}}{\vdash^*}(q_1,\varepsilon,U_r(x_1)Z)$. $q_1\in F\Rightarrow\underline{w\in L(\A)}$
\end{enumerate}
\item Пусть $|x_1|=0$. Тогда $w=\tau^Rc\tau y_1,\,y_1\in\{a,b\}^*$. $x^R\neq y\Rightarrow \tau\neq\tau y_1\Rightarrow |y_1|>0\Rightarrow y_1=\varkappa\varPsi,\,\varkappa\in\{a,b\}$\begin{enumerate}
\item $|\tau|>0\Rightarrow \tau=\tau_0\sigma,\,\sigma\in\{a,b\}$. Получаем $(q_0,w,Z)\equiv(q_0,\tau^Rc\tau y_1,Z)\underset{|\tau|>0}{\overset{\ref{1.1adding}}{\vdash^*}}(q_0,c\tau y_1,U_r(\tau)Z)\underset{|\tau|>0}{\overset{\ref{1.1c}}{\vdash}}(q_1,\tau y_1,U_r(\tau)Z)\equiv(q_1,\tau_0\sigma y_1,\tau_0U_r(\sigma)Z)\overset{\ref{1.1removing}}{\vdash^*}(q_1,\sigma y_1,U_r(\sigma)Z)\overset{\ref{1.1q1q3}}{\vdash}(q_3,y_1,Z)\equiv(q_3,\varkappa\varPsi,Z)\overset{\ref{1.1q3q4}}{\vdash}(q_4,\varPsi,Z)\overset{\ref{1.1q4eat}}{\vdash^*}(q_4,\varepsilon,Z)$. $q_4\in F\Rightarrow \underline{w\in L(\A)}$
\item $|\tau|=0\Rightarrow w=x_1^R\tau^Rc\tau y_1\equiv cy_1\Rightarrow (q_0,w,Z)\equiv(q_0,cy_1,Z)\overset{\ref{q1q3jump}}{\vdash}(q_3,y_1,Z)\equiv(q_3,\varkappa\varPsi,Z)\overset{\ref{1.1q3q4}}{\vdash}(q_4,\varPsi,Z)\overset{\ref{1.1q4eat}}{\vdash^*}(q_4,\varepsilon,Z)$. $q_4\in F\Rightarrow \underline{w\in L(\A)}$
\end{enumerate}
\end{enumerate}
\end{enumerate}
\item Докажем, что $L(\A)\subseteq L$. Пусть $w\in L(\A)\Rightarrow (q_0,w,Z)\vdash^*(q,\varepsilon,\gamma)$, $q\in F$:\begin{enumerate}
\item \label{1.2q1} $q=q_1$. В $q_1$ прочитываются $a,b$. Переходы в $q_1$ есть только из $q_0$ по $c$. В $q_0$ прочитываются символы $a,b$. Значит, $w=xcy,\,x,y\in\{a,b\}^*$. Если $x=\varepsilon$, то был совершен переход $q_0\overset{c,Z/Z}{\longrightarrow}q_3$~--- противоречие. Автомат детерминированный, поэтому цепочка конфигураций при выводе $w$ имеет вид $(q_0,w,Z)\equiv(q_0,xcy,Z)\underset{|x|>0}{\overset{\ref{1.1adding}}{\vdash}}(q_0,cy,U_r(x^R)Z)\underset{|x|>0}{\overset{\ref{1.1c}}{\vdash}}(q_1,y,U_r(x^R)Z)\boxed{=}$. Выделим максимальную общую часть от начала для слов $x^R$ и $y$: $x^R=\tau x_1$, $y=\tau y_1$, $x_1\neq y_1$.\begin{enumerate}
\item $|\tau|=0,|x_1|=0\Rightarrow |x|=0$~--- \underline{противоречие}
\item $|\tau|>0,|x_1|=0\Rightarrow\tau=\tau_0\sigma,\,\sigma\in\{a,b\}$. $\boxed{=}(q_1,\tau_0\sigma y_1,\tau_0 U_r(\sigma)Z)\overset{\ref{1.1removing}}{\vdash^*}(q_1,\sigma y_1,U_r(\sigma)Z)\overset{\ref{1.1q1q3}}{\vdash}(q_3,...)$~--- \underline{противоречие}, из $q_3$ нет переходов в $q_1$.
\item $|\tau|\geqslant 0,|x_1|>0$. Тогда $\boxed{=}(q_1,\tau y_1,\tau U_r(x_1)Z)\overset{\ref{1.1removing}}{\vdash^*}(q_1,y_1,U_r(x_1)Z)\boxed{=_1}$.\begin{enumerate}[a.]
\item $|y_1|=0\Rightarrow\boxed{=_1}(q_1,\varepsilon,U_r(x_1))$. Тогда $w=\underbrace{x_1^R\tau^R}_{x}c\underbrace{\tau y_1^R}_y,\,x^R=\tau x_1\underset{|x_1|>0}{\neq}\tau=y\Rightarrow\underline{w\in L}$.
\item $|y_1|>0$. Тогда $x_1[1]\neq y_1[1]$, и $\boxed{=_1}(q_1,y_1,U_r(x_1)Z)\underset{x_1[1]\neq y_1[1]}{\overset{\ref{1.1q1q2}}{\vdash}}(q_3,...)$~--- \underline{противоречие}, из $q_3$ нет переходов в $q_1$.
\end{enumerate}
\end{enumerate}
\item \label{1.2q2} $q=q_2$. В $q_2$ есть переходы только из $q_1$, в $q_2$ прочитываются $a,b$. $\ref{1.2q1}\Rightarrow w=xcy,\,|x|\neq 0,\,x,y\in\{a,b\}^*$. При переходе в $q_2$ прочитывается символ, поэтому $|y|>0$. Аналогично $\ref{1.2q1}$ выделим общую часть $x^R=\tau x_1,y=\tau y_1$. Аналогично $\ref{1.2q1}$ ($|x|>0$) получаем $(q_0,w,Z)\vdash^*(q_1,\tau y_1,U_r(\tau x_1)Z)\boxed{=}$. Рассмотрим случаи:\begin{enumerate}
\item $|\tau|=0,|x_1|=0\Rightarrow|x|=0$~--- \underline{противоречие}
\item $|\tau|>0,|x_1|=0\Rightarrow\tau=\tau_0\sigma,\,\sigma\in\{a,b\}$. $\boxed{=}(q_1,\tau_0\sigma y_1,\tau_0 U_r(\sigma)Z)\overset{\ref{1.1removing}}{\vdash^*}(q_1,\sigma y_1,U_r(\sigma)Z)\overset{\ref{1.1q1q3}}{\vdash}(q_3,...)$~--- \underline{противоречие}, из $q_3$ нет переходов в $q_2$.
\item $|\tau|\geqslant 0,|x_1|>0$. Тогда $\boxed{=}(q_1,\tau y_1,\tau U_r(x_1)Z)\overset{\ref{1.1removing}}{\vdash^*}(q_1,y_1,U_r(x_1)Z)\boxed{=_1}$.\begin{enumerate}[a.]
\item $|y_1|=0\Rightarrow\boxed{=_1}(q_1,\varepsilon,U_r(x_1))$. В $\ref{1.2q1}$ было показано, что автомат остановится в $q_1$~--- \underline{противоречие}.
\item $|y_1|>0$. Тогда $x_1[1]\neq y_1[1]$. Обозначим $x_1=\sigma_1 x_1^0,\,y_1=\sigma_2 y^0_1$, и $\boxed{=_1}(q_1,\sigma_1 y^0_1,U_r(\sigma_2 x^0_1)Z)\underset{x_1[1]\neq y_1[1]}{\overset{\ref{1.1q1q2}}{\vdash}}\newline(q_3,y^0_1,U_r(x^0_1)Z)\overset{\ref{1.1q2eat}}{\vdash^*}(q_3,\varepsilon,U_r(x^0_1)Z)$ {\em (последние переходы возможны только при $x^0_1\neq\varepsilon$)}.\newline
Получаем $x_1\neq y_1\Rightarrow x^R\neq y\Rightarrow \underline{w\in L}$.
\end{enumerate}
\end{enumerate}
\item $q=q_4$. В $q_4$ прочитываются $a$, $b$; в $q_4$ есть переходы только из $q_3\overset{\substack{a,Z/Z\\b,Z/Z}}{\longrightarrow} q_4$, в $q_3$ есть переходы только из $p\in\{q_0,q_1\}$. Рассмотрим случаи:\begin{enumerate}
\item $p=q_0$. Если в $q_0$ были прочитаны символы из $\{a,b\}$, то на верхушке стека не $Z\Rightarrow$ переход в $q_3$ не мог быть совершен. Получаем, что $w=cy,\,y\in\{a,b\}^*$. Но при переходе в $q_4$ из $q_3$ прочитывается хотя бы один символ, поэтому $|y|>0\Rightarrow \underline{w\in L}$
\item $p=q_1$. $\ref{1.2q2}\Rightarrow w=xcy,\,|x|>0,\,x,y\in\{a,b\}^*$. Аналогично $\ref{1.2q2}$ разобьем $x^R=\tau x_1,\,y=\tau y_1$, рассмотрим случаи:\begin{enumerate}
\item $|\tau|=0,|x_1|=0\Rightarrow|x|=0$~--- \underline{противоречие}
\item $|\tau|>0,|x_1|=0\Rightarrow\tau=\tau_0\sigma,\,\sigma\in\{a,b\}$. Аналогочино $\ref{1.2q2}$ получим $(q_0,w,Z)\vdash^*(q_1,\tau_0\sigma y_1,\tau_0 U_r(\sigma)Z)\overset{\ref{1.1removing}}{\vdash^*}(q_1,\sigma y_1,U_r(\sigma)Z)\overset{\ref{1.1q1q3}}{\vdash}(q_3,y_1,Z)$. При переходе из $q_3$ в $q_4$ был прочитан символ, поэтому $|y_1|>0$. Имеем $x^R\equiv\tau x_1\equiv\tau\underset{|y_1|>0}{\neq}\tau y_1\equiv y\Rightarrow\underline{w\in L}$.
\item $|\tau|\geqslant 0,|x_1|>0$.\begin{enumerate}[a.]
\item $|y_1|=0$. В $\ref{1.2q1}$ было показано, что автомат остановится в $q_1$~--- \underline{противоречие}.
\item $|y_1|>0$. В $\ref{1.2q2}$ было показано, что автомат остановится в $q_2$~--- \underline{противоречие}.
\end{enumerate}
\end{enumerate}
\end{enumerate}
\end{enumerate}
\end{enumerate}
%\begin{tikzpicture}[shorten >=1pt,node distance=2cm,on grid,auto,every node/.style={text centered},initial text=]
%	\node [state,initial,accepting] (q_0)	{$q_0$}; 
%	\node [state] (q_1) [right = 2.5cm of q_0 ] {$q_1$};
%	\path[->]
%		(q_0) edge node [swap] {$\varepsilon,Z_0/Z_0$} (q_1);
%\end{tikzpicture}
%{\fontsize{400}{1} df}
\section*{Задача 2}
\section*{Задача 3}
$\Sigma\eqdef\{a,b\}$, $\Gamma\eqdef(N,\Sigma,P,S)$. $N\eqdef\{A,B,C,D,E,F,G\}$ $P:$
\vspace{-7ex}
\begin{multicols}{2}
	\begin{align*}
		S &\to A|B|C|E|AG \\
		A &\to C |aABC|\eps\\
		B &\to bABa|aCbDaGb|\eps\\
	\end{align*}

	\begin{align*}
		C &\to BaAbC|aGD|\eps\\
		F &\to aBaaCbA|aGE\\
		E &\to A\\			
	\end{align*}	
\end{multicols}
\vspace{-7ex}
\begin{enumerate}
\item Удалим бесплодные символы (для упрощения):\begin{enumerate}
\item $V_0\eqdef \{a,b\}$
\item $V_1=V_0\cup\{A,B,C\}=\{a,b,A,B,C\}$
\item $V_2=V_1\cup\{S,F,E\}=\{a,b,S,A,B,C,F,E\}$
\item $V_3=V_2\cup\varnothing$
\end{enumerate}
Тогда $V_3\setminus\Sigma=\{S,A,B,C,F,E\}$. Удалим нетерминалы $N\setminus V_3=\{D,G\}$ и правила, их содержащие: $N'\eqdef N\setminus V_3=\{S,A,B,C,F,E\}$, $P'$:
\vspace{-7ex}
\begin{multicols}{2}
	\begin{align*}
		S &\to A|B|C|E|\cancel{AG} \\
		A &\to C |aABC|\eps\\
		B &\to bABa|\cancel{aCbDaGb}|\eps\\
	\end{align*}

	\begin{align*}
		C &\to BaAbC|\cancel{aGD}|\eps\\
		F &\to aBaaCbA|\cancel{aGE}\\
		E &\to A\\			
	\end{align*}	
\end{multicols}
\vspace{-7ex}
\item Удалим недостижимые символы (для упрощения):\begin{enumerate}
\item $V_0\eqdef\{S\}$
\item $V_1=V_0\cup\{A,B,C,E\}$
\item $V_2=V_1\cup\varnothing$
\end{enumerate}
$N''\eqdef\{A,B,C,E,S\}$, $P''$:
\vspace{-7ex}
\begin{multicols}{2}
	\begin{align*}
		S &\to A|B|C|E|\cancel{AG} \\
		A &\to C |aABC|\eps\\
		B &\to bABa|\cancel{aCbDaGb}|\eps\\
	\end{align*}

	\begin{align*}
		C &\to BaAbC|\cancel{aGD}|\eps\\
		F &\to \cancel{aBaaCbA|aGE}\\
		E &\to A\\			
	\end{align*}	
\end{multicols}
\vspace{-7ex}
\item[1,2.] Имеем $P''$:
\vspace{-7ex}
\begin{multicols}{2}
	\begin{align*}
		S &\to A|B|C|E\\
		A &\to C |aABC|\eps\\
		B &\to bABa|\eps\\
	\end{align*}

	\begin{align*}
		C &\to BaAbC|\eps\\
		E &\to A\\			
	\end{align*}	
\end{multicols}
\vspace{-7ex}
\item Удалим $\varepsilon$-правила:\begin{enumerate}
\item $A,B,C$~--- $\varepsilon$-порождающие.
\item $S,E$~--- $\varepsilon$-порождающие ($S\rightarrow A$, $E\rightarrow A$)
\end{enumerate}
Перепишем правила, содержащие $\varepsilon$-порождающие нетерминалы справа ($2^k$ правил для каждого правила, содержащего $k$ $\varepsilon$-порождающих нетерминалов). $P'''$:
\vspace{-7ex}
\begin{multicols}{2}
	\begin{align*}
		S &\to A|B|C|E\\
		A &\to C|a|aC|aB|aBC|aA|aAC|aAB|aABC\\
		B &\to ba|bBa|bAa|bABa\\
	\end{align*}

	\begin{align*}
		C &\to ab|abC|aAb|aAbC|Bab|BabC|BaAbC\\
		E &\to A\\			
	\end{align*}	
\end{multicols}
\vspace{-7ex}
Грамматика с такими правилами порождает язык $L(\Gamma)\setminus\{\varepsilon\}$.
\item Найдем цепные пары (множества пар соответствуют добавлениям на шагах алгоритма):\begin{enumerate}
\item $(S,S),(A,A),(B,B),(C,C),(E,E)$
\item $(S,A),(S,B),(S,C),(S,E);(A,C);(E,A)$
\item $(S,C);\cancel{(S,A)};(E,C)$
\end{enumerate}
\item Выпишем новое множество правил $P''''$:\newline
$\begin{array}{|l|lll|}
\hline
\mbox{Цепная пара} & \multicolumn{3}{c|}{\mbox{Правила}}\\\hline
(S,S) & \multicolumn{3}{c|}{\varnothing}\\\hline
(A,A) & A & \to & a|aC|aB|aBC|aA|aAC|aAB|aABC\\\hline
(B,B) & B & \to & ba|bBa|bAa|bABa\\\hline
(C,C) & C & \to & ab|abC|aAb|aAbC|Bab|BabC|BaAbC\\\hline
(E,E) & \multicolumn{3}{c|}{\varnothing}\\\hline
(S,A) & S & \to & a|aC|aB|aBC|aA|aAC|aAB|aABC\\\hline
(S,B) & S & \to & ba|bBa|bAa|bABa\\\hline
(S,C) & S & \to & ab|abC|aAb|aAbC|Bab|BabC|BaAbC\\\hline
(S,E) & \multicolumn{3}{c|}{\varnothing}\\\hline
(A,C) & A & \to & ab|abC|aAb|aAbC|Bab|BabC|BaAbC\\\hline
(E,A) & E & \to & a|aC|aB|aBC|aA|aAC|aAB|aABC\\\hline
(S,C) & S & \to & ab|abC|aAb|aAbC|Bab|BabC|BaAbC\\\hline
(E,C) & E & \to & ab|abC|aAb|aAbC|Bab|BabC|BaAbC\\\hline
\end{array}$
\item Нетерминалы $A,B,C,E,S$ не являются бесплодными: $A\to a,B\to ba,C\to ab,E\to a,S\to ab$.
\item Удалим недостижимые:\begin{enumerate}
\item $V_0\eqdef\{S\}$
\item $V_1\eqdef\{S,A,B,C\}$
\item $V_2=V_1$
\end{enumerate}
Удаляем $E$. $P^{(5)}$:
\vspace{-7ex}
\begin{multicols}{2}
	\begin{align*}
		A &\to a|aC|aB|aBC|aA|aAC|aAB|aABC|ab|abC|aAb|aAbC|Bab|BabC|BaAbC\\
		B &\to ba|bBa|bAa|bABa\\
		C &\to ab|abC|aAb|aAbC|Bab|BabC|BaAbC\\
		S &\to a|aC|aB|aBC|aA|aAC|aAB|aABC|ba|bBa|bAa|bABa|ab|abC|aAb|aAbC|Bab|BabC|BaAbC\\
	\end{align*}	
\end{multicols}
\vspace{-7ex}
\item Приведем к нормальной форме Хомского. Добавим нетерминалы $A',B'$, $A'\to a,B'\to b$. Заменим в правилах $a$ на $A'$, $b$ на $B'$. Подчеркнем слова из нетерминалов длины 2 в правых частях правил, которые заменим на новые нетерминалы:
\vspace{-7ex}
\begin{multicols}{2}
	\begin{align*}
		A &\to a|A'C|A'B|\underline{A'B}C|A'A|\underline{A'A}C|\underline{A'A}B|\underline{A'A}\,\underline{BC}|A'B'|\underline{A'B'}C|\underline{A'A}B'|\underline{A'A}\,\underline{B'C}|\underline{BA'}B'|\underline{BA'}\,\underline{B'C}|\underline{BA'}\,\underline{AB'}C\\
		B &\to B'A'|\underline{B'B}A'|\underline{B'A}A'|\underline{B'A}\,\underline{BA'}\\
		C &\to A'B'|\underline{A'B'}C|\underline{A'A}B'|\underline{A'A}\,\underline{B'C}|\underline{BA'}B'|\underline{BA'}\,\underline{B'C}|\underline{BA'}\,\underline{AB'}\,C\\
		S &\to a|A'C|A'B|\underline{A'B}C|A'A|\underline{A'A}C|\underline{A'A}B|\underline{A'A}\,\underline{BC}|B'A'|\underline{B'B}A'\\
		S &\to \underline{B'A}A'|\underline{B'A}\,\underline{BA'}|A'B'|\underline{A'B'}\,C|\underline{A'A}B'|\underline{A'A}\,\underline{B'C}|\underline{BA'}B'|\underline{BA'}\,\underline{B'C}|\underline{BA'}\,\underline{AB'}C\\			
		A' &\to a\\
		B' &\to b\\
	\end{align*}	
\end{multicols}
\vspace{-7ex}
Заменим подчеркнутые слова на новые нетерминалы:
\vspace{-6ex}
\begin{multicols}{2}
	\begin{align*}
		A &\to a|A'C|A'B|X_0C|A'A|X_1C|X_1B|X_1X_2|A'B'|X_3C|X_1B'|X_1X_4|X_5B'|X_5X_4|X_9C\\
		B &\to B'A'|X_7A'|X_8A'|X_8X_5\\
		C &\to A'B'|X_3C|X_1B'|X_1X_4|X_5B'|X_5X_4|X_9C\\
		S &\to a|A'C|A'B|X_0C|A'A|X_1C|X_1B|X_1X_2|B'A'|X_7A'|X_8A'|X_8X_5|A'B'|X_3C|X_1B'|X_1X_4|X_5B'|X_5X_4|X_9C\\			
	\end{align*}
\end{multicols}
\vspace{-6ex}
\vspace{-6ex}
\begin{multicols}{3}
	\begin{align*}
		A' &\to a\\
		B' &\to b\\
		X_0 &\to A'B\\
		X_1 &\to A'A\\
	\end{align*}	
	
	\begin{align*}
		X_2 &\to BC\\
		X_3 &\to A'B'\\
		X_4 &\to B'C\\
		X_5 &\to BA'\\		
	\end{align*}	
	
	\begin{align*}
		X_6 &\to AB'\\
		X_7 &\to B'B\\
		X_8 &\to B'A\\
		X_9 &\to X_5X_6\\
	\end{align*}	
\end{multicols}
\vspace{-7ex}
\end{enumerate}
\section*{Задача 4}
\section*{Задача 5}
$\Sigma_2\eqdef\{[_1,[_2\},\,\overline{\Sigma}_2\eqdef\{]_1,]_2\}$. $D_2\overset{\mbox{\tiny def}}{\mbox{ --- }}$ язык ПСП над $\Sigma\eqdef \Sigma_2\cup\overline{\Sigma}_2$. $\Delta\eqdef\{a,b\}$. $\varphi\colon \Sigma^*\longrightarrow \Delta^*$, $\varphi([_1)\eqdef a$, $\varphi([_2)\eqdef b$, $\varphi(]_1)\eqdef b$, $\varphi(]_2)\eqdef a$. Доопределим $\varphi$ до морфизма (см. решение упр. 2 из задания 3). $L\eqdef \varphi(D_2\cap \Sigma^*)\equiv \varphi(D_2)$.
\begin{enumerate}
\item Докажем, что $L\subseteq L'$. Пусть $\underline{y\in L}\equiv \varphi(D_2)$. Тогда $\exists x\in D_2\colon y=\varphi(x)$. $x$~--- ПСП $\Rightarrow$ $\forall i\in\overline{1,2}\hookrightarrow |x|_{[_i}=|x|_{]_i}$. Сложим равенства, получим: $|x|_{[_1}+|x|_{]_2}=|x|_{]_1}+|x|_{[_2}$. Пусть $x=x_1...x_m,\,\forall i\in\overline{1,m}\hookrightarrow x_i\in\Sigma$. Тогда $y=\varphi(x)=\varphi(x_1)...\varphi(x_m)=y_1...y_m,\,\forall i\in\overline{1,m}\hookrightarrow y_i=\varphi(x_i)\in\Delta$. Но из определения $\varphi$ имеем $[_1,]_2\overset{\varphi}{\rightarrow}a;\,]_1,[_2\overset{\varphi}{\rightarrow}b$. Тогда $|y|_a=|x|_{[_1}+|x|_{]_2}\equiv|x|_{]_1}+|x|_{[_2}=|y|_b\Rightarrow \underline{y\in L'}$ $\blacksquare$
\item Докажем, что $L'\subseteq L$ индукцией по длине $y\in L'$: $P(n)\eqdef\big[\forall y\in L'\colon |y|\leqslant n\hookrightarrow y\in L\big]$.\newline
Заметим, что $y\in L\Leftrightarrow y\in \varphi(D_2)\Leftrightarrow \varphi^{-1}(y)\cap D_2\neq\varnothing$. Поэтому будем искать прообраз слова $y$, принадлежащий $D_2$.\begin{enumerate}
\item $n=0\Rightarrow |y|=0\Rightarrow y=\varepsilon\in L'$. Пусть $x\eqdef\varepsilon\in D_2$ (так как пустое слово~--- ПСП). Тогда $y=\varepsilon\equiv\varphi(x)\Rightarrow y\in \varphi(D_2)\equiv L\Rightarrow P(0)$
\item Фиксируем $n>0$. Пусть $P(n-1)$. Пусть $y\in L'\colon |y|=n$. Поскольку $|y|=n>0$, и $|y|$~--- четно (см. решение задачи 3 из задания 6), то $|y|\geqslant 2$. Рассмотрим первый и последний символы $\sigma_l$ и $\sigma_r$ слова $y\equiv \sigma_ly_1\sigma_r$:\begin{enumerate}
\item $\sigma_l=a,\,\sigma_r=b$. Тогда $y=ay_1b$. $|y_1|=n-2\leqslant n-1\overset{P(n-1)}{\Rightarrow}\exists x_1\in D_2\colon \varphi(x_1)=y_1$. Определим $x=[_1x_1]_1$. $x_1\in D_2\Rightarrow x_1$~--- ПСП $\Rightarrow x$~--- ПСП, так как получен из ПСП добавленим скобок типа $1$ слева и справа $\Rightarrow x\in D_2$. Но $\varphi(x)\equiv\varphi([_1x_1]_1)=\varphi([_1)\varphi(x_1)\varphi(]_1)=ay_1b\equiv y$. Получаем $\varphi^{-1}(y)\cap D_2\ni x\Rightarrow \varphi^{-1}(y)\cap D_2\neq\varnothing$.
\item $\sigma_l=b,\,\sigma_r=b$. Тогда $y=by_1a$. $|y_1|=n-2\leqslant n-1\overset{P(n-1)}{\Rightarrow}\exists x_1\in D_2\colon \varphi(x_1)=y_1$. Определим $x=[_2x_1]_2$. $x_1\in D_2\Rightarrow x_1$~--- ПСП $\Rightarrow x$~--- ПСП, так как получен из ПСП добавленим скобок типа $2$ слева и справа $\Rightarrow x\in D_2$. Но $\varphi(x)\equiv\varphi([_2x_1]_2)=\varphi([_2)\varphi(x_1)\varphi(]_2)=by_1a\equiv y$. Получаем $\varphi^{-1}(y)\cap D_2\ni x\Rightarrow \varphi^{-1}(y)\cap D_2\neq\varnothing$.
\item $\sigma_l=\sigma_r$. Тогда $y=\sigma y_1\sigma\in L'$. Воспользуемся утверждением в рамочке из решения задачи 3 задания 6:
$$\boxed{y=\sigma y_1\sigma\in L'\Rightarrow\exists y_l,y_r\colon y=y_ly_r, |y_l|, |y_r|\in\overline{1, |y|-2}, y_l,y_r\in L'}$$
Но $|y_l|,|y_r|\leqslant |y|-2=n-2\leqslant n-1\overset{P(n-1)}{\Rightarrow}\exists x_l,x_r\in D_2\colon y_l=\varphi(x_l),y_r=\varphi(x_r)$. Определим $x\eqdef x_lx_r$. Тогда $x\in D_2$ (конкатенация ПСП~--- ПСП), и $\varphi(x)=\varphi(x_lx_r)=\varphi(x_l)\varphi(x_r)=y_ly_r=y\Rightarrow \varphi^{-1}(y)\cap D_2\ni x\Rightarrow\varphi^{-1}(y)\cap D_2\neq\varnothing$
\end{enumerate}
\end{enumerate}
$\blacksquare$
\end{enumerate}
\section*{Задача 6}
Автомат $\A=(\Sigma,\Gamma,Q,q_0,Z_0,\delta,\varnothing)$ из $7$-го задания:\newline
\begin{tabular}{cc}
\begin{minipage}{0.3\textwidth}
\begin{enumerate}
\item $\Sigma=\{a,b\}$
\item $\Gamma=\{a,Z\}$
\item $Q=\{q_0,q_1\}$
\item $\delta$ изображена справа
\end{enumerate}
\end{minipage}
&
\begin{minipage}{0.3\textwidth}
\begin{tikzpicture}[shorten >=1pt,node distance=2cm,on grid,auto,every node/.style={text centered},initial text=]
	\node [state,initial] (q_0)	{$q_0$};
	\node [state] (q_1) [right = 4cm of q_0 ] {$q_1$};
	\path[->]
		(q_0) edge [in=30,out=150,loop] node {$a,Z_0/aZ_0\  a,a/aa$} (q_0)
			  edge node {$b,a/\eps$} (q_1)			
		(q_1) edge [in=30,out=150,loop] node {$b,a/\eps\ \eps,Z_0/\eps$} (q_1);
\end{tikzpicture}
\end{minipage}\\
\end{tabular}
\newline
Определим грамматику $G=(N,\Sigma,P,S)$. $N=\{S\}\cup\{[qZp]\big|q,p\in Q,Z\in \Gamma\}$
\begin{enumerate}
\item Добавим правила $S\to [q_0Z_0q_0]|[q_0Z_0q_1]$
\item Рассмотрим переходы из $\delta$, добавим правила\begin{enumerate}
\item $\delta\ni q_0\overset{a,Z_0/aZ_0}{\longrightarrow}q_0$: $[q_0Z_0q_0]\to a[q_0aq_0][q_0Z_0q_0]\big|a[q_0aq_1][q_1Z_0q_0]$, $[q_0Z_0q_1]\to a[q_0aq_0][q_0Z_0q_1]\big|a[q_0aq_1][q_1Z_0q_1]$
\item $\delta\ni q_0\overset{a,a/aa}{\longrightarrow}q_0$: $[q_0aq_0]\to a[q_0aq_0][q_0aq_0]\big|a[q_0aq_1][q_1aq_0]$, $[q_0aq_1]\to a[q_0aq_0][q_0aq_1]\big|a[q_0aq_1][q_1aq_1]$
\item $\delta\ni q_0\overset{b,a/\varepsilon}{\longrightarrow}q_1$: $[q_0aq_1]\to b$
\item $\delta\ni q_1\overset{b,a/\varepsilon}{\longrightarrow}q_1$: $[q_1aq_1]\to b$
\item $\delta\ni q_0\overset{\varepsilon,Z_0/\varepsilon}{\longrightarrow}q_1$: $[q_1Z_0q_1]\to \varepsilon$
\end{enumerate}
\item Удалим бесплодные нетерминалы:\begin{enumerate}
\item $V_0=\{a,b\}$
\item $V_1=V_0\cup\{[q_0aq_1],[q_1aq_1],[q_1Z_0q_1]\}$
\item $V_2=V_1\cup\{[q_0Z_0q_1]\}$
\item $V_3=V_2\cup\{S\}$
\item $V_4=V_3$.
\end{enumerate}
Имеем правила $S\to [q_0Z_0q_1]$, $[q_0Z_0q_1]\to a[q_0aq_1][q_1Z_0q_1]$,  $[q_0aq_1]\to a[q_0aq_1][q_1aq_1]\big|b$, $[q_1aq_1]\to b$, $[q_1Z_0q_1]\to \varepsilon$
\item Удалим недостижимые нетерминалы:\begin{enumerate}
\item $V_0=\{S\}$
\item $V_1=V_0\cup\{[q_0Z_0q_1]\}$
\item $V_2=V_1\cup\{[q_0aq_1],[q_1Z_0q_1]\}$
\item $V_3=V_2\cup\{[q_1aq_1]\}$
\item $V_4=V_3$
\end{enumerate}
(все достижимы)
\item Переобозначим: $$S\to \underbrace{[q_0Z_0q_1]}_A, \underbrace{[q_0Z_0q_1]}_A\to a\underbrace{[q_0aq_1]}_B\underbrace{[q_1Z_0q_1]}_C, \underbrace{[q_0aq_1]}_B\to a\underbrace{[q_0aq_1]}_B\underbrace{[q_1aq_1]}_D\big|b, \underbrace{[q_1aq_1]}_D\to b, \underbrace{[q_1Z_0q_1]}_C\to \varepsilon,$$
получим $$S\to A, A\to aBC, B\to aBD|b, D\to b, C\to \varepsilon$$
Из $D,C$ есть только правила $D\to b,C\to\varepsilon$, поэтому они раскрываются единственным образом. Уберем их, получим грамматику $G'$, причем $G'$~--- однозначная $\Leftrightarrow$ $G$~--- однозначная:
$$S\to A, A\to aB, B\to aBb|b$$
Аналогично для $S\to A$ (раскрывается единственным образом). Получим $G''\colon$ $G''$~--- однозначная $\Leftrightarrow$ $G$~--- однозначная:
$$S\to aB, B\to aBb|b$$
После примения правила $B\to b$ нельзя применить правило $B\to aBb$, и каждое применение $B\to aBb$ увеличивает количество символов $a$ и $b$ на 1. Поэтому количество его применений фиксировано. Отсюда получаем, что грамматика $G''$~--- однозначная $\Rightarrow$ $G'$~--- однозначная $\Rightarrow$ $G$~--- однозначная.
\\[5pt]
\end{enumerate}
Ответ:\begin{enumerate}
\item $G\colon S\to \underbrace{[q_0Z_0q_1]}_A, \underbrace{[q_0Z_0q_1]}_A\to a\underbrace{[q_0aq_1]}_B\underbrace{[q_1Z_0q_1]}_C, \underbrace{[q_0aq_1]}_B\to a\underbrace{[q_0aq_1]}_B\underbrace{[q_1aq_1]}_D\big|b, \underbrace{[q_1aq_1]}_D\to b, \underbrace{[q_1Z_0q_1]}_C\to \varepsilon$, после переобозначения $S\to A, A\to aBC, B\to aBD|b, D\to b, C\to \varepsilon$
\item $G$~--- однозначная.
\end{enumerate}
\end{document}