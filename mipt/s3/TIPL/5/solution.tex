\documentclass[a4paper]{article}
\usepackage[a4paper, left=5mm, right=5mm, top=5mm, bottom=5mm]{geometry}
%\geometry{paperwidth=210mm, paperheight=2000pt, left=5pt, top=5pt}
\usepackage[utf8]{inputenc}
\usepackage[english,russian]{babel}
\usepackage{indentfirst}
\usepackage{tikz} %Рисование автоматов
\usetikzlibrary{automata,positioning,arrows}
\usepackage{amsmath}
\usepackage{enumerate}
\usepackage{amsfonts}
\usepackage{amssymb}
\usepackage{wasysym}
\title{Теория и реализация языков программирования.\\Задание 5: Регулярные грамматики}
\date{задано 2013.10.02}
\author{Сергей~Володин, 272 гр.}
\newcommand{\matrixl}{\left|\left|}
\newcommand{\matrixr}{\right|\right|}
% названия автоматов
\def\A{{\cal A}}
\def\B{{\cal B}}
\def\C{{\cal C}}

%+= и -=, иначе разъезжаются...
\newcommand{\peq}{\mathrel{+}=}
\newcommand{\meq}{\mathrel{-}=}
\newcommand{\deq}{\mathrel{:}=}
\newcommand{\plpl}{\mathrel{+}+}

% регулярные языки
\def\REG{{\mathsf{REG}}}
\def\eqdef{\overset{\mbox{\tiny def}}{=}}
\def\taskoneautomataB{\begin{tikzpicture}[shorten >=1pt,node distance=2cm,on grid,auto,initial text=]
	\node[state, initial, accepting]	(Q_0)							{$Q_0$};
	\node[state, accepting]	(Q_1) [above right = of Q_0]	{$Q_1$};
	\node[state] 	(Q_2) [below right = of Q_0]			{$Q_2$};
	\node[state, accepting]	(Q_3) [right = of Q_1]			{$Q_3$};
	\node[state, accepting]	(Q_4) [right = of Q_2]	{$Q_4$};
	\path[->]
		(Q_0)	edge	node {$a$}	(Q_1)
				edge	node {$b$}	(Q_2)
		(Q_1)	edge  [loop below]	node[swap] {$a$} (Q_1)
				edge	node {$b$}		(Q_3)
		(Q_2)	edge	node {$b$}		(Q_4)
		(Q_3)	edge [bend left=10]	node {$a$}		(Q_0)
				edge node {$b$}		(Q_4)
		(Q_4)	edge [bend right=10]	node[swap] {$a$}		(Q_0);
\end{tikzpicture}}
\newcommand{\niton}{\not\owns}

\begin{document}
\maketitle
\section*{Упражнение 1}
\begin{enumerate}
\item Докажем, что $\forall G_1$~--- ПРГ $\exists G_2$~--- ПГ$\colon L(G_1)=L(G_2)$: каждая ПРГ является ПГ, так как $T\subset T^*$, $\varepsilon\in T^*$. Поэтому можно взять $G_2\eqdef G_1$.
\item Докажем, что $\forall G_2$~--- ПГ $\exists G_1$~--- ПРГ$\colon L(G_1)=L(G_2)$:\begin{enumerate}[1.]
\item Построим по $G_2$ НКА $\A\colon L(\A)=L(G_2)$ (алгоритм и доказательство в задаче 2).
\item По НКА $\A$ построим ПРГ $G_1\colon L(G_1)=L(\A)$ (алгоритм и доказательство в задаче 1).
\end{enumerate}
\end{enumerate}
Это доказывает, что ПРГ и ПГ порождают одно и то же множество языков.
\section*{Упражнение 2}
\section*{Задача 1}
\begin{tabular}{ll}
\begin{minipage}{0.6\textwidth}
Нет, предложенный алгоритм может построить грамматику, которая не будет праволинейной регулярной. Например, для автомата $\A_0$ из условия переход $q_0\overset{\varepsilon}{\longrightarrow}q_1$ по алгоритму должен соответствовать правилу $q_0\longrightarrow\varepsilon q_1$, но это правило не имеет вид $A\longrightarrow xB$ ($\varepsilon=x\notin \Sigma$) или $A\longrightarrow x$ или $A\longrightarrow \varepsilon$.
\end{minipage}
\begin{minipage}{0.3\textwidth}
\begin{tikzpicture}[shorten >=1pt,node distance=2cm,on grid,auto,initial text=]
	\node[state, initial]	(q_0)							{$q_0$};
	\node[state] 		  	(q_1) [above right = of q_0]	{$q_1$};
	\node[state, accepting] 	(q_2) [right = of q_1]			{$q_2$};
	\node[state]			(q_3) [right = 2.5cm of q_0]			{$q_3$};
	\node[state, accepting]	(q_4) [below right = of q_0]	{$q_4$};
	\path[->]
		(q_0)	edge	node {$\varepsilon$}	(q_1)
				edge	node {$\varepsilon$}	(q_4)
				edge	node {$a$} 		(q_3)
		(q_1)	edge	node {$a$}		(q_2)
				edge	node {$b$}		(q_3)
		(q_2)	edge	node {$a$}		(q_3)
		(q_3)	edge	node {$b$}		(q_4)
		(q_4)	edge [bend left]	node {$a$} 		(q_0);
\end{tikzpicture}
\end{minipage}
\end{tabular}\newline
{\em Далее $\A$~--- абстрактный входной автомат}\newline
Заметим, что проблему можно решить, преобразовав НКА $\A$ в ДКА $\B$. Тогда $\varepsilon$-переходов не будет. Остается один случай, в котором $q_0\in F$, и в $q_0$ есть переходы из других состояний: $\exists q_1\colon \delta(q_1,\sigma)=q_0$. Соответствующими правилами были бы $q_0\longrightarrow\varepsilon,\,q_1\longrightarrow\sigma q_0$, которые не подходят для праволинейной регулярной грамматики (аксиома $q_0$ встречается в правой части при том, что есть переход $q_0\longrightarrow\varepsilon$)
\\[5pt]\newpage
Алгоритм:
\begin{enumerate}
\item Преобразуем данный НКА $\A$ в ДКА $\B$. $L(\A)=L(\B)$.
\item По ДКА $\B=(Q,\Sigma,\delta,q_0,F)$ построим другой ДКА $\C=(Q',\Sigma,\delta',q_s,F')$:\begin{enumerate}
\item $Q'\eqdef Q\cup\{q_s\}$ (добавим состояние $q_s$)
\item $\delta'(q,\sigma)\eqdef\begin{cases}
\delta(q_0,\sigma), & q=q_s\\
\delta(q,\sigma),   & \mbox{иначе}
\end{cases}$ (добавим такие же переходы из $q_s$, как из $q_0$)
\item $F'\eqdef F\cup\big(q_0\in F\,?\,\{q_s\}\,:\,\varnothing\big)$ (если $q_0\in F$, то сделаем $q_s$ принимающим)
\end{enumerate}
Пример построения:\newline
\begin{tabular}{|c|c|c|}
\hline
$\B$ & $\longrightarrow$ & $\C$\\
\hline
\begin{minipage}{0.25\textwidth}
\begin{tikzpicture}[shorten >=1pt,node distance=2cm,on grid,auto,initial text=]
	\node[state, initial,accepting]	(q_0)				{$q_0$};
	\node[state] (q_2) [right = of q_0]			{$q_2$};
	\node[state] (q_1) [above = of q_2]	{$q_1$};
	\node[state] (q_k) [below = of q_2]	{$q_k$};
	\node (dots) [right = 1cm of q_2] {$...$};
	\path[->]
		(q_0)	edge	node {$\sigma_1$}	(q_1)
				edge	node {$\sigma_2$}	(q_2)
				edge	node[swap] {$\sigma_k$} 	(q_k);
\end{tikzpicture}
\end{minipage}&&
\begin{minipage}{0.35\textwidth}
\begin{tikzpicture}[shorten >=1pt,node distance=2cm,on grid,auto,initial text=]
	\node[state, initial, accepting]	(q_s)				{$q_s$};
	\node[state, accepting]	(q_0) [right of=q_s]				{$q_0$};
	\node[state] (q_2) [right = of q_0]			{$q_2$};
	\node[state] (q_1) [above = of q_2]	{$q_1$};
	\node[state] (q_k) [below = of q_2]	{$q_k$};
	\node (dots) [right = 1cm of q_2] {$...$};
	\path[->]
		(q_0)	edge	node {$\sigma_1$}	(q_1)
				edge	node {$\sigma_2$}	(q_2)
				edge	node[swap] {$\sigma_k$} 	(q_k)
		(q_s)	edge [bend left]	node {$\sigma_1$}	(q_1)
				edge [bend left]	node {$\sigma_2$}	(q_2)
				edge [bend right]	node[swap] {$\sigma_k$} 	(q_k);
\end{tikzpicture}
\end{minipage}\\
\hline
\end{tabular}
\\[5pt]
Докажем, что $L(\B)=L(\C)$:
\begin{enumerate}[a.]
\item $L(\B)\subset L(\C)$. Пусть $w\in L(\B)$.\begin{enumerate}[a.]
\item $|w|=0\Rightarrow w=\varepsilon\Rightarrow q_0\in F\Rightarrow q_s\in F'\Rightarrow w\in L(\C)\,\blacksquare$
\item $|w|>0\Rightarrow w=\sigma x,\,\sigma\in\Sigma,\,x\in\Sigma^*$. Тогда $\delta'(q_s,w)\equiv\delta'(q_s,\sigma x)=\delta'(\delta'(q_s,\sigma),x)$. Обозначим $q_i\eqdef \delta'(q_s,\sigma)\overset{\mbox{\tiny def } \delta'}{\equiv}\delta(q_0,\sigma)$.\newline
Очевидно, что $q_i\neq q_s$ (иначе получим переход $q_0\overset{\sigma}{\longrightarrow}q_s$ в $\B$, но $q_s\notin F$). Значит, $\delta'(q_i,x)=\delta(q_i,x)\Rightarrow\delta'(q_s,w)\equiv\delta'(q_i,x)\equiv\delta(q_i,x)\equiv\delta(q_0,\sigma x)\equiv\delta(q_0,w)\in F$, т.к. $w\in L(\B)\Rightarrow w\in L(\C)\,\blacksquare$
\end{enumerate}
\item $L(\C)\subset L(\B)$. Пусть $w\in L(\C)$.\begin{enumerate}[a.]
\item $|w|=0\Rightarrow w=\varepsilon\Rightarrow q_s\in F\Rightarrow q_0\in F\Rightarrow\delta(q_0,w)\equiv\delta(q_0,\varepsilon)=q_0\in F\Rightarrow w\in L(\B)\,\blacksquare$
\item $|w|>0\Rightarrow w=\sigma x,\,\sigma\in\Sigma,\,x\in\Sigma^*$. $F'\ni\delta'(q_s,w)\equiv\delta'(q_s,\sigma x)\equiv\delta'(\delta'(q_s,\sigma),x)$. Аналогично $q_i\eqdef \delta'(q_s,\sigma)\equiv\delta(q_0,\sigma)\in Q\Rightarrow \delta'(q_i,x)=\delta(q_i,x)$\newline
Получаем $F'\ni\delta(q_s,w)=\delta(q_i,x)=\delta(q_0,w)\Rightarrow\delta(q_0,w)\in F\Rightarrow w\in L(\B)\,\blacksquare$
\end{enumerate}
\end{enumerate}
\item По $\C$ построим ПРГ $G=(Q',\Sigma,P,q_s)$:\begin{enumerate}[1.]
\item $P=\underbrace{\{q_a\longrightarrow\sigma q_b\big|\delta'(q_a,\sigma)=q_b\}}_{(1)}\cup\underbrace{\{q_a\longrightarrow\sigma\big|\delta'(q_a,\sigma)=q_b\in F'\}}_{(2)}\cup\underbrace{\big(q_s\in F\,?\,\{q_s\longrightarrow\varepsilon\}\,:\,\varnothing\big)}_{(3)}$.\newline
То есть, каждому переходу $q_a\overset{\sigma}{\longrightarrow}q_b$ в $\C$ соответствует правило $q_a\longrightarrow\sigma q_b\big|\sigma$, причем вторая часть имеется тогда и только тогда, когда $q_b\in F'$. Если $q_s$~--- принимающее, то добавляется правило $q_s\longrightarrow\varepsilon$.
\item $G$~--- ПРГ. Действительно, правила имеют один из видов ($\sigma\in\Sigma$) $A\longrightarrow\sigma B$, $A\longrightarrow\sigma$. Поскольку переходов в $q_s$ в $\C$ нет, то аксиома $q_s$ не будет встречаться в правых частях.
\newline
Также отметим некоторые свойства правил:\begin{enumerate}[a.]
\item По построению $\forall P\ni p\equiv(\alpha,\beta)\hookrightarrow |\beta|_\Sigma-|\alpha|_\Sigma\in\overline{0,1}$, то есть, количество нетерминальных символов справа либо такое же, как слева, либо на 1 больше. Действительно, для групп правил $(1)$ и $(2)$ эта разница равна $1$, а для $(3)$ (если там есть правила) разница равна $0$.
\item Также по построению $\C$ для всех правил справа нет $q_s$, так как в противном случае в $\C$ были бы переходы в $q_s$, что невозможно.
%\item Правила из $(1)$ и $(2)$ сохраняют количество нетерминальных символов, а $(3)$ уменьшает его на $1$.
\item \label{leftone} В правилах слева всегда один нетерминал.
\item \label{w1wninLGthen} Если $w_1...w_n\equiv w\in L(G),\,n>0,\,\forall i\in\overline{1,n}\hookrightarrow w_i\in\Sigma$, то есть, $q_s\Longrightarrow^*w_1...w_n$, то оно было получено применением $n-1$ правил из $(1)$, а затем одного правила из $(2)$, то есть, $q_s\Longrightarrow w_1q_1\Longrightarrow w_1w_2q_2\Longrightarrow...\Longrightarrow w_1w_2...w_{n-1}q_{n-1}\Longrightarrow w_1...w_n$.\newline
Действительно,\begin{enumerate}[a.]
\item \label{sum12} Количество нетерминальных символов в конце равно $n$, значит, было применено $n$ правил из $(1)$ и $(2)$
\item \label{thirdfirst} Если первым было применено правило из $(3)$, то количество нетерминалов стало равно $0$, и далее ни одно правило не могло быть применено (см. \ref{leftone}). При этом количество нетерминалов осталось бы равно $0\neq n$~--- противоречие.
\item \label{thirdnotfirst} Правило из $(3)$ не могло быть применено и далее, так как тогда бы получили $q_s$ в правой части некоторого (предыдущего по выводу) правила.
\item Из \ref{thirdfirst} и \ref{thirdnotfirst} получаем, что применялись только правила из $(1)$ и $(2)$, а из \ref{sum12}~--- что всего таких применений было $n$.
\item Применение правила из $(2)$ ранее, чем на последнем шаге привело бы к тому, что количество нетерминальных символов стало бы равно $0$, после чего (см. \ref{leftone}) правила бы применяться не могли. Однако, количество нетерминальных символов было бы меньше $n$~--- противоречие.
\end{enumerate}
\end{enumerate}
\item Докажем, что $L(G)=L(\C)$, т.е. $\forall w\in\Sigma^*\hookrightarrow w\in L(G)\Leftrightarrow w\in L(\C)$:
\begin{enumerate}[a.]
\item $|w|=0\Rightarrow w=\varepsilon$\begin{enumerate}[a.]
\item $w\in L(\C)\Rightarrow q_s\in F\Rightarrow$ правило $q_s\longrightarrow\varepsilon\in P$. Значит, $\varepsilon\equiv w\in L(G)\,\blacksquare$
\item $w\in L(G)\Rightarrow$ правило $q_s\longrightarrow\varepsilon\in P$. Тогда $q_s\in F'\Rightarrow w\in L(\C)\,\blacksquare$
\end{enumerate}
\item $n\eqdef |w|>0\Rightarrow w=w_1...w_n,\,\forall i\in\overline{1,n}\hookrightarrow w_i\in\Sigma$.\begin{enumerate}[a.]
\item $w\in L(\C)\Rightarrow (q_s,w)\equiv(q_s,w_1...w_n)\vdash(q_1,w_2...w_n)\vdash(q_2,w_3...w_n)\vdash...\vdash(q_{n-1},w_n)\vdash(q_n,\varepsilon),\,q_n\in F'$. Тогда, по построению $G$ имеем $P\ni q_s\longrightarrow w_1q_1,\,q_1\longrightarrow w_2q_2,\,...,\,q_{n-1}\longrightarrow w_n$. Значит,\newline
$q_s\longrightarrow w_1q_1\longrightarrow w_1w_2q_2\longrightarrow...\longrightarrow w_1w_2...w_{n-1}q_{n-1}\longrightarrow w_1...w_n\in\Sigma^*\Rightarrow w\in L(G)\,\blacksquare$
\item $w\in L(G)\overset{\ref{w1wninLGthen}}{\Rightarrow}q_s\longrightarrow w_1q_1\longrightarrow w_1w_2q_2\longrightarrow...\longrightarrow w_1w_2...w_{n-1}q_{n-1}\longrightarrow w_1...w_n$, и были применены правила $q_s\Rightarrow w_1q_1,\,...,\,q_{n-1}\Rightarrow w_n$ (также см. \ref{w1wninLGthen}).\newline
Отсюда $\delta'(q_s,w_1)=q_1,\,\delta'(q_1,w_2)=q_2,\,...,\,\delta'(q_{n-1},w_n)\in F\Rightarrow\delta'(q_s,w)\in F\Rightarrow w\in L(\C)\,\blacksquare$
\end{enumerate}
\end{enumerate}
%$P(n)=\big[\forall w\in\Sigma^*\colon |w|=n\hookrightarrow w\in L(\C)\Leftrightarrow w\in L(G)\big]$. Докажем $\forall n\geqslant 0\hookrightarrow P(n)$ по индукции:
%\begin{enumerate}[1.]
%\item $n=0\Rightarrow w=\varepsilon$\begin{enumerate}[a.]
%\item $w\in L(\C)\Rightarrow q_s\in F\Rightarrow$ правило $q_s\longrightarrow\varepsilon\in P$. Значит, $\varepsilon\equiv w\in L(G)\,\blacksquare$
%\item $w\in L(G)\Rightarrow$ правило $q_s\longrightarrow\varepsilon\in P$. Тогда $q_s\in F'\Rightarrow w\in L(\C)\,\blacksquare$
%\end{enumerate}
%Таким образом, $P(0)\,\blacksquare$
%\item Пусть $P(n)$. $P(n)\Rightarrow \forall w_0\in\Sigma^*\colon |w_0|=n\hookrightarrow w_0\in L(\C)\Leftrightarrow w_0\in L(G)$. Докажем $P(n+1)\Leftrightarrow\big[\forall w\in\Sigma^*\colon |w|=n+1\hookrightarrow w\in L(\C)\Leftrightarrow w\in L(G)\big]$. Фиксируем $w\colon |w|=n+1$.\begin{enumerate}
%\item $w\in L(\C)$
%\end{enumerate}
%\end{enumerate}
\end{enumerate}
\end{enumerate}
Применим описанный алгоритм для автомата $\A_0$ из условия:
\begin{enumerate}
\item Построим по $\A_0$ ДКА $\B_0$:\newline
\begin{tabular}{|c|c|c|}
\hline
$\A_0$ & Построение & $\B_0$\\\hline
\begin{minipage}{0.3\textwidth}
\begin{tikzpicture}[shorten >=1pt,node distance=2cm,on grid,auto,initial text=]
	\node[state, initial]	(q_0)							{$q_0$};
	\node[state] 		  	(q_1) [above right = of q_0]	{$q_1$};
	\node[state, accepting] 	(q_2) [right = of q_1]			{$q_2$};
	\node[state]			(q_3) [right = 2.5cm of q_0]			{$q_3$};
	\node[state, accepting]	(q_4) [below right = of q_0]	{$q_4$};
	\path[->]
		(q_0)	edge	node {$\varepsilon$}	(q_1)
				edge	node {$\varepsilon$}	(q_4)
				edge	node {$a$} 		(q_3)
		(q_1)	edge	node {$a$}		(q_2)
				edge	node {$b$}		(q_3)
		(q_2)	edge	node {$a$}		(q_3)
		(q_3)	edge	node {$b$}		(q_4)
		(q_4)	edge [bend left]	node {$a$} 		(q_0);
\end{tikzpicture}
\end{minipage} &
\begin{minipage}{0.3\textwidth}
\small (отмечены начальная и принимающие)
$$
\begin{array}{|l|l|c|c|}
\hline
Q & q & a & b\\\hline
\rightarrow Q_0 \rightarrow & q_0,q_1,q_4 & Q_1 & Q_2\\\hline
Q_1\rightarrow & q_0,q_1,q_2,q_3,q_4 & Q_1 & Q_3\\\hline
Q_2 & q_3 & \varnothing & Q_4\\\hline
Q_3\rightarrow & q_3, q_4 & Q_0 & Q_4\\\hline
Q_4\rightarrow & q_4 & Q_0 & \varnothing\\
\hline
\end{array}
$$
\end{minipage} &
\begin{minipage}{0.3\textwidth}
\taskoneautomataB
\end{minipage}\\\hline
\end{tabular}
\item Построим $\C_0$ по $\B_0$, как описано в алгоритме:\newline
\begin{tabular}{|c|c|c|}
\hline
$\B_0$ & $\C_0$\\\hline
\begin{minipage}{0.3\textwidth}
\taskoneautomataB
\end{minipage} &
\begin{minipage}{0.4\textwidth}
\begin{tikzpicture}[shorten >=1pt,node distance=2cm,on grid,auto,initial text=]
	\node[state, initial, accepting]	(Q_s)							{$Q_s$};
	\node[state, accepting]	(Q_0) [right of = Q_s]			{$Q_0$};
	\node[state, accepting]	(Q_1) [above right = of Q_0]	{$Q_1$};
	\node[state] 	(Q_2) [below right = of Q_0]			{$Q_2$};
	\node[state, accepting]	(Q_3) [right = of Q_1]			{$Q_3$};
	\node[state, accepting]	(Q_4) [right = of Q_2]	{$Q_4$};
	\path[->]
		(Q_s)	edge [bend left]	node {$a$}	(Q_1)
				edge [bend right]	node[swap] {$b$}	(Q_2)
		(Q_0)	edge	node {$a$}	(Q_1)
				edge	node {$b$}	(Q_2)
		(Q_1)	edge  [loop below]	node[swap] {$a$} (Q_1)
				edge	node {$b$}		(Q_3)
		(Q_2)	edge	node {$b$}		(Q_4)
		(Q_3)	edge [bend left=10]	node {$a$}		(Q_0)
				edge node {$b$}		(Q_4)
		(Q_4)	edge [bend right=10]	node[swap] {$a$}		(Q_0);
\end{tikzpicture}
\end{minipage}\\\hline
\end{tabular}
\item Определим грамматику $G=(\Sigma, Q, P, Q_s)$. Выпишем правила $p\in P$:\newline
\begin{tabular}{|c|l|l|l|}
\hline
$Q$ & $\varepsilon$ & $a$ & $b$\\\hline
$Q_s$ & $Q_s\longrightarrow\varepsilon$ & $Q_s\longrightarrow aQ_1|a$ & $Q_s\longrightarrow bQ_2$ \\\hline
$Q_0$ & & $Q_0\longrightarrow aQ_1|a$ & $Q_0\longrightarrow bQ_2$ \\\hline
$Q_1$ & & $Q_1\longrightarrow aQ_1|a$ & $Q_1\longrightarrow bQ_3|b$ \\\hline
$Q_2$ & & & $Q_2\longrightarrow bQ_4|b$\\\hline
$Q_3$ & & $Q_3\longrightarrow aQ_0|a$ & $Q_3\longrightarrow bQ_4|b$ \\\hline
$Q_4$ & & $Q_4\longrightarrow aQ_0|a$ & \\\hline
\end{tabular}
\end{enumerate}
\newpage
\section*{Задача 2}
\begin{enumerate}
\item Пусть дана ПГ $G=(N,T,P,S)$. Построим по ней НКА $\A=(Q,T,\delta,q_0,F)\colon L(\A)=L(G)$.\begin{enumerate}[1.]
\item Определим \begin{enumerate}[a.]
\item $Q=N\cup\{q_f\}$
\item $q_0\eqdef S$
\item $F\eqdef\{q_f\}$
\item $\delta\eqdef\varnothing$
\end{enumerate}
\item Рассмотрим переходы $P\ni p_i=A\longrightarrow xB$ (правила типа 1). Для них добавим\begin{enumerate}[a.]
\item Состояния $q^i_1,...,q^i_{|x|-1}$, если $|x|>1$.
\item Переходы: \label{stepAB}\begin{enumerate}[1.]
\item $A\overset{\varepsilon}{\longrightarrow}B$, если $x=\varepsilon$.
\item $A\overset{x_1}{\longrightarrow}B$, если $|x|=1$, $x_1\equiv x$.
\item $A\overset{x_1}{\longrightarrow}q^i_1\overset{x_2}{\longrightarrow} q^i_2\longrightarrow...\longrightarrow q^i_{n-1}\overset{x_n}{\longrightarrow}B$, если $|x|>1$, $x=x_1...x_n$.
\end{enumerate}
\end{enumerate}
\item Рассмотрим переходы $P\ni p_i=A\longrightarrow x$ (правила типа 2). Для них добавим\begin{enumerate}[a.]
\item Состояния $q^i_1,...,q^i_{|x|-1}$, если $|x|>1$.
\item Переходы: \label{stepAqf}\begin{enumerate}[1.]
\item $A\overset{\varepsilon}{\longrightarrow}q_f$, если $x=\varepsilon$.
\item $A\overset{x_1}{\longrightarrow}q_f$, если $|x|=1$, $x_1\equiv x$.
\item $A\overset{x_1}{\longrightarrow}q^i_1\overset{x_2}{\longrightarrow} q^i_2\longrightarrow...\longrightarrow q^i_{n-1}\overset{x_n}{\longrightarrow}q_f$, если $|x|>1$, $x=x_1...x_n$.
\end{enumerate}
\end{enumerate}
\end{enumerate}
\item[1.5.] Докажем, что $L(G)=L(\A)$\begin{enumerate}[a.]
\item Пусть $w\in L(G), n=|w|>0,w\equiv w_1...w_n$. Правила типа 1 не изменяют количество нетерминальных символов, а правила типа 2~--- уменьшают на 1. Аксиома~--- один нетерминал. В левой части каждого правила один нетерминал. Поэтому вывод слова $w$ имеет следующий вид: $m-1$ применений правил типа 1, затем применение правила типа 2~--- т.к. после правила (2) нельзя применить никакое правило. Также каждое правило не уменьшает количество нетерминальных символов. То есть, $S=q_0\longrightarrow w_1w_2...w_{i_1}q_1\longrightarrow w_1w_2...w_{i_1}...w_{i_2}q_2\longrightarrow...\longrightarrow w_1...w_{i_{m-1}}q_{m-1}\longrightarrow w_1...w_n$. По построению имеем $(q_0,w)\vdash^*(q_1,w_{i_1+1}...w_n)\vdash^*(q_2,w_{i_2+1}...w_n)\vdash^*(q_{m-1},w_{i_{m-1}}...w_n)\vdash(q_f,\varepsilon)$, $q_f\in F\Rightarrow w\in L(\A)$.
\item Пусть $w\in L(G), |w|=0\Rightarrow w=\varepsilon$. Тогда во всех $m$ примененных правилах $x_i=\varepsilon$, а последнее~--- типа 2: $q_0\longrightarrow q_{i_1}\longrightarrow...\longrightarrow q_{i_{m-1}} \varepsilon$. По построению, $\delta(q_0,\varepsilon)=q_{i_1},...,\delta(q_{i_{m-1}},\varepsilon)=q_f$. Получаем $w\in L(\A)$
\item Пусть $w\in L(\A)$. Тогда $F\ni\delta(q_0,w)=q_f$ (т.к. $F=\{q_f\}$). Рассмотрим цепочку конфигураций $(q_0,w)\vdash^*(q_f,\varepsilon)$. Выпишем оттуда все состояния $A_i\colon A_i\in N$. Тогда переходы $A_i\overset{x_i}{\longrightarrow}A_{i+1}$ между ними были добавлены на шагах \ref{stepAB}, а последний переход $A_m\overset{x_m}{\longrightarrow} q_f$~--- на шаге \ref{stepAqf}. Поэтому в грамматике есть правила $S\Rightarrow x_1A_1,...,A_m\Rightarrow x_m$. Отсюда $w\in L(G)$.
\end{enumerate}
\item Построим НКА $\A$ по грамматике $G$: $S\longrightarrow abaA|abB|\varepsilon,\ A\longrightarrow  aB|aa,\ B\longrightarrow bA|aS$
\begin{center}
\begin{tikzpicture}[shorten >=1pt,node distance=2cm,on grid,auto,initial text=]
	\node[state, initial]	(q_0)							{$q_0$};
	\node[state, accepting] (q_f)  [right = of q_0]	{$q_f$};
	\node[state] 			(q_11) [above = of q_f]	{$q^1_1$};
	\node[state] 			(q_21) [below = of q_f]	{$q^2_1$};
	\node[state] 			(q_12) [right = of q_11]	{$q^1_2$};
	\node[state] 			(q_51) [right = of q_f]	{$q^5_1$};
	\node[state] 			(q_B)  [right = of q_21]	{$B$};
	\node[state] 			(q_A)  [right = of q_51]	{$A$};
	\path[->]
		(q_0)	edge	node {$a$}	(q_11)
				edge	node {$\varepsilon$}	(q_f)
				edge	node {$a$} 		(q_21)
		(q_11)	edge	node {$b$}	(q_12)
		(q_12)	edge	node {$a$}	(q_A)
		(q_A)	edge	node {$a$}	(q_51)
				edge [bend left=10]	node {$a$}	(q_B)
		(q_51)	edge	node {$a$}	(q_f)
		(q_21)	edge	node {$b$}	(q_B)
		(q_B)	edge [bend left=70]	node {$a$}	(q_0)
				edge [bend left=10]	node {$b$}	(q_A);
\end{tikzpicture}
\end{center}
\item Запишем систему уравнений:
$\begin{cases}
(1) & S=abaA+abB+\varepsilon\\
(2) & A=aB+aa\\
(3) & B=bA+aS\\
\end{cases}$. Подставим $(3)$ в $(2)$, получим $A=abA+aaS+aa$. Отсюда $A=(ab)^*(aaS+aa)$. Подставим в $(3)$: $B=b(ab)^*(aaS+aa)+aS$. Подставим $A,B$ в $(1)$: $$S=aba(ab)^*(aaS+aa)+ab(b(ab)^*(aaS+aa)+aS)+\varepsilon\equiv aba(ab)^*aaS+aba(ab)^*aa+abb(ab)^*aaS+abb(ab)^*aa+abaS+\varepsilon\equiv$$
$$\equiv(aba(ab)^*aa+abb(ab)^*aa+aba)S+aba(ab)^*aa+abb(ab)^*aa+\varepsilon\Rightarrow$$
$S=(aba(ab)^*aa+abb(ab)^*aa+aba)^*(aba(ab)^*aa+abb(ab)^*aa+\varepsilon)$.
\item Нет: $S\longrightarrow abaA\longrightarrow abaaa$, $S\longrightarrow abaA\longrightarrow abaaB\longrightarrow abaaaS\longrightarrow abaaa$.
\end{enumerate}
\section*{Задача 3}
Для приведенного мной алгоритма это неверно. Возьмем грамматику $G\colon P=\{S\longrightarrow a,S\longrightarrow aa\}$. Она будет праволинейной (все правила имеют вид $S\longrightarrow x,\,x\in\Sigma^*$) и однозначной: она порождает язык $\{a,aa\}$, каждое слово может быть получено единственным способом. Но алгоритм построит автомат $\A$, который не будет детерминированным, так как $\delta(q_0,a)=\{q_f,q^2_1\}$.
\begin{center}
\begin{tikzpicture}[shorten >=1pt,node distance=2cm,on grid,auto,initial text=]
	\node[state, initial]	(q_0)							{$q_0$};
	\node[state, accepting] (q_f)  [right = 2.9cm of q_0]	{$q_f$};
	\node[state] 			(q_21) [below right = of q_0]	{$q^2_1$};
	\path[->]
		(q_0)	edge [bend right]	node[swap] {$a$}	(q_21)
				edge	node {$a$}	(q_f)
		(q_21)	edge [bend right]	node[swap] {$a$}	(q_f);
\end{tikzpicture}
\end{center}
\section*{Задача 4}
Рассмотрим грамматику $G\colon P=\{S\longrightarrow 0S1,S\longrightarrow \varepsilon\}$. $L\eqdef\{0^n1^n\big| n\geqslant 0\}$. Докажем, что $w\in L(G)\Leftrightarrow w\in L$.
\begin{enumerate}
\item $w\in L\Rightarrow w=0^n1^n$\begin{enumerate}[a.]
\item $n=0\Rightarrow w=\varepsilon$. Но $P\ni S\longrightarrow \varepsilon\Rightarrow w\equiv\varepsilon\in L(G)$.
\item $n>0$. Применим первое правило $n$ раз, после чего применим второе: $S\longrightarrow 0S1\longrightarrow 00S11\longrightarrow...\longrightarrow \underbrace{0...0}_{n}S\underbrace{1...1}_n\longrightarrow 0^n1^n\Rightarrow w\in L(G)$
\end{enumerate}
\item $w\in L(G)$. Первое правило сохраняет количество нетерминалов, второе уменьшает на 1. Поэтому в выводе сначала $n$ применений первого правила, а затем одно применение второго. Количество терминалов не уменьшается. Вывод имеет вид $S\longrightarrow 0S1\longrightarrow...\longrightarrow \underbrace{0...0}_nS\underbrace{1...1}_n\longrightarrow 0^n1^n\in L$
\end{enumerate}
$G$~--- линейная, так как в правых частях правил не более одного нетерминала. Но $L\equiv L(G)\notin\REG$~--- было доказано на семинаре. Поэтому получаем, что утверждение $\forall G\mbox{~--- линеная}\hookrightarrow L(G)\in\REG$~--- неверно.
\section*{Задача 5}
Пусть $\Sigma=\{\sigma_1,...\sigma_n\}\supseteq L,L_{\sigma_1},...,L_{\sigma_n}\in\REG$. Докажем, что подстановка $L_{\sigma_1},...,L_{\sigma_n}$ в $L$ $L'=\{L_{w_1}...L_{w_k}\big| w_1...w_k\in L\}\in\REG$ индукцией по $R_3(L)$~--- количеству применений третьего пункта определения при получении языка $L$ (см. решение предыдущего задания)\begin{enumerate}
\item $R_3(L)=0\Rightarrow$\begin{enumerate}
\item $L=\varnothing$. Тогда $L'=\varnothing$, так как ни одно слово $w$ не в $L$. Получаем $L'\in\REG\,\blacksquare$
\item $L=\{\varepsilon\}$. Тогда $L'=\varepsilon$ (конкатенация из $0$ строк). Получаем $L'\in\REG\,\blacksquare$
\item $L=\{\sigma\}$. Тогда $L'=L_\sigma\in\REG\,\blacksquare$
\end{enumerate}
\item Пусть $\forall L\colon R_3(L)\leqslant n\hookrightarrow L'\in\REG$. Докажем для $n+1$. Варианты:\begin{enumerate}
\item $L=XY$, $X,Y\in\REG$. Тогда подстановка в $X$ $L'_X\in\REG$ по предположению индукции. Аналогично для $L'_Y$. Но $L'=L'_XL'_Y$: $L'=\{L_{w_1}...L_{w_k}\big|w_1...w_k\in XY\}=\{L_{w_1}...L_{w_m}...L_{w_k}\big|w_1...w_m\in X,w_{m+1}...w_k\in Y\}=\{L_{w_1}...L_{w_m}\big|w_1...w_m\in X\}\cdot \{L_{w_{m+1}}...L_{w_k}\big|w_{m+1}...w_k\in Y\}=L'_XL'_Y$, поэтому $L'\in\REG$.
\item $L=X|Y,X,Y\in\REG$. Аналогично $L'_X,L'_Y\in\REG$. Но $L'=\{L_{w_1}...L_{w_k}\big|w_1...w_k\in X|Y\}=\{L_{w_1}...L_{w_k}\big|w_1...w_k\in X\}\cup\{L_{w_1}...L_{w_k}\big|w_1...w_k\in Y\}=L'_XL'_Y\in\REG\,\blacksquare$
\item $L=X^*,X\in\REG$. По предположению, $L'_X\in\REG$. Но $L'=\{L_{w_1}...L_{w_k}\big|w_1...w_k\in X^*\}=\{(L_{w_1}...L_{w_k})^*\big|w_1...w_k\in X\}={L'}_X^*\in\REG\,\blacksquare$
\end{enumerate}
\end{enumerate}
\section*{Задача 6}
\section*{Задача 7}
\end{document}