\documentclass[a4paper]{article}
\usepackage[a4paper, left=5mm, right=5mm, top=5mm, bottom=5mm]{geometry}
%\geometry{paperwidth=210mm, paperheight=2000pt, left=5pt, top=5pt}
\usepackage[utf8]{inputenc}
\usepackage[english,russian]{babel}
\usepackage{indentfirst}
\usepackage{tikz} %Рисование автоматов
\usetikzlibrary{automata,positioning,arrows,trees}
\usepackage{amsmath}
\usepackage[makeroom]{cancel} % зачеркивание
\usepackage{multicol,multirow} %Несколько колонок
\usepackage{hyperref}
\usepackage{tabularx}
\usepackage{amsfonts}
\usepackage{amssymb}
\DeclareMathOperator*{\argmin}{arg\,min}
\usepackage{wasysym}
\title{Алгоритмы и модели вычислений.\\Задание 2: Арифметические операции и линейные рекуррентные последовательности}
\date{задано 2014.02.20}
\author{Сергей~Володин, 272 гр.}
\newcommand{\matrixl}{\left|\left|}
\newcommand{\matrixr}{\right|\right|}
% названия автоматов  (rubtsov)
\def\A{{\cal A}}
\def\B{{\cal B}}
\def\C{{\cal C}}

% алгоритмы (Рубцов)
\usepackage{verbatim}
\usepackage{listings}
\usepackage{algorithm2e}

%+= и -=, иначе разъезжаются...
\newcommand{\peq}{\mathrel{+}=}
\newcommand{\meq}{\mathrel{-}=}
\newcommand{\deq}{\mathrel{:}=}
\newcommand{\plpl}{\mathrel{+}+}

% пустое слово  (rubtsov)
\def\eps{\varepsilon}

% регулярные языки  (rubtsov)
\def\REG{{\mathsf{REG}}}
\def\CFL{{\mathsf{CFL}}}
\def\eqdef{\overset{\mbox{\tiny def}}{=}}
\newcommand{\niton}{\not\owns}

%FIRST & FOLLOW (rubtsov)
\def\first{\mathrm{ FIRST} }
\def\follow{\mathrm{ FOLLOW} }

% LL LR (rubtsov)
\def\LL{{\mathrm{LL}}}
\def\LR{{\mathrm{LR}}}

\newcounter{rowItemCount}
\newcounter{subRowItemCount}
\newcommand\rowItem{
    \setcounter{subRowItemCount}{0}
    \arabic{rowItemCount}.\addtocounter{rowItemCount}{1}}
\newcommand\subRowItem{
    \addtocounter{subRowItemCount}{1}
    \addtocounter{rowItemCount}{-1}
    \arabic{rowItemCount}.\arabic{subRowItemCount}.\addtocounter{rowItemCount}{1}}
    
\newcommand{\smalll}[1]{\overline{\overline{#1}}}
\newcommand{\smallo}{\bar{\bar{o}}}

\begin{document}
\maketitle
\subsection*{(каноническое) Задача 6}
$T(n)=7T(\frac{n}{2})+f(n)$, $f(n)=O(n^2)$. Дерево рекурсии:
\newline
\begin{tikzpicture}[scale=0.75,transform shape,level/.style={sibling distance = 5cm/#1, level distance = 1.5cm}]
\node [circle,draw] (z){$n^2$}
  child {node [circle,draw] (a) {$\frac{n^2}{2^2}$}
    child {node [circle,draw] (b) {$\frac{n^2}{2^4}$}
      child {node {$\vdots$}
        child {node [circle,draw] (d) {$T(1)$}}
        child {node [circle,draw] (e) {$T(1)$}}
      } 
      child {node {$\vdots$}}
    }
    child {node [circle,draw] (g) {$\frac{n^2}{2^4}$}
      child {node {$\vdots$}}
      child {node {$\vdots$}}
    }
  }
  child {node [circle,draw] (j) {$\frac{n^2}{2^2}$}
    child {node [circle,draw] (k) {$\frac{n^2}{2^4}$}
      child {node {$\vdots$}}
      child {node {$\vdots$}}
    }
  child {node [circle,draw] (l) {$\frac{n^2}{2^4}$}
    child {node {$\vdots$}}
    child {node (c){$\vdots$}
      child {node [circle,draw] (o) {$T(1)$}}
      child {node [circle,draw] (p) {$T(1)$}
          child [grow=right] {node (q) {$7^hT(1)$} edge from parent[draw=none]
          child [grow=up, level distance=0.7cm] {node (r) {$\vdots$} edge from parent[draw=none]
          child [grow=up, level distance=0.7cm] {node (r) {$7^k\frac{n^2}{2^{2k}}$} edge from parent[draw=none]
            child [grow=up, level distance=1cm] {node (r) {$\vdots$} edge from parent[draw=none]
              child [grow=up, level distance=0.6cm] {node (s) {$7^2\frac{n^2}{2^4}$} edge from parent[draw=none]
                child [grow=up, level distance=1.5cm] {node (t) {$7\frac{n^2}{2^2}$} edge from parent[draw=none]
                  child [grow=up, level distance=1.5cm] {node (u) {$n^2$} edge from parent[draw=none]}
                }
              }
              }
              }
            }
          }
        }
    }
  }
};
\end{tikzpicture}\newline
Высота дерева $h=\log_2 n$. $T(n)=\sum\limits_{k=0}^{h-1}7^kf(\frac{n}{2^k})+7^hT(1)\boxed{\leqslant}$. Из определения $O$ $\exists C>0\,\exists n_0\colon\forall n\geqslant n_0\hookrightarrow f(n)\leqslant Cn^2$, откуда первая сумма
$\sum\limits_{k=0}^{h-1}7^kf(\frac{n^2}{2^{2k}})\leqslant Cn^2\sum\limits_{k=0}^{h-1}(\frac{7}{4})^k=Cn^2\frac{(7/4)^{h-1}-1}{7/4-1}=C_1n^2((7/4)^{\log_2 n}-C_2)=C_1n^2n^{\log_2\frac{7}{4}}-C_3n^2=C_1n^{\log_2 7}-C_3n^2$. Второе слагаемое $7^hT(1)=7^{\log_2 n}T(1)=Cn^{\log_2 7}$\newline
Поэтому $T(n)\leqslant n^{\log_2 7}-C_5n^2$
Ответ: $\boxed{T(n)=O(n^{\log_2 7})}$
\end{document}