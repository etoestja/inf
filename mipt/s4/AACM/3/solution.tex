\documentclass[a4paper]{article}
\usepackage[a4paper, left=5mm, right=5mm, top=5mm, bottom=5mm]{geometry}
%\geometry{paperwidth=210mm, paperheight=2000pt, left=5pt, top=5pt}
\usepackage[utf8]{inputenc}
\usepackage[english,russian]{babel}
\usepackage{indentfirst}
\usepackage{tikz} %Рисование автоматов
\usetikzlibrary{automata,positioning,arrows,trees}
\usepackage{amsmath}
\usepackage[makeroom]{cancel} % зачеркивание
\usepackage{multicol,multirow} %Несколько колонок
\usepackage{hyperref}
\usepackage{tabularx}
\usepackage{amsfonts}
\usepackage{amssymb}
\DeclareMathOperator*{\argmin}{arg\,min}
\usepackage{listings}
\usepackage{wasysym}
\date{задано 2014.02.27}
\author{Сергей~Володин, 272 гр.}
\newcommand{\matrixl}{\left|\left|}
\newcommand{\matrixr}{\right|\right|}
% названия автоматов  (rubtsov)
\def\A{{\cal A}}
\def\B{{\cal B}}
\def\C{{\cal C}}

%классы сложности (rubtsov)
\def\PP{{\mathsf{P}}}
\def\NP{{\mathsf{NP}}}
\def\NPc{{\mathsf{NP}}\text{-}{\rm c}}
\def\coNP{{\rm co}\text{-}{\mathsf{NP}}}
\def\DTIME{{\mathsf{DTIME}}}
\def\NTIME{{\mathsf{NTIME}}}
\def\HALT{{\rm{HALT}}}
\def\SAT{{\rm{SAT}}}
\def\3SAT{{\rm{3\text{-}SAT}}}
\def\2SAT{{\rm{2\text{-}SAT}}}
\def\UNSAT{{\rm{UNSAT}}}
\def\LL{{\mathrm{LL}}}
\def\poly{{\rm{poly}}}

\title{Алгоритмы и модели вычислений.\\Задание 3: Сложность вычислений, классы $\PP$, $\NP$}

% алгоритмы (Рубцов)
\usepackage{verbatim}
\usepackage{listings}
\usepackage{algorithm2e}

%+= и -=, иначе разъезжаются...
\newcommand{\peq}{\mathrel{+}=}
\newcommand{\meq}{\mathrel{-}=}
\newcommand{\deq}{\mathrel{:}=}
\newcommand{\plpl}{\mathrel{+}+}

% пустое слово  (rubtsov)
\def\eps{\varepsilon}

% регулярные языки  (rubtsov)
\def\REG{{\mathsf{REG}}}
\def\CFL{{\mathsf{CFL}}}
\def\eqdef{\overset{\mbox{\tiny def}}{=}}
\newcommand{\niton}{\not\owns}

%FIRST & FOLLOW (rubtsov)
\def\first{\mathrm{ FIRST} }
\def\follow{\mathrm{ FOLLOW} }

% LL LR (rubtsov)
\def\LL{{\mathrm{LL}}}
\def\LR{{\mathrm{LR}}}

\newcounter{rowItemCount}
\newcounter{subRowItemCount}
\newcommand\rowItem{
    \setcounter{subRowItemCount}{0}
    \arabic{rowItemCount}.\addtocounter{rowItemCount}{1}}
\newcommand\subRowItem{
    \addtocounter{subRowItemCount}{1}
    \addtocounter{rowItemCount}{-1}
    \arabic{rowItemCount}.\arabic{subRowItemCount}.\addtocounter{rowItemCount}{1}}
    
\newcommand{\smalll}[1]{\overline{\overline{#1}}}
\newcommand{\smallo}{\bar{\bar{o}}}

\begin{document}
\maketitle
\subsection*{(каноническое) Задача 11}
$\mbox{M}^{\mathbb{Z}, S}_{p\times q}$~--- множество матриц $\matrixl a_{ij} \matrixr$ размера $p\times q$ с целыми коэффициентами, такими, что $|a_{ij}|\leqslant S$. $S=10000, m=2014$. Язык $\{0,1\}^*\supset L_{m\times n}=\{\mbox{bin}(m,n,A,b)\big|(A,b)\in\mbox{M}^{\mathbb{Z}, S}_{m\times n}\times \mbox{M}^{\mathbb{Z}, S}_{m\times 1},\,Ax=b\mbox{~--- несовместна}\}$~--- двоичные записи несовместных систем линейных уравнений с целыми коэффициентами.
% {\em (имеются в виду %$L'\subset(\overline{-L,L}^{m\cdot n\cdot m\cdot 1})^*$) двоичные записи}
\begin{enumerate}
\item Рассмотрим $w^i_j=\big(\matrixl\begin{array}{cccc}
i & 0 & ... & 0\\
\end{array}\matrixr,\,\matrixl
\begin{array}{c}
j\\
\end{array}\matrixr\big)$. При $i=0,\,j\in\{1,2\}$ система несовместна, поэтому $w^0_1,\,w^0_2\in L_{2014\times 1}$. При $i=1,\,j\in\{1,2\}$ система совместна, поэтому $w^1_1,\,w^1_2\notin L_{2014\times 1}$
\item Применим метод Гаусса. Для оценки времени работы приведен псевдокод:\newline

\lstset{
    language=C,
    basicstyle=\ttfamily\small,
    breaklines=true,
    prebreak=\raisebox{0ex}[0ex][0ex]{\ensuremath{\hookleftarrow}},
    frame=lines,
    showtabs=false,
    showspaces=false,
    showstringspaces=false,
    keywordstyle=\color{red}\bfseries,
    stringstyle=\color{green!50!black},
    commentstyle=\color{gray}\itshape,
    numbers=left,
    captionpos=t,
    escapeinside={\%*}{*)}
}

%
% a11  a1n
% am1  amn
%

\begin{lstlisting}
//M[i][j] - matrix A|b
for(i = 1; i <= m; i++) // rows i=1...m
{
  for(j = 1; j <= n; j++) // find j: aij != 0
  {
    if(M[i][j] != 0) // found
    {
      C = M[i][j];

      // dividing i-th row by non-zero element
      for(k = 1; i <= n + 1; i++)
        M[i][k] /= C;

      for(k = 1; k <= m; k++) // subtracting from row k
      {
        if(k != i)
        {
          C = M[k][j];
          for(l = 1; l <= n + 1; l++) // column l
            M[k][l] -= M[i][l] * C;
        }
      }
      
      break;
    }
  }
}
\end{lstlisting}
Предположение индукции: после $i$ шагов внешнего цикла (обработано $i$ строк) элементы ???\newline
Храним рациональные числа как числитель и знаменатель.\newline
Рассмотрим обработку одного столбца в методе Гаусса ($a$~--- ведущий элемент в строке).\newline
$\matrixl
\begin{array}{ccccc}
... & a & ... & b & ...\\
\multicolumn{5}{c}{...}\\
... & c & ... & d & ...\\
\end{array}
\matrixr\sim
\matrixl
\begin{array}{ccccc}
... & 1 & ... & \frac{b}{a} & ...\\
\multicolumn{5}{c}{...}\\
... & c & ... & d & ...\\
\end{array}
\matrixr\sim
\matrixl
\begin{array}{ccccc}
... & 1 & ... & \frac{b}{a} & ...\\
\multicolumn{5}{c}{...}\\
... & 0 & ... & d-\frac{bc}{a} & ...\\
\end{array}
\matrixr
$

Для вычитания нужно вычислить для каждого элемента строки $d-\frac{bc}{a}=\frac{ad-bc}{a}$.\newline
\textbf{WTF Тарасов нес какую-то фигню про то, что миноры не меняются после элементарных операций, и поэтому все ок}
\subsection*{(каноническое) Задача 12}
\begin{enumerate}
\item Используем быстрое возведение в степень по модулю $d$. Умножаем числе не более чем $|d|$ бит. Остаток от деления считается за квадрат битовой записи.
\item Слова, соответсвующие $(1,1,1,2),\,(1,2,1,2)\in L$, $(1,1,2,2),\,(1,2,2,2)\notin L$
\end{enumerate}
\end{enumerate}
\subsection*{(каноническое) Задача 13}
Бинпоиском ищем корень 2014 степени. $L=1$, $R=$вход. Шагов $\log_{2} R$=$\log_2 2^t=t$, возводим числа $<= 2^t$ в 2014 степень за $\log^{2014} 2^t=t^{2014}$???
\end{document}