\documentclass[a4paper]{article}
\usepackage[a4paper, left=5mm, right=5mm, top=5mm, bottom=5mm]{geometry}
%\geometry{paperwidth=210mm, paperheight=2000pt, left=5pt, top=5pt}
\usepackage[utf8]{inputenc}
\usepackage[english,russian]{babel}
\usepackage{indentfirst}
\usepackage{tikz} %Рисование автоматов
\usetikzlibrary{automata,positioning,arrows,trees}
\usepackage{amsmath}
\usepackage[makeroom]{cancel} % зачеркивание
\usepackage{multicol,multirow} %Несколько колонок
\usepackage{hyperref}
\usepackage{tabularx}
\usepackage{amsfonts}
\usepackage{amssymb}
\DeclareMathOperator*{\argmin}{arg\,min}
\usepackage{listings}
\usepackage{wasysym}
\date{задано 2014.03.06}
\author{Сергей~Володин, 272 гр.}
\newcommand{\matrixl}{\left|\left|}
\newcommand{\matrixr}{\right|\right|}
% названия автоматов  (rubtsov)
\def\A{{\cal A}}
\def\B{{\cal B}}
\def\C{{\cal C}}

%классы сложности (rubtsov)
\def\PP{{\mathsf{P}}}
\def\NP{{\mathsf{NP}}}
\def\NPc{{\mathsf{NP}}\text{-}{\rm c}}
\def\coNP{{\rm co}\text{-}{\mathsf{NP}}}
\def\DTIME{{\mathsf{DTIME}}}
\def\NTIME{{\mathsf{NTIME}}}
\def\HALT{{\rm{HALT}}}
\def\SAT{{\rm{SAT}}}
\def\3SAT{{\rm{3\text{-}SAT}}}
\def\2SAT{{\rm{2\text{-}SAT}}}
\def\UNSAT{{\rm{UNSAT}}}
\def\LL{{\mathrm{LL}}}
\def\poly{{\rm{poly}}}

\title{Алгоритмы и модели вычислений.\\Задание 4: Сложность вычислений, классы $\PP$, $\NP$ и $\coNP$}

% алгоритмы (Рубцов)
\usepackage{verbatim}
\usepackage{listings}
\usepackage{algorithm2e}

%+= и -=, иначе разъезжаются...
\newcommand{\peq}{\mathrel{+}=}
\newcommand{\meq}{\mathrel{-}=}
\newcommand{\deq}{\mathrel{:}=}
\newcommand{\plpl}{\mathrel{+}+}

% пустое слово  (rubtsov)
\def\eps{\varepsilon}

% регулярные языки  (rubtsov)
\def\REG{{\mathsf{REG}}}
\def\CFL{{\mathsf{CFL}}}
\def\eqdef{\overset{\mbox{\tiny def}}{=}}
\newcommand{\niton}{\not\owns}

%FIRST & FOLLOW (rubtsov)
\def\first{\mathrm{ FIRST} }
\def\follow{\mathrm{ FOLLOW} }

% LL LR (rubtsov)
\def\LL{{\mathrm{LL}}}
\def\LR{{\mathrm{LR}}}

\newcounter{rowItemCount}
\newcounter{subRowItemCount}
\newcommand\rowItem{
    \setcounter{subRowItemCount}{0}
    \arabic{rowItemCount}.\addtocounter{rowItemCount}{1}}
\newcommand\subRowItem{
    \addtocounter{subRowItemCount}{1}
    \addtocounter{rowItemCount}{-1}
    \arabic{rowItemCount}.\arabic{subRowItemCount}.\addtocounter{rowItemCount}{1}}
    
\newcommand{\smalll}[1]{\overline{\overline{#1}}}
\newcommand{\smallo}{\bar{\bar{o}}}

\begin{document}
\maketitle
\subsection*{Задача 1}
\begin{enumerate}
\item \label{sat3sat} Докажем, что $\SAT\leqslant_m^p \3SAT$.
\item Теорема $\Rightarrow$ $\SAT\in\NPc$, $\ref{sat3sat}\Rightarrow\SAT\leqslant_m^p \3SAT\in \NP$. Поэтому из \ref{abnpc} следует, что $\3SAT\in \NPc$
\end{enumerate}
\subsection*{Задача 2}
\subsection*{(каноническое) Задача 16}
\begin{enumerate}
\item $\begin{Vmatrix}
A & b
\end{Vmatrix}^\Box=\begin{Vmatrix}
\underline{1/1} & 0/1 & 0/1 & 4/1\\
3/1 & 4/1 & 0/1 & 16/1\\
9/1 & 3/1 & 1/1 & 64/1\\
\end{Vmatrix}
\overset{r1 *= \frac{1}{1}}{\sim}
\begin{Vmatrix}
1/1 & 0/1 & 0/1 & 4/1\\
3/1 & 4/1 & 0/1 & 16/1\\
9/1 & 3/1 & 1/1 & 64/1\\
\end{Vmatrix}
\underset{\substack{\frac{16}{1}-\frac{4}{1}\frac{3}{1}=\frac{4}{1}\\\frac{64}{1}-\frac{4}{1}\frac{9}{1}=\frac{28}{1}}}{\overset{s 1 2 \frac{3}{1}, s 1 3 \frac{9}{1}}{\sim}}
\begin{Vmatrix}
1/1 & 0/1 & 0/1 & 4/1\\
0/1 & \underline{4/1} & 0/1 & 4/1\\
0/1 & 3/1 & 1/1 & 28/1\\
\end{Vmatrix}
\overset{r2 *= \frac{1}{4}}{\sim}\newline\sim
\begin{Vmatrix}
1/1 & 0/1 & 0/1 & 4/1\\
0/1 & 1/1 & 0/1 & 1/1\\
0/1 & 3/1 & 1/1 & 28/1\\
\end{Vmatrix}
\underset{\frac{28}{1}-\frac{1}{1}\frac{3}{1}=\frac{25}{1}}{\overset{s 2 3 \frac{3}{1}}{\sim}}
\begin{Vmatrix}
1/1 & 0/1 & 0/1 & 4/1\\
0/1 & 1/1 & 0/1 & 1/1\\
0/1 & 0/1 & 1/1 & 25/1\\
\end{Vmatrix}
$
\begin{enumerate}
\item $r_i *= \frac{c_1}{c_2}$~--- умножение строки $i$ на дробь $\frac{c_1}{c_2}$
\item $s i j \frac{c_1}{c_2}$~--- вычитание $i$-й строки, умноженной на дробь $\frac{c_1}{c_2}$ из $j$-й.
\end{enumerate}
\end{enumerate}
\subsection*{(каноническое) Задача 17}
\subsection*{(каноническое) Задача 18}
\subsection*{(каноническое) Задача 19}
\subsection*{(каноническое) Задача 20}
\subsection*{Вспомогательные утверждения}
\begin{enumerate}
\item \label{trans} $\leqslant_m^p$~--- транзитивно. Действительно, пусть $\Sigma_1^*\supseteq A\leqslant_m^p B\subseteq \Sigma_2^*$, $B\leqslant_m^p C\subseteq\Sigma_3^*$. Тогда существуют полиномиально-вычислимые функции $f_1\colon \Sigma_1^*\to \Sigma_2^*$, $f_2\colon \Sigma_2^*\to \Sigma_3^*$, причем $\forall x\in\Sigma_1^*\left(x\in A\Leftrightarrow f_1(x)\in B\right)$, $\forall y\in \Sigma_2^*\left(y\in B\Leftrightarrow f_2(y)\in C\right)$\newline
Фиксируем $x\in\Sigma_1^*$, определим $y=f_1(x)$. Тогда $x\in A\Leftrightarrow \underbrace{f_1(x)}_y\in B\Leftrightarrow f_2(f_1(x))\in C$\newline
Функция $g(x)\colon \Sigma_1^*\to\Sigma_3^*$ $g=f_2\circ f_1$ полиномиально-вычислима (как композиция полиномиально-вычислимых). Получаем, что существует полиномиально-вычислимая $g(x)$, такая что $\forall x\in\Sigma_1^*\left(x\in A\Leftrightarrow g(x)\in C\right)$, откуда $A\leqslant_m^p C\,\blacksquare$
\item \label{abnpc} Пусть $A\in\NPc$, и $A\leqslant_m^p B\in \NP$. Тогда $B\in \NPc$. Действительно, $A\in\NPc\Rightarrow\forall C\in \NP\hookrightarrow C\leqslant_m^p A$. Фиксируем $C\in\NP$. $A\leqslant_m^p B$, поэтому из \ref{trans} следует, что $C\leqslant_m^p B$. Поэтому $\forall C\in\NP\hookrightarrow C\leqslant_m^p B$. Значит, $B\in\NPc\,\blacksquare$
\end{enumerate}
\end{document}