\documentclass{letter}
\usepackage{geometry,amsmath,amssymb,enumerate,bbzm}
\geometry{letterpaper}

%%%%%%%%%% Start TeXmacs macros
\usepackage[utf8]{inputenc}
\usepackage[russian]{babel}
\newcommand{\nosymbol}{}
\newcommand{\tmem}[1]{{\em #1\/}}
\newcommand{\tmname}[1]{\textsc{#1}}
\newcommand{\tmop}[1]{\ensuremath{\operatorname{#1}}}
\newcommand{\tmsamp}[1]{\textsf{#1}}
\newcommand{\tmstrong}[1]{\textbf{#1}}
\newcommand{\tmtextbf}[1]{{\bfseries{#1}}}
\newcommand{\tmtexttt}[1]{{\ttfamily{#1}}}
\newenvironment{enumeratenumeric}{\begin{enumerate}[1.] }{\end{enumerate}}
\newenvironment{tmindent}{\begin{tmparmod}{1.5em}{0pt}{0pt} }{\end{tmparmod}}
\newenvironment{tmparmod}[3]{\begin{list}{}{\setlength{\topsep}{0pt}\setlength{\leftmargin}{#1}\setlength{\rightmargin}{#2}\setlength{\parindent}{#3}\setlength{\listparindent}{\parindent}\setlength{\itemindent}{\parindent}\setlength{\parsep}{\parskip}} \item[]}{\end{list}}
\newenvironment{tmparsep}[1]{\begingroup\setlength{\parskip}{#1}}{\endgroup}
%%%%%%%%%% End TeXmacs macros

\begin{document}

\begin{enumeratenumeric}
  \item \tmtextbf{Быстрое умножение.}
  
  Пусть $f \left( a, b \right) \overset{\tmop{def}}{=} \tmop{ab}$.
  Заметим, что
  
  $f \left( a, b \right) \underset{a \vdots 2}{=} \frac{a}{2} b + \frac{a}{2}
  b${\tmstrong{$\vee$}}$f \left( a, b \right) \underset{a \tmop{mod} 2 = 1}{=}
  \frac{a - 1}{2} b + \frac{a - 1}{2} b$,
  
  где деление целочисленное.
  
  Код на C:
  
  {\noindent}\begin{tmindent}
    \begin{tmparsep}{0em}
      int f(int a, int b)
      
      \{
      
      \ \ \ int c;
      
      \ \ \ if(a == 0 || b == 0) return(0);
      
      \ \ \ else
      
      \ \ \ \{
      
      \ \ \ \ \ \ \ if(a \% 2)
      
      \ \ \ \ \ \ \ \{
      
      \ \ \ \ \ \ \ \ \ \ \ c = f((a - 1) / 2, b);
      
      \ \ \ \ \ \ \ \ \ \ \ return(c + c + b);
      
      \ \ \ \ \ \ \ \}
      
      \ \ \ \ \ \ \ else
      
      \ \ \ \ \ \ \ \{
      
      \ \ \ \ \ \ \ \ \ \ \ c = f(a / 2, b);
      
      \ \ \ \ \ \ \ \ \ \ \ return(c +{\text{}} c);
      
      \ \ \ \ \ \ \ \}
      
      \ \ \ \}
      
      \}
    \end{tmparsep}
  \end{tmindent}{\hspace*{\fill}}{\medskip}
  
  На python:
  
  {\noindent}\begin{tmindent}
    \begin{tmparsep}{0em}
      def f(a, b):
      
      if a == 0 or b == 0:
      
      \ return 0
      
      else:
      
      \ if a \% 2 == 1:
      
      \ \ c = f((a - 1) / 2, b)
      
      \ \ return c + c + b
      
      \ else:
      
      \ \ c = f(a / 2, b)
      
      \ \ return c + c
    \end{tmparsep}
  \end{tmindent}{\hspace*{\fill}}{\medskip}
  
  \item \tmtextbf{Решето Эратосфена}
  \begin{enumerate}
    \item Доказательство возможности замены
    2i$\rightarrow i^2$:
    
    Пусть $P (i) =$''$\forall k \in \mathbbm{N}$, k$\geqslant 2 \forall n
    \in \mathbbm{N} \cap \left[ 2 ; i \right]$новый алгоритм
    вычеркнул все числа вида
    $\tmop{nk}$''\tmtexttt{{\tmstrong{{\tmem{{\tmsamp{}}}}}}}
    \begin{enumeratenumeric}
      \item Для i=2 (минимальное) 2i=i$^2$
      $\Rightarrow$доказываемое верно
      
      \item Пусть $P (i)$. Докажем, что $P (i + 1)$:
      
      $P (i) \Rightarrow \forall k \in \mathbbm{N}, k \geqslant 2 \forall n
      \in \mathbbm{N} \cap \left[ 2 ; i \right] $вычеркнуты
      числа вида nk$\Rightarrow$ вычеркнуты числа
      вида $n (i + 1)$, где \mathup{$n$\tmtexttt{{\tmsamp{{\tmname{$\in
      \mathbbm{N} \cap \left[ 2 ; i \right]$}}}}}}. Тогда
      алгоритм вычеркнет числа вида $(i + 1) (i +
      1)$, (i+2)(i+1)..., т.е. будут вычеркнуты числа
      n(i+1), где n\tmtexttt{{\tmsamp{{\tmname{$\in \mathbbm{N}, n
      \geqslant i + 1 \nosymbol \Rightarrow$}}}}}будут
      вычеркнуты все числа вида $n (i + 1)$, где
      $n${\tmname{$\in \mathbbm{N} \cap \left( \left[ 2 ; i \right] \cup
      \left[ i + 1 ; + \infty \right] \right)$}}$\Box$
      
      \item По аксиоме об индукции $\forall i \geqslant 2 P
      \left( i \right)$
    \end{enumeratenumeric}
    \item Внутренний цикл совершает $N - i^2$
    операций (при $i < \left\lfloor \sqrt{N} \right\rfloor$, при
    больших $i$ внутренний цикл занимает
    константное время, следовательно, к
    общему времени прибавляется $O (N - \sqrt{N})$,
    не учитываем) $\Rightarrow T_{\tmop{inner}} \sim \underset{i =
    2}{\overset{\left\lfloor \sqrt{N} \right\rfloor}{\sum}} \left( N - i^2
    \right) =$N$\left\lfloor \sqrt{N} \right\rfloor - \frac{\left\lfloor
    \sqrt{N} \right\rfloor \left( \left\lfloor \sqrt{N} \right\rfloor + 1
    \right) \left( 2 \left\lfloor \sqrt{N} \right\rfloor + 1 \right)}{6} + 1
    \overset{m = \sqrt{N}}{\leqslant} m^2 m - \frac{m \left( m + 1 \right)
    \left( 2 m + 1 \right)}{6} + 1 \leqslant m^3 \Rightarrow T_{\tmop{inner}}
    \in O \left( N^{3 / 2} \right)$
    
    \item На практике результат улучшится, так
    как будет совершаться заведомо меньше
    операций.
    
    \item Может возникнуть проблема
    переполнения переменной j (например, 2i
    около ограничения типа int, тогда $i^2$
    точно будет за пределами. Это приведет к
    ошибочному вычеркиванию чисел.
  \end{enumerate}
  \item \tmtextbf{Инверсии}
  
  
\end{enumeratenumeric}


\end{document}
