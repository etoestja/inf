\documentclass[a4paper]{article}
\usepackage[a4paper, left=5mm, right=5mm, top=5mm, bottom=5mm]{geometry}
%\geometry{paperwidth=210mm, paperheight=2000pt, left=5pt, top=5pt}
\usepackage[utf8]{inputenc}
\usepackage[english,russian]{babel}
\usepackage{indentfirst}
\usepackage{tikz}
\usetikzlibrary{automata,positioning,arrows}
\usepackage{amsmath}
\usepackage{enumerate}
\usepackage{hyperref}
\usepackage{amsfonts}
\usepackage{amssymb}
\DeclareMathOperator*{\argmin}{arg\,min}
\usepackage{wasysym}
\title{Статистическое обучение\\Задание 1}
\date{задано 2017.02.19}
\author{Сергей~Володин, 374 гр.}
\newcommand{\matrixl}{\left|\left|}
\newcommand{\matrixr}{\right|\right|}

\newcommand{\peq}{\mathrel{+}=}
\newcommand{\meq}{\mathrel{-}=}
\newcommand{\deq}{\mathrel{:}=}
\newcommand{\plpl}{\mathrel{+}+}

% пустое слово
\def\eps{\varepsilon}

% регулярные языки
\def\eqdef{\overset{\mbox{\tiny def}}{=}}
\newcommand{\niton}{\not\owns}

\begin{document}
\maketitle
\section*{Упражнение 1}
\begin{enumerate}
\item Неравенство Маркова: Если $X\geqslant 0$, то $P(X\geqslant \eps)\leqslant \frac{\mathbb{E}X}{\eps}$. Нужно: $P(X\geqslant \eps)=\frac{\mathbb{E}X}{\eps}$. Найдем $P(X<\eps)=1-\frac{\mathbb{E}X}{\eps}$, $f_X(x)=\frac{\mathbb{E}X}{x^2}$. Тогда $\mathbb{E}X=\int\limits_0^\infty xf_X(x)dx=\int\limits_0^\infty \mathbb{E}X\frac{dx}{x}$. Поскольку интеграл $\int\limits_0^\infty\frac{dx}{x}$ расходится, то $\mathbb{E}X=0$. Значит, $\boxed{X=0}$. Проверим: $0=P(0\geqslant\eps)=\frac{0}{\eps}\blacksquare$
\item Неравенство Чебышева: $P(|X-\mathbb{E}X|\geqslant a)\leqslant \frac{\sigma^2}{a^2}$. Если обозначить $\eta=|X-\mathbb{E}X|^2$, то получим неравенство Маркова. Возьмем предыдущий пример $\Rightarrow$ $\eta=0$ $\Rightarrow$ $X=c$ (константа). Проверим: $0=P(0\geqslant a)=\frac{0}{a^2}$ (для константы $\sigma=0$)
\end{enumerate}
\section*{Упражнение 2.1}
Имеем: $Y\geqslant 0$~--- случайная величина, числа $A\geqslant 2$, $B>0$. $\forall \eps\geqslant 0\hookrightarrow P(Y\geqslant \eps)\leqslant A\exp(-\frac{\eps^2}{B^2})$.

\begin{enumerate}
\item Оценим $\mathbb{E}e^{\lambda Y^2}=1+\int\limits_1^\infty P(e^{\lambda Y^2}>x)dx$. Перепишем $e^{\lambda Y^2}>x\Leftrightarrow \lambda Y^2>\ln x\Leftrightarrow Y>\sqrt{\frac{\ln x}{\lambda}}$. Значит, $\mathbb{E}e^{\lambda Y^2}\leqslant 1+A\int\limits_1^\infty x^{-1/\lambda B^2}dx=1+A\frac{1}{1/\lambda B^2-1}$ при условии $\lambda\in(0,1/B^2)$. Берём $\lambda=1/2B^2$. Тогда $\mathbb{E}e^{\lambda Y^2}\leqslant 1+A\leqslant 2A$ при $A\geqslant 2$
\item $\mathbb{E}Y=\sqrt{\frac{1}{\lambda}\ln e^{\lambda (\mathbb{E}Y)^2}}\underbrace{\leqslant}_{\mbox{\small Йенс. } e^{\lambda x^2}}\sqrt{\frac{1}{\lambda}\ln \mathbb{E}e^{\lambda Y^2}}\underbrace{\leqslant}_{(1)}\sqrt{2B^2\ln 2A}=\sqrt{2}B\sqrt{\ln 2A}$. Заметим, что при $A\geqslant 2$, $\sqrt{\ln 2A}\leqslant \sqrt{2\ln A}$. Тогда $\mathbb{E}Y\leqslant \boxed{2B\sqrt{\ln A}}$. То есть, проведено доказательство для $C=2$.
\end{enumerate}
\section*{Упражнение 2.2}
Имеем: $Y\geqslant 0$~--- случайная величина, числа $A\geqslant 2$, $B>0$. $\forall \eps\geqslant 0\hookrightarrow P(Y\geqslant \eps)\leqslant A\exp(-\frac{\eps}{B})$.
\begin{enumerate}
\item Оценим $\mathbb{E}e^{\lambda Y}=1+\int\limits_1^\infty P(e^{\lambda Y}>x)dx$. Рассмотрим $e^{\lambda Y}>x\Leftrightarrow Y>\frac{\ln x}{\lambda}$. $P(Y>\frac{\ln x}{\lambda})\leqslant Ae^{-\frac{\ln x}{\lambda B}}=Ax^{-1/\lambda B}$. Тогда $\mathbb{E}e^{\lambda Y}\leqslant 1+A\int\limits_1^\infty x^{-1/\lambda B}dx$ при $\lambda B<1$. Берем $\lambda=1/2B$. Тогда $\mathbb{E}e^{\lambda Y}\leqslant 1+A\leqslant 2A$
\item $\mathbb{E}Y=\frac{1}{\lambda}\ln e^{\lambda \mathbb{E}Y}\leqslant \frac{1}{\lambda}\ln \mathbb{E} e^{\lambda Y}\leqslant \frac{1}{\lambda}2A=2B\ln 2A\leqslant \boxed{4B\ln A}$
\end{enumerate}
\section*{Упражнение 3}
Случайная величина $X$~--- субгауссовская с параметром $\sigma$ $\Leftrightarrow$ $\mathbb{E}e^{\lambda X}\leqslant e^{\frac{\lambda^2\sigma^2}{2}}$.

Пусть $X_1,\,X_2$~--- субгауссовские с параметрами $\sigma_1$ и $\sigma_2$. $Y=X_1+X_2$. Доказать: $Y$~--- субгауссовская для некоторого $\sigma$.
\section*{Упражнение 4}
Плотность нормального распределения: $\psi(x)=f_{N(0,1)}(x)=\frac{1}{\sqrt{2\pi}}e^{-\frac{x^2}{2}}$. Тогда $\frac{d}{dx}\psi(x)=\frac{1}{\sqrt{2\pi}}(-2x/2)e^{-\frac{x^2}{2}}=-x\psi(x)$.

Значит, $x\psi(x)+\psi'(x)=0$
\section*{Упражнение 5}
Конспект
\section*{Задача 1}
?
\section*{Задача 2}
Конспект 1-е занятие
\end{document}