\documentclass[a4paper]{article}
\usepackage[a4paper, left=5mm, right=5mm, top=5mm, bottom=5mm]{geometry}
%\geometry{paperwidth=210mm, paperheight=2000pt, left=5pt, top=5pt}
\usepackage[utf8]{inputenc}
\usepackage[english,russian]{babel}
\usepackage{indentfirst}
\usepackage{tikz} %Рисование автоматов
\usetikzlibrary{automata,positioning,arrows,trees}
\usepackage{amsmath}
\usepackage[makeroom]{cancel} % зачеркивание
\usepackage{multicol,multirow} %Несколько колонок
\usepackage{hyperref}
\usepackage{tabularx}
\usepackage{amsfonts}
\usepackage{amssymb}
\DeclareMathOperator*{\argmin}{arg\,min}
\usepackage{listings}
\usepackage{wasysym}
\date{задано 2014.03.13}
\author{Сергей~Володин, 272 гр.}
\newcommand{\matrixl}{\left|\left|}
\newcommand{\matrixr}{\right|\right|}
% названия автоматов  (rubtsov)
\def\A{{\cal A}}
\def\B{{\cal B}}
\def\C{{\cal C}}

%классы сложности (rubtsov)
\def\PP{{\mathsf{P}}}
\def\NP{{\mathsf{NP}}}
\def\NPc{{\mathsf{NP}}\text{-}{\rm c}}
\def\coNP{{\rm co}\text{-}{\mathsf{NP}}}
\def\DTIME{{\mathsf{DTIME}}}
\def\NTIME{{\mathsf{NTIME}}}
\def\HALT{{\rm{HALT}}}
\def\SAT{{\rm{SAT}}}
\def\3SAT{{\rm{3\text{-}SAT}}}
\def\2SAT{{\rm{2\text{-}SAT}}}
\def\UNSAT{{\rm{UNSAT}}}
\def\LL{{\mathrm{LL}}}
\def\poly{{\rm{poly}}}

\title{Алгоритмы и модели вычислений.\\Задание 5: сложность вычислений: классы $\PP$, $\NP$, $\coNP$ II}

% алгоритмы (Рубцов)
\usepackage{verbatim}
\usepackage{listings}
\usepackage{algorithm2e}

%+= и -=, иначе разъезжаются...
\newcommand{\peq}{\mathrel{+}=}
\newcommand{\meq}{\mathrel{-}=}
\newcommand{\deq}{\mathrel{:}=}
\newcommand{\plpl}{\mathrel{+}+}

% пустое слово  (rubtsov)
\def\eps{\varepsilon}

% регулярные языки  (rubtsov)
\def\REG{{\mathsf{REG}}}
\def\CFL{{\mathsf{CFL}}}
\def\eqdef{\overset{\mbox{\tiny def}}{=}}
\newcommand{\niton}{\not\owns}

%FIRST & FOLLOW (rubtsov)
\def\first{\mathrm{ FIRST} }
\def\follow{\mathrm{ FOLLOW} }

% LL LR (rubtsov)
\def\LL{{\mathrm{LL}}}
\def\LR{{\mathrm{LR}}}

\newcounter{rowItemCount}
\newcounter{subRowItemCount}
\newcommand\rowItem{
    \setcounter{subRowItemCount}{0}
    \arabic{rowItemCount}.\addtocounter{rowItemCount}{1}}
\newcommand\subRowItem{
    \addtocounter{subRowItemCount}{1}
    \addtocounter{rowItemCount}{-1}
    \arabic{rowItemCount}.\arabic{subRowItemCount}.\addtocounter{rowItemCount}{1}}
    
\newcommand{\smalll}[1]{\overline{\overline{#1}}}
\newcommand{\smallo}{\bar{\bar{o}}}

\begin{document}
\maketitle
\subsection*{Задача 1}
\subsection*{Задача 2}
См. (каноническое) 21
\subsection*{Задача 3}
\begin{enumerate}
\item $\C\supset \NP\cup\coNP$.\begin{enumerate}
\item Пусть $L\in\NP$. Тогда (семинар) $L\leqslant_m^p\SAT\Leftrightarrow \exists f\colon\Sigma^*\to\Sigma^*\colon\forall x(x\in L\Leftrightarrow f(x)\in\SAT)$, $f$~--- вычислима за полиномиальное время. Определим $M_{\SAT}$: вычисляем за полиномиальное время (определение сводимости) $f(x)$ ($x$~--- вход), спрашиваем оракула $f(x)\overset{?}{\in}\SAT$ за $O(1)$. Ответ~--- ответ оракула (корректно из определения сводимости).  Время работы полиномиально: $T(|x|)=\poly(|x|)+O(1)=\poly(|x|)$.
\item Пусть $L\in\coNP$. Тогда $\overline{L}\leqslant_m^p\SAT\Leftrightarrow \exists f\colon\Sigma^*\to\Sigma^*\colon\forall x(x\in \overline{L}\Leftrightarrow f(x)\in\SAT)\Leftrightarrow\forall x(x\in L\Leftrightarrow f(x)\notin\SAT)$, $f$~--- вычислима за полиномиальное время. Определим $M_{\SAT}$: вычисляем за полиномиальное время (определение сводимости) $f(x)$ ($x$~--- вход), спрашиваем оракула $f(x)\overset{?}{\in}\SAT$ за $O(1)$. Ответ~--- \underline{противоположный} ответу оракула (корректно из определения сводимости). Время работы полиномиально: $T(|x|)=\poly(|x|)+O(1)=\poly(|x|)$.
\end{enumerate}
\item $\C\subset\NP\cup\coNP$
\end{enumerate}
\subsection*{(каноническое) Задача 21}
\subsection*{(каноническое) Задача 23}
\begin{enumerate}
\item $\Psi(x_1,x_2)\eqdef (\urcorner x_1\vee x_2)$. Базовое множество ($n=2$) $\{x_1,x_2,\urcorner x_1,\urcorner x_2\}$.\newline
Семейство подмножеств $A_\Psi=\big\{\{x_1,\urcorner x_1\},\,\{x_2,\urcorner x_2\},\,\{\urcorner x_1,x_2\}\big\}$.\newline
$\varangle A\eqdef\{x_1,\,x_2\}$. Получаем $A\cap\{x_1,\urcorner x_1\}\ni x_1$, $A\cap \{x_2,\urcorner x_2\}\ni x_2$, $A\cap \{\urcorner x_1, x_2\}\ni x_2$.\newline
Значит, $A$~--- протыкающее множество для $A_\Psi$, и $|A|=2$.
\item $\chi(x_1,x_2,x_3)\eqdef (x_1\vee x_2\vee x_3)\wedge(\urcorner x_1\vee \urcorner x_2)\wedge(x_1\vee \urcorner x_2)\wedge (\urcorner x_1\vee x_2\vee x_3)\wedge\urcorner x_3$. Семейство подмножеств ($n=3$) $A_\chi=\big\{\{x_1,\urcorner x_1\},\{x_2,\urcorner x_2\},\{x_3,\urcorner x_3\},\{x_1,x_2,x_3\},\{\urcorner x_1,\urcorner x_2\},\{x_1,\urcorner x_2\},\{\urcorner x_1,x_2,x_3\},\{\urcorner x_3\}\big\}$. Пусть $A$~--- протыкающее множество. Тогда $A\cap \{\urcorner x_3\}\neq \varnothing\Rightarrow A\ni \urcorner x_3$. Также $A\cap\{x_1,\urcorner x_1\}\neq\varnothing$, поэтому $A$ содержит $x_1$ или $\urcorner x_1$. Аналогично $x_2\in A$ или $\urcorner x_2\in A$. Получаем, что $A$ содержит не менее трех элементов. Предположим, что их ровно 3. Рассмотрим все возможные 4 случая (или$\times$или раньше по тексту):\begin{enumerate}
\item $A=\{x_1,x_2,\urcorner x_3\}$. Тогда $A\cap \{\urcorner x_1,\urcorner x_2\}=\varnothing$~--- противоречие.
\item $A=\{x_1,\urcorner x_2,\urcorner x_3\}$. Тогда $A\cap \{\urcorner x_1,x_2,x_3\}=\varnothing$~--- противоречие.
\item $A=\{\urcorner x_1,x_2,\urcorner x_3\}$. Тогда $A\cap \{x_1,\urcorner x_2\}=\varnothing$~--- противоречие.
\item $A=\{\urcorner x_1,\urcorner x_2,\urcorner x_3\}$. Тогда $A\cap\{x_1,x_2,x_3\}=\varnothing$~--- противоречие.
\end{enumerate}
Получаем, что $A$ содержит более трех элементов $\blacksquare$
\end{enumerate}
\end{document}