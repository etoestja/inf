\documentclass[a4paper]{article}
\usepackage[a4paper, left=5mm, right=5mm, top=5mm, bottom=5mm]{geometry}
%\geometry{paperwidth=210mm, paperheight=2000pt, left=5pt, top=5pt}
\usepackage[utf8]{inputenc}
\usepackage[english,russian]{babel}
\usepackage{indentfirst}
\usepackage{tikz}
\usepackage{cancel}
\usetikzlibrary{automata,positioning,arrows}
\usepackage{amsmath}
\usepackage{enumerate}
\usepackage{hyperref}
\usepackage{amsfonts}
\usepackage{amssymb}
\DeclareMathOperator*{\argmin}{arg\,min}
\usepackage{wasysym}
\title{Статистическое обучение\\Задание 2}
\date{задано 2017.02.12}
\author{Сергей~Володин, 374 гр.}
\newcommand{\matrixl}{\left|\left|}
\newcommand{\matrixr}{\right|\right|}

\newcommand{\peq}{\mathrel{+}=}
\newcommand{\meq}{\mathrel{-}=}
\newcommand{\deq}{\mathrel{:}=}
\newcommand{\VC}{\mbox{VC}}
\newcommand{\plpl}{\mathrel{+}+}
\newcommand{\sign}{\mbox{sign}\,}
\newcommand{\F}{\mathcal{F}}
\newcommand{\R}{\mathbb{R}}
\newcommand{\conv}{\mbox{conv}\,}
\newcommand{\E}{\mathbb{E}}
\newcommand{\D}{\mathbb{D}}

% пустое слово
\def\eps{\varepsilon}

% регулярные языки
\def\eqdef{\overset{\mbox{\tiny def}}{=}}
\newcommand{\niton}{\not\owns}

\begin{document}
\maketitle
\section*{Meta}
Делал один. Список ссылок:
\begin{enumerate}
\item \href{http://www.jmlr.org/proceedings/papers/v32/mohri14-supp.pdf}{Contraction Lemma}
\item \href{https://ocw.mit.edu/courses/mathematics/18-657-mathematics-of-machine-learning-fall-2015/assignments/MIT18_657F15_PS2.pdf}{Desymmetrization inequality step-by-step tasks}
\item \href{http://www.cs.huji.ac.il/~shais/UnderstandingMachineLearning/understanding-machine-learning-theory-algorithms.pdf}{UML}
\end{enumerate}
\section*{Упражнение 1.1}
Класс функций $S=\{f\colon [0,1]\to\{0,1\}\big| \#\{x|f(x)=1\}<+\infty\}$

$R(S)=\frac{1}{n}\mathbb{E}_{x_i}\mathbb{E}_{\eps_i}\sup\limits_{f\in S}\big|\sum\limits_{i=1}^n\eps_if(x_i)\big|$

Фиксируем $x_i$ и $\eps_i$. Поскольку ${x_i}$~--- конечный набор точек, то для каждого набора из $\{0,1\}^n$ существует $f\in S$, дающая эти значения.

Необходимо максимизировать $|\sum \eps_i q_i|$, где $\eps_i\in\{-1,1\}$, $q_i\in\{0,1\}$. Пусть $k$~--- количество $+1$ в $\{\eps_i\}$.

Тогда $\sup|\sum \eps_i q_i|=\max\{k,n-k\}\geqslant n/2$. Пусть иначе. Тогда $k<n/2$ и $k>n/2$~--- противоречие.

Тогда $R(S)=\frac{1}{n}\mathbb{E}_{x_i}\mathbb{E}_{\eps_i}\sup\limits_{f\in S}\big|\sum\limits_{i=1}^n\eps_if(x_i)\big| \geqslant \frac{1}{n}\mathbb{E}_{x_i}\mathbb{E}_{\eps_i}\frac{n}{2}=\frac{1}{2}$ $\blacksquare$
\section*{Упражнение 1.2}
Для любого $n$, для любого распределения $P(X,Y)$ имеем
$$\forall \eps \,P(\sup\limits_{f\in S}|\E f(x)-1/n\sum\limits_{i=1}^n f(x_i)|\geqslant\eps)\geqslant 1/2-1/\sqrt{n}\underset{n\to\infty}{\not\to}0$$
Значит, оба варианта (по вероятности, п.н.) РЗБЧ не выполнены.
\section*{Упражнение 2.1}
$n\leqslant d,\,\mathcal{F}=\{x\to \sign(\theta^Tx)\big| \theta\in\R^d\}$

Рассмотрим $R_n(\F)=\frac{1}{n}\mathbb{E}_{\eps_i}\sup\limits_{f\in\F}|\sum\limits_{i=1}^n \eps_i f(x_i)|$

Заметим, что $\boxed{R_n(\F)\leqslant 1}$.

Фиксируем $\{x_i\}$ и $\{\eps_i\}$. Выберем $\theta$: $\underbrace{\matrixl\begin{array}{c}
x_1^T\\
...\\
x_n^T
\end{array}\matrixr}_X\theta=\matrixl\begin{array}{c}
\eps_1\\
...\\
\eps_n
\end{array}
\matrixr$. Это можно сделать по теореме Кронекера-Капелли: матрица $X\colon n\times d$ имеет ранг $n\leqslant d$, значит, при приписывании столбца $(\eps_1,...,\eps_n)$ ранг также будет равен $n$.

Мы нашли $f\in\F\colon $ $|\sum\limits_{i=1}^n \eps_i f(x_i)|=|\sum\limits_{i=1}^n \eps_i \sign(\theta^Tx_i)|=\sum\limits_{i=1}^n \eps_i^2=n$. Значит, $\boxed{R_n(\F)\geqslant 1}$ $\blacksquare$
\section*{Упражнение 2.2}
Если разрешить брать $>d$ точек в обучающую выборку, то класс обучаем, так как линейных классов известно, что $\VC(\F)<+\infty$.
\section*{Упражнение 3}
$$
\begin{array}{ccccc}
G(F)&=&\frac{1}{n}\E_X&\E_{g_i\sim N(0,1)}&\sup\limits_{f\in \F}|\sum\limits_{i=1}^n g_if(x_i)|\\
R(F)&=&\frac{1}{n}\E_X&\E_{\eps}&\sup\limits_{f\in \F}|\sum\limits_{i=1}^n \eps_if(x_i)|
\end{array}$$

Фиксируем $\{x_i\}$. Рассмотрим $\varphi(\vec{t})=\sup\limits_{f\in\F}|\sum\limits_{i=1}^n f(x_i)\eps_i t_i|$. Тогда $\varphi(\alpha\vec{t}_1+(1-\alpha)\vec{t}_2)=\sup\limits_{f\in\F}|\alpha\sum\limits_{i=1}^n f(x_i)\eps_i t_{1i}+(1-\alpha)\sum\limits_{i=1}^n f(x_i)\eps_i t_{2i}|\boxed{\leqslant}$. $|x|$~--- выпуклая, поэтому $\boxed{\leqslant}\sup\limits_{f\in\F}\alpha|\sum\limits_{i=1}^n f(x_i)\eps_i t_{1i}|+(1-\alpha)|\sum\limits_{i=1}^n f(x_i)\eps_i t_{2i}|\leqslant \alpha\varphi(\vec{t}_1)+(1-\alpha)\varphi(\vec{t}_2)$.

Пусть $\{g_i\}_{i=1}^n\colon g_i\sim N(0,1)$ и $\{g_i\}$~--- i.i.d.

Обозначим $c=\mathbb{E}|g_i|=2\int\limits_{x=0}^\infty x e^{-x^2/2}dx >0$. Тогда $\E g_i/c=1$.

Рассмотрим $\varphi(\vec{1})=\varphi(\E\vec{g}/c)=\sup\limits_{f\in F}|\sum\limits_{i=1}^n \eps_i f(x_i) \underbrace{\frac{\E g_i}{c}}_1|\leqslant \E_{g}\frac{1}{c}\sup\limits_{f\in \F}|\sum\limits_{i=1}^n \eps_if(x_i)|g_i||$.

Рассмотрим $R_n(\F)=\frac{1}{n}\E_x\E_\eps\sup\limits_{f\in\F}|\sum\limits_{i=1}^n \eps_if(x_i)|=\frac{1}{n}\E_x\E_\eps \varphi(\vec{1})\leqslant \frac{1}{nc}\E_x\E_\eps\E_g \sup\limits_{f\in \F}|\sum\limits_{i=1}^n \eps_if(x_i)|g_i||\overset{d}{=}$. Поскольку $\eps |g|\overset{d}{=}g$,

$\overset{d}{=}\frac{1}{nc}\E_x\E_g\sup\limits_{f\in \F}|\sum\limits_{i=1}^n f(x_i)g_i|=\frac{1}{c}G(\F)$.

Значит, $R(\F)\leqslant CG(\F)$, где $C=1/c$.
\section*{Упражнение 4.1}
\begin{enumerate}
\item Все функции $X\to \{0,1\}$, то есть, $\F=\{0,1\}^X$. Для любого набора из $n$ точек найдется $2^n$ функций из $\F$, принимающих все возможные значения из $\{0,1\}^n$ (по построению). Значит, $\VC(\F)=\infty$. Верхняя оценка выполняется как равенство: $$S_\F(n)=2^n=\sum\limits_{k=0}^\infty C_n^k$$
\item $\F = \{0\}\cup \{[x=x_0]\big|x_0\in X \}$. Рассмотрим выборку $\{x_i\}\subseteq\F$. Применение всевозможных функций из $\F$ даст матрицу
$$\matrixl\begin{array}{ccccc}
0 & 0 & 0 & ... & 0\\
1 & 0 & 0 & ... & 0\\
0 & 1 & 0 & ... & 0\\
0 & 0 & 1 & ... & 0\\
&&...&&\\
0 & 0 & 0 & ... & 1\\
\end{array}\matrixr$$
\end{enumerate}
То есть, $S_\F(n)=n+1$. Значит, $d=\VC(\F)=1$. Проверим: $S_\F(n)=n+1\overset{?}{=}\sum\limits_{k=0}^dC_n^k=1+n$ $\blacksquare$
\section*{Упражнение 4.2}
$\F=\{[x\in A]\big| A\in\R^d \mbox{~--- выпуклое, замкнутое}\}$.

Доказать: $VC(\F)=\infty$. $VC(\F)=\sup\{d\big| S_\F (d)=2^d\}$.

$S_\F (d)=\sup\limits_{\{z_i\}_{i=1}^d} n_{\mbox{\small unique}}(\{z_i\})$.

То есть, необходимо найти $X_n=\{x_i\}_{i=1}^n\subseteq \R^d$, такие что для каждого подмножества $\hat{X}_n\subseteq X_n$ существует выпуклое замкнутое множество $A_n$, содержащее точки $\hat{X}_n$ и не содержащее $X_n\setminus \hat{X}_n$.

Возьмем $n$ точек, расположенных на окружности $C=\{(\cos\varphi,\sin\varphi, 0,...,0)\big| \varphi\in\R\}$ на равном расстоянии друг от друга: $x_k=(\cos \frac{2\pi k}{n},\sin \frac{2\pi k}{n},0,...,0)$. Фиксируем подмножество $\hat{X}_n$.

Возьмем $A_n=\conv\hat{X}_n$~--- выпуклое и замкнутое. Тогда $\hat{X}_n\subseteq A_n$. Поскольку $A_n\cap C=\hat{X}_n$, $A_n\cap (X_n\setminus \hat{X}_n)=\varnothing$ $\blacksquare$
\section*{Упражнение 4.3}
\section*{Упражнение 4.4}
\section*{Упражнение 4.5}
\section*{Упражнение 5.1}
{\bf Лемма} (\url{http://www.jmlr.org/proceedings/papers/v32/mohri14-supp.pdf} Lemma 8): Пусть $\{\varphi(x_i)\}_{i=1}^n$~--- $L$-липшицевы. Тогда
$$\frac{1}{n}\E_\eps \sup\limits_{f\in F}|\sum\limits_{i=1}^n \eps_i \varphi_i(f(x_i))|\leqslant \frac{L}{n}\E_\eps \sup\limits_{f\in F}|\sum\limits_{i=1}^n \eps_i f(x_i)|$$

Определим $\varphi_i(x)=[y_i\neq f(x)]$. Тогда $|\varphi_i(x_k)-\varphi_i(x_l)|\leqslant \frac{1}{2}|f(x_k)-f(x_l)|$, так как в случае $f(x_k)\neq f(x_l)$ имеем $|f(x_k)-f(x_l)|=2$, $|\varphi_i(x_k)-\varphi(x_l)|=1$

В обозначениях курса $(l\cdot f)(x_i)=[y_i\neq f(x_i)]$ (то есть, это не совсем композиция)

Значит, 

$$\boxed{R(l\cdot \F)\leqslant \frac{1}{2}R(\F)}$$

\section*{Упражнение 5.2}
Обозначим $l(y,\hat{y})=(y-\hat{y})^2$,  $\varphi_i(x)=l(y_i, f(x))$. Рассмотрим $|\varphi_i(x)-\varphi_i(x')|=|(y-f(x))^2-(y-f(x'))^2|=|(y-f(x)-y+f(x')|\cdot |y-f(x)+y-f(x')|=|f(x')-f(x)|\cdot |2y-f(x)-f(x')|\leqslant 2(a+b)|f(x)-f(x')|$, так как $|f(x)|\leqslant \sup\limits_{f\in \F}\|f\|_\infty\leqslant b$, $|Y|\leqslant a$.

Значит, $\varphi_i$~--- $2(a+b)$-липшицева. Значит,
$$
\boxed{R(l\cdot \F)\leqslant 2(a+b)R(\F)}
$$
\section*{Задача 1}
\section*{Задача 2}
\begin{enumerate}
\item (2)$\Rightarrow$(3). Пусть $\VC(\F)=d\leqslant\infty$. Тогда $\forall n$ $S_\F(n)\leqslant \sum\limits_{i=0}^d\left(\begin{array}{c}n\\i\end{array}\right)\leqslant (n+1)^d$.

Рассмотрим $R(F)=\frac{1}{n}\E_\eps\E_z\sup\limits_{f\in F}|\sum\limits_{i=1}^n f(x_i)\eps_i|$

Фиксируем $\{z_i\}$. Определим $A=\{\left(f(x_1),...,f(x_n)\right) \big| f\in \F\}$. Поскольку $S_\F(n)\leqslant (n+1)^d$, $N=|A|\leqslant (n+1)^d$.

Рассмотрим $R(A)=\frac{1}{n}\E_\eps\sup\limits_{a\in A}|(\eps, a)|\leqslant\frac{1}{n}\max\limits_{a\in A}\|a\|_2\sqrt{2\ln 2N}\leqslant \frac{1}{\sqrt{n}}\sqrt{2\ln 2N}\leqslant\sqrt{\frac{2d\ln (n+1)}{n}}\to 0,\,n\to\infty$

Первое неравенство основано на следующих фактах: $\E e^{\lambda a_i\eps_i}\leqslant e^{\lambda^2 a_i^2/2}$, $\E\max |y_i|\leqslant\sigma\sqrt{2\ln 2N}$, где $\sigma=\max \|a\|_2\leqslant\sqrt{n}$

Тогда $R(F)=\E_z R(A(z))\leqslant\sqrt{\frac{2d\ln (n+1)}{n}}\to 0,\,n\to\infty$.

Раccмотрим $\mathbb{E}_x\sup\limits_{f\in \F}|Pf-P_nf|=\E_x\sup\limits_{f\in\F}|\E f-\frac{1}{n}\sum\limits_{i=1}^n f(x_i)|=\E_x\sup\limits_{f\in\F}|\E'_x \frac{1}{n}\sum\limits_{i=1}^n (f(x_i)-f(x_i'))\leqslant\E'\E_x\sup\limits_{f\in\F}| \frac{1}{n}\sum\limits_{i=1}^n (f(x_i)-f(x_i')|=\E'_x\E_x\E_\eps\sup\limits_{f\in\F}| \frac{1}{n}\sum\limits_{i=1}^n (\eps_i(f(x_i)-f(x_i'))|\leqslant 2R(\F)\to 0,\,n\to\infty$. Поскольку $\xi_n=\sup\limits_{f\in \F}|Pf-P_nf|$ удовлетворяет условию ограниченных разностей, $P(|\E \xi_n-\xi_n|\geqslant t)\leqslant 2e^{-nt^2/2/c^2}$, $c=||f||_\infty=1$.

Рассмотрим $P(\sup\limits_{m\geqslant n}|\xi_m-\E\xi_m|\geqslant t)\leqslant 2\sum\limits_{m=n}^\infty (\underbrace{\exp(-t^2/2c)}_{q<1})^m=\frac{2q^n}{1-q}\to 0,\,n\to\infty$. Тогда $P(\sup\limits_{m\geqslant n}\xi_m\geqslant t-\E\xi_m)\to 0$

Значит, $\lim\limits_{n\to\infty}\sup\limits_P P\{\sup\limits_{m\geqslant n}\sup\limits_{f\in\F}|Pf-P_mf|\geqslant\eps\}=0$. Значит, верно (3)~--- РЗБЧ для $\F$.

\item Докажем, что $(3)\Rightarrow(1)$, то есть, из РЗБЧ следует обучаемость. Рассмотрим $\xi_n=\sup\limits_{f\in\F}|Pf-P_nf|$. Имеем: $\sup\limits_PP(\sup\limits_{m\geqslant n}\xi_m\geqslant\eps)\to 0,\,n\to\infty$. Значит, что $\sup\limits_P P(\xi_m\geqslant \eps)\to 0$, $n\to\infty$. Значит, $\E\xi_n\to 0$.

Рассмотрим $\sup\limits_{g\in l\cdot \F}|Pg-P_ng|\leqslant 2R(l\cdot \F)\leqslant 2R(\F)\underbrace{\leqslant}_{\mbox{N3}}\frac{2\sup\|f\|_\infty}{\sqrt{n}}+4\mathbb{E}\sup\limits_{f\in\F}|Pf-P_nf|=2/\sqrt{n}+4\E \xi_n\to 0$

Пусть $\hat{f}$~--- МЭР. Рассмотрим $L(\hat{f})-L(f^*_\F)=L(\hat{f})-L_n(\hat{f})+\underbrace{L_n(\hat{f})-L_n(f^*_\F)}_{\leqslant 0}+L_n(f^*_\F)-L(f^*_\F)\leqslant 2\sup\limits_{f\in\F}|L(f)-L_n(f)|=2\sup\limits_{g\in l\cdot\F} |Pg-P_ng|\to 0$. Значит, верно (1)~--- обучаемость $\F$.

\item $(1)\Rightarrow(2)$.
\end{enumerate}
\section*{Задача 3.1}

\section*{Задача 3.2}
Пусть $\F$~--- класс всех линейных классификаторов. Имеем: $\VC(\F)=p+1<\infty$.

Рассмотрим набор из $p+2$ точек в общем положении.
\end{document}