\documentclass[a4paper]{article}
\usepackage[a4paper, left=5mm, right=5mm, top=5mm, bottom=5mm]{geometry}
%\geometry{paperwidth=210mm, paperheight=2000pt, left=5pt, top=5pt}
\usepackage[utf8]{inputenc}
\usepackage[english,russian]{babel}
\usepackage{indentfirst}
\usepackage{tikz}
\usepackage{cancel}
\usetikzlibrary{automata,positioning,arrows}
\usepackage{amsmath}
\usepackage{enumerate}
\usepackage{hyperref}
\usepackage{amsfonts}
\usepackage{amssymb}
\DeclareMathOperator*{\argmin}{arg\,min}
\DeclareMathOperator*{\argmax}{arg\,max}
\usepackage{wasysym}
\title{Статистическое обучение\\Задание 4}
\date{задано 2017.05.08}
\author{Сергей~Володин, 374 гр.}
\newcommand{\matrixl}{\left|\left|}
\newcommand{\matrixr}{\right|\right|}

\newcommand{\peq}{\mathrel{+}=}
\newcommand{\meq}{\mathrel{-}=}
\newcommand{\deq}{\mathrel{:}=}
\newcommand{\VC}{\mbox{VC}}
\newcommand{\plpl}{\mathrel{+}+}
\newcommand{\sign}{\mbox{sign}\,}
\newcommand{\F}{\mathcal{F}}
\newcommand{\R}{\mathbb{R}}
\newcommand{\conv}{\mbox{conv}\,}
\newcommand{\E}{\mathbb{E}}
\newcommand{\D}{\mathbb{D}}

% пустое слово
\def\eps{\varepsilon}

% регулярные языки
\def\eqdef{\overset{\mbox{\tiny def}}{=}}
\newcommand{\niton}{\not\owns}

\begin{document}
\maketitle
\section*{Meta}
Делал один. Список ссылок:
\begin{enumerate}
\item filled later
\end{enumerate}
\section*{Упражнение 1}

\section*{Упражнение 2}
Имеем матрицу $X\colon n\times d$, параметр $\theta\in\R^d$, отклик $Y\in\mathbb{R}^n$. Функция потерь:

$$L(\theta)=\frac{1}{n}\|Y-X\theta\|_2^2+\tau\|\theta\|_2^2=\frac{1}{n}\sum\limits_{i=1}^n(y_i-\sum\limits_{s=1}^dx_{is}\theta_s)^2+\tau\sum\limits_{k=1}^d\theta_k^2$$

$L$ дифференцируема как композиция дифференцируемых функций. Найдём

$$\frac{\partial L(\theta)}{\partial\theta_j}=\frac{\partial}{\partial\theta_j}\left(\frac{1}{n}\sum\limits_{i=1}^n(y_i-\sum\limits_{s=1}^dx_{is}\theta_s)^2+\tau\sum\limits_{k=1}^d\theta_k^2\right)=2\theta_j\tau-\frac{1}{n}\sum\limits_{i=1}^n2(y_i-(X\theta)_i)x_{ij}=2\theta_j\tau-\frac{2}{n}(X^T(Y-X\theta))_j$$

Значит,
$$\nabla L(\theta)=2\theta\tau-\frac{2}{n}X^TY+\frac{2}{n}X^TX\theta$$

Решим
$$
\nabla L(\theta)=0\Leftrightarrow (n\tau E+X^TX)\theta=X^TY
$$
Поскольку $X^TX$~--- симметричная матрица, ее минимальное собственное число неотрицательное: $\lambda_{\min}(X^TX)\geqslant 0$. Значит, при $\tau>0$ матрица $n\tau E+X^TX$ обратима.

Получаем $$\hat{\theta}=(n\tau E+X^TX)^{-1}X^TY$$

Поскольку это единственный ноль градиента выпуклой функции, глобальный минимум находится в $\hat{\theta}$

Обозначим $Q=(X^TX+\beta I)^{-1}$, $\beta=\tau n$. Рассмотрим $\E \hat{\theta}=\E QX^TY=QX^T\E Y$. Поскольку $Y=X\theta^*+\eps$, $\E Y=X\theta^*$.

Тогда $\E\hat{\theta}=QX^TX\theta^*$.

Рассмотрим смещение $\Delta=\theta^*-\E\hat{\theta}=QQ^{-1}\theta^*QX^TX\theta^*=Q(X^TX+\beta I-X^TX)\theta^*=Q\beta\theta^*$.

$\|\Delta\|_2\leqslant \beta \|Q\|_2\|\theta^*\|_2$. Рассмотрим $\|Q\|_2=\|(X^TX+\beta I)^{-1}\|_2=\frac{1}{\|X^TX+\beta I\|_2}\leqslant \beta$.

Значит, $\|\Delta\|_2\leqslant \|\theta^*\|_2$
\section*{Упражнение 3}
\section*{Упражнение 4}
\section*{Задача 1}
\section*{Задача 2}
\end{document}