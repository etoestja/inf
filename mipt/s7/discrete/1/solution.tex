\documentclass[a4paper]{article}
\usepackage[a4paper, left=25mm, right=25mm, top=25mm, bottom=25mm]{geometry}
%\geometry{paperwidth=210mm, paperheight=2000pt, left=5pt, top=5pt}
\usepackage[utf8]{inputenc}
\usepackage[english,russian]{babel}
\usepackage{indentfirst}
\usepackage{tikz} %Рисование автоматов
\usetikzlibrary{automata,positioning,arrows,trees}
\usepackage{amsmath}
\usepackage[makeroom]{cancel} % зачеркивание
\usepackage{multicol,multirow} %Несколько колонок
\usepackage{hyperref}
\usepackage{tabularx}
\usepackage{amsfonts}
\usepackage{amssymb}
\DeclareMathOperator*{\argmin}{arg\,min}
\usepackage{wasysym}
\title{Дискретная оптимизация.\\Задание 1}
\date{задано 2016.09.07}
\author{Сергей~Володин, 374 гр.}
\newcommand{\matrixl}{\left|\left|}
\newcommand{\matrixr}{\right|\right|}

% алгоритмы (Рубцов)
\usepackage{verbatim}
\usepackage{listings}
\usepackage{algorithm2e}

%+= и -=, иначе разъезжаются...
\newcommand{\peq}{\mathrel{+}=}
\newcommand{\meq}{\mathrel{-}=}
\newcommand{\deq}{\mathrel{:}=}
\newcommand{\plpl}{\mathrel{+}+}

% пустое слово  (rubtsov)
\def\eps{\varepsilon}
\newcommand{\smalll}[1]{\overline{\overline{#1}}}
\newcommand{\smallo}{\bar{\bar{o}}}

\begin{document}
\maketitle

\subsection*{Сложные, задача 6}
Условие: есть 4 проекта, в каждый можно вложить от 0 до 5 млн. Сумма вложений ограничена.

Обозначим $x_{ij}\in\{0,1\}$~--- вкладываем ли $j$ млн в $i$-й проект, $i\in\overline{1,4}$, $j\in\overline{0,5}$.

Если $x_{ij}=1$, то отдаем $j$ млн в $i$-й проект, а затем получаем $q_{ij}$ денег. В условии дана прибыль, т.е. величина $q_{ij}-j$, причем в процентах от вложения.

Прибыль от вложения $j$ млн в $i$-й проект в процентах от вложения:

$A_{ij}=\matrixl\begin{array}{ccccccc}
0.00 & 0.28 & 0.45 & 0.65 & 0.78 & 0.90\\
0.00 & 0.25 & 0.41 & 0.55 & 0.65 & 0.75\\
0.00 & 0.15 & 0.25 & 0.40 & 0.50 & 0.62\\
0.00 & 0.20 & 0.35 & 0.42 & 0.48 & 0.53\\
\end{array}
\matrixr$

Суммарная прибыль $Q(x)=\sum\limits_{ij}jA_{ij}x_{ij}$

Ограничения: вложить в каждый проект можно только 1 раз: $\sum\limits_{j} x_{ij}\leqslant 1$, а также сумма вложения не превышает $X$ (ресурсы ограничены, условие требует их распределения).

Задача оптимизации:
$$\sum\limits_{ij}jx_{ij}A_{ij}\to\max_{\begin{cases}
	x_{ij}\in\{0,1\}\\
	\forall i \sum\limits_{j} x_{ij}\leqslant 1\\
	\sum\limits_{i,j}jx_{ij}\leqslant X
	\end{cases}}$$
Это задача ЦЛП.

Решение в AMPL \url{http://ampl.com/cgi-bin/ampl/amplcgi}:
\begin{verbatim}
set Projects;
set Investments;
param Income {i in Projects, j in Investments};
param MaxInvestment;

var x{i in Projects, j in Investments} binary;

maximize Profit:
sum{i in Projects, j in Investments} j*Income[i,j]*x[i,j];
subject to const1 {i in Projects}:
sum{j in Investments} x[i,j] <= 1;
subject to const2:
sum {i in Projects, j in Investments} x[i,j] * j <= MaxInvestment;

data;  ############ DATA STARTS HERE ############


set Projects := p1 p2 p3 p4;
set Investments := 0 1 2 3 4 5;

param Income : 0 1 2 3 4 5 :=
p1 0 0.28 0.45 0.65 0.78 0.90
p2 0 0.25 0.41 0.55 0.65 0.75
p3 0 0.15 0.25 0.40 0.50 0.62
p4 0 0.20 0.35 0.42 0.48 0.53;

param MaxInvestment := 6;

LP_SOLVE 4.0.1.0: optimal, objective 4.75
42 simplex iterations
25 branch & bound nodes: depth 13
:     _varname   _var    :=
1    "x['p1',0]"   0
2    "x['p1',1]"   0
3    "x['p1',2]"   0
4    "x['p1',3]"   0
5    "x['p1',4]"   0
6    "x['p1',5]"   1
7    "x['p2',0]"   0
8    "x['p2',1]"   1
9    "x['p2',2]"   0
10   "x['p2',3]"   0
11   "x['p2',4]"   0
12   "x['p2',5]"   0
13   "x['p3',0]"   0
14   "x['p3',1]"   0
15   "x['p3',2]"   0
16   "x['p3',3]"   0
17   "x['p3',4]"   0
18   "x['p3',5]"   0
19   "x['p4',0]"   0
20   "x['p4',1]"   0
21   "x['p4',2]"   0
22   "x['p4',3]"   0
23   "x['p4',4]"   0
24   "x['p4',5]"   0
;
== 21 ==========================

\end{verbatim}

В примере $X=6$, оптимальное инвестирование~--- 5 млн в 1-й проект и еще 1 во второй.

\subsection*{Простые, задача 13}
Обозначим $c=(200,230,100)$~--- стоимости, $A=\matrixl\begin{array}{ccc}
40 & 40 & 20\\
5 & 3 & 1
\end{array}
\matrixr$~--- ресурсы по клею и коже на изделие.
Обозначим $b=(3000000, 4000000)$~--- имеющиеся ресурсы.

Обозначим $x=(x_1, x_2, x_3)$~--- количество обуви каждого из типов соответственно.

Тогда задача оптимизации принимает вид:
$$c^Tx\to\max_{\begin{cases}
Ax\leqslant b\\x\in\mathbb{N}^3
	\end{cases}}$$
То есть, нужно максимизировать прибыль $c^Tx$, укладываясь в ограничения по ресурсам $Ax\leqslant b$.


Или, в действительных переменных:
$$c^Tx\to\max_{\begin{cases}
	Ax\leqslant b\\x\geqslant 0\\x\in\mathbb{R}^3
	\end{cases}}$$

Двойственная задача:
$$b^Ty\to\min_{\begin{cases}
A^Ty\geqslant c\\
y\geqslant 0\\
y\in\mathbb{R}^2\\
\end{cases}}$$
Смысл: какие минимальные цены $y$ установить на ресурсы, чтобы чтобы покупателю было уже невыгодно собирать из них обувь?

Решение в AMPL \url{http://ampl.com/cgi-bin/ampl/amplcgi} (прямая):
\begin{verbatim}
var x1;
var x2;
var x3;
maximize Profit: 200 * x1 + 230 * x2 + 100 * x3;
subject to glue: 40 * x1 + 40 * x2 + 20 * x3 <= 3000000;
subject to leather: 5 * x1 + 3 * x2 + 1 * x3 <= 4000000;
subject to x1_zero: x1 >= 0;
subject to x2_zero: x2 >= 0;
subject to x3_zero: x3 >= 0;
MINOS 5.51: optimal solution found.
1 iterations, objective 17250000
: _varname   _var     :=
1   x1           0
2   x2       75000
3   x3           0
;
\end{verbatim}
~--- производим только женскую обувь на все ресурсы. Прибыль 17 250 000.

Решение в AMPL (двойственная):
\begin{verbatim}
var y1;
var y2;
minimize Cost: 3000000 * y1 + 4000000 * y2;
subject to type_male: 40 * y1 + 5 * y2 >= 200;
subject to type_female: 40 * y1 + 3 * y2 >= 230;
subject to type_child: 20 * y1 + 1 * y2 >= 100;
subject to y1_zero: y1 >= 0;
subject to y2_zero: y2 >= 0;

MINOS 5.51: optimal solution found.
1 iterations, objective 17250000
: _varname   _var    :=
1   y1       5.75
2   y2       0
;
\end{verbatim}
~--- продаем клей по 5.75, а кожу отдаем даром. В итоге получаем столько же прибыли, как от оптимального изготовления обуви (17 250 000).
\end{document}