\documentclass[a4paper]{article}
\usepackage[a4paper, left=5mm, right=5mm, top=5mm, bottom=5mm]{geometry}
\usepackage[utf8]{inputenc}
\usepackage[english,russian]{babel}
\usepackage{graphicx}
\usepackage{indentfirst}
\usepackage{tikz} %Рисование автоматов
\usetikzlibrary{automata,positioning}
\usepackage{amsmath}
\usepackage{floatflt}
\usepackage{enumerate}
\usepackage{hyperref}
\usepackage{amsfonts}
\usepackage{amssymb}
\title{Теория и реализация языков программирования.\\Задание 2: НКА и алгоритмы поиска подстрок}
\date{задано 2013.09.11}
\author{Сергей~Володин, 272 гр.}

% названия автоматов
\def\A{{\cal A}}
\def\B{{\cal B}}
\def\C{{\cal C}}

\begin{document}
\maketitle
\section*{Упражнение 0}
Автомат, распознающий $\varnothing$:
\begin{center}
\begin{tikzpicture}[shorten >=1pt,node distance=2cm,on grid,auto,initial text=]
	  \node[state,initial]  (q_0)   {$q_0=q_F$};
\end{tikzpicture}
\end{center}
У него нет принимающих состояний, поэтому ни одно слово не будет принято.
\\[10pt]
Автомат, распознающий $\{\sigma\}$:
\begin{center}
\begin{tikzpicture}[shorten >=1pt,node distance=2cm,on grid,auto,initial text=]
	  \node[state,initial]  (q_0)   {$q_0$};
  	  \node[state,accepting]  (q_1) [right = of q_0]   {$q_1$};
  	  \path[->] 
  	  			(q_0)	edge		node	{$\sigma$}	(q_1);
\end{tikzpicture}
\end{center}
Очевидно, автомат примет $\sigma$: $(q_0,\sigma)\vdash(q_1,\varepsilon)$, и $q_1\in F$. Пусть автомат принял слово $w$. Поскольку начальных и принимающих верший по одной, и между ними один переход $\delta(q_0,\sigma)=\{q_1\}$, получаем $(q_0,w)\vdash(q_1,\varepsilon)$. Отсюда $w=\sigma$.
\section*{Упражнение 1}
$L(\A)=X,L(\B)=Y$. Докажем, $L(\C)=X\cdot Y$, $\C$~--- автомат из условия:
\begin{center}
\begin{tikzpicture}[shorten >=1pt,node distance=2cm,on grid,auto,initial text=]
	  %\draw[help lines] (0,0) grid (3,2);
	  \draw (1,0) ellipse (16mm and 10mm);
	  \draw (3,0) ellipse (16mm and 10mm);
	  \node at (1,0) {$\A$};
	  \node at (3,0) {$\B$};
	  \node[state,initial]  (q_0)                      {$q^\A_0$};
	  \node[state]          (q_1) [ right = of q_0] {\tiny\tiny$q_F^\A,q_0^\B$};
	  \node[state, accepting]          (q_2) [ right = of q_1] {$q^\B_F$};
	 % \node[state,accepting](q_3) [below right=of q_1] {$q_3$};
\end{tikzpicture}
\end{center}
Докажем, что $w\in X\cdot Y\Rightarrow w\in L(\C)$:\newline
$w\in X\cdot Y\Rightarrow w=xy, x\in X, y\in Y$. Тогда $x\in L(\A)$, то есть, $(q_0^\A,x)\vdash^*(q_F^\A,\varepsilon)$. Аналогично $(q_0^\B,y)\vdash^*(q_F^\B,\varepsilon)$.
\\[5pt]
Пусть для некоторого автомата $(q_1,cx)\vdash(q_2,x)$. Это значит, что $\delta(q_1,c)\ni q_2$. Но отсюда $(q_1,cxy)\vdash(q_2,xy)$, $x,y$~--- некоторые слова. Пусть $(q_1,x)\vdash^*(q_2,y)$. Это значит, что существует цепочка $(q_1,x)\vdash\dots\vdash(q_2,y)$. Применяя утверждение выше, получаем $(q_1,xz)\vdash\dots\vdash(q_2,yz)$, а отсюда $(q_1,xz)\vdash^*(q_2,yz)$. Очевидно, рассуждение верно и в обратную сторону: если $(q_1,xz)\vdash^*(q_2,yz)$, то $(q_1,x)\vdash^*(q_2,y)$.
\\[5pt]
Из предыдущего утверждения получаем для автомата $\A$ $(q_0^\A,xy)\vdash^*(q_F^\A,y)\equiv(q_0^\B,y)\vdash^*(q^\B_F,\varepsilon)$, то есть, $(q_0^\A,w)\vdash^*(q_F^\B,\varepsilon)\Rightarrow w\in L(\C)$.\newline
Докажем обратное: $w\in L(\C)\Rightarrow w\in X\cdot Y$.\newline
Пусть $w\in L(\C)$. Тогда $(q_0^\A,w)\vdash\dots\vdash(q_F^\B,\varepsilon)$. Докажем, что в этой цепочке встретилось состояние $(q_F^\A,y)$.\newline
В этой цепочке был переход из $q_\A\in Q^\A$ в $q_\B\in Q^\B$, т.к. иначе получим, что все переходы были внутри множества состояний $Q^\A$. Поэтому $\exists q^\A\,\exists\sigma\in\Sigma:\delta(q^\A,\sigma)\ni q^\B$. Но изначально множество $Q^\A$~--- множество состояний автомата $\A$, поэтому $\delta(Q^\A\times\Sigma)\subseteq 2^{Q^\A}$. Получаем $q^\B\in Q^\A$. По условию $Q^\A\cap Q^\B=\{q_F^\A\}$, откуда получаем требуемое: в цепочке встретилось $(q^\A,\sigma y)\vdash(q_F^\A,y)$.\newline
Таким образом, $(q_0^\A,w)\vdash^*(q_F^\A,y)\vdash^*(q_F^\B,\varepsilon)$. Поскольку $y$~--- суффикс $w$, то $w=xy$. Из доказанного ранее получаем $(q_0^\A,x)\vdash^*(q_F^\A,\varepsilon)$, откуда $x\in L(\A)=X$. Аналогично $y\in L(\B)=Y$. Найдены $x\in X$ и $y\in Y$: $w=xy$. Отсюда $w\in X\cdot Y$.
\section*{Упражнение 2}
\section*{Упражнение 3}
\newpage
\section*{Задача 1}
Пользуясь доказанным ранее, построим автомат для $ab$:
\begin{center}
\begin{tikzpicture}[shorten >=1pt,node distance=2cm,on grid,auto,initial text=]
	  \node[state,initial]  (q_0)   {$q_0$};
  	  \node[state]  (q_1) [right = of q_0]   {$q_1$};
   	  \node[state,accepting]  (q_2) [right = of q_1]   {$q_2$};
  	  \path[->] 
  	  			(q_0)	edge		node	{$a$}	(q_1)
  	  			(q_1)	edge		node	{$b$}	(q_2);
\end{tikzpicture}
\end{center}
Для $ab|b$:
\begin{center}
\begin{tikzpicture}[shorten >=1pt,node distance=2cm,on grid,auto,initial text=]
	  \node[state,initial]  (q_0)   {$q_0$};
  	  \node[state]  (q_1) [above right = of q_0]   {$q_1$};
   	  \node[state]  (q_2) [below right = of q_0]   {$q_2$};
   	  \node[state]  (q_3) [right = of q_1]   {$q_3$};
  	  \node[state]  (q_4) [right = of q_3]   {$q_4$};
  	  \node[state]  (q_5) [right = of q_2]   {$q_5$};
  	  \node[state, accepting]  (q_6) [below right = of q_4]   {$q_6$};
  	  \path[->] 
  	  			(q_0)	edge		node	{$\varepsilon$}	(q_1)
  	  			(q_1)	edge		node	{$a$}	(q_3)
  	  			(q_3)	edge		node	{$b$}	(q_4)
  	  			(q_2)	edge		node	{$b$}	(q_5)
  	  			(q_5)	edge		node	{$\varepsilon$}	(q_6)
  	  			(q_4)	edge		node	{$\varepsilon$}	(q_6)
  	  			(q_0)	edge		node	{$\varepsilon$}	(q_2);
\end{tikzpicture}
\end{center}
Для $(ab|b)^*$
\begin{center}
\begin{tikzpicture}[shorten >=1pt,node distance=2cm,on grid,auto,initial text=]
	  \node[state]  (q_0)   {};
  	  \node[state]  (q_1) [above right = of q_0]   {};
   	  \node[state]  (q_2) [below right = of q_0]   {};
   	  \node[state]  (q_3) [right = of q_1]   {};
  	  \node[state]  (q_4) [right = of q_3]   {};
  	  \node[state]  (q_5) [right = of q_2]   {};
  	  \node[state,initial]  (q_7) [left = of q_0]   {};
  	  \node[state,accepting]  (q_8) [right = of q_6]   {};
  	  \node[state]  (q_6) [below right = of q_4]   {};
  	  \path[->] 
  	  			(q_0)	edge		node	{$\varepsilon$}	(q_1)
  	  			(q_1)	edge		node	{$a$}	(q_3)
  	  			(q_3)	edge		node	{$b$}	(q_4)
  	  			(q_2)	edge		node	{$b$}	(q_5)
  	  			(q_5)	edge		node	{$\varepsilon$}	(q_6)
  	  			(q_7)	edge		node	{$\varepsilon$}	(q_0)
  	  			(q_6)	edge		node	{$\varepsilon$}	(q_8)
  	  			(q_4)	edge		node	{$\varepsilon$}	(q_6)
  	  			(q_0)	edge		node	{$\varepsilon$}	(q_2)
  	  			(q_6)	edge		node	{$\varepsilon$}	(q_0)
  	  			(q_7)	edge [bend right=50]	node	{$\varepsilon$}	(q_8);
\end{tikzpicture}
\end{center}
Для $a(ab|b)^*b$:
\begin{center}
\begin{tikzpicture}[shorten >=1pt,node distance=2cm,on grid,auto,initial text=]
	  \node[state]  (q_0)   {};
  	  \node[state]  (q_1) [above right = of q_0]   {};
   	  \node[state]  (q_2) [below right = of q_0]   {};
   	  \node[state]  (q_3) [right = of q_1]   {};
  	  \node[state]  (q_4) [right = of q_3]   {};
  	  \node[state]  (q_5) [right = of q_2]   {};
  	  \node[state]  (q_7) [left = of q_0]   {};
  	  \node[state]  (q_8) [right = of q_6]   {};
  	  \node[state]  (q_6) [below right = of q_4]   {};
  	  \node[state,initial]  (q_11) [left = of q_7]   {};
  	  \node[state,accepting]  (q_12) [right = of q_8]   {};
  	  \path[->] 
  	  			(q_0)	edge		node	{$\varepsilon$}	(q_1)
  	  			(q_1)	edge		node	{$a$}	(q_3)
  	  			(q_3)	edge		node	{$b$}	(q_4)
  	  			(q_2)	edge		node	{$b$}	(q_5)
  	  			(q_5)	edge		node	{$\varepsilon$}	(q_6)
  	  			(q_7)	edge		node	{$\varepsilon$}	(q_0)
  	  			(q_6)	edge		node	{$\varepsilon$}	(q_8)
  	  			(q_4)	edge		node	{$\varepsilon$}	(q_6)
  	  			(q_0)	edge		node	{$\varepsilon$}	(q_2)
  	  			(q_6)	edge		node	{$\varepsilon$}	(q_0)
  	  			(q_11)	edge		node	{$a$}	(q_7)
  	  			(q_8)	edge		node	{$b$}	(q_12)
  	  			(q_7)	edge [bend right=50]	node	{$\varepsilon$}	(q_8);
\end{tikzpicture}
\end{center}
\section*{Задача 2}
\begin{enumerate}[a.]
\item Докажем, что автомат $\A$ распознает $L_3$:
\begin{center}
\begin{tikzpicture}[shorten >=1pt,node distance=2cm,on grid,auto,initial text=]
	  \node[state,initial]  (q_0)   {$q_0$};
  	  \node[state]  (q_1) [right = of q_0]   {$q_1$};
   	  \node[state]  (q_2) [right = of q_1]   {$q_2$};
   	  \node[state]  (q_3) [right = of q_2]   {$q_3$};
  	  \node[state,accepting]  (q_4) [right = of q_3]   {$q_4$};
  	  \path[->] 
			(q_0)	edge[loop]		node	{$0,1$}	(q_0)
			(q_0)	edge		node	{$\varepsilon$}	(q_1)
			(q_1)	edge		node	{$1$}	(q_2)
			(q_2)	edge		node	{$0,1$}	(q_3)
			(q_3)	edge		node	{$0,1$}	(q_4);
\end{tikzpicture}
\end{center}
\begin{enumerate}[1.]
\item $w\in L_3$. Тогда $w[n-3]=1\Leftrightarrow w=x1\sigma_1\sigma_2$. $x\in \Sigma^*\Rightarrow$ после обработки $x$ автомат может оказаться в $q_0$: $(q_0,x1\sigma_1\sigma_2)\vdash^*(q_0,1\sigma_1\sigma_2)$. Далее $(q_0,1\sigma_1\sigma_2)\vdash(q_1,1\sigma_1\sigma_2)\vdash(q_2,\sigma_1\sigma_2)$. $\sigma_1,\sigma_2\in\Sigma\Rightarrow(q_2,\sigma_1\sigma_2)\vdash(q_3,\sigma_2)\vdash(q_4,\varepsilon)$. Таким образом, $(q_0,w)\vdash^*(q_4,\varepsilon)\Rightarrow w\in L(\A)$.
\item $w\in L(\A)\Rightarrow(q_0,w)\vdash^*(q_4,\varepsilon)$. Поскольку переход в $q_4$ только один: $\exists!q:\delta(q,\sigma)\ni q_4,q=q_3$, получаем, что в цепочке конфигураций на последнем месте переход $(q_3,\sigma)\vdash(q_4,\varepsilon)$, $\sigma$~--- суффикс $w$. Аналогично получаем $(q_1,1\sigma_1\sigma_2)\vdash^*(q_4,\varepsilon)$, $1\sigma_1\sigma_2$~--- суффикс $w$, откуда $w\in L_3$.
\end{enumerate}
\item Построим ДКА $\B$ по НКА $\A$:\newline
$\begin{array}{|c|c|c|c|c|c|}
\hline
Q & q & 0\,(q) & 0\,(Q) & 1\,(q) & 1\,(Q)\\\hline\hline
Q_0 & q_0, q_1					& q_0, q_1				& Q_0 & q_0, q_1, q_2			& Q_1\\\hline
Q_1 & q_0, q_1, q_2				& q_0, q_1, q_3			& Q_2 & q_0, q_1, q_2, q_3		& Q_3\\\hline
Q_2 & q_0, q_1, q_3				& q_0, q_1, q_4			& Q_4 & q_0, q_1, q_2, q_4		& Q_5\\\hline
Q_3 & q_0, q_1, q_2, q_3		& q_0, q_1, q_3, q_4 	& Q_6 & q_0, q_1, q_2, q_3, q_4 & Q_7\\\hline
Q_4 & q_0, q_1, q_4				& q_0, q_1 				& Q_0 & q_0, q_1, q_2			& Q_1\\\hline
Q_5 & q_0, q_1, q_2, q_4		& q_0, q_1, q_3			& Q_2 & q_0, q_1, q_2, q_3		& Q_3\\\hline
Q_6 & q_0, q_1, q_3, q_4		& q_0, q_1, q_4			& Q_4 & q_0, q_1, q_2, q_4		& Q_5\\\hline
Q_7 & q_0, q_1, q_2, q_3, q_4 	& q_0, q_1, q_3, q_4	& Q_6 & q_0, q_1, q_2, q_3, q_4	& Q_7\\\hline
\end{array}$
\begin{center}
\begin{tikzpicture}[shorten >=0pt,node distance=3cm,on grid,auto,initial text=]
	  \node[state,initial]  (q_0)   {$Q_0$};
  	  \node[state]  (q_1) [right = of q_0]   {$Q_1$};
   	  \node[state]  (q_2) [above right = of q_1]   {$Q_2$};
   	  \node[state]  (q_3) [below right = of q_1]   {$Q_3$};
   	  \node[state, accepting]  (q_4) [above right = of q_2]   {$Q_4$};
   	  \node[state, accepting]  (q_5) [right = of q_2]   {$Q_5$};
   	  \node[state, accepting]  (q_6) [right = of q_3]   {$Q_6$};
   	  \node[state, accepting]  (q_7) [below right = of q_3]   {$Q_7$};
  	  \path[->] 
			(q_0)	edge[loop below]		node	{$0$}	(q_0)
			(q_0)	edge		node	{$1$}	(q_1)
			(q_1)	edge		node	{$0$}	(q_2)
			(q_1)	edge		node	{$1$}	(q_3)
			(q_2)	edge		node	{$0$}	(q_4)
			(q_2)	edge[bend right=25]		node	{$1$}	(q_5)
			(q_3)	edge		node	{$0$}	(q_6)
			(q_3)	edge		node	{$1$}	(q_7)
			(q_4)	edge[bend right=30]		node	{$0$}	(q_0)
			(q_4)	edge[bend right=40]		node	{$1$}	(q_1)
			(q_5)	edge[bend right=25]		node	{$0$}	(q_2)
			(q_5)	edge		node	{$1$}	(q_3)
			(q_6)	edge[bend right=50]		node	{$0$}	(q_4)
			(q_6)	edge		node	{$1$}	(q_5)
			(q_7)	edge		node	{$0$}	(q_6)
			(q_7)	edge[loop below]		node	{$1$}	(q_7);
\end{tikzpicture}
\end{center}
\end{enumerate}
Заметим, что в построенном автомате $8=2^3$ состояний.
\section*{Задача 3}
В исходном автомате $\A: L(\A)=L_n$ будет $n+2$ состояний (при $n=3$ состояний $5$, при увеличении $n$ на $1$ количество состояний также будет увеличиваться на 1, так как в конце слова должен стоять еще один произвольный символ).
\begin{center}
\begin{tikzpicture}[shorten >=1pt,node distance=2cm,on grid,auto,initial text=]
	  \node[state,initial]  (q_0)   {$q_0$};
  	  \node[state]  (q_1) [right = of q_0]   {$q_1$};
   	  \node[state]  (q_2) [right = of q_1]   {$q_2$};
   	  \node			(q_dots) [right of = q_2]{$\cdots$};
   	  \node[state]  (q_3) [right = of q_dots]   {$q_n$};
  	  \node[state,accepting]  (q_4) [right = of q_3]   {$q_{n+1}$};
  	  \path[->] 
			(q_0)	edge[loop]		node	{$0,1$}	(q_0)
			(q_0)	edge		node	{$\varepsilon$}	(q_1)
			(q_1)	edge		node	{$1$}	(q_2)
			(q_2)	edge		node	{$0,1$}	(q_dots)
			(q_dots)	edge		node	{$0,1$}	(q_3)
			(q_3)	edge		node	{$0,1$}	(q_4);

\end{tikzpicture}
\end{center}
{\em{(Идея из Хопкрофта, 2.3.6: плохой случай для конструкции подмножеств)}}
\\[3pt]
Предположим, что в ДКА $\B:L(\B)=L_n$ меньше $2^n$ состояний. Тогда по принципу Дирихле (т.к. состояний меньше, чем строк из $n$ символов) $\exists a=a_1\dots a_n\,\exists b=b_1\dots b_n:a\neq b$ и $(Q_0,q)\vdash^*(Q,x)\,,(Q_0,b)\vdash^*(Q,y)$, то есть, после обработки $a$ или $b$ автомат переходит в состояние $Q$. Поскольку $a\neq b$, $\exists i:a_i\neq b_i$.
\begin{enumerate}[1.]
\item $i=1$. Это значит, что отличаются первые символы строк. Без ограничения общности, $a_1=1, b_1=0$. Но тогда $a\in L(\B)$, так как $n$-й символ с конца строки $a$ равен $1$. Поэтому $Q\in F_\B$.\newline
Аналогично, так как $b_1=0$, $b\notin L(\B)\Rightarrow Q\notin F_\B$~--- противоречие.
\item $i>1$. Тогда, без ограничения общности, $a_i=1, b_i=0$. Дополним строки $a$ и $b$ чем-либо (например, нулями) справа так, чтобы $i$-й символ стал $n$-м:
$a'=a_1\dots a_i\dots a_n\underbrace{0\dots 0}_{i-1}, b'$~--- аналогично. Тогда $a'\in L(\B), b'\notin L(\B)$.\newline
Но $(Q_0,a')\vdash^*(Q,0\dots 0)\vdash^*(P,\varepsilon)$,\newline
$(Q_0,b')\vdash^*(Q,0\dots 0)\vdash^*(P,\varepsilon)$, откуда состояние $P$ должно быть как принимающим, так и не принимающим~--- противоречие.
\end{enumerate}
Получаем, что в НКА $\A$ $O(n)$ состояний, а в любом ДКА $\B: L(\B)=L_n$ $2^n$ состояний. В том числе $2^n$ состояний будет у того ДКА, который построен по $\A$ при помощи алгоритма$\,\blacksquare$
\section*{Задача 4}
Построим ДКА $\B$ по НКА $\A$:\newline
$\begin{array}{|c|c|c|c|}
\hline
Q & q & a & b\\\hline\hline
Q_0 & q_0						& Q_1  & Q_0 \\\hline
Q_1 & q_0, q_1					& Q_1  & Q_2 \\\hline
Q_2 & q_0, q_2					& Q_3  & Q_0 \\\hline
Q_3 & q_0, q_1, q_3				& Q_1  & Q_4 \\\hline
Q_4 & q_0, q_2, q_4				& Q_5  & Q_0 \\\hline
Q_5 & q_0, q_1, q_3, q_5		& Q_1  & Q_6 \\\hline
Q_6 & q_0, q_2, q_4, q_6		& Q_7  & Q_8 \\\hline
Q_7 & q_0, q_1, q_3, q_5, q_6	& Q_9  & Q_6 \\\hline
Q_8 & q_0, q_6					& Q_9  & Q_8 \\\hline
Q_9 & q_0, q_1, q_6				& Q_9  & Q_{10}\\\hline
Q_{10} & q_0, q_2, q_6				& Q_{11} & Q_8 \\\hline
Q_{11} & q_0, q_1, q_3, q_6		& Q_9  & Q_6 \\\hline
\end{array}$
\newline
Начальное состояние~--- $Q_0$, принимающие~--- $\{Q_6, Q_7, Q_8, Q_9, Q_{10}, Q_{11}\}$.
\section*{Задача 5}
\section*{Задача 6}
%Пименов, Алгоритмы: построение и анализ
Пусть $n=|w|$. Тогда определим $Q=\{q_0,q_1,\dots,q_n\}$, $F=\{q_n\}$.
\begin{center}
\begin{tikzpicture}[shorten >=1pt,node distance=2cm,on grid,auto,initial text=]
	  \node[state,initial]  (q_0)   {$q_0$};
  	  \node[state]  (q_1) [right = of q_0]   {$q_1$};
   	  \node[state]  (q_2) [right = of q_1]   {$q_2$};
   	  \node			(q_dots) [right of = q_2]{$\cdots$};
   	  \node[state]  (q_3) [right = of q_dots]   {$q_{n-1}$};
  	  \node[state,accepting]  (q_4) [right = of q_3]   {$q_n$};
  	  \path[->] 
			(q_0)	edge		node	{$w_1$}	(q_1)
			(q_1)	edge		node	{$w_2$}	(q_2)
			(q_2)	edge		node	{$w_3$}	(q_dots)
			(q_dots)	edge		node	{$w_{n-1}$}	(q_3)
			(q_3)	edge		node	{$w_n$}	(q_4);
\end{tikzpicture}
\end{center}
Определим $l'(i)=|l(w_1\dots w_i)|$.\newline
Определим функцию $f(i,\sigma)$~--- длина максимального префикса слова $w_1\dots w_i$, который совпадает с его суффиксом, причем после префикса стоит символ $\sigma$. Рассмотрим все префиксы, которые являются суффиксами (включая собственный) в порядке убывания длины, используя префикс-функцию (рассматривая последовательность $w[1,i],l(w[1,i]),l(l(w[1,i])),\dots$). Выберем первый префикс, после которого стоит символ $\sigma$. Если такого префикса ненулевой длины нет, и $w_1\neq \sigma$, то определим $f(i,\sigma):=-1$.\newline
Определим функцию переходов $\delta$ через f: $\delta(q_i,\sigma)=
\begin{cases}
\{q_{i+1}\} & \mbox{если } w_{i+1}=\sigma \\
\{q_{f(i,\sigma)+1}\} & \mbox{иначе} \\
\end{cases}$.\newline
{\em Свойство: номер состояния~--- длина максимального префикса слова $w$, который входит во входное слово, причем вхождение заканчивается на рассматриваемом автоматом символе.}\newline
Первый случай, очевидно, корректный (сохраняет свойство выше): если совпало $i$ символов, и следующий символ $\sigma$ совпадает с $w_{i+1}$, то всего совпадает $i+1$ символ (и не больше) $\Rightarrow$ переход в $q_{i+1}$.\newline
Во втором случае, когда следующий символ входного слова отличается от $w_{i+1}$, совершается переход в состояние с номером $f(i,\sigma)+1$, так как необходимо найти максимальный префикс слова $w$, который совпадает с суффиксом слова $w[1,i]$, причем после префикса стоит $\sigma$. Длина этого префикса~--- $f(i,\sigma)$. Тогда перейти нужно вершину со следующим номером (еще символ $\sigma$).
\section*{Задача 7}
\begin{verbatim}
fillPrefix at i=0, looking at symbol a
fillPrefix at i=1, looking at symbol b
  zero prefix[0, *]
fillPrefix at i=2, looking at symbol b
  zero prefix[0, *]
fillPrefix at i=3, looking at symbol a
  got prefix[0, *] . 'a'
fillPrefix at i=4, looking at symbol #
  zero prefix[0, *]
fillPrefix at i=5, looking at symbol a
  got prefix[0, *] . 'a'
fillPrefix at i=6, looking at symbol b
  got prefix[1, a] . 'b'
fillPrefix at i=7, looking at symbol b
  got prefix[2, b] . 'b'
fillPrefix at i=8, looking at symbol b
  zero prefix[0, *]
fillPrefix at i=9, looking at symbol a
  got prefix[0, *] . 'a'
fillPrefix at i=10, looking at symbol b
  got prefix[1, a] . 'b'
fillPrefix at i=11, looking at symbol a
  zero prefix[0, *]
  got prefix[0, *] . 'a'
fillPrefix at i=12, looking at symbol b
  got prefix[1, a] . 'b'
fillPrefix at i=13, looking at symbol b
  got prefix[2, b] . 'b'
fillPrefix at i=14, looking at symbol a
  got prefix[3, b] . 'a'
fillPrefix at i=15, looking at symbol b
  going deeper[1, a]
  got prefix[1, a] . 'b'
n	0	1	2	3	4	5	6	7	8	9	A	B	C	D	E	F	G	
c	*	a	b	b	a	#	a	b	b	b	a	b	a	b	b	a	b	
p	0	0	0	0	1	0	1	2	3	0	1	2	1	2	3	4	2	
findPInT: found P in T starting from position 12
\end{verbatim}
\href{https://bitbucket.org/etoestja/inf/raw/HEAD/mipt/s3/TIPL/2/KMP\_stupid.c}{https://bitbucket.org/etoestja/inf/raw/HEAD/mipt/s3/TIPL/2/KMP\_stupid.c}~--- исходный код. Запускать:
\begin{verbatim}
$ cc KMP_stupid.c -o KMP
$ echo "abba abbbababbab" | ./KMP
\end{verbatim}
\end{document}
