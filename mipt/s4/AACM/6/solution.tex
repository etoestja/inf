\documentclass[a4paper]{article}
\usepackage[a4paper, left=5mm, right=5mm, top=5mm, bottom=5mm]{geometry}
%\geometry{paperwidth=210mm, paperheight=2000pt, left=5pt, top=5pt}
\usepackage[utf8]{inputenc}
\usepackage[english,russian]{babel}
\usepackage{indentfirst}
\usepackage{tikz} %Рисование автоматов
\usetikzlibrary{automata,positioning,arrows,trees,calc}
\usepackage{amsmath}
\usepackage[makeroom]{cancel} % зачеркивание
\usepackage{multicol,multirow} %Несколько колонок
\usepackage{hyperref}
\usepackage{tabularx}
\usepackage{amsfonts}
\usepackage{amssymb}
\DeclareMathOperator*{\argmin}{arg\,min}
\usepackage{listings}
\usepackage{wasysym}
\date{задано 2014.03.20}
\author{Сергей~Володин, 272 гр.}
\newcommand{\matrixl}{\left|\left|}
\newcommand{\matrixr}{\right|\right|}
% названия автоматов  (rubtsov)
\def\A{{\cal A}}
\def\B{{\cal B}}
\def\C{{\cal C}}

%классы сложности (rubtsov)
\def\PP{{\mathsf{P}}}
\def\NP{{\mathsf{NP}}}
\def\NPc{{\mathsf{NP}}\text{-}{\rm c}}
\def\coNP{{\rm co}\text{-}{\mathsf{NP}}}
\def\DTIME{{\mathsf{DTIME}}}
\def\NTIME{{\mathsf{NTIME}}}
\def\CLIQUE{{\mathsf{CLIQUE}}}
\def\HALT{{\rm{HALT}}}
\def\SAT{{\rm{SAT}}}
\def\3SAT{{\rm{3\text{-}SAT}}}
\def\2SAT{{\rm{2\text{-}SAT}}}
\def\UNSAT{{\rm{UNSAT}}}
\def\HP{{\rm{HAMPATH}}}
\def\UHP{{\rm{UHAMPATH}}}
\def\LL{{\mathrm{LL}}}
\def\poly{{\rm{poly}}}
\def\GC{{\mbox{ГЦ}}}
\def\GP{{\mbox{ГП}}}

\title{Алгоритмы и модели вычислений.\\Задание 6: всякая хуйня}

% алгоритмы (Рубцов)
\usepackage{verbatim}
\usepackage{listings}
\usepackage{algorithm2e}

%+= и -=, иначе разъезжаются...
\newcommand{\peq}{\mathrel{+}=}
\newcommand{\meq}{\mathrel{-}=}
\newcommand{\deq}{\mathrel{:}=}
\newcommand{\plpl}{\mathrel{+}+}

\newcommand{\todo}{{\em todo}}

% пустое слово  (rubtsov)
\def\eps{\varepsilon}

% регулярные языки  (rubtsov)
\def\REG{{\mathsf{REG}}}
\def\CFL{{\mathsf{CFL}}}
\def\eqdef{\overset{\mbox{\tiny def}}{=}}
\newcommand{\niton}{\not\owns}

%FIRST & FOLLOW (rubtsov)
\def\first{\mathrm{ FIRST} }
\def\follow{\mathrm{ FOLLOW} }

% LL LR (rubtsov)
\def\LL{{\mathrm{LL}}}
\def\LR{{\mathrm{LR}}}

\newcounter{rowItemCount}
\newcounter{subRowItemCount}
\newcommand\rowItem{
    \setcounter{subRowItemCount}{0}
    \arabic{rowItemCount}.\addtocounter{rowItemCount}{1}}
\newcommand\subRowItem{
    \addtocounter{subRowItemCount}{1}
    \addtocounter{rowItemCount}{-1}
    \arabic{rowItemCount}.\arabic{subRowItemCount}.\addtocounter{rowItemCount}{1}}
    
\newcommand{\smalll}[1]{\overline{\overline{#1}}}
\newcommand{\smallo}{\bar{\bar{o}}}

\begin{document}
\maketitle
\subsection*{(каноническое) Задача 24}
$\psi=\overline{x_1}\vee x_2$. $\psi'=(\overline{x_1}\vee x_2\vee y)\wedge(\overline{x_1}\vee x_2\vee \overline{y})$. Граф $W_{\psi'}$ с раскраской:

\begin{tikzpicture}[shorten >=1pt,node distance=1.3cm,on grid,auto,initial text=]
	  \node[state, fill=gray] (x_1) {$x_1$};
	  \node[state, fill=green] (nx_1)  [right = of x_1] {$\overline{x_1}$};
	  \node[state, fill=green] (x_2) [right = of nx_1]  {$x_2$};
	  \node[state, fill=gray] (nx_2)  [right = of x_2] {$\overline{x_2}$};
	  \node[state, fill=green] (y)  [right = of nx_2] {$y$};
	  \node[state, fill=gray] (ny)  [right = of y] {$\overline{y}$};
	  \node[state, fill=gray] (F) [above of = x_2] {$F$};
	  \node[state, fill=red] (R) [right = of F] {$R$};
	  \node[state, fill=red] (t2) [below = 1.7cm of nx_1] {};
	  \node[state, fill=gray] (t1) [left = of t2] {};
	  \node[state, fill=red] (t5) [below = 1.7cm of x_2] {};
	  \node[state, fill=gray] (t6) [right = of t5] {};
	  \node (t1b) [below of = t1] {};
	  \node (t2b) [below of = t2] {};
	  \node[state, fill=green] at ($(t1b)!0.5!(t2b)$) (t3) {};
	  \node (t5b) [below of = t5] {};
	  \node (t6b) [below of = t6] {};
	  \node[state, fill=green] at ($(t5b)!0.5!(t6b)$) (t7) {};
	  \node[state, fill=gray] (t4) [below = of t3] {};
	  \node[state, fill=gray] (t10) [below = of t7] {};
	  \node[state, fill=red] (t8) [below = 4.3cm of y] {};
	  \node[state, fill=red] (t9) [below = 4.3cm of ny] {};
	  \node (Fb) [below = 8cm of F] {};
	  \node (Rb) [below = 8cm of R] {};
	  \node[state, fill=green] at ($(Fb)!0.5!(Rb)$) (T) {$T$};
  	  \path[-] 
			(x_1)	edge	(nx_1)
			(x_2)	edge	(nx_2)
			(y)	edge	(ny)
			(F) edge (R)
			(nx_1) edge (t1)
			(x_2) edge [bend left = 10] (t2)
			(t1) edge (t2)
			(t1) edge (t3)
			(t2) edge (t3)
			(t3) edge (t4)
			(nx_1) edge [bend left = 10] (t5)
			(x_2) edge (t6)
			(t5) edge (t6)
			(t5) edge (t7)
			(t6) edge (t7)
			(t7) edge (t10)
			(y) edge (t8)
			(ny) edge (t9)
			(t4) edge [bend left = 20] (t8)
			(t10) edge [bend left = 30] (t9)
			(t4) edge (T)
			(t10) edge (T)
			(t8) edge (T)
			(t9) edge (T)
			(F) edge [in = 100, out = 240] (T)
			(R) edge [in = 70, out = 300] (T)
			(R) edge [in=30, out=216] (x_1)
			(R) edge (nx_1)
			(R) edge (x_2)
			(R) edge (nx_2)
			(R) edge (y)
			(R) edge (ny)
			;
\end{tikzpicture}

\subsection*{(каноническое) Задача 25}
\begin{enumerate}
\item $\psi=\overline{x_1}\vee x_2$, $\psi'=(\overline{x_1}\vee x_2\vee y)\wedge (\overline{x_1}\vee x_2\vee \overline{y})$. $n=3$, $m=2$. Граф $Q_{\psi'}$. Клика мощности $s=m=2$ выделена красным цветом.\newline
\begin{tikzpicture}[shorten >=1pt,node distance=1.3cm,on grid,auto,initial text=]
	  \node[state, fill=gray] (nx11) {$\overline{x_1}$};
	  \node[state, fill=red] (x21) [right = of nx11] {$x_2$};
	  \node[state, fill=gray] (y1) [right = of x21] {$y$};
	  
	  \node[state, fill=red] (nx12) [below = of nx11] {$\overline{x_1}$};
	  \node[state, fill=gray] (x22) [right = of nx12] {$x_2$};
	  \node[state, fill=gray] (ny2) [right = of x22] {$\overline{y}$};

  	  \path[-] 
			(nx11)	edge	(nx12)
			(x21)	edge	(x22)
			(nx11)	edge	(x22)
			(x21)	edge	(nx12)
			(x21)	edge	(ny2)
			(nx11)	edge	(ny2)
			(y1)	edge	(nx12)
			(y1)	edge	(x22)
			;
\end{tikzpicture}
\item {\em (доказано на семинаре)} $\3SAT \leqslant_m^p\CLIQUE$. Формула $\chi=(x_1\vee x_2\vee x_3)\wedge(\overline{x_1}\vee\overline{x_2})\wedge(x_1\vee\overline{x_2})\wedge(\overline{x_1}\vee x_2\vee x_3)\wedge\overline{x_3}$,\newline $\chi'=(x_1\vee x_2\vee x_3)\wedge(\overline{x_1}\vee\overline{x_2}\vee y_1)\wedge(\overline{x_1}\vee\overline{x_2}\vee \overline{y_1})\wedge(x_1\vee\overline{x_2}\vee y_2)\wedge (x_1\vee\overline{x_2}\vee \overline{y_2})\wedge(\overline{x_1}\vee x_2\vee x_3)\wedge (\overline{x_3}\vee y_3\vee y_4)\wedge(\overline{x_3}\vee y_3\vee \overline{y_4})\wedge(\overline{x_3}\vee \overline{y_3}\vee y_4)\wedge(\overline{x_3}\vee \overline{y_3}\vee \overline{y_4})$. $n=7$, $t=m=10$. $f(x)=(G,t)$~--- граф, построенный по $\chi'$ (и число $10$~--- мощность искомой клики), $f$~--- функция из сводимости. Пусть в $G$ существует клика мощности $\geqslant t$. Тогда существует клика мощности $t$ (любой подграф из $t$ вершин исходной клики). Тогда $f(x)\in\CLIQUE\overset{\mbox{\tiny сводимость}}{\Longrightarrow}$ $\chi'\in\3SAT$ $\Rightarrow$ $\chi'$~--- выполнима $\overset{\mbox{\tiny эквив. формул}}{\Longrightarrow}$ $\chi$~--- выполнима~--- противоречие. Значит, в графе образа $\chi'$ нет клики мощности $\geqslant t\equiv 10\,\blacksquare$
\end{enumerate}
\subsection*{(каноническое) Задача 26}
\begin{enumerate}
\item [0. ] Исходный дизъюнкт $w=(a_i\vee b_i\vee c_i)$. Не будем писать индексы (рассматриваем один дизъюнкт).\newline Рассмотрим $L\eqdef\{a,b,c,d,\overline{a}\vee\overline{b},\overline{a}\vee\overline{c},\overline{b}\vee\overline{c},a\vee\overline{d},b\vee\overline{d},c\vee\overline{d}\}$.
\begin{enumerate}
\item Пусть $w$~--- не выполнена. Тогда $a=b=c=0$. Найдем $q\colon$ $\forall d\in\{0,1\}$ в $L$ менее $q$ формулы выполнены. Случаи:\begin{enumerate}
\item $d=0$. Рассмотрим $L$, $\boxed{\mbox{выделим}}$ выполненные дизъюнкции: $\{\cancel{a},\cancel{b},\cancel{c},\cancel{d},\boxed{\overline{a}\vee\overline{b}},\boxed{\overline{a}\vee\overline{c}},\boxed{\overline{b}\vee\overline{c}},\boxed{a\vee\overline{d}},\boxed{b\vee\overline{d}},\boxed{c\vee\overline{d}}\}$. Выполнены 6 дизъюнкции.
\item $d=1$. $\{\cancel{a},\cancel{b},\cancel{c},\boxed{d},\boxed{\overline{a}\vee\overline{b}},\boxed{\overline{a}\vee\overline{c}},\boxed{\overline{b}\vee\overline{c}},\cancel{a\vee\overline{d}},\cancel{b\vee\overline{d}},\cancel{c\vee\overline{d}}\}$. Выполнено 4 дизъюнкции.
\end{enumerate}
Значит, $q>\max(4,6)$. Возьмем $q=7$ и докажем вторую часть.
\item Пусть $w$~--- выполнена. Тогда $\left[\begin{array}{ccc}
a & = & 1\\
b & = & 1\\
c & = & 1\\
\end{array}\right.$. Рассмотрим различные случаи, и подберем $d$ так, чтобы было выполнено $\geqslant q\equiv 7$ дизъюнкций. Поскольку $w$ и $L$ симметричны относительно замены переменных (например, $a\leftrightarrow b$), разделим случаи по количеству $a+b+c$ (количество единиц в наборе).\begin{enumerate}
\item $(a,b,c)=(1,0,0)$. Возьмем $d=0$, получим $\{\boxed{a},\cancel{b},\cancel{c},\cancel{d},\boxed{\overline{a}\vee\overline{b}},\boxed{\overline{a}\vee\overline{c}},\boxed{\overline{b}\vee\overline{c}},\boxed{a\vee\overline{d}},\boxed{b\vee\overline{d}},\boxed{c\vee\overline{d}}\}$~--- выполнено $7\geqslant 7\equiv q$ дизъюнкций.
\item $(a,b,c)=(1,1,0)$. Возьмем $d=0$, получим $\{\boxed{a},\boxed{b},\cancel{c},\cancel{d},\cancel{\overline{a}\vee\overline{b}},\boxed{\overline{a}\vee\overline{c}},\boxed{\overline{b}\vee\overline{c}},\boxed{a\vee\overline{d}},\boxed{b\vee\overline{d}},\boxed{c\vee\overline{d}}\}$~--- выполнено $7\geqslant 7\equiv q$ дизъюнкций.
\item $(a,b,c)=(1,1,1)$. Возьмем $d=1$, получим $\{\boxed{a},\boxed{b},\boxed{c},\boxed{d},\cancel{\overline{a}\vee\overline{b}},\cancel{\overline{a}\vee\overline{c}},\cancel{\overline{b}\vee\overline{c}},\boxed{a\vee\overline{d}},\boxed{b\vee\overline{d}},\boxed{c\vee\overline{d}}\}$~--- выполнено $7\geqslant 7\equiv q$ дизъюнкций.
\end{enumerate} 
\end{enumerate}
\item $\psi'=(\overline{x_1}\vee x_2\vee y)\wedge(\overline{x_1}\vee x_2\vee \overline{y})$.\newline $L_1=\{\overline{x_1},x_2,y,d_1,x_1\vee\overline{x_2},x_1\vee\overline{y},\overline{x_2}\vee\overline{y},\overline{x_1}\vee\overline{d_1},x_2\vee\overline{d_1},y\vee\overline{d_1}\}$\newline
$L_2=\{\overline{x_1},x_2,\overline{y},d_2,x_1\vee\overline{x_2},x_1\vee y,\overline{x_2}\vee y,\overline{x_1}\vee\overline{d_2},x_2\vee\overline{d_2},\overline{y}\vee\overline{d_2}\}$. Образ $\widetilde{\psi'}=L_1\cup L_2$. $k=2$ (количество дизъюнктов), Пороговое значение $kq=2\times 7=14$.
\item Возьмем набор $(x_1,x_2,y,d_1,d_2)=(0,1,1,1,0)$.\newline
Рассмотрим $L_1=\{\boxed{\overline{x_1}},\boxed{x_2},\boxed{y},\boxed{d_1},\cancel{x_1\vee\overline{x_2}},\cancel{x_1\vee\overline{y}},\cancel{\overline{x_2}\vee\overline{y}},\boxed{\overline{x_1}\vee\overline{d_1}},\boxed{x_2\vee\overline{d_1}},\boxed{y\vee\overline{d_1}}\}$~--- выполнено $7$ дизъюнкций\newline
Рассмотрим $L_2=\{\boxed{\overline{x_1}},\boxed{x_2},\cancel{\overline{y}},\cancel{d_2},\cancel{x_1\vee\overline{x_2}},\boxed{x_1\vee y},\boxed{\overline{x_2}\vee y},\boxed{\overline{x_1}\vee\overline{d_2}},\boxed{x_2\vee\overline{d_2}},\boxed{\overline{y}\vee\overline{d_2}}\}$~--- выполнено $7$ дизъюнкций\newline
Получаем, что в $\widetilde{\psi'}$ выполнено $2\times 7=14\geqslant 14\equiv kq$ дизъюнкций $\blacksquare$.
\end{enumerate}
\subsection*{(каноническое) Задача 27}
Пусть $f\colon \GC\subset\Sigma^*\to\{0,1\}$, $f(x)=1\Leftrightarrow x\in\GC$, и $T_f(x)=\poly(|x|)$ ($f$ вычислима за полиномиальное по $|x|$ время).
\begin{enumerate}
\item Построим алгоритм поиска гамильтонова цикла (если он существует), использующий $f$. Фиксируем граф $G$, его описание $x\in\Sigma^*$. Обозначим за $h(G, v)$ граф, полученный из $G$ удалением вершины $v$ и перенаправлением ребер, инцидентных $v$: Если в $G$ есть ребра $(u,v), (v, w)$, в $h(G,v)$ есть ребро $(v, w)$.\begin{enumerate}
\item Если в $G$ нет гамильтонова цикла, за один вызов $f(x)$ выдадим ответ. Время $T_f(x)$~--- полиномиально по $|x|$.
\item Пусть первый вызов вернул $1$, т.е. в графе есть гамильтонов цикл. Найдем его. Фиксируем произвольную вершину $v$ графа $G$. Рассмотрим граф $h(G,v)$. Он также гамильтонов: пусть в $G$ есть цикл $s\leftrightarrow...\leftrightarrow u\leftrightarrow v\leftrightarrow w\leftrightarrow...\leftrightarrow s$ Тогда (по построению) в $h(G,v)$ есть цикл $s\leftrightarrow...\leftrightarrow u\leftrightarrow w\leftrightarrow...\leftrightarrow s$. Он гамильтонов, так как содержит все вершины (все вершины $G$ кроме $v$~--- это вершины $h(G,v)$).\newline
Будем пребирать все вершины $v'$ графа $h(G,v)$, смежные с $v$ в $G$ и рассматривать $h(h(G,v),v')$.\begin{enumerate}
\item Хотя бы один из них будет гамильтоновым. Действительно, в гамильтоновом цикле в $G$ содержится $v$. Значит, содержится некоторая инцидентная ей вершина $v'$ (они стоят рядом). Удалением $v$ и $v'$ получаем гамильтонов путь в $h(h(G,v),v')$.
\item Докажем обратное. Пусть при некоторой $v'$ граф $h(h(G,v),v')$~--- гамильтонов. Тогда в $G$ в некотором гамильтоновом цикле $v$ и $v'$ стояли рядом. Предположим противное, (в $h(G,v)$ нет гамильтонова цикла с ребром $(v,v')$). Фиксируем Рассмотрим ГЦ в $G$: $s\leftrightarrow...\leftrightarrow$
\end{enumerate}
Значит, в некотором гамильтоновом пути в $G$ вершины $u$ и $v$ стояли рядом \todo. Продолжим этот процесс, пока не останутся две вершины $v_1$ и $v_2$. Они стоят рядом. Полученная последовательность $(v,u,u',...,u^{(l)},v_1,v_2,v)$~--- искомый гамильтонов цикл.
\end{enumerate}
\item Псевдокод
\lstset{
    language=C,
    basicstyle=\ttfamily\small,
    breaklines=true,
    prebreak=\raisebox{0ex}[0ex][0ex]{\ensuremath{\hookleftarrow}},
    frame=lines,
    showtabs=false,
    showspaces=false,
    showstringspaces=false,
    keywordstyle=\color{red}\bfseries,
    stringstyle=\color{green!50!black},
    commentstyle=\color{gray}\itshape,
    numbers=left,
    captionpos=t,
    escapeinside={\%*}{*)}
}
\begin{lstlisting}
path(x)
{
    if(!f(x)) return(empty); // no path
    else
    {
    }
}
\end{lstlisting}
\item Время работы. Всего в графе $n$ вершин, для каждой перебираем не более, чем $n$, откуда сложность $T(x)=O(n^2)$. Поскольку $|x|=\Omega(n)$, то $T(x)=O(|x|^2)=\poly(|x|)$. Более точно, используя псевдокод: \todo.
\end{enumerate}
\end{document}