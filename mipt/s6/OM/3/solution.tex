\documentclass[a4paper]{article}
\usepackage[a4paper, left=5mm, right=5mm, top=5mm, bottom=5mm]{geometry}
\usepackage[utf8]{inputenc}
\usepackage[english,russian]{babel}
\usepackage{graphicx}
\usepackage{indentfirst}
\usepackage{tikz} %Рисование автоматов
\usetikzlibrary{automata,positioning}
\usepackage{amsmath}
\usepackage{floatflt}
\usepackage{enumerate}
\usepackage{amsfonts}
\usepackage{amssymb}
\newcommand{\matrixr}{\right|\right|}
\newcommand{\matrixl}{\left|\left|}
\def\eqdef{\overset{\mbox{\tiny def}}{=}}
\title{Методы оптимизации. Задание 3}
\date{задано 2016.04.12}
\author{Сергей~Володин, 374 гр.}

\begin{document}
\maketitle
\section*{Задача 1}
(2016.04.05)
Доказать, что метод возможных направлений с $S\eqdef \{s| ||s||^2\leqslant r\}$ эквивалентен задаче квадратичного программирования
$$\begin{cases}
\sigma+\gamma ||s||^2\to\min\\
\sigma\geqslant(\nabla f,s)\\
\sigma \geqslant (\nabla g_i,s)
\end{cases}$$
при некотором $\gamma$
\section*{Задача 2}
(2016.04.12)
Пусть $\varphi$~--- дифференцируемая. Доказать или опровергнуть:
$\nabla\varphi$~--- липшицев с константой $L$ $\Leftrightarrow$ $\varphi^*$~--- выпуклая/сильно выпуклая
\section*{Задача 3}
(2016.04.12)
$\varphi(x)\eqdef f(y)-\nabla^Tf(x)\cdot y$~--- выпуклая (?), если $f$~--- выпуклая
\section*{Задача 4}
Доказать\begin{enumerate}
\item $\frac{1}{\epsilon_{k+1}}-\frac{1}{\epsilon_{k}}\geqslant \omega_k$
\item $\forall k \omega_k\geqslant \omega_1$.
\end{enumerate}
Обозначения~--- метод быстрых градиентов
\section*{Задача 5}
Пусть $x_{k+1}\eqdef \mbox{prox}_{D_\varPhi}(C,y_{k+1})$. Доказать $\forall x$ $(\nabla\varPhi(x_{k+1})-\nabla\varPhi(y_{k+1}))^T(x_{k+1}-x)\leqslant 0$
\section*{Задача 6}
Исследовать, выполняется ли неравенство треугольника (с обратным знаком) для $D_{\varPhi}$ для произвольных трех точек
\section*{Задача 7}
Выбрать наилучшее $\gamma$ для полученной на семинаре оценки $\sum (f(x_j)-f(x))$. Оценить $D_{\varPhi}(x,x_1)$ через $$R^2\eqdef \sup_{x\in C}|\varPhi(x)-\varPhi(x_1)|$$
\end{document}
