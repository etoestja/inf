\documentclass[a4paper]{article}
\usepackage[a4paper, left=5mm, right=5mm, top=5mm, bottom=5mm]{geometry}
\usepackage[utf8]{inputenc}
\usepackage[english,russian]{babel}
\usepackage{indentfirst}
\usepackage{tikz} %Рисование автоматов
\usetikzlibrary{automata,positioning}
\usepackage{amsmath}
\usepackage{enumerate}
\usepackage{amsfonts}
\usepackage{amssymb}
\usepackage{wasysym}
\title{Теория и реализация языков программирования.\\Задание 2: НКА и алгоритмы поиска подстрок}
\date{задано 2013.09.18}
\author{Сергей~Володин, 272 гр.}

% названия автоматов
\def\A{{\cal A}}
\def\B{{\cal B}}
\def\C{{\cal C}}

% регулярные языки
\def\REG{{\mathsf{REG}}}

\newcommand{\niton}{\not\owns}

\begin{document}
\maketitle
\section*{Упражнение 1}
Пусть $\sim\,\subset X\times X$. $C(x)=\{z\in X|x\sim z\}$, $C(y)=\{w\in X|y\sim z\}$.
\\[1pt]
Пусть $\exists z\in C(x)\cap C(y)$. Тогда $x\sim z,y\sim z$, и $w\in C(x)\overset{\mathrm{def}}{\Leftrightarrow}x\sim w\underset{\substack{\mbox{\tiny транз.} \\ \mbox{\tiny симм.}}}{\overset{z\sim x}{\Leftrightarrow}}z\sim w\overset{y\sim z}{\Leftrightarrow}y\sim w\overset{\mathrm{def}}{\Leftrightarrow}w \in C(y)$, то есть, $C(x)=C(y)$.\newline
В противном случае $\rceil (\exists z\in C(x)\cap C(y))\Leftrightarrow C(x)\cap C(y)=\varnothing$.
Получаем, что возможны два случая:
\begin{enumerate}[1.]
\item $C(x)\cap C(y)=\varnothing$ {\em (не пересекаются)}
\item $C(x)=C(y)$ {\em (совпадают)}
\end{enumerate}
\section*{Упражнение 2}
Пусть $\varphi\colon\Sigma^*\supseteq X\longrightarrow\Delta^*$. $\varphi(\sigma_i)\overset{\mbox{\tiny def}}{=}\delta_i\in\Delta^*$, $|\sigma_i|=1$.\newline
\begin{enumerate}
\item {\em (единственность)} Предположим, что существует такое $\varphi$~--- морфизм. Тогда $\forall w=w_1...w_n\in X, |w_i|=1\hookrightarrow\varphi(w)\equiv \varphi(w_1...w_n)=\varphi(w_1)\cdot\varphi(w_2...w_n)=...=\varphi(w_1)\cdot...\cdot\varphi(w_n)\in\Delta^*$. Для $w=\varepsilon$ получаем $\varphi(\varepsilon)=\varepsilon$, так как $\varphi$~--- морфизм: $w_0\overset{\mathrm{def}}{=}\varphi(\varepsilon)=\varepsilon$. $\varphi(\varepsilon)\equiv\varphi(\varepsilon\varepsilon)=\varphi(\varepsilon)\varphi(\varepsilon)=w_0w_0\Rightarrow w_0=w_0w_0\Rightarrow |w_0|=|w_0||w_0|\Rightarrow w_0=\varphi(\varepsilon)=\varepsilon$.
\\[5pt]
Таким образом, получаем, что такой морфизм единственный (если существует).
\item {\em (существование)} Докажем, что определенное выше отображение $\varphi$~--- морфизм: пусть $x,y\in X$. Рассмотрим случаи:
\begin{enumerate}[a.]
\item $|x|=0,|y|=0\Rightarrow\varphi(xy)=\varphi(\varepsilon\varepsilon)=\varphi(\varepsilon)=\varepsilon=\varepsilon\varepsilon=\varphi(\varepsilon)\varphi(\varepsilon)$
\item $|x|=0,|y|>0\Rightarrow\varphi(xy)=\varphi(y)=\varepsilon\varphi(y)=\varphi(x)\varphi(y)$
\item $|x|>0,|y|=0\Rightarrow\varphi(xy)=\varphi(x)=\varphi(x)\varepsilon=\varphi(x)\varphi(y)$
\item $|x|>0,|y|>0\Rightarrow\varphi(xy)=\varphi(x_1...x_my_1...y_n)=\underbrace{\varphi(x_1)...\varphi(x_m)}\underbrace{\varphi(y_1)...\varphi(y_n)}=\varphi(x)\varphi(y)$.
\end{enumerate}
\end{enumerate}
Таким образом, если заданы значения $\varphi(\sigma_i),\sigma_i\in X\subset\Sigma$, то морфизм $\varphi\colon \Sigma^*\supseteq X\longrightarrow\Delta^*$ с этими значениями существует и единственнен.
\section*{Задача 1}
Определим $R_3:\REG\ni X\longrightarrow \mathbb{N}\cup\{0\}$~--- количество применений правила $3$ из определения регулярности $X$. В случае $X=AB$ или $X=A|B, A,B\in\REG$ $R_3(X)\overset{\mathrm{def}}{=}1+R_3(A)+R_3(B)$. В случае $X=A^*, A\in\REG$, определим $R_3(X)\overset{\mathrm{def}}{=}1+R_3(A)$. В случае $X=\varnothing$ или $X=\{\sigma\}$ определим $R_3(X)\overset{\mathrm{def}}{=}0$.\newline
Пусть $\varphi\colon\Sigma^*\supset X\longrightarrow Y\subset\Delta^*$~--- морфизм, $X\in\REG$. Докажем, что $Y\equiv\varphi(X)\in\REG$ индукцией по $R_3(X)$:
\\[5pt]
$P(i)=(\forall X\in\REG\colon R_3(X)\leqslant i\,\,\forall\varphi\mbox{~--- морфизм}\hookrightarrow \varphi(X)\in\REG)$.
\\[1pt]
\begin{enumerate}[1.]
\item Докажем $P(0)$: пусть $X\in\REG\colon R_3(X)=0$. Тогда $X$ получен без применения третьего правила. Значит, $\forall\varphi$~--- морфизм либо $X=\varnothing\Rightarrow\varphi(X)=\varnothing$, либо $X=\{\sigma\}\Rightarrow\varphi(X)=\{\varphi(\sigma)\}=\{w\}, w\in \Delta^*$.\newline
\\[5pt]
Докажем, что $\Delta^*\supset\{w\}\in\REG$. $\{w\}\equiv\{\sigma_1...\sigma_n\}\equiv\{\sigma_1\}\cdot...\cdot\{\sigma_n\}$. Поскольку $\{\sigma_i\}\in\REG$, и регулярные языки замкнуты относительно конкатенации (по определению), получаем требуемое.
\\[5pt]
Итак, $\varphi(X)\in\REG\,\blacksquare$
\item Пусть $P(n)$. Докажем $P(n+1)$. Пусть $\REG\ni X\colon R_3(X)\leqslant n+1$. Если $R_3(X)<n+1$, $P(n)\Rightarrow X\in\REG$.\newline
$\varangle X\colon R_3(X)=n+1$. Возможны случаи:
\begin{enumerate}[a.]
\item $X=WZ$, $W,Z\in\REG$. Тогда $\varphi(X)\equiv\varphi(WZ)=\{\varphi(wz)|w\in W,z\in Z\}=\{\varphi(w)\varphi(z)|w\in W,z\in Z\}=\\\{\varphi(w)|w\in W\}\cdot\{\varphi(z)|z\in Z\}=\varphi(W)\varphi(Z)$. $R_3(X)=1+R_3(W)+R_3(Z)=n+1\Rightarrow R_3(W),R_3(Z)\leqslant n\overset{P(n)}{\Rightarrow} \varphi(W),\varphi(Z)\in\REG\Rightarrow \varphi(X)=\varphi(W)\varphi(Z)\in\REG$.
\item $X=W|Z$, $W,Z\in\REG$. Тогда $\varphi(X)\equiv\varphi(W|Z)\equiv\varphi(W)|\varphi(Z)$. Аналогично $R_3(W),R_3(Z)\leqslant n\overset{P(n)}{\Rightarrow}\varphi(W),\varphi(Z)\in\REG\Rightarrow\varphi(X)=\varphi(W)|\varphi(Z)\in\REG$.
\item $X=W^*, W\in\REG$. Тогда $R_3(X)=1+R_3(W)=n+1\Rightarrow R_3(W)=n\overset{P(n)}{\Rightarrow}\varphi(W)\in\REG\Rightarrow \varphi(W^*)=\varphi(\varepsilon|W|WW|...)=\varphi(\varepsilon)|\varphi(W)|\varphi(WW)...\overset{\varphi(\varepsilon)=\varepsilon}{=}\varepsilon|\varphi(W)|\varphi(WW)...=\varphi(W)^*\in\REG$.
\end{enumerate}
\end{enumerate}
Получаем $\forall i\geqslant 0\hookrightarrow P(i)\Rightarrow\forall X\in\REG\,\forall\varphi\mbox{~--- морфизм}\hookrightarrow\varphi(X)\in\REG\,\blacksquare$
\section*{Задача 2}
\begin{enumerate}[1.]
%Почему \varphi^-1 - морфизм, если \varphi - морфизм?
\item Нет. Пусть $\Sigma=\{0,1\}$, $L=\Sigma^*$. Определим $\varphi\colon L\longrightarrow L\colon \forall w\in L\hookrightarrow\varphi(w)=\varepsilon$. В этом случае $\varphi$~--- морфизм, так как $\forall x\in L\,\forall y\in L\hookrightarrow\varphi(xy)=\varepsilon=\varepsilon\varepsilon=\varphi(x)\varphi(y)$. Тогда $\forall \varnothing\neq X\subset L\hookrightarrow\varphi(X)=\{\varepsilon\}$, так как $\forall w\in L\hookrightarrow\varphi(w)=\varepsilon$. Поскольку $\varphi(\varepsilon)=\varepsilon\in L$, $\varphi^{-1}(\varepsilon)\ni\varepsilon\Rightarrow \varphi^{-1}(L)\supset\{\varepsilon\}\neq\varnothing\Rightarrow\varphi^{-1}(L)\neq\varnothing\Rightarrow\varphi(\varphi^{-1}(L))=\{\varepsilon\}\neq L$.\newline
Таким образом, $\exists L\subseteq\Sigma^*\,\exists\varphi\mbox{~--- морфизм}\colon \varphi(\varphi^{-1}(L))\neq L$.
\item Нет. Пусть $\Sigma=\{a,b\}$, $L=\{b\}^*$, $\varphi(a)\overset{\mathrm{def}}{=}\varphi(b)\overset{\mathrm{def}}{=}a$. Доопределим $\varphi$ так, чтобы оно было морфизмом (это возможно, см. упражнение 2). Тогда $\varphi(L)\equiv\varphi(\{b^*\})\ni\varphi(b)=a\Rightarrow\varphi^{-1}(\varphi(L))\supset\varphi^{-1}(a)\ni a\notin L\Rightarrow \varphi^{-1}(\varphi(L))\nsubseteq L\Rightarrow\varphi^{-1}(\varphi(L))\neq L$.\newline
Таким образом, $\exists L\subseteq\Sigma^*\,\exists\varphi\mbox{~--- морфизм}\colon \varphi^{-1}(\varphi(L))\neq L$.
\item Нет. Пусть $\Sigma=\{a,b\}$, $L=\{ab\}$, морфизм $\varphi\colon\Sigma^*\longrightarrow\Sigma^*$~--- из предыдущего пункта. Тогда $\varphi(L)=\{\varphi(ab)\}=\{\varphi(a)\varphi(b)\}=\{aa\}$, $\varphi^{-1}(L)=\{x\in\Sigma^*|\varphi(x)\in\{ab\}\}=\{x\in\Sigma^*|\varphi(x)=ab\}=\varnothing$, так как $\varphi(\Sigma^*)=\varphi((a|b)^*)\overset{1.2.c}{=}(\varphi(a|b))^*=\{\varphi(a),\varphi(b)\}^*=\{a\}^*=a^*\niton ab$. Тогда $\varphi(\varphi^{-1}(L))=\varphi(\varnothing)=\varnothing\niton aa \in\varphi^{-1}(aa)=\varphi^{-1}(\varphi(L))$.\newline
Таким образом, $\exists L\subseteq\Sigma^*\,\exists\varphi\mbox{~--- морфизм}\colon \varphi(\varphi^{-1}(L))\neq \varphi^{-1}(\varphi(L))$.
\end{enumerate}
\end{document}