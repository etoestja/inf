\documentclass[a4paper]{article}
\usepackage[a4paper, left=5mm, right=5mm, top=5mm, bottom=5mm]{geometry}
%\geometry{paperwidth=210mm, paperheight=2000pt, left=5pt, top=5pt}
\usepackage[utf8]{inputenc}
\usepackage[english,russian]{babel}
\usepackage{indentfirst}
\usepackage{tikz} %Рисование автоматов
\usetikzlibrary{automata,positioning,arrows}
\usepackage{amsmath}
\usepackage{enumerate}
\usepackage{amsfonts}
\usepackage{amssymb}
\usepackage{wasysym}
\title{Теория и реализация языков программирования.\\Задание 6: Грамматики}
\date{задано 2013.10.09}
\author{Сергей~Володин, 272 гр.}
\newcommand{\matrixl}{\left|\left|}
\newcommand{\matrixr}{\right|\right|}
% названия автоматов
\def\A{{\cal A}}
\def\B{{\cal B}}
\def\C{{\cal C}}

%+= и -=, иначе разъезжаются...
\newcommand{\peq}{\mathrel{+}=}
\newcommand{\meq}{\mathrel{-}=}
\newcommand{\deq}{\mathrel{:}=}
\newcommand{\plpl}{\mathrel{+}+}

% регулярные языки
\def\REG{{\mathsf{REG}}}
\def\eqdef{\overset{\mbox{\tiny def}}{=}}
\newcommand{\niton}{\not\owns}

\begin{document}
\maketitle
\section*{Задача 1}
\section*{Задача 2}
$\Sigma\eqdef\{a,b\}$, $\Sigma^*\supset L\eqdef\{w\in\Sigma^*\big|w=w^R\}$~--- язык палиндромов из $a,b$.
\begin{enumerate}
\def\ansba{$\Gamma\eqdef(\{S\},\Sigma,P,S)$, $P\eqdef\big\{\underbrace{S\longrightarrow\varepsilon}_{(1)},\,\underbrace{S\longrightarrow aSa}_{(2)},\,\underbrace{S\longrightarrow bSb}_{(3)},\,\underbrace{S\longrightarrow a}_{(4)},\,\underbrace{S\longrightarrow b}_{(5)}\big\}$}
\def\ansbac{\Gamma\eqdef(\{S\},\Sigma,P,S)$, $P\eqdef\big\{S\longrightarrow\varepsilon,\,S\longrightarrow aSa,\,S\longrightarrow bSb,\,S\longrightarrow a,\,S\longrightarrow b\big\}}
\item Определим КС-грамматику \ansba.\newline
%\begin{enumerate}[a.]
%\item $S\longrightarrow\varepsilon$,
%\item $S\longrightarrow aSa$,
%\item $S\longrightarrow bSb\big\}$
%\end{enumerate}
Докажем, что $L(\Gamma)=L$:
\begin{enumerate}
\item $L(\Gamma)\subseteq L$. Пусть $w\in L(\Gamma)\Rightarrow S\Longrightarrow^* w$. $|w|=n$. Рассмотрим последовательность $\{w_k\}^I_{k=0}\subset (N\cup\Sigma)^*$ слов в выводе. $w_0=S,\,w_I=w$. Индукцией по $k$ докажем $P(k)=\big[w_k=w_k^R,\forall i\colon w_k[i]=S\hookrightarrow i=\frac{|w_k|+1}{2}\big]$.\begin{enumerate}[1.]
\item $k=0\Rightarrow w_k\equiv w_0=S$. Поэтому $\exists ! i=1\colon w_0[i]=S$. Но $1\equiv\frac{1+1}{2}$ и $w_0^R=S^R=S=w_0\Rightarrow P(0)\,\blacksquare$
\item Пусть $P(n),\,n<I$. Докажем, $P(n+1)$. $P(n)\Rightarrow w_n=w_n^R,\,\forall i\colon w_n[i]=S\hookrightarrow i=\frac{|w_n|+1}{2}$.
\\[1pt]
Предположим, что $\nexists i\colon w_n[i]=S\Rightarrow w\in \Sigma^*$. Тогда ни одно правило не может быть применено, так как в левой части каждого правила $S\in N$. Но $n<I$ (это не конец вывода) $\Rightarrow$ противоречие.
\\[1pt]
Значит, $\exists i\colon w_n[i]=S$. Но $P(n)\Rightarrow \forall i\colon w_n[i]=S\hookrightarrow i=\frac{|w_n|+1}{2}$. Поэтому $\exists!i=\frac{|w_n|+1}{2}\colon w_n[i]=S$. Значит, $w_n=xSy$, $|x|=|y|$, $x,y\in\Sigma^*$. $w_n^R=y^RSx^R$. $S$ в $w_n$ входит один раз $\Rightarrow$ $x=y^R$.\newline
Рассмотрим правила $(1)$---$(4)$:\begin{enumerate}[(1).]
%\item $w_n[1]...w_n[\frac{n-1}{2}]Sw_n[\frac{n+3}{2}]...w_n[n]\overset{(1)}{\Longrightarrow}w_n[1]...w_n[\frac{n-1}{2}]w_n[\frac{n+3}{2}]...w_n[n]\equiv w_{n+1}$.\newline
%Но $w_n^R=w_n\Rightarrow$ $w_n[1]=w_n{n},...,w_n[\frac{n-1}{2}]=w_n[\frac{n+3}{2}]$. Поэтому $w_{n+1}^R=w_{n+1}]$.\newline
%Также $S$ не встречается в $w_{n+1}\Rightarrow P(n+1)\,\blacksquare$
\item $w_n=xSy\overset{(1)}{\Longrightarrow} x\varepsilon y\equiv xy=xx^R=w_{n+1}$~--- палиндром: $(xx^R)^R={x^R}^Rx^R=xx^R$. $w_{n+1}=xy\in\Sigma^*\Rightarrow\forall i\hookrightarrow w_{n+1}[i]\neq S\Rightarrow P(n+1)\,\blacksquare$
\item $w_n=xSx^R\overset{(2)}{\Longrightarrow}xaSax^R=w_{n+1}$. $w_{n+1}^R={x^R}^Ra^RS^Ra^Rx^R=xaSax^R\equiv w_{n+1}$. $a\neq S\Rightarrow \exists!i\colon w_{n+1}=S$,\newline$i=\frac{|w_n|+1}{2}+1=\frac{|w_n|+3}{2}\equiv\frac{|w_{n+1}|+1}{2}\Rightarrow P(n+1)\,\blacksquare$.
\item $w_n=xSx^R\overset{(3)}{\Longrightarrow}xbSbx^R=w_{n+1}$. $w_{n+1}^R={x^R}^Rb^RS^Rb^Rx^R=xbSbx^R\equiv w_{n+1}$. $b\neq S\Rightarrow \exists!i\colon w_{n+1}=S$,\newline$i=\frac{|w_n|+1}{2}+1=\frac{|w_n|+3}{2}\equiv\frac{|w_{n+1}|+1}{2}\Rightarrow P(n+1)\,\blacksquare$.
\item $w_n=xSx^R\overset{(4)}{\Longrightarrow}xax^R=w_{n+1}$. $w_{n+1}^R={x^R}^Ra^Rx^R=xax^R\equiv w_{n+1}$. $w_{n+1}=xax^R\in\Sigma^*\Rightarrow \forall i\hookrightarrow w_{n+1}[i]\neq S\Rightarrow P(n+1)\,\blacksquare$
\item $w_n=xSx^R\overset{(5)}{\Longrightarrow}xbx^R=w_{n+1}$. $w_{n+1}^R={x^R}^Rb^Rx^R=xbx^R\equiv w_{n+1}$. $w_{n+1}=xbx^R\in\Sigma^*\Rightarrow \forall i\hookrightarrow w_{n+1}[i]\neq S\Rightarrow P(n+1)\,\blacksquare$
\end{enumerate}
Итак, доказано $\forall k\in\overline{0,I}\hookrightarrow P(k)\Rightarrow P(I)\Rightarrow w\equiv w_I\overset{P(I)}{=}w_I^R\equiv w^R\Rightarrow w\in L\,\blacksquare$ 
\end{enumerate}
\item $L\subseteq L(\Gamma)$. Пусть $w\in L\Rightarrow w^R=w$. $|w|=n$. Рассмотрим $n\mod 2$:\begin{enumerate}[0.]
\item $n\mod 2=0\Rightarrow w=xy,\,|x|=|y|$. $w=w^R\Rightarrow xy=y^Rx^R$. Поскольку $|x|=|y|$, $y=x^R\Rightarrow\boxed{w=xx^R}$.
\item $n\mod 2=1\Rightarrow w=x\sigma y,\, |x|=|y|,\,\sigma\in\Sigma$. $w=w^R\Rightarrow x\sigma y=y^R\sigma^Rx^R=y^R\sigma x^R$. Так как $|x|=|y|$, $y=x^R\Rightarrow\boxed{w=x\sigma x^R}$.
\end{enumerate}
Значит, $L=\{xx^R,\,xax^R,\,xbx^R\big| x\in\Sigma^*\}$.
\\[5pt]
Построим вывод $S\Longrightarrow^*xSx^R$:\begin{enumerate}[a.]
\item Пусть $x=\varepsilon$. $S\overset{(1)}{\Longrightarrow}\varepsilon=\varepsilon\varepsilon^R=w\Rightarrow w\in L(\Gamma)\,\blacksquare$. 
\item Иначе $x=x_1...x_m,\,\forall i\in\overline{1,m}\hookrightarrow x_i\in\Sigma$. Рассмотрим символы $x_m,...,x_1$. Применим правило $(2)$, если $x_i=a$ и $(3)$ иначе. Примененное правило обозначим за $R(i)$ Получим $S\overset{(R(m))}{\Longrightarrow}x_mSx_m\Longrightarrow...\overset{(R(1))}{\Longrightarrow}x_1...x_mSx_m...x_1$.
\end{enumerate}
Теперь покажем, как получить $w$:\begin{enumerate}[1.]
\item $w=xx^R$. Было получено $S\Longrightarrow^*xSx^R$. Тогда $S\Longrightarrow^*xSx^R\overset{(1)}{\Longrightarrow}xx^R\,\blacksquare$
\item $w=xax^R$. Было получено $S\Longrightarrow^*xSx^R$. Тогда $S\Longrightarrow^*xSx^R\overset{(4)}{\Longrightarrow}xax^R\,\blacksquare$
\item $w=xbx^R$. Было получено $S\Longrightarrow^*xSx^R$. Тогда $S\Longrightarrow^*xSx^R\overset{(5)}{\Longrightarrow}xbx^R\,\blacksquare$
\end{enumerate}
Получаем $w\in L(\Gamma)$.\newline
$\boxed{\mbox{Ответ: }\ansbac}$
\end{enumerate}
\item Определим грамматику $\overline{\Gamma}$
\end{enumerate}
\section*{Задача 3}
$\Sigma\eqdef\{a,b\}$. $\Sigma^*\supset L\eqdef L^=\eqdef\{w\in\Sigma^*\big||w|_a=|w|_b\}$. КС-грамматика $\Gamma=\{N,\Sigma,P,S\}$,\newline
$P\eqdef\big\{\underbrace{S\longrightarrow SS}_{(1)},\,\underbrace{S\longrightarrow aSb}_{(2)},\,\underbrace{S\longrightarrow bSa}_{(3)},\,\underbrace{S\longrightarrow \varepsilon}_{(4)}\big\}$.\newline
Докажем, что $L(\Gamma)=L^=$:\begin{itemize}
\item $L(\Gamma)\subset L$. $w\in L(\Gamma)\Rightarrow S\Longrightarrow^* w$. Пусть $\{w_k\}_{k=0}^I\subset (N\cup \Sigma)^*$~--- последовательность слов при выводе.\newline
$P(k)\eqdef\big[|w_k|_a=|w_k|_b\big]$. Докажем, что $\forall k\in\overline{0,I}\hookrightarrow P(k)$:\begin{enumerate}[1.]
\item $k=0\Rightarrow w_k\equiv w_0=S$. $|w_k|_a=|S|_a=0=|S|_b=|w_k|_b\Rightarrow P(0)$.
\item $P(n)\Rightarrow |w_n|_a=|w_n|_b$. $n<I$. Пусть $w_n\overset{(i)}{\Longrightarrow}w_{n+1}$. Каждое из правил $(1)$---$(4)$ сохраняет равенство между $|w|_a$ и $|w|_b$:\newline$(1)$ и $(4)$ не изменяют их, а $(2)$ и $(3)$ увеличивают каждое на 1 $\Rightarrow |w_{n+1}|_a=|w_{n+1}|_b\Rightarrow P(n+1)$
\end{enumerate}
Получаем $P(I)\Rightarrow |w|_a\equiv|w_I|_a\overset{P(I)}{=}|w_I|_b\equiv|w|_b\Rightarrow w\in L^=$.
\item $L\subset L(\Gamma)$.\newline
Определим $S\colon \Sigma^*\longrightarrow {\mathbb Z}\colon w\in\Sigma^*\Rightarrow S(w)=|w|_a-|w|_b$.
$w\in L^=\Leftrightarrow |w|_a=|w|_b\Leftrightarrow S(w)=0$. $|w|=n$. $w\in\Sigma^*\Rightarrow|w|=|w|_a+|w|_b=2|w|_a\Rightarrow |w|$~--- четно $\Rightarrow n=2m$.
\\[5pt]
Пусть $L\ni w=axa$. Тогда $0=S(w)=|axa|_a-|axa|_b=2+S(x)\Rightarrow S(x)=-2$. Отсюда следует, что $|x|\geqslant 0$. Пусть $|x|=t$, $x=x_1...x_t,\forall i\in\overline{1,t}\hookrightarrow x_i\in\Sigma$. Обозначим $f(t)\colon \overline{1,t}\longrightarrow {\mathbb Z}\colon f(i)=S(ax_1...x_i)$. Тогда $f(0)\equiv S(a)=1$, $f(t)\equiv S(ax_1...x_t)=1+S(x)=1-2=-1$. Заметим, что $|f(t+1)-f(t)|=1$ {\em (<<аналог непрерывности>>)}. Поэтому $\exists i\in\overline{1,t}\colon f(t)=0$ {\em <<принимает промежуточное значение>>}. Получаем, что $w=ax_1...x_ix_{i+1}...x_ta=w_lw_r$. Поскольку $0=S(w)=S(w_l)+S(w_r)$ и $S(w_l)\equiv f(i)=0$, $S(w_r)=0$. $S(w_l)=S(w_r)=0\Rightarrow w_l,w_r\in L$. Поскольку $|w_l|,|w_r|\geqslant |a|=1, |w|_l,|w_r|\leqslant |w|-1$. Но $w,w_l,w_r\in L\Rightarrow |w|,|w_l|,|w_r|$~--- четные. значит, $|w_l|,|w_r|\leqslant |w|-2$. Итак, $w=axa\in L\Rightarrow w=w_lw_r, |w_l|, |w_r|\in\overline{1, |w|-2}, w_l,w_r\in L$. Аналогично доказываем для $L\ni w=bxb$. Получаем $\boxed{w=\sigma x\sigma\in L\Rightarrow w=w_lw_r, |w_l|, |w_r|\in\overline{1, |w|-2}, w_l,w_r\in L}$.
\\[5pt]
$P(m)\eqdef \big[\forall w\in L\colon |w|=2m\hookrightarrow w\in L(\Gamma)\big]$. Докажем $\forall i\geqslant 0\hookrightarrow P(i)$:\begin{enumerate}
\item $m=0\Rightarrow w=\varepsilon$. $S\overset{(4)}{\Longrightarrow} \varepsilon=w\Rightarrow w\in L(\Gamma)$
\item Пусть $P(m)$. Докажем $P(m+1)$. Рассмотрим $w\in L\colon |w|=2(m+1)>2$. Значит, $w=\sigma_1x\sigma_2$. Заметим, что $|x|=2m$. Рассмотрим варианты для $(\sigma_1,\sigma_2)$:\begin{enumerate}[1.]
\item $\sigma_1=a,\sigma_2=b$. Тогда $w\in L\Rightarrow 0=S(w)=|axb|_a-|axb|_b=1+|x|_a-|x|_b-1=S(x)$. Как было замечено, $|x|=2m$, поэтому, по предположению индукции, $S\overset{P(m)}{\Longrightarrow^*}x$. Но $S\overset{(2)}{\Longrightarrow}aSb\overset{P(m)}{\Longrightarrow^*}axb\Rightarrow w\in L(\Gamma)$
\item $\sigma_1=b,\sigma_2=a$. Аналогично получаем $w=bxa$, $x\in L(\Gamma)\Rightarrow S\overset{(3)}{\Longrightarrow}bSa\overset{P(m)}{\Longrightarrow^*}bxa\Rightarrow w\in L(\Gamma)$
\item[3, 4.] $\sigma\eqdef\sigma_1=\sigma_2$. Разобьем слово $w=\sigma x\sigma$ на {\em подслова} $w=w_1...w_k,\newline\forall i\in\overline{1,k}\hookrightarrow w_i\in\Sigma^*\cap L,\, |w_i|\leqslant |w|-2,\,w_i[1]\neq w_i[|w_i|]$.\newline
 Для этого воспользуемся утверждением в рамочке (см. выше): разобьем $w=w_lw_r$, потом, если первый и последний символы $w_l$ совпадают, повторим для него (возможно, так как $w_l\in L$ по построению): $w_l=w_{ll}w_{lr}$, аналогично для $w_r$. Всего разбиений будет не больше $|w|$, так как части разбиения непустые (см. утверждение) $\Rightarrow$ алгоритм конечен. Каждое разбиение дает подслова из $L$~--- также см. утверждение. И части разбиения не длиннее исходного слова, а также $w_l,w_r\leqslant |w|-2$. Значит, $w_i\leqslant |w|-2$. Поэтому $S\overset{P(m)}{\Longrightarrow^*}w_l$, $S\overset{P(m)}{\Longrightarrow^*}w_r$~--- по предположению индукции. Покажем, как вывести $w$ из $S$: воспользуемся правилом $(1)$ $k-1$ раз:\newline
 $S\overset{(1)}{\Longrightarrow}SS\overset{(1)}{\Longrightarrow}...\overset{(1)}{\Longrightarrow}S^k$.
 Далее воспользуемся выводами $w_i$: $S^k\overset{\mbox{\tiny вывод }w_1}{\Longrightarrow^*}w_1S^{k-1}\Longrightarrow^*...\overset{\mbox{\tiny вывод }w_k}{\Longrightarrow^*}w_1...w_k\equiv w\Rightarrow w\in L(\Gamma)$
\end{enumerate}
% ура, блин... долбаные 3 часа... долбаная рамочка...
%Тогда $w=x_1...x_m,\,\forall i\in\overline{1,m}\hookrightarrow |x_i|=2$. Применим правило $(1)$ $m$ раз. Получим $S\Longrightarrow^*S^m$.
\end{enumerate}
\end{itemize}
%\begin{enumerate}[a.]
%\item $S\longrightarrow SS$,
%\item $S\longrightarrow aSb$,
%\item $S\longrightarrow bSa\big\}$
%\end{enumerate}
\section*{Задача 4}
$\Sigma\eqdef\{a,b\}$, $\Sigma^*\supset L\eqdef\{w\in\Sigma^*\big||w|_b\leqslant|w|_a\}\equiv\{w\in\Sigma^*\big|S(w)\geqslant 0\}$ {\em (определение $S(w)$ см. в задаче 3)}.\newline
Определим КС-грамматику $\Gamma=\{\{S\},\Sigma,P,S\}$,\newline
$P\eqdef\big\{\underbrace{S\longrightarrow SS}_{(1)},\,\underbrace{S\longrightarrow aSb}_{(2)},\,\underbrace{S\longrightarrow bSa}_{(3)},\,\underbrace{S\longrightarrow \varepsilon}_{(4)},\,\underbrace{S\longrightarrow aS}_{(5)},\,\underbrace{S\longrightarrow Sa}_{(6)}\big\}$\newline (добавим в грамматику из предыдущей задачи правила $S\longrightarrow Sa$ и $S\longrightarrow aS$).\newline
Докажем, что $L(\Gamma)=L$:\begin{itemize}
\item $L(\Gamma)\subset L$. $w\in L(\Gamma)\Rightarrow S\Longrightarrow^* w$. Пусть $\{w_k\}_{k=0}^I\subset (N\cup \Sigma)^*$~--- последовательность слов при выводе.\newline
$P(k)\eqdef\big[S(w_k)\geqslant 0\big]$. Докажем, что $\forall k\in\overline{0,I}\hookrightarrow P(k)$:\begin{enumerate}[1.]
\item $k=0\Rightarrow w_k\equiv w_0=S$. $|w_k|_a=|S|_a=0=|S|_b=|w_k|_b\Rightarrow P(0)$.
\item $P(n)\Rightarrow S(w_n)\geqslant 0$. $n<I$. Пусть $w_n\overset{(i)}{\Longrightarrow}w_{n+1}$. Каждое из правил $(1)$---$(4)$ не уменьшает разницу $|w|_a-|w|_b\equiv S(w)$:\newline$(1)$ и $(4)$ не изменяют операнды, $(2)$ и $(3)$ увеличивают каждое на 1, $(5)$ и $(6)$ увеличивает разницу на 1$\Rightarrow S(w_{n+1})\geqslant 0\Rightarrow P(n+1)$
\end{enumerate}
Получаем $P(I)\Rightarrow S(w)\equiv S(w_I)\geqslant 0\Rightarrow w\in L$.
\item $L\subset L(\Gamma)$. Докажем, что $\forall w\in L\,\exists w^=\in L^=\colon w$~--- слово $w_0$ с добавленными в некоторые позиции символами $a$. Действительно, рассмотрим $w$, удалим из него $S(w)$ любых символов $a$, получим $w_0$, $S(w_0)=0\Rightarrow w_0\in L^=$, и $w$ получается из $w_0$ добавлением символов $a$ (в те же позиции, с которых они были удалены).
\\[5pt]
\def\neqv{n^=}
\def\weqv{w^=}
\def\ieq{^=}
Фиксируем $w\in L$, $|w|=n$; $w^=$ найдем из доказанного утверждения выше. $|w^=|=\neqv$, $w^=\in L^=\Rightarrow \exists \{x_i\}^I_{i=0}\subset\{S\}\cup \Sigma^*$~--- последовательность слов при выводе слова $w^=$ в грамматике $\Gamma^=$ {\em из предыдущей задачи}, $\{p_i\}_{i=1}^I$~--- примененные правила.\newline
Определим $f\colon \overline{1,\neqv}\longrightarrow {\mathbb N}\cup\{0\}$: $f(i)$~--- количество букв $a$, которые нужно добавить после $i$-го символа $\weqv$, чтобы получить после всех добавлений $w$, то есть, $w=\weqv_1a^{f(1)}\weqv_2a^{f(2)}...\weqv_na^{f(\neqv)}$.
\\[5pt]
Модифицируем вывод, добавив буквы $a$.
\\[5pt]
Заметим, что если в $\Gamma$ было применено правило $\alpha_1\alpha_2S\alpha_3\Longrightarrow\alpha_1\alpha_2\beta\alpha_3$, то после добавления символа $a$ то же правило из $\Gamma$ также может быть применено: $\alpha_1a\alpha_2S\alpha_3\Longrightarrow\alpha_1a\alpha_2\beta\alpha_3$, аналогично в случае, где $a$ добавлено после $S$. Иными словами, добавление букв $a$ оставляет возможность применить те же правила к тем же символам $S$ в последующих шагах вывода, причем результатом будет слово с добавленной буквой $a$. Такие же рассуждения применимы к добавлению многих букв $a$.
\\[5pt]
Каждый символ $\weqv_i$ был получен из правил $(2)$ или $(3)$ грамматики $\Gamma\ieq$ (остальные правила не добавляют терминалов). Пусть это произошло на $j(i)$-м шаге вывода. Покажем, как модифицировать этот шаг, чтобы после $i$-го символа $\weqv$ добавить $f(i)$ букв $a$:\begin{enumerate}
\item $\weqv_i=a,\,p_{j(i)}=(2)$. То есть $x_{j(i)-1}\equiv\alpha S\beta\overset{(2)}{\Longrightarrow}\alpha \underline{a}Sb\beta\equiv x_{j(i)}$. Заменим это на\newline
$\alpha S\beta\overset{(2)}{\Longrightarrow}\alpha \underline{a}Sb\beta\underbrace{\overset{(5)}{\Longrightarrow}\alpha \underline{a}aSb\beta\overset{(5)}{\Longrightarrow}...\overset{(5)}{\Longrightarrow}\alpha \underline{a}a^{f(i)}Sb\beta}_{f(i)\mbox{ раз}}$~--- верное применение правил в $\Gamma$.
\item $\weqv_i=a,\,p_{j(i)}=(3)$. То есть $x_{j(i)-1}\equiv\alpha S\beta\overset{(3)}{\Longrightarrow}\alpha  bS\underline{a}\beta\equiv x_{j(i)}$. Заменим это на\newline
$\alpha S\beta\overset{(1)}{\Longrightarrow}\alpha SS\beta\overset{(2)}{\Longrightarrow}\alpha bS\underline{a}S\beta\underbrace{\overset{(5)}{\Longrightarrow}\alpha bS\underline{a}aS\beta\overset{(5)}{\Longrightarrow}...\overset{(5)}{\Longrightarrow}\alpha bS\underline{a}a^{f(i)}S\beta}_{f(i)\mbox{ раз}}\overset{(4)}{\Longrightarrow}\alpha bS\underline{a}a^{f(i)}\beta$~--- верное применение правил в $\Gamma$.
\item $\weqv_i=b,\,p_{j(i)}=(3)$. То есть $x_{j(i)-1}\equiv\alpha S\beta\overset{(3)}{\Longrightarrow}\alpha \underline{b}Sa\beta\equiv x_{j(i)}$. Заменим это на\newline
$\alpha S\beta\overset{(3)}{\Longrightarrow}\alpha \underline{b}Sa\beta\underbrace{\overset{(5)}{\Longrightarrow}\alpha \underline{b}aSa\beta\overset{(5)}{\Longrightarrow}...\overset{(5)}{\Longrightarrow}\alpha \underline{b}a^{f(i)}Sa\beta}_{f(i)\mbox{ раз}}$~--- верное применение правил в $\Gamma$.
\item $\weqv_i=b,\,p_{j(i)}=(2)$. То есть $x_{j(i)-1}\equiv\alpha S\beta\overset{(2)}{\Longrightarrow}\alpha  aS\underline{b}\beta\equiv x_{j(i)}$. Заменим это на\newline
$\alpha S\beta\overset{(1)}{\Longrightarrow}\alpha SS\beta\overset{(3)}{\Longrightarrow}\alpha aS\underline{b}S\beta\underbrace{\overset{(5)}{\Longrightarrow}\alpha aS\underline{b}aS\beta\overset{(5)}{\Longrightarrow}...\overset{(5)}{\Longrightarrow}\alpha aS\underline{b}a^{f(i)}S\beta}_{f(i)\mbox{ раз}}\overset{(4)}{\Longrightarrow}\alpha aS\underline{b}a^{f(i)}\beta$~--- верное применение правил в $\Gamma$.
\end{enumerate}
Дальнейшее применение правил (после этой измененной части) останется возможным (см. выше), результатом будет <<старый>> результат, с $f(i)$ буквами $a$ после соответствующего символа (также показано ранее).\newline
Таким образом получено слово $\acute{w}=\weqv_1...\weqv_ia^{f(i)}...\weqv_{\neqv}$, получен его вывод в $\Gamma$. Применим такие же рассуждения для остальных символов, получим вывод $w$ в $\Gamma$ $\Rightarrow w\in L(\Gamma)\,\blacksquare$
\end{itemize}
\section*{Задача 5}
\end{document}
