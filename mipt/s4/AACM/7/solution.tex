\documentclass[a4paper]{article}
\usepackage[a4paper, left=5mm, right=5mm, top=5mm, bottom=5mm]{geometry}
%\geometry{paperwidth=210mm, paperheight=2000pt, left=5pt, top=5pt}
\usepackage[utf8]{inputenc}
\usepackage[english,russian]{babel}
\usepackage{indentfirst}
\usepackage{tikz} %Рисование автоматов
\usetikzlibrary{automata,positioning,arrows,trees,calc}
\usepackage{amsmath}
\usepackage[makeroom]{cancel} % зачеркивание
\usepackage{multicol,multirow} %Несколько колонок
\usepackage{hyperref}
\usepackage{tabularx}
\usepackage{amsfonts}
\usepackage{amssymb}
\DeclareMathOperator*{\argmin}{arg\,min}
\usepackage{listings}
\usepackage{wasysym}
\date{задано 2014.03.27}
\author{Сергей~Володин, 272 гр.}
\newcommand{\matrixl}{\left|\left|}
\newcommand{\matrixr}{\right|\right|}
% названия автоматов  (rubtsov)
\def\A{{\cal A}}
\def\B{{\cal B}}
\def\C{{\cal C}}

%классы сложности (rubtsov)
\def\PP{{\mathsf{P}}}
\def\NP{{\mathsf{NP}}}
\def\NPc{{\mathsf{NP}}\text{-}{\rm c}}
\def\coNP{{\rm co}\text{-}{\mathsf{NP}}}
\def\DTIME{{\mathsf{DTIME}}}
\def\NTIME{{\mathsf{NTIME}}}
\def\CLIQUE{{\mathsf{CLIQUE}}}
\def\HALT{{\rm{HALT}}}
\def\SAT{{\rm{SAT}}}
\def\3SAT{{\rm{3\text{-}SAT}}}
\def\2SAT{{\rm{2\text{-}SAT}}}
\def\UNSAT{{\rm{UNSAT}}}
\def\HP{{\rm{HAMPATH}}}
\def\UHP{{\rm{UHAMPATH}}}
\def\LL{{\mathrm{LL}}}
\def\poly{{\rm{poly}}}
\def\GC{{\mbox{ГЦ}}}
\def\GP{{\mbox{ГП}}}

\title{Алгоритмы и модели вычислений.\\Задание 7: потоки}

% алгоритмы (Рубцов)
\usepackage{verbatim}
\usepackage{listings}
\usepackage{algorithm2e}

%+= и -=, иначе разъезжаются...
\newcommand{\peq}{\mathrel{+}=}
\newcommand{\meq}{\mathrel{-}=}
\newcommand{\deq}{\mathrel{:}=}
\newcommand{\plpl}{\mathrel{+}+}

\newcommand{\todo}{{\em todo}}

% пустое слово  (rubtsov)
\def\eps{\varepsilon}

% регулярные языки  (rubtsov)
\def\REG{{\mathsf{REG}}}
\def\CFL{{\mathsf{CFL}}}
\def\eqdef{\overset{\mbox{\tiny def}}{=}}
\newcommand{\niton}{\not\owns}

%FIRST & FOLLOW (rubtsov)
\def\first{\mathrm{ FIRST} }
\def\follow{\mathrm{ FOLLOW} }

% LL LR (rubtsov)
\def\LL{{\mathrm{LL}}}
\def\LR{{\mathrm{LR}}}

\newcounter{rowItemCount}
\newcounter{subRowItemCount}
\newcommand\rowItem{
    \setcounter{subRowItemCount}{0}
    \arabic{rowItemCount}.\addtocounter{rowItemCount}{1}}
\newcommand\subRowItem{
    \addtocounter{subRowItemCount}{1}
    \addtocounter{rowItemCount}{-1}
    \arabic{rowItemCount}.\arabic{subRowItemCount}.\addtocounter{rowItemCount}{1}}
    
\newcommand{\smalll}[1]{\overline{\overline{#1}}}
\newcommand{\smallo}{\bar{\bar{o}}}
\newcommand{\ZZ}{\mathbb{Z}}
\newcommand{\Nz}{\mathbb{N}\cup\{0\}}
\begin{document}
\maketitle
\subsection*{Определения}
{\em (сюда будут ссылки)}\newline
$(G(V,E)$, $c\colon V^2\to\Nz, s, t)$~--- транспортная сеть $\Leftrightarrow$ \begin{enumerate}
\item \label{defNCg0}$c(u,v)\geqslant 0$
\item \label{defNCE} $\forall (u,v)\in V^2\hookrightarrow \big((u,v)\in E\Leftrightarrow c(u,v)>0\big)$
\end{enumerate}
$f\colon V^2\to\ZZ$~--- поток в этой сети $\Leftrightarrow$\begin{enumerate}
\item \label{defFl} $\forall(u,v)\in V^2\hookrightarrow \big(f(u,v)\leqslant c(u,v)\big)$
\item \label{defFs} $\forall(u,v)\in V^2\hookrightarrow \big(f(u,v)=-f(v,u)\big)$
\item \label{defFc} $\forall u\in V^2\setminus\{s,t\}\hookrightarrow f(u,V)=0$
\end{enumerate}
\subsection*{Упражнение 0}
\begin{enumerate}
\item \label{uvvu0} Пусть $(G(V,E)$, $c\colon V^2\to\Nz, s, t)$~--- транспортная сеть. Пусть $(u,v)\notin E$, $(v,u)\notin E$. Тогда $f(u,v)=f(v,u)=0$.\newline
$(u,v)\notin E\overset{\ref{defNCE}}{\Rightarrow} c(u,v)=0$. $(v,u)\notin E\overset{\ref{defNCE}}{\Rightarrow} c(v,u)=0$. Но $-0=-c(v,u)\overset{\ref{defFl}}{\leqslant} -f(v,u)\overset{\ref{defFs}}{=}\underline{f(u,v)}\overset{\ref{defFl}}{\leqslant} c(u,v)=0$, откуда $f(u,v)=f(v,u)=0\,\blacksquare$
\end{enumerate}
\subsection*{Упражнение 1}
Пусть $(G(V,E)$, $c\colon V^2\to\Nz, s, t)$~--- транспортная сеть. Фиксируем $u\notin\{s,t\}$. Пусть $L=\{v\in V\big| (v,u)\in E\}$, $R=\{v\in V\big|(u,v)\in E\}$~--- вершины, из которых (в которые, соответственно) есть ребра в фиксированную. Тогда $f(L,u)=f(u,R)$.\newline
Найдем $$0\overset{\ref{defFc}}{=}f(u,V)\equiv\sum\limits_{v\in V}f(u,v)=
\underbrace{\sum\limits_{\substack{v\in V\\ (u,v)\in E\\ (v,u)\in E}}f(u,v)}_{S_1}+
\underbrace{\sum\limits_{\substack{v\in V\\ (u,v)\in E\\ (v,u)\notin E}}f(u,v)}_{S_2}+
\underbrace{\sum\limits_{\substack{v\in V\\ (u,v)\notin E\\ (v,u)\in E}}f(u,v)}_{S_3}+
\underbrace{\sum\limits_{\substack{v\in V\\ (u,v)\notin E\\ (v,u)\notin E}}f(u,v)}_{S_4}$$
$(u,v)\notin E$, $(v,u)\notin E$ $\overset{\ref{uvvu0}}{\Rightarrow}$ $f(u,v)=0$, поэтому $S_4=0$. Рассмотрим $S_1=\sum\limits_{\substack{v\in V\\ (u,v)\in E\\ (v,u)\in E}}f(u,v)\overset{\ref{defFs}}{=}\sum\limits_{\substack{v\in V\\ (u,v)\in E\\ (v,u)\in E}}(-f(v,u))=-\sum\limits_{\substack{v\in V\\ (u,v)\in E\\ (v,u)\in E}}f(v,u)\boxed{=}$. Переобозначим вершины, получим $\boxed{=}-\sum\limits_{\substack{u\in V\\ (v,u)\in E\\ (u,v)\in E}}f(u,v)=-S_1$, откуда $S_1=0$.\newline
Рассмотрим $f(L,u)=\sum\limits_{(v,u)\in E}f(v,u)=-\sum\limits_{(v,u)\in E}f(u,v)=-(S_1+S_3)\overset{S_1=0}{\equiv} -S_3$\newline
Рассмотрим $f(u,R)=\sum\limits_{(u,v)\in E}f(u,v)=S_1+S_2\overset{S_1=0}{\equiv}S_2$.\newline
Из $(*)$ получаем $0\overset{S_1=0}{\underset{S_4=0}{=}}S_2+S_3$, откуда $S_2=-S_3$, и $f(L,u)=f(u,R)\,\blacksquare$
\newpage
\subsection*{Упражнение 2}
\label{fst} Пусть $(G(V,E)$, $c\colon V^2\to\Nz, s, t)$~--- транспортная сеть. $f$~--- поток в ней.\newline
Рассмотрим $A\eqdef\sum\limits_{\substack{u\in V\\v\in V}} f(u,v)$. Переобозначим, получим $A=\sum\limits_{\substack{v\in V\\u\in V}}f(v,u)\overset{\ref{defFs}}{=}-\sum\limits_{\substack{v\in V\\u\in V}}f(u,v)=-A$, откуда $A=0$\newline
Но $A=\underbrace{\sum\limits_{\substack{u=s\\v\in V}}f(u,v)}_{S_1}+\underbrace{\sum\limits_{\substack{u=t\\v\in V}}f(u,v)}_{S_2}+\underbrace{\sum\limits_{\substack{u\in V\setminus \{s,t\}\\v\in V}}f(u,v)}_{S_3}$.\newline
Рассмотрим $S_3=\sum\limits_{u\in V\setminus\{s,t\}}\underline{\sum\limits_{v\in V}f(u,v)}$. По свойству $\ref{defFc}$ каждая подчеркнутая часть равна $0$, и $S_3=0$\newline
Рассмотрим $S_1=\sum\limits_{v\in V}f(s,v)\equiv |f|$\newline
Рассмотрим $S_2=\sum\limits_{v\in V}f(t,v)\overset{\ref{defFs}}{=} -\sum\limits_{v\in V}f(v,t)=-f(V,t)$.\newline
Поскольку $0=A=S_1+S_2$, получаем $|f|=f(V,t)\,\blacksquare$
\subsection*{Задача 1}
Пусть $(G(V,E)$, $c\colon V^2\to\Nz, s, t)$~--- транспортная сеть. $f$~--- поток в ней.
\begin{enumerate}
\item \label{FAXX} Пусть $X\subseteq V$. Рассмотрим $A\eqdef f(X,X)\equiv \sum\limits_{\substack{u\in X\\v\in X}}f(u,v)$. Переобозначим, получим $$A=\sum\limits_{\substack{v\in X\\u\in X}}f(v,u)\overset{\ref{defFs}}{=}-\sum\limits_{\substack{v\in X\\u\in X}}f(u,v)=-A,$$
откуда $A=0\,\blacksquare$
\item \label{FAs} Пусть $X,\,Y\subseteq V$. Рассмотрим $f(X, Y)\equiv\sum\limits_{\substack{x\in X\\y\in Y}}f(x,y)\overset{\ref{defFs}}{=}-\sum\limits_{\substack{x\in X\\y\in Y}}f(y,x)\equiv -f(Y,X)\,\blacksquare$
\item \label{FAu} Пусть $X,\,Y,\,Z\subseteq V$, $X\cap Y=\varnothing$. Рассмотрим $f(X\cup Y, Z)\overset{(*)}{\equiv}\sum\limits_{\substack{u\in X\cup Y\\ v\in Z}}f(u,v)=
\underbrace{\sum\limits_{\substack{u\in X\\u\in Y\\ v\in Z}}f(u,v)}_{S_1}+
\underbrace{\sum\limits_{\substack{u\in X\\u\notin Y\\ v\in Z}}f(u,v)}_{S_2}+
\underbrace{\sum\limits_{\substack{u\notin X\\u\in Y\\ v\in Z}}f(u,v)}_{S_3}$.\newline
$S_1=0$, так как $u\in X\,\wedge\,u\in Y\Leftrightarrow u\in X\cap Y\Leftrightarrow u\in\varnothing$\newline
По определению, $f(X,Z)=\sum\limits_{\substack{u\in X\\u\in Y\\v\in Z}}f(u,v)+\sum\limits_{\substack{u\in X\\u\notin Y\\v\in Z}}f(u,v)\equiv S_1+S_2\overset{S_1=0}{=}S_2$\newline
По определению, $f(Y,Z)=\sum\limits_{\substack{u\in Y\\u\in X\\v\in Z}}f(u,v)+\sum\limits_{\substack{u\in Y\\u\notin X\\v\in Z}}f(u,v)\equiv S_1+S_3\overset{S_1=0}{=}S_3$\newline
Тогда из $(*)$ получаем $f(X\cup Y,Z)=S_2+S_3=f(X,Z)+f(Y,Z)$.
\item Пусть $X,\,Y,\,Z\subseteq V$, $X\cap Y=\varnothing$. Тогда $f(Z,X\cup Y)\overset{\ref{FAs}}{=}-f(X\cup Y,Z)\overset{\ref{FAu}}{=}-(f(X,Z)+f(Y,Z)\equiv -f(X,Z)-f(Y,Z)\overset{\ref{FAs}}{=}f(Z,X)+f(Z,Y)$
\end{enumerate}
\subsection*{Задача 2}
Нет, не обязательно. Пример. Рассмотрим $(G(V,E)$, $c\colon V^2\to\Nz, s, t)$~--- транспортная сеть. $f$~--- поток в ней:\newline
\begin{tikzpicture}[shorten >=1pt,node distance=2cm,on grid,auto,initial text=]
	  \node[state] (s) {$s$};
	  \node[state] (t)  [right = of s] {$t$};
  	  \path[->] 
			(s)	edge node {$1/1$}	(t)
			;
\end{tikzpicture}\newline
Определим $V\supseteq X\eqdef\{s\}$, $Y\eqdef X$. Тогда $A=f(X,Y)\overset{X=Y}{\equiv} f(X,X)\overset{\ref{FAXX}}{=}0$.\newline
Рассмотрим $B=-f(V-X,Y)\equiv f(\{t\},\{s\})=-\sum\limits_{\substack{u\in\{t\}\\v\in\{s\}}}f(u,v)\equiv -f(t,s)\overset{\ref{defFs}}{=}f(s,t)=1$\newline
Получаем $A=0\neq 1=B\,\blacksquare$
\subsection*{Упражнение 3}
\label{Fsum} Пусть $(G(V,E)$, $c\colon V^2\to\Nz, s, t)$~--- транспортная сеть. $f_1$ и $f_2$~--- потоки, для которых выполнено $\ref{defFc}$, $\ref{defFs}$ (заметим, что функция $c$ не участвует в этой части определения).\newline
Определим функцию $f\colon V^2\to \mathbb{R}$ как $f(u,v)\eqdef f_1(u,v)+f_2(u,v)$. По определению, $f$~--- поток в данной транспортной сети $\Leftrightarrow$ \begin{enumerate}
\item [3.] \ref{defFc}. Фиксируем $u\in V$. Рассмотрим $f(u,V)=\sum\limits_{\substack{v\in V}}f(u,v)=\sum\limits_{\substack{v\in V}}\big[f_1(u,v)+f_2(u,v)\big]\equiv\sum\limits_{\substack{v\in V}}f_1(u,v)+\sum\limits_{\substack{v\in V}}f_2(u,v)\equiv \cancelto{0}{f_1(u,V)}+\cancelto{0}{f_2(u,V)}=0$~--- выполнено всегда (зачеркнуто по свойству $\ref{defFc}$).
\item [2.] \ref{defFs}. Фиксируем $(u,v)\in V^2$. Рассмотрим $f(u,v)\equiv f_1(u,v)+f_2(u,v)\overset{\ref{defFs}}{=}-f_1(v,u)-f_2(v,u)\equiv -(f_1(v,u)+f_2(v,u))=-f(v,u)$~--- выполнено всегда.
\item [1.] \ref{defFl}. Нужно: $\forall (u,v)\in V^2\hookrightarrow f(u,v)\leqslant c(u,v)$. Поэтому третье свойство выполнено для $f$ $\Leftrightarrow$ $\forall (u,v)\in V^2\hookrightarrow f_1(u,v)+f_2(u,v)\leqslant c(u,v)$.
\end{enumerate}
Поэтому сумма потоков $f_1+f_2$~--- поток $\Leftrightarrow$ $\boxed{\forall (u,v)\in V^2\hookrightarrow f_1(u,v)+f_2(u,v)\leqslant c(u,v)}$.
\subsection*{Упражнение 4}
Пусть $N=(G(V,E)$, $c\colon V^2\to\Nz, s, t)$~--- транспортная сеть. Пусть $f_1$~--- поток в ней. Пусть $N'=(G'(u,v), c',s,t)$~--- остаточная сеть для $N$ и $f_1$. Пусть найден увеличивающий путь в остаточной сети, т.е. последовательность вершин $s\equiv v_0\to v_1\to ...\to v_{k-1}\to v_k\equiv t$, такая, что
$M\eqdef \min\limits_{i\in\overline{0,k-1}}c'(v_i,v_{i+1})>0$. Считаем путь простым (если путь не простой, выкенем цикл, получится простой путь). Определим функцию $f_2(u,v)=\sum\limits_{i=0}^{k-1}\left\{\begin{array}{rc}
 M, & (v_i,v_{i+1})=(u,v)\\
-M, & (v_i,v_{i+1})=(v,u)\\
\end{array}\right.$. Поскольку путь простой, то каждое (неориентированное) ребро встречается в нем только один раз. Значит, в сумме максимум один элемент ненулевой, и получаем $f_2(u,v)=\left\{\begin{array}{rl}
M, & \exists i\colon (u,v)=(v_i,v_{i+1})\\
-M, & \exists i\colon (v,u)=(v_i,v_{i+1})\\
0, & \mbox{иначе}
\end{array}\right.$
:\begin{enumerate}
\item $f_2(u,v)=\left\{\begin{array}{rl}
M, & \exists i\colon (u,v)=(v_i,v_{i+1})\\
-M, & \exists i\colon (v,u)=(v_i,v_{i+1})\\
0, & \mbox{иначе}
\end{array}\right.=\left\{\begin{array}{rl}
-M, & \exists i\colon (v,u)=(v_i,v_{i+1})\\
M, & \exists i\colon (u,v)=(v_i,v_{i+1})\\
0, & \mbox{иначе}
\end{array}\right.=-\left\{\begin{array}{rl}
M, & \exists i\colon (v,u)=(v_i,v_{i+1})\\
-M, & \exists i\colon (u,v)=(v_i,v_{i+1})\\
0, & \mbox{иначе}
\end{array}\right.=\\=-f_2(v,u)$, поэтому для $f_2$ и $N$ выполнено свойство \ref{defFs}
\item Фиксируем $u\in V\setminus\{t,s\}$. \begin{enumerate}
\item Пусть $u$ не входит в увеличивающий путь. Тогда $\forall v\in V\,\forall i\in\overline{0,k-1}\hookrightarrow (u,v)\neq(v_i,v_{i+1})$, значит, $f_2(u,v)=0$, и $\sum\limits_{v\in V}f_2(u,v)=0$.
\item Пусть $u$ входит в увеличивающий путь. $u\neq s\,\wedge u\neq t$, поэтому $u$~--- не первая, и не последняя вершина в пути. Значит, $\exists v_1,v_2\colon (v_1,u),\, (u,v_2)$~--- смежные ребра из пути, и других ребер из пути, инцидентных $u$ нет (путь простой). Тогда $\sum\limits_{v\in V}f_2(u,v)=0+...+0+f_2(u,v_1)+f_2(u,v_2)+0+...+0=(-M)+M=0\,\blacksquare$
\end{enumerate}
Получаем для $f_2$ свойство \ref{defFc}
\item $f_2(u,v)=\left\{\begin{array}{rlc}
M, & \exists i\colon (u,v)=(v_i,v_{i+1}) & (1)\\
-M, & \exists i\colon (v,u)=(v_i,v_{i+1}) & (2)\\
0, & \mbox{иначе} & (3)
\end{array}\right.$\begin{enumerate}
\item [(1).] $\exists i\colon (u,v)=(v_i,v_{i+1})$. $f_2(u,v)=M=\min\limits_{j\in\overline{0,k-1}}c'(v_j,v_{j+1})\leqslant c'(v_i,v_{i+1})$ (минимум меньше каждого)
\item [(2).] $\exists i\colon (v,u)=(v_i,v_{i+1})$. $f_2(u,v)=-M<0\leqslant c'(u,v)$ (пропускная способность $c'=c-f_1$ неотрицательна, так как $f_1$~--- поток в $N$, откуда $f_1\leqslant c$).
\item [(3).] $f_2(u,v)=0\leqslant c'(u,v)$ (пропускная способность неотрицательна)
\end{enumerate}
Получаем, что для $f_2$ выполнено свойство $\ref{defFl}$ для сети $N'$
\end{enumerate}
Получаем, что $f_2$~--- поток в $N'$. Докажем, что $f_1+f_2$~--- поток в $N$. По \ref{Fsum} это выполнено, если $\forall (u,v)\in V^2\hookrightarrow f_1(u,v)+f_2(u,v)\leqslant c(u,v)$. Фиксируем $(u,v)\in V^2$. $f_2$~--- поток в $N'$, поэтому $f_2(u,v)\leqslant c'(u,v)\equiv c(u,v)-f_1(u,v)$, поэтому $f_1(u,v)+f_2(u,v)\leqslant f_1(u,v)+c(u,v)-f_1(u,v)\equiv c(u,v)\,\blacksquare$\newline
Докажем, что $f_1+f_2$~--- поток в исходной сети $N$ после этой итерации ФФ: алгоритм добавляет к $f_1(v_i,v_{i+1})$ величину $M$, вычитает из $f_1(v_{i+1},v_i)$ $M$. Рассмотрим разность $(f_1+f_2)-f_1=f_2$, которая как равна этой величине ($M$ в случае $(v_i,v_{i+1})$ в пути, $-M$ в случае $(v_{i+1},v_i)$ в пути, $0$ иначе) $\blacksquare$
\subsection*{(каноническое) Задача 28}
Транспортная сеть $N(G(V,E),c,s,t)$ и поток $f$ в ней (см. картинку ниже)\newline
\begin{enumerate}
\item $|f|=2+2+3=7$.
\item Нет (см. далее).
\item $k$-я итерация алгоритма. $M$~--- величина увеличивающего пути:\newline
\begin{tabular}{lcll}
Сеть и поток&  & Остаточная сеть & $M$\\
\begin{minipage}{0.35\textwidth}
\begin{tikzpicture}[shorten >=1pt,node distance=2cm,on grid,auto,initial text=]
 \tikzstyle{every node}=[font=\small]
	  \node[state] (s) {$s$};
	  \node[state] (v2) [right = of s] {$v_2$};
	  \node[state] (v1) [above = of v2] {$v_1$};
	  \node[state] (v3) [below = of v2] {$v_3$};
	  \node[state] (v4) [right = of v1] {$v_4$};
	  \node[state] (v5) [below = of v4] {$v_5$};
	  \node[state] (v6) [below = of v5] {$v_6$};
	  \node[state] (t)  [right = of v5] {$t$};
	  \node[state] (v7) [above = of t] {$v_7$};
	  \node[state] (v8) [below = of t] {$v_8$};
	  \begin{scope}[every node/.style={scale=.5}]
  	  \path[->] 
			(s)	edge node {$2/5$}	(v1)
			(s)	edge node {$2/5$}	(v2)
			(s)	edge node[swap] {$3/8$}	(v3)
			(v1)edge node[swap] {$1/4$}	(v2)
			(v1)edge node {$1/3$}	(v4)
			(v2)edge node {$2/3$}	(v4)
			(v2)edge node {$1/3$}	(v5)
			(v3)edge node {$0/4$}	(v2)
			(v3)edge node[swap] {$3/5$}	(v6)
			(v4)edge node {$2/3$}	(v7)
			(v4)edge node[swap] {$1/3$}	(v5)
			(v5)edge node {$1/3$}	(v7)
			(v5)edge node {$2/3$}	(t)
			(v6)edge node {$1/2$}	(t)
			(v6)edge node[swap] {$1/5$}	(v8)
			(v6)edge node {$1/3$}	(v5)
			(v7)edge node {$3/6$}	(t)
			(v8)edge node[swap] {$1/3$}	(t)
			;
			\end{scope}
\end{tikzpicture}
\end{minipage}&$\to$&
\begin{minipage}{0.35\textwidth}
\begin{tikzpicture}[shorten >=1pt,node distance=2cm,on grid,auto,initial text=]
 \tikzstyle{every node}=[font=\small]
	  \node[state] (s) {$s$};
	  \node[state] (v2) [right = of s] {$v_2$};
	  \node[state] (v1) [above = of v2] {$v_1$};
	  \node[state] (v3) [below = of v2] {$v_3$};
	  \node[state] (v4) [right = of v1] {$v_4$};
	  \node[state] (v5) [below = of v4] {$v_5$};
	  \node[state] (v6) [below = of v5] {$v_6$};
	  \node[state] (t)  [right = of v5] {$t$};
	  \node[state] (v7) [above = of t] {$v_7$};
	  \node[state] (v8) [below = of t] {$v_8$};
	  \begin{scope}[every node/.style={scale=.5}]
  	  \path[->] 
			(s)	edge [bend left=10] node {$3$}	(v1)
			(v1)edge [bend left=10] node {$2$}	(s)
			
			(s)	edge [bend left=10]node {$3$}	(v2)
			(v2)edge [bend left=10]node {$2$}	(s)
			
			(s)	edge [color=red,bend right=10]node[swap] {$5$}	(v3)
			(v3)edge [bend right=10]node[swap] {$3$}	(s)
			
			(v1)edge [bend right=10]node[swap] {$3$}	(v2)
			(v2)edge [bend right=10]node[swap] {$1$}	(v1)
			
			(v1)edge [bend left=10]node {$2$}	(v4)
			(v4)edge [bend left=10]node {$1$}	(v1)
			
			(v2)edge [bend left=10]node {$1$}	(v4)
			(v4)edge [bend left=10]node {$2$}	(v2)
			
			(v2)edge [bend left=10]node {$2$}	(v5)
			(v5)edge [bend left=10]node {$1$}	(v2)
			
			(v3)edge [bend left=10]node {$4$}	(v2)
%			(v2)edge [bend left=10]node {$0$}	(v3)
			
			(v3)edge [color=red,bend right=10]node[swap] {$2$}	(v6)
			(v6)edge [bend right=10]node[swap] {$3$}	(v3)
			
			(v4)edge [bend left=10]node {$1$}	(v7)
			(v7)edge [bend left=10]node {$2$}	(v4)
			
			(v4)edge [bend right=10]node[swap] {$2$}	(v5)
			(v5)edge [bend right=10]node[swap] {$1$}	(v4)
			
			(v5)edge [bend left=10]node {$2$}	(v7)
			(v7)edge [bend left=10]node {$1$}	(v5)
			
			(v5)edge [bend left=10]node {$1$}	(t)
			(t)edge [bend left=10]node {$2$}	(v5)
			
			(v6)edge [bend left=10] node {$1$}	(t)
			(t)edge [bend left=10]node {$1$}	(v6)
			
			(v6)edge [color=red,bend right=10]node[swap] {$4$}	(v8)
			(v8)edge[bend right=10] node[swap] {$1$}	(v6)
			
			(v6)edge [bend left=10]node {$2$}	(v5)
			(v5)edge [bend left=10]node {$1$}	(v6)
			
			(v7)edge[bend left=10] node {$3$}	(t)
			(t)edge[bend left=10] node {$3$}	(v7)
			
			(v8)edge[color=red,bend right=10] node[swap] {$2$}	(t)
			(t)edge [bend right=10]node[swap] {$1$}	(v8)
			;
			\end{scope}
\end{tikzpicture}
\end{minipage}& 2\\
\end{tabular}
\item $k+1$-я итерация алгоритма. $M$~--- величина увеличивающего пути:\newline
\begin{tabular}{lcll}
Сеть и поток&  & Остаточная сеть & $M$\\
\begin{minipage}{0.35\textwidth}
\begin{tikzpicture}[shorten >=1pt,node distance=2cm,on grid,auto,initial text=]
 \tikzstyle{every node}=[font=\small]
	  \node[state] (s) {$s$};
	  \node[state] (v2) [right = of s] {$v_2$};
	  \node[state] (v1) [above = of v2] {$v_1$};
	  \node[state] (v3) [below = of v2] {$v_3$};
	  \node[state] (v4) [right = of v1] {$v_4$};
	  \node[state] (v5) [below = of v4] {$v_5$};
	  \node[state] (v6) [below = of v5] {$v_6$};
	  \node[state] (t)  [right = of v5] {$t$};
	  \node[state] (v7) [above = of t] {$v_7$};
	  \node[state] (v8) [below = of t] {$v_8$};
	  \begin{scope}[every node/.style={scale=.5}]
  	  \path[->] 
			(s)	edge node {$2/5$}	(v1)
			(s)	edge node {$2/5$}	(v2)
			(s)	edge node[swap] {$5/8$}	(v3)
			(v1)edge node[swap] {$1/4$}	(v2)
			(v1)edge node {$1/3$}	(v4)
			(v2)edge node {$2/3$}	(v4)
			(v2)edge node {$1/3$}	(v5)
			(v3)edge node {$0/4$}	(v2)
			(v3)edge node[swap] {$5/5$}	(v6)
			(v4)edge node {$2/3$}	(v7)
			(v4)edge node[swap] {$1/3$}	(v5)
			(v5)edge node {$1/3$}	(v7)
			(v5)edge node {$2/3$}	(t)
			(v6)edge node {$1/2$}	(t)
			(v6)edge node[swap] {$3/5$}	(v8)
			(v6)edge node {$1/3$}	(v5)
			(v7)edge node {$3/6$}	(t)
			(v8)edge node[swap] {$3/3$}	(t)
			;
			\end{scope}
\end{tikzpicture}
\end{minipage}&$\to$&
\begin{minipage}{0.35\textwidth}
\begin{tikzpicture}[shorten >=1pt,node distance=2cm,on grid,auto,initial text=]
 \tikzstyle{every node}=[font=\small]
	  \node[state] (s) {$s$};
	  \node[state] (v2) [right = of s] {$v_2$};
	  \node[state] (v1) [above = of v2] {$v_1$};
	  \node[state] (v3) [below = of v2] {$v_3$};
	  \node[state] (v4) [right = of v1] {$v_4$};
	  \node[state] (v5) [below = of v4] {$v_5$};
	  \node[state] (v6) [below = of v5] {$v_6$};
	  \node[state] (t)  [right = of v5] {$t$};
	  \node[state] (v7) [above = of t] {$v_7$};
	  \node[state] (v8) [below = of t] {$v_8$};
	  \begin{scope}[every node/.style={scale=.5}]
  	  \path[->] 
			(s)	edge [color=red,bend left=10] node {$3$}	(v1)
			(v1)edge [bend left=10] node {$2$}	(s)
			
			(s)	edge [bend left=10]node {$3$}	(v2)
			(v2)edge [bend left=10]node {$2$}	(s)
			
			(s)	edge [bend right=10]node[swap] {$3$}	(v3)
			(v3)edge [bend right=10]node[swap] {$5$}	(s)
			
			(v1)edge [bend right=10]node[swap] {$3$}	(v2)
			(v2)edge [bend right=10]node[swap] {$1$}	(v1)
			
			(v1)edge [color=red,bend left=10]node {$2$}	(v4)
			(v4)edge [bend left=10]node {$1$}	(v1)
			
			(v2)edge [bend left=10]node {$1$}	(v4)
			(v4)edge [bend left=10]node {$2$}	(v2)
			
			(v2)edge [bend left=10]node {$2$}	(v5)
			(v5)edge [bend left=10]node {$1$}	(v2)
			
			(v3)edge [bend left=10]node {$4$}	(v2)
%			(v2)edge [bend left=10]node {$0$}	(v3)
			
%			(v3)edge [color=red,bend right=10]node[swap] {$2$}	(v6)
			(v6)edge [bend right=10]node[swap] {$5$}	(v3)
			
			(v4)edge [bend left=10]node {$1$}	(v7)
			(v7)edge [bend left=10]node {$2$}	(v4)
			
			(v4)edge [color=red,bend right=10]node[swap] {$2$}	(v5)
			(v5)edge [bend right=10]node[swap] {$1$}	(v4)
			
			(v5)edge [color=red,bend left=10]node {$2$}	(v7)
			(v7)edge [bend left=10]node {$1$}	(v5)
			
			(v5)edge [bend left=10]node {$1$}	(t)
			(t)edge [bend left=10]node {$2$}	(v5)
			
			(v6)edge [bend left=10] node {$1$}	(t)
			(t)edge [bend left=10]node {$1$}	(v6)
			
			(v6)edge [bend right=10]node[swap] {$2$}	(v8)
			(v8)edge[bend right=10] node[swap] {$3$}	(v6)
			
			(v6)edge [bend left=10]node {$2$}	(v5)
			(v5)edge [bend left=10]node {$1$}	(v6)
			
			(v7)edge[color=red,bend left=10] node {$3$}	(t)
			(t)edge[bend left=10] node {$3$}	(v7)
			
%			(v8)edge[color=red,bend right=10] node[swap] {$2$}	(t)
			(t)edge [bend right=10]node[swap] {$3$}	(v8)
			;
			\end{scope}
\end{tikzpicture}
\end{minipage}& 2\\
\end{tabular}
\item $k+2$-я итерация алгоритма. $M$~--- величина увеличивающего пути:\newline
\begin{tabular}{lcll}
Сеть и поток&  & Остаточная сеть & $M$\\
\begin{minipage}{0.35\textwidth}
\begin{tikzpicture}[shorten >=1pt,node distance=2cm,on grid,auto,initial text=]
 \tikzstyle{every node}=[font=\small]
	  \node[state] (s) {$s$};
	  \node[state] (v2) [right = of s] {$v_2$};
	  \node[state] (v1) [above = of v2] {$v_1$};
	  \node[state] (v3) [below = of v2] {$v_3$};
	  \node[state] (v4) [right = of v1] {$v_4$};
	  \node[state] (v5) [below = of v4] {$v_5$};
	  \node[state] (v6) [below = of v5] {$v_6$};
	  \node[state] (t)  [right = of v5] {$t$};
	  \node[state] (v7) [above = of t] {$v_7$};
	  \node[state] (v8) [below = of t] {$v_8$};
	  \begin{scope}[every node/.style={scale=.5}]
  	  \path[->] 
			(s)	edge node {$4/5$}	(v1)
			(s)	edge node {$2/5$}	(v2)
			(s)	edge node[swap] {$5/8$}	(v3)
			(v1)edge node[swap] {$1/4$}	(v2)
			(v1)edge node {$3/3$}	(v4)
			(v2)edge node {$2/3$}	(v4)
			(v2)edge node {$1/3$}	(v5)
			(v3)edge node {$0/4$}	(v2)
			(v3)edge node[swap] {$5/5$}	(v6)
			(v4)edge node {$2/3$}	(v7)
			(v4)edge node[swap] {$3/3$}	(v5)
			(v5)edge node {$3/3$}	(v7)
			(v5)edge node {$2/3$}	(t)
			(v6)edge node {$1/2$}	(t)
			(v6)edge node[swap] {$3/5$}	(v8)
			(v6)edge node {$1/3$}	(v5)
			(v7)edge node {$5/6$}	(t)
			(v8)edge node[swap] {$3/3$}	(t)
			;
			\end{scope}
\end{tikzpicture}
\end{minipage}&$\to$&
\begin{minipage}{0.35\textwidth}
\begin{tikzpicture}[shorten >=1pt,node distance=2cm,on grid,auto,initial text=]
 \tikzstyle{every node}=[font=\small]
	  \node[state] (s) {$s$};
	  \node[state] (v2) [right = of s] {$v_2$};
	  \node[state] (v1) [above = of v2] {$v_1$};
	  \node[state] (v3) [below = of v2] {$v_3$};
	  \node[state] (v4) [right = of v1] {$v_4$};
	  \node[state] (v5) [below = of v4] {$v_5$};
	  \node[state] (v6) [below = of v5] {$v_6$};
	  \node[state] (t)  [right = of v5] {$t$};
	  \node[state] (v7) [above = of t] {$v_7$};
	  \node[state] (v8) [below = of t] {$v_8$};
	  \begin{scope}[every node/.style={scale=.5}]
  	  \path[->] 
			(s)	edge [bend left=10] node {$1$}	(v1)
			(v1)edge [bend left=10] node {$4$}	(s)
			
			(s)	edge [color=red,bend left=10]node {$3$}	(v2)
			(v2)edge [bend left=10]node {$2$}	(s)
			
			(s)	edge [bend right=10]node[swap] {$3$}	(v3)
			(v3)edge [bend right=10]node[swap] {$5$}	(s)
			
			(v1)edge [bend right=10]node[swap] {$3$}	(v2)
			(v2)edge [bend right=10]node[swap] {$1$}	(v1)
			
%			(v1)edge [color=red,bend left=10]node {$2$}	(v4)
			(v4)edge [bend left=10]node {$3$}	(v1)
			
			(v2)edge [bend left=10]node {$1$}	(v4)
			(v4)edge [bend left=10]node {$2$}	(v2)
			
			(v2)edge [color=red,bend left=10]node {$2$}	(v5)
			(v5)edge [bend left=10]node {$1$}	(v2)
			
			(v3)edge [bend left=10]node {$4$}	(v2)
%			(v2)edge [bend left=10]node {$0$}	(v3)
			
%			(v3)edge [color=red,bend right=10]node[swap] {$2$}	(v6)
			(v6)edge [bend right=10]node[swap] {$5$}	(v3)
			
			(v4)edge [bend left=10]node {$1$}	(v7)
			(v7)edge [bend left=10]node {$2$}	(v4)
			
%			(v4)edge [color=red,bend right=10]node[swap] {$2$}	(v5)
			(v5)edge [bend right=10]node[swap] {$3$}	(v4)
			
%			(v5)edge [color=red,bend left=10]node {$2$}	(v7)
			(v7)edge [bend left=10]node {$3$}	(v5)
			
			(v5)edge [color=red,bend left=10]node {$1$}	(t)
			(t)edge [bend left=10]node {$2$}	(v5)
			
			(v6)edge [bend left=10] node {$1$}	(t)
			(t)edge [bend left=10]node {$1$}	(v6)
			
			(v6)edge [bend right=10]node[swap] {$2$}	(v8)
			(v8)edge[bend right=10] node[swap] {$3$}	(v6)
			
			(v6)edge [bend left=10]node {$2$}	(v5)
			(v5)edge [bend left=10]node {$1$}	(v6)
			
			(v7)edge[bend left=10] node {$1$}	(t)
			(t)edge[bend left=10] node {$5$}	(v7)
			
%			(v8)edge[color=red,bend right=10] node[swap] {$2$}	(t)
			(t)edge [bend right=10]node[swap] {$3$}	(v8)
			;
			\end{scope}
\end{tikzpicture}
\end{minipage}& 1\\
\end{tabular}

\item $k+3$-я итерация алгоритма. $M$~--- величина увеличивающего пути:\newline
\begin{tabular}{lcll}
Сеть и поток&  & Остаточная сеть & $M$\\
\begin{minipage}{0.35\textwidth}
\begin{tikzpicture}[shorten >=1pt,node distance=2cm,on grid,auto,initial text=]
 \tikzstyle{every node}=[font=\small]
	  \node[state] (s) {$s$};
	  \node[state] (v2) [right = of s] {$v_2$};
	  \node[state] (v1) [above = of v2] {$v_1$};
	  \node[state] (v3) [below = of v2] {$v_3$};
	  \node[state] (v4) [right = of v1] {$v_4$};
	  \node[state] (v5) [below = of v4] {$v_5$};
	  \node[state] (v6) [below = of v5] {$v_6$};
	  \node[state] (t)  [right = of v5] {$t$};
	  \node[state] (v7) [above = of t] {$v_7$};
	  \node[state] (v8) [below = of t] {$v_8$};
	  \begin{scope}[every node/.style={scale=.5}]
  	  \path[->] 
			(s)	edge node {$4/5$}	(v1)
			(s)	edge node {$3/5$}	(v2)
			(s)	edge node[swap] {$5/8$}	(v3)
			(v1)edge node[swap] {$1/4$}	(v2)
			(v1)edge node {$3/3$}	(v4)
			(v2)edge node {$2/3$}	(v4)
			(v2)edge node {$2/3$}	(v5)
			(v3)edge node {$0/4$}	(v2)
			(v3)edge node[swap] {$5/5$}	(v6)
			(v4)edge node {$2/3$}	(v7)
			(v4)edge node[swap] {$3/3$}	(v5)
			(v5)edge node {$3/3$}	(v7)
			(v5)edge node {$3/3$}	(t)
			(v6)edge node {$1/2$}	(t)
			(v6)edge node[swap] {$3/5$}	(v8)
			(v6)edge node {$1/3$}	(v5)
			(v7)edge node {$5/6$}	(t)
			(v8)edge node[swap] {$3/3$}	(t)
			;
			\end{scope}
\end{tikzpicture}
\end{minipage}&$\to$&
\begin{minipage}{0.35\textwidth}
\begin{tikzpicture}[shorten >=1pt,node distance=2cm,on grid,auto,initial text=]
 \tikzstyle{every node}=[font=\small]
	  \node[state] (s) {$s$};
	  \node[state] (v2) [right = of s] {$v_2$};
	  \node[state] (v1) [above = of v2] {$v_1$};
	  \node[state] (v3) [below = of v2] {$v_3$};
	  \node[state] (v4) [right = of v1] {$v_4$};
	  \node[state] (v5) [below = of v4] {$v_5$};
	  \node[state] (v6) [below = of v5] {$v_6$};
	  \node[state] (t)  [right = of v5] {$t$};
	  \node[state] (v7) [above = of t] {$v_7$};
	  \node[state] (v8) [below = of t] {$v_8$};
	  \begin{scope}[every node/.style={scale=.5}]
  	  \path[->] 
			(s)	edge [bend left=10] node {$1$}	(v1)
			(v1)edge [bend left=10] node {$4$}	(s)
			
			(s)	edge [color=red,bend left=10]node {$2$}	(v2)
			(v2)edge [bend left=10]node {$3$}	(s)
			
			(s)	edge [bend right=10]node[swap] {$3$}	(v3)
			(v3)edge [bend right=10]node[swap] {$5$}	(s)
			
			(v1)edge [bend right=10]node[swap] {$3$}	(v2)
			(v2)edge [bend right=10]node[swap] {$1$}	(v1)
			
%			(v1)edge [color=red,bend left=10]node {$2$}	(v4)
			(v4)edge [bend left=10]node {$3$}	(v1)
			
			(v2)edge [color=red,bend left=10]node {$1$}	(v4)
			(v4)edge [bend left=10]node {$2$}	(v2)
			
			(v2)edge [bend left=10]node {$1$}	(v5)
			(v5)edge [bend left=10]node {$2$}	(v2)
			
			(v3)edge [bend left=10]node {$4$}	(v2)
%			(v2)edge [bend left=10]node {$0$}	(v3)
			
%			(v3)edge [color=red,bend right=10]node[swap] {$2$}	(v6)
			(v6)edge [bend right=10]node[swap] {$5$}	(v3)
			
			(v4)edge [color=red,bend left=10]node {$1$}	(v7)
			(v7)edge [bend left=10]node {$2$}	(v4)
			
%			(v4)edge [color=red,bend right=10]node[swap] {$2$}	(v5)
			(v5)edge [bend right=10]node[swap] {$3$}	(v4)
			
%			(v5)edge [color=red,bend left=10]node {$2$}	(v7)
			(v7)edge [bend left=10]node {$3$}	(v5)
			
%			(v5)edge [color=red,bend left=10]node {$1$}	(t)
			(t)edge [bend left=10]node {$3$}	(v5)
			
			(v6)edge [bend left=10] node {$1$}	(t)
			(t)edge [bend left=10]node {$1$}	(v6)
			
			(v6)edge [bend right=10]node[swap] {$2$}	(v8)
			(v8)edge[bend right=10] node[swap] {$3$}	(v6)
			
			(v6)edge [bend left=10]node {$2$}	(v5)
			(v5)edge [bend left=10]node {$1$}	(v6)
			
			(v7)edge[color=red,bend left=10] node {$1$}	(t)
			(t)edge[bend left=10] node {$5$}	(v7)
			
%			(v8)edge[color=red,bend right=10] node[swap] {$2$}	(t)
			(t)edge [bend right=10]node[swap] {$3$}	(v8)
			;
			\end{scope}
\end{tikzpicture}
\end{minipage}& 1\\
\end{tabular}


\item $k+4$-я итерация алгоритма. $M$~--- величина увеличивающего пути:\newline
\begin{tabular}{lcll}
Сеть и поток&  & Остаточная сеть & $M$\\
\begin{minipage}{0.35\textwidth}
\begin{tikzpicture}[shorten >=1pt,node distance=2cm,on grid,auto,initial text=]
 \tikzstyle{every node}=[font=\small]
	  \node[state] (s) {$s$};
	  \node[state] (v2) [right = of s] {$v_2$};
	  \node[state] (v1) [above = of v2] {$v_1$};
	  \node[state] (v3) [below = of v2] {$v_3$};
	  \node[state] (v4) [right = of v1] {$v_4$};
	  \node[state] (v5) [below = of v4] {$v_5$};
	  \node[state] (v6) [below = of v5] {$v_6$};
	  \node[state] (t)  [right = of v5] {$t$};
	  \node[state] (v7) [above = of t] {$v_7$};
	  \node[state] (v8) [below = of t] {$v_8$};
	  \begin{scope}[every node/.style={scale=.5}]
  	  \path[->] 
			(s)	edge node {$4/5$}	(v1)
			(s)	edge node {$4/5$}	(v2)
			(s)	edge node[swap] {$5/8$}	(v3)
			(v1)edge node[swap] {$1/4$}	(v2)
			(v1)edge node {$3/3$}	(v4)
			(v2)edge node {$3/3$}	(v4)
			(v2)edge node {$2/3$}	(v5)
			(v3)edge node {$0/4$}	(v2)
			(v3)edge node[swap] {$5/5$}	(v6)
			(v4)edge node {$3/3$}	(v7)
			(v4)edge node[swap] {$3/3$}	(v5)
			(v5)edge node {$3/3$}	(v7)
			(v5)edge node {$3/3$}	(t)
			(v6)edge node {$1/2$}	(t)
			(v6)edge node[swap] {$3/5$}	(v8)
			(v6)edge node {$1/3$}	(v5)
			(v7)edge node {$6/6$}	(t)
			(v8)edge node[swap] {$3/3$}	(t)
			;
			\end{scope}
\end{tikzpicture}
\end{minipage}&$\to$&
\begin{minipage}{0.35\textwidth}
\begin{tikzpicture}[shorten >=1pt,node distance=2cm,on grid,auto,initial text=]
 \tikzstyle{every node}=[font=\small]
	  \node[state] (s) {$s$};
	  \node[state] (v2) [right = of s] {$v_2$};
	  \node[state] (v1) [above = of v2] {$v_1$};
	  \node[state] (v3) [below = of v2] {$v_3$};
	  \node[state] (v4) [right = of v1] {$v_4$};
	  \node[state] (v5) [below = of v4] {$v_5$};
	  \node[state] (v6) [below = of v5] {$v_6$};
	  \node[state] (t)  [right = of v5] {$t$};
	  \node[state] (v7) [above = of t] {$v_7$};
	  \node[state] (v8) [below = of t] {$v_8$};
	  \begin{scope}[every node/.style={scale=.5}]
  	  \path[->] 
			(s)	edge [bend left=10] node {$1$}	(v1)
			(v1)edge [bend left=10] node {$4$}	(s)
			
			(s)	edge [color=red,bend left=10]node {$1$}	(v2)
			(v2)edge [bend left=10]node {$4$}	(s)
			
			(s)	edge [bend right=10]node[swap] {$3$}	(v3)
			(v3)edge [bend right=10]node[swap] {$5$}	(s)
			
			(v1)edge [bend right=10]node[swap] {$3$}	(v2)
			(v2)edge [bend right=10]node[swap] {$1$}	(v1)
			
%			(v1)edge [color=red,bend left=10]node {$2$}	(v4)
			(v4)edge [bend left=10]node {$3$}	(v1)
			
%			(v2)edge [color=red,bend left=10]node {$1$}	(v4)
			(v4)edge [bend left=10]node {$3$}	(v2)
			
			(v2)edge [color=red,bend left=10]node {$1$}	(v5)
			(v5)edge [bend left=10]node {$2$}	(v2)
			
			(v3)edge [bend left=10]node {$4$}	(v2)
%			(v2)edge [bend left=10]node {$0$}	(v3)
			
%			(v3)edge [color=red,bend right=10]node[swap] {$2$}	(v6)
			(v6)edge [bend right=10]node[swap] {$5$}	(v3)
			
%			(v4)edge [color=red,bend left=10]node {$1$}	(v7)
			(v7)edge [bend left=10]node {$3$}	(v4)
			
%			(v4)edge [color=red,bend right=10]node[swap] {$2$}	(v5)
			(v5)edge [bend right=10]node[swap] {$3$}	(v4)
			
%			(v5)edge [color=red,bend left=10]node {$2$}	(v7)
			(v7)edge [bend left=10]node {$3$}	(v5)
			
%			(v5)edge [color=red,bend left=10]node {$1$}	(t)
			(t)edge [bend left=10]node {$3$}	(v5)
			
			(v6)edge [color=red,bend left=10] node {$1$}	(t)
			(t)edge [bend left=10]node {$1$}	(v6)
			
			(v6)edge [bend right=10]node[swap] {$2$}	(v8)
			(v8)edge[bend right=10] node[swap] {$3$}	(v6)
			
			(v6)edge [bend left=10]node {$2$}	(v5)
			(v5)edge [color=red,bend left=10]node {$1$}	(v6)
			
%			(v7)edge[color=red,bend left=10] node {$1$}	(t)
			(t)edge[bend left=10] node {$6$}	(v7)
			
%			(v8)edge[color=red,bend right=10] node[swap] {$2$}	(t)
			(t)edge [bend right=10]node[swap] {$3$}	(v8)
			;
			\end{scope}
\end{tikzpicture}
\end{minipage}& 1\\
\end{tabular}




\item $k+5$-я итерация алгоритма. $M$~--- величина увеличивающего пути:\newline
\begin{tabular}{lcll}
Сеть и поток&  & Остаточная сеть & $M$\\
\begin{minipage}{0.35\textwidth}
\begin{tikzpicture}[shorten >=1pt,node distance=2cm,on grid,auto,initial text=]
 \tikzstyle{every node}=[font=\small]
	  \node[state] (s) {$s$};
	  \node[state] (v2) [right = of s] {$v_2$};
	  \node[state] (v1) [above = of v2] {$v_1$};
	  \node[state] (v3) [below = of v2] {$v_3$};
	  \node[state] (v4) [right = of v1] {$v_4$};
	  \node[state] (v5) [below = of v4] {$v_5$};
	  \node[state] (v6) [below = of v5] {$v_6$};
	  \node[state] (t)  [right = of v5] {$t$};
	  \node[state] (v7) [above = of t] {$v_7$};
	  \node[state] (v8) [below = of t] {$v_8$};
	  \begin{scope}[every node/.style={scale=.5}]
  	  \path[->] 
			(s)	edge node {$4/5$}	(v1)
			(s)	edge node {$5/5$}	(v2)
			(s)	edge node[swap] {$5/8$}	(v3)
			(v1)edge node[swap] {$1/4$}	(v2)
			(v1)edge node {$3/3$}	(v4)
			(v2)edge node {$3/3$}	(v4)
			(v2)edge node {$3/3$}	(v5)
			(v3)edge node {$0/4$}	(v2)
			(v3)edge node[swap] {$5/5$}	(v6)
			(v4)edge node {$3/3$}	(v7)
			(v4)edge node[swap] {$3/3$}	(v5)
			(v5)edge node {$3/3$}	(v7)
			(v5)edge node {$3/3$}	(t)
			(v6)edge node {$2/2$}	(t)
			(v6)edge node[swap] {$3/5$}	(v8)
			(v6)edge node {$0/3$}	(v5)
			(v7)edge node {$6/6$}	(t)
			(v8)edge node[swap] {$3/3$}	(t)
			;
			\end{scope}
\end{tikzpicture}
\end{minipage}&$\to$&
\begin{minipage}{0.35\textwidth}
\begin{tikzpicture}[shorten >=1pt,node distance=2cm,on grid,auto,initial text=]
 \tikzstyle{every node}=[font=\small]
	  \node[state,color=green] (s) {$s$};
	  \node[state,color=green] (v2) [right = of s] {$v_2$};
	  \node[state,color=green] (v1) [above = of v2] {$v_1$};
	  \node[state,color=green] (v3) [below = of v2] {$v_3$};
	  \node[state] (v4) [right = of v1] {$v_4$};
	  \node[state] (v5) [below = of v4] {$v_5$};
	  \node[state] (v6) [below = of v5] {$v_6$};
	  \node[state] (t)  [right = of v5] {$t$};
	  \node[state] (v7) [above = of t] {$v_7$};
	  \node[state] (v8) [below = of t] {$v_8$};
	  \begin{scope}[every node/.style={scale=.5}]
  	  \path[->] 
			(s)	edge [bend left=10] node {$1$}	(v1)
			(v1)edge [bend left=10] node {$4$}	(s)
			
%			(s)	edge [color=red,bend left=10]node {$0$}	(v2)
			(v2)edge [bend left=10]node {$5$}	(s)
			
			(s)	edge [bend right=10]node[swap] {$3$}	(v3)
			(v3)edge [bend right=10]node[swap] {$5$}	(s)
			
			(v1)edge [bend right=10]node[swap] {$3$}	(v2)
			(v2)edge [bend right=10]node[swap] {$1$}	(v1)
			
%			(v1)edge [color=red,bend left=10]node {$2$}	(v4)
			(v4)edge [bend left=10]node {$3$}	(v1)
			
%			(v2)edge [color=red,bend left=10]node {$1$}	(v4)
			(v4)edge [bend left=10]node {$3$}	(v2)
			
%			(v2)edge [color=red,bend left=10]node {$1$}	(v5)
			(v5)edge [bend left=10]node {$3$}	(v2)
			
			(v3)edge [bend left=10]node {$4$}	(v2)
%			(v2)edge [bend left=10]node {$0$}	(v3)
			
%			(v3)edge [color=red,bend right=10]node[swap] {$2$}	(v6)
			(v6)edge [bend right=10]node[swap] {$5$}	(v3)
			
%			(v4)edge [color=red,bend left=10]node {$1$}	(v7)
			(v7)edge [bend left=10]node {$3$}	(v4)
			
%			(v4)edge [color=red,bend right=10]node[swap] {$2$}	(v5)
			(v5)edge [bend right=10]node[swap] {$3$}	(v4)
			
%			(v5)edge [color=red,bend left=10]node {$2$}	(v7)
			(v7)edge [bend left=10]node {$3$}	(v5)
			
%			(v5)edge [color=red,bend left=10]node {$1$}	(t)
			(t)edge [bend left=10]node {$3$}	(v5)
			
%			(v6)edge [color=red,bend left=10] node {$1$}	(t)
			(t)edge [bend left=10]node {$2$}	(v6)
			
			(v6)edge [bend right=10]node[swap] {$2$}	(v8)
			(v8)edge[bend right=10] node[swap] {$3$}	(v6)
			
			(v6)edge [bend left=10]node {$3$}	(v5)
%			(v5)edge [color=red,bend left=10]node {$1$}	(v6)
			
%			(v7)edge[color=red,bend left=10] node {$1$}	(t)
			(t)edge[bend left=10] node {$6$}	(v7)
			
%			(v8)edge[color=red,bend right=10] node[swap] {$2$}	(t)
			(t)edge [bend right=10]node[swap] {$3$}	(v8)
			;
			\end{scope}
\end{tikzpicture}
\end{minipage}& 0\\
\end{tabular}\newline
Зеленым выделены достижимые из $s$ вершины, $t$ не зеленая, алгоритм останавливается.
\item Максимальный поток равен $|f|=14$
\item (Теорема, семинар, Кормен) максимальный поток равен минимальному разрезу, который строится следующим образом: $S$~--- достижимые из $s$, $T=V\setminus S$. Алгоритм: ищем максимальный поток (ФФ), на последнем шаге (нет увеличивающего пути) строим $S$ и $T$.
\item Для данного случая $S=\{s,v_1,v_2,v_3\}$, $T=\{v_4,v_5,v_6,v_7,t,v_8\}$, $c(S,T)=3+3+3+5=14$
\end{enumerate}
\newpage




\subsection*{(каноническое) Задача 29}
{\em (Кормен)}\newline
Рассмотрим транспортную сеть:\newline
\begin{tabular}{ll}
Сеть & Остаточная сеть\\
\begin{minipage}{0.35\textwidth}
\begin{tikzpicture}[shorten >=1pt,node distance=1.5cm,on grid,auto,initial text=]
	  \node[state] (s) {$s$};
	  \node (c) [right = 3cm of s] {$$};
	  \node[state] (u) [above = of c] {$u$};
	  \node[state] (v) [below = of c] {$v$};
	  \node[state] (t) [right = 3cm of c] {$t$};
  	  \path[->] 
			(s)	edge node {$0/H$}	(u)
			(s)	edge node[swap] {$0/H$}	(v)
			(u)	edge node {$0/H$}	(t)
			(v)	edge node[swap] {$0/H$}	(t)
			(u)	edge node {$0/1$}	(v)
			;
\end{tikzpicture}
\end{minipage}
&
\begin{minipage}{0.4\textwidth}
\begin{tikzpicture}[shorten >=1pt,node distance=1.5cm,on grid,auto,initial text=]
	  \node[state] (s) {$s$};
	  \node (c) [right = 3cm of s] {$$};
	  \node[state] (u) [above = of c] {$u$};
	  \node[state] (v) [below = of c] {$v$};
	  \node[state] (t) [right = 3cm of c] {$t$};
  	  \path[->] 
			(s)	edge node {$H$}	(u)
			(s)	edge node[swap] {$H$}	(v)
			(u)	edge node {$H$}	(t)
			(v)	edge node[swap] {$H$}	(t)
			(u)	edge node {$1$}	(v)
			;
\end{tikzpicture}
\end{minipage}\\
\end{tabular}
\begin{enumerate}
\item На каждом шаге алгоритм выбирает увеличивающий путь.
\item Пусть на первом шаге выбран увеличивающий путь $s\to u\to v\to t$ величины 1. После первой итерации исходная сеть и остаточная сеть:\newline
\begin{tabular}{ll}
Сеть & Остаточная сеть\\
\begin{minipage}{0.35\textwidth}
\begin{tikzpicture}[shorten >=1pt,node distance=1.5cm,on grid,auto,initial text=]
	  \node[state] (s) {$s$};
	  \node (c) [right = 3cm of s] {$$};
	  \node[state] (u) [above = of c] {$u$};
	  \node[state] (v) [below = of c] {$v$};
	  \node[state] (t) [right = 3cm of c] {$t$};
  	  \path[->] 
			(s)	edge node {$1/H$}	(u)
			(s)	edge node[swap] {$0/H$}	(v)
			(u)	edge node {$0/H$}	(t)
			(v)	edge node[swap] {$1/H$}	(t)
			(u)	edge node {$1/1$}	(v)
			;
\end{tikzpicture}
\end{minipage}
&
\begin{minipage}{0.4\textwidth}
\begin{tikzpicture}[shorten >=1pt,node distance=1.5cm,on grid,auto,initial text=]
	  \node[state] (s) {$s$};
	  \node (c) [right = 3cm of s] {$$};
	  \node[state] (u) [above = of c] {$u$};
	  \node[state] (v) [below = of c] {$v$};
	  \node[state] (t) [right = 3cm of c] {$t$};
  	  \path[->] 
			(s)	edge [bend left=10] node {$H-1$}	(u)
			(u)	edge [bend left=10] node {$1$}	(s)
			(s)	edge node[swap] {$H$}	(v)
			(u)	edge node {$H$}	(t)
			(v)	edge [bend right=10] node[swap] {$H-1$}	(t)
			(t)	edge [bend right=10] node[swap] {$1$}	(v)
			(v)	edge node {$1$}	(u)
			;
\end{tikzpicture}
\end{minipage}\\
\end{tabular}
\item Пусть на втором шаге выбран увеличивающий путь $s\to v\to u\to t$ величины 1. После второй итерации исходная сеть и остаточная сеть:\newline
\begin{tabular}{ll}
Сеть & Остаточная сеть\\
\begin{minipage}{0.35\textwidth}
\begin{tikzpicture}[shorten >=1pt,node distance=1.5cm,on grid,auto,initial text=]
	  \node[state] (s) {$s$};
	  \node (c) [right = 3cm of s] {$$};
	  \node[state] (u) [above = of c] {$u$};
	  \node[state] (v) [below = of c] {$v$};
	  \node[state] (t) [right = 3cm of c] {$t$};
  	  \path[->] 
			(s)	edge node {$1/H$}	(u)
			(s)	edge node[swap] {$1/H$}	(v)
			(u)	edge node {$1/H$}	(t)
			(v)	edge node[swap] {$1/H$}	(t)
			(u)	edge node {$0/1$}	(v)
			;
\end{tikzpicture}
\end{minipage}
&
\begin{minipage}{0.4\textwidth}
\begin{tikzpicture}[shorten >=1pt,node distance=1.5cm,on grid,auto,initial text=]
	  \node[state] (s) {$s$};
	  \node (c) [right = 3cm of s] {$$};
	  \node[state] (u) [above = of c] {$u$};
	  \node[state] (v) [below = of c] {$v$};
	  \node[state] (t) [right = 3cm of c] {$t$};
  	  \path[->] 
			(s)	edge [bend left=10] node {$H-1$}	(u)
			(u)	edge [bend left=10] node {$1$}	(s)
			(s)	edge [bend left=10] node {$H-1$}	(v)
			(v)	edge [bend left=10] node {$1$}	(s)
			(u)	edge [bend left=10] node {$H-1$}	(t)
			(t)	edge [bend left=10] node {$1$}	(u)
			(v)	edge [bend right=10] node[swap] {$H-1$}	(t)
			(t)	edge [bend right=10] node[swap] {$1$}	(v)
			(u)	edge node {$1$}	(v)
			;
\end{tikzpicture}
\end{minipage}\\
\end{tabular}
\item Остаточная сеть после второй итерации содержит остаточную сеть для входной сети при $H-1$, поэтому при таком выборе путей будет совершено $2H$ итераций (максимальный поток равен $2H$).
\item Размер входа $f(H)=\Theta(\log H)$ (описание сети~--- константа), время $T(H)=\Omega(H)$ (по количеству итераций), откуда $$\forall c\geqslant \hookrightarrow\lim\limits_{H\to\infty} \frac{T(H)}{f^c(H)}\geqslant \lim\limits_{H\to\infty}\frac{C_1\cdot H}{C_2^c\log^c H} =\lim\limits_{H\to\infty}c_3\frac{H}{\log^c H}=+\infty\Rightarrow T(H)\neq\poly(f(H)).$$
То есть, время работы неполиномиально по длине битовой записи входа.
\end{enumerate}
\newpage
\subsection*{(каноническое) Задача 30}
Пусть $G(V,E)$~--- двудольный граф с долями $L$ и $R$. Считаем, что $E\subseteq L\times R$ (только слева направо). Задача: найти максимальное по мощности $E_0\subseteq E$, такое что любые два ребра из $E_0$ не смежны, то есть, каждая вершина $v\in V$ инцидентна не более, чем одному ребру из $E'$. Определим $G'(V',E')$: $$V'=V\cup\{s,t\}.\,E'=(\bigcup\limits_{l\in L}\{(s,l)\})\cup E\cup (\bigcup\limits_{r\in R}\{(r,t)\}).$$
Задададим все пропускные способности $c(u,v)=\begin{cases}
1, & (u,v)\in E'\\
0, & (u,v)\notin E'\\
\end{cases}$. Тогда $N\eqdef (G',c,s,t)$~--- транспортная сеть.\newline
Обозначим $a=|L|,\,b=|R|$. Поясняющая картинка:\newline
\begin{tikzpicture}[shorten >=1pt,node distance=1.3cm,on grid,auto,initial text=]
	  \node[state] (l1) {$l_1$};
	  \node[state] (l2) [below = of l1] {$l_2$};
	  \node (ld) [below = of l2] {$...$};
	  \node[state] (lm1) [below = of ld] {$l_{a-1}$};
	  \node[state] (l) [below = of lm1] {$l_{a}$};
	  \node[state] (s) [left = 3cm of ld] {$s$};
	  \node[state] (r1) [right = 3cm of l1] {$r_1$};
	  \node[state] (r2) [below = of r1] {$r_2$};
	  \node (rd) [below = of r2] {$...$};
	  \node[state] (rm1) [below = of rd] {$r_{b-1}$};
	  \node[state] (r) [below = of rm1] {$r_{b}$};
	  \node[state] (t) [right = 3cm of rd] {$t$};
  	  \path[->] 
			(s)	edge node {$1$}	(l1)
			(s)	edge node[swap] {$1$}	(l2)
			(s)	edge node {$1$}	(lm1)
			(s)	edge node {$1$}	(l)
			(r1)	edge node {$1$}	(t)
			(r2)	edge node[swap] {$1$}	(t)
			(rm1)	edge node {$1$}	(t)
			(r)	edge node {$1$}	(t)
			(l1)	edge node {$1$}	(r2)
			(l)	edge [out=50,in=200] node {$1$}	(r1)
			(l2)	edge [out=-40,in=190] node {$1$}	(rm1)
			;
\end{tikzpicture}
\begin{enumerate}
\item \label{l1} Пусть $f$~--- поток в $N$. Тогда $|f|$ задает некоторое паросочетание $E_0$, причем $|f|=|E_0|$\begin{enumerate}
\item Вершина $s$ соединена только с вершинами из $L$. Фиксируем одну $l\in L$. Предположим, что $f(s,l)<0$. Тогда (\ref{defFs}) $f(l,s)>0$. Но $c(l,s)=0$, так как так как $(l,s)\notin E'$. Получаем противоречие (\ref{defFl}): $f(l,s)>0=c(l,s)$. Значит, $f(s,l)\geqslant 0$, т.е. $f(s,l)\in \{0,1\}$.
\item \label{tl5} Обозначим $L_0=\{l\in L\big| f(s,l)=1\}$. Тогда $f(s,L)=|L_0|$, так как $c(\cdot,\cdot)\in\{0,1\}$. $s$ инцидентна только вершинам из $L$, поэтому для остальных вершин $v\notin L$ по \ref{uvvu0} имеем $f(s,v)=0$. Значит, $f(s,V)=|L_0|$. Аналогично получаем $f(r,t)\geqslant 0$, обозначим $R_0=\{r\in R\big| f(r,t)=1\}$, и $f(R,t)\equiv f(V,t)=|R_0|$. Но по \ref{fst} $f(s,V)=f(V,t)$, откуда $|L_0|=|R_0|$.
\item \label{tl2} Фиксируем $l\in L$. Пусть $l\in L_0$. Тогда $\exists!r\in R\colon f(l,r)=1$:\begin{enumerate}
\item \label{tl1} Фиксируем $r\in R$. Пусть $f(l,r)<0$. Тогда $f(r,l)>0$. Но $(r,l)\notin E\subseteq L\times R$, откуда $c(r,l)=0<f(r,l)$~--- противоречие (\ref{defFl})
\item ($\exists$) Пусть иначе. Тогда $\forall r\in R\hookrightarrow f(l,r)\leqslant 0$. Из \ref{tl1} получаем, что $f(l,r)=0$. Тогда $f(l,R)=0$. Но $l$ и $t$ не смежны, поэтому (свойство \ref{uvvu0}) $f(l,V\setminus\{s\})=0$. Получим $0\overset{\ref{defFc}}{=}f(l,V)\overset{\ref{FAu}}{=}f(l,s)+f(l,V\setminus\{s\})$. Первое слагаемое равно $-1$, так как $f(s,l)=1$ ($l\in L_0$), второе равно нулю, получаем $0=-1$~--- противоречие $\,\blacksquare$
\item ($!$) Пусть иначе. Поскольку $\forall r\in R\hookrightarrow f(l,r)\geqslant 0$ (\ref{tl1}), найдем $0\overset{\ref{defFc}}{=}f(l,V)\overset{\ref{FAu}}{=}\underbrace{f(l,s)}_{=-1}+\cancelto{0}{f(l,t)}+\cancelto{0}{f(l,L)}+\underbrace{f(l,R)}_{\geqslant 2}\geqslant 1$~--- противоречие.
\end{enumerate}
\item \label{tl3} Пусть $l\in L_0$, $r\in R$: $f(l,r)>0$. Тогда $r\in R_0$. Пусть иначе. Тогда $f(r,t)=0$ (ребра $(t,r)$ нет в $E'$). Получаем $f(r,V)=f(r,l)+f(r,t)=-1\neq 0$~--- противоречие с $\ref{defFc}$.
\item \label{tl4} Пусть $r\in R_0$. Тогда $\exists l\in L\colon f(l,r)=1$. Пусть иначе. $r$ смежна (возможно) только с вершинами из $L$, поэтому $f(r,V)=\underbrace{f(r,t)}_{=1}+\cancelto{0}{f(r,L)}=1$~--- противоречие. По \ref{tl3} эта существующая $l\in L_0$.
\item Построена функция $E_0\colon L_0\to R_0$. Действительно, для каждой $l\in L$ найдена единственная (\ref{tl2}) вершина $r\in R_0$ (\ref{tl3}). По \ref{tl4} эта функция сюръективная (все значения достигаются), и по \ref{tl5} она~--- биекция ($|R_0|=|L_0|$). Значит, $E_0$~--- паросочетание $\blacksquare$
\item Было доказано (\ref{tl5}), что $|L_0|=|R_0|=f(s,V)\equiv |f|$, откуда мощность паросочетания равна величине потока $\blacksquare$
\end{enumerate}
\item \label{l2} Пусть $E_0\subseteq E\subseteq L\times R$~--- паросочетание. Тогда существует поток $f$ в $N$, причем $|f|=|E_0|$\begin{enumerate}
\item Определим\begin{itemize}
\item $f(E_0)=1$ (для каждой пары)
\item $f(s,L_0)=1$, где $L_0=\{l\in L\big|\exists r\in R\colon (l,r)\in E_0\}$
\item $f(R_0,t)=1$, где $R_0=\{r\in R\big|\exists l\in L\colon (l,r)\in E_0\}$
\item $f(E_0^T)=-1$ ($E_0^T=\{(r,l)\big|(l,r)\in E_0\}$)
\item $f(L_0,s)=-1$
\item $f(t,R_0)=-1$
\item $f(u,v)=0$ в остальных случаях
\end{itemize}
\item Тогда $\forall (u,v)\in E'\hookrightarrow f(u,v)=-f(v,u)$
\item Единицы добавлены только на существующих ребрах, поэтому $f(u,v)\leqslant c(u,v)$
\item $E_0$~--- паросочетание, поэтому функция $E_0\colon L_0\to R_0$~--- биекция.
\item Рассмотрим $l\in L\setminus L_0$. Получаем, что (рассматриваем только существующие ребра) $f(l,V)=\cancelto{0}{f(l,s)}+\cancelto{0}{f(l,R)}=0$
\item Рассмотрим $l\in L$. Получаем, что (рассматриваем только существующие ребра) $f(l,V)=\underbrace{f(l,s)}_{=-1}+\underbrace{f(l,R)}_{=1}=0$. $f(l,R)=1$, так как $E_0$~--- биекция.
\item Аналогично для $r\in R$: $\forall r\in R\hookrightarrow f(r,V)=0$
\item Получаем свойство $\ref{defFc}$
\item Получаем, что $f$~--- поток в $N\,\blacksquare$
\item Найдем $|f|=f(s,V)=f(s,L)=f(s,L_0)=|L_0|=|E_0|\,\blacksquare$
\end{enumerate}
\end{enumerate}
Алгоритм: По $G(V,E)$ строим сеть $N$ (конструкция выше), ищем максимальный поток $f$, по нему построим паросочетание $E_0$ (см. \ref{l1}).\newline
Оно будет максимально. Пусть иначе. Тогда по большему паросочетанию $E_0'\colon |E_0'|>|E_0|$ найдем поток $f'$, такой что $f'=|E_0'|$ (см. \ref{l2}). Получим $|f'|\overset{\ref{l2}}{=}|E_0'|>|E_0|\overset{\ref{l1}}{=}|f|$, т.е. $f$~--- не максимальный поток~--- противоречие.
\subsection*{(каноническое) Задача 31.1}
\begin{enumerate}
\item Пусть заданы $n\in\mathbb{N}$, $\{\alpha_i\}_{i=1}^n$, $\{\beta_i\}_{i=1}^n$, $\{\gamma_{ij}\}_{i,j=1}^{n}$, $\forall i\in\overline{1,n}\hookrightarrow\gamma_{ii}=0$. Построим сеть $(G(V,E),c,s,t)$:\begin{enumerate}
\item $V\eqdef\{V_i\}_{i=1}^n\cup\{s,t\}$
\item $c(s,V_i)\eqdef \beta_i$
\item $c(V_i,t)\eqdef \alpha_i$
\item $c(V_i,V_j)\eqdef \gamma_{ij}$
\item $c(u,v)=0$ в остальных случаях
\end{enumerate}
Определим $E=\{(u,v)\in E\big|c(u,v)>0\}$.
\item Рассмотрим некоторый разрез $S$, $T=V\setminus S$. $s\in S$, $t\in T$. Пусть $X=S\cap\{V_i\}_{i=1}^n$, $Y\eqdef\{V_i\}_{i=1}^n\setminus X$. Тогда величина разреза $c(S,T)=\sum\limits_{\substack{u\in S\\v\in T}}c(u,v)=\sum\limits_{y\in Y}c(s,y)+\sum\limits_{x\in X}c(x,t)+\sum\limits_{\substack{x\in X\\y\in Y}}c(x,y)=\sum\limits_{V_i\in X}\alpha_i+\sum\limits_{V_i\in Y}\beta_i+\sum\limits_{\substack{V_i\in X\\V_j\in Y}}\gamma_{ij}$. Обозначим $A\eqdef\{i\big|V_i\in X\}$, $B\eqdef\{j\big|V_j\in Y\}$. Тогда $c(S,T)=\sum\limits_{i\in A}\alpha_i+\sum\limits_{j\in B}\beta_j+\sum\limits_{\substack{i\in A\\j\in B}}\gamma_{ij}=g(A,B)$, где $g(A,B)$~--- функция из условия, которую нужно минимизировать.
\item Фиксируем $A,\,B\colon A\cap B=\varnothing, A\cup B=\overline{1,n}$~--- распределение программ. Заметим, что тогда $S=\{s\}\cup V_A$, $T=\{t\}\cup V_B$~--- разрез. Тогда (предыдущее рассуждение) для него верно равенство $c(S,T)=g(A,B)$.
\item Алгоритм: строим сеть по входу $(n,\alpha_i,\beta_i,\gamma_{ij})$, ищем минимальный разрез, по нему строим ответ. Пусть найденный ответ не минимальный. Тогда существует лучшее распределение, значит, существует меньший разрез~--- противоречие.
\end{enumerate}
\subsection*{(каноническое) Задача 31.2}
\item $1$-я итерация алгоритма. $M$~--- величина увеличивающего пути:\newline
\begin{tabular}{lcll}
Сеть и поток&  & Остаточная сеть & $M$\\
\begin{minipage}{0.35\textwidth}
\begin{tikzpicture}[shorten >=1pt,node distance=2cm,on grid,auto,initial text=]
 \tikzstyle{every node}=[font=\small]
	  \node[state] (s) {$s$};
	  \node[state] (v2) [right = of s] {$v_2$};
	  \node[state] (v1) [above = of v2] {$v_1$};
	  \node[state] (v3) [below = of v2] {$v_3$};
	  \node[state] (t) [right = of s] {$t$};
	  \begin{scope}[every node/.style={scale=.5}]
  	  \path[->] 
			(s)	edge node {$4/5$}	(v1)
			;
			\end{scope}
\end{tikzpicture}
\end{minipage}&$\to$&
\begin{minipage}{0.35\textwidth}
\begin{tikzpicture}[shorten >=1pt,node distance=2cm,on grid,auto,initial text=]
 \tikzstyle{every node}=[font=\small]
	  \node[state,color=green] (s) {$s$};
	  \node[state,color=green] (v2) [right = of s] {$v_2$};
	  \begin{scope}[every node/.style={scale=.5}]
  	  \path[->] 
			(s)	edge [bend left=10] node {$1$}	(v1)
			;
			\end{scope}
\end{tikzpicture}
\end{minipage}& 0\\
\end{tabular}\newline
\end{document}