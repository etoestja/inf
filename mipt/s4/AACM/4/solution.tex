\documentclass[a4paper]{article}
\usepackage[a4paper, left=5mm, right=5mm, top=5mm, bottom=5mm]{geometry}
%\geometry{paperwidth=210mm, paperheight=2000pt, left=5pt, top=5pt}
\usepackage[utf8]{inputenc}
\usepackage[english,russian]{babel}
\usepackage{indentfirst}
\usepackage{tikz} %Рисование автоматов
\usetikzlibrary{automata,positioning,arrows,trees}
\usepackage{amsmath}
\usepackage[makeroom]{cancel} % зачеркивание
\usepackage{multicol,multirow} %Несколько колонок
\usepackage{hyperref}
\usepackage{tabularx}
\usepackage{amsfonts}
\usepackage{amssymb}
\DeclareMathOperator*{\argmin}{arg\,min}
\usepackage{listings}
\usepackage{wasysym}
\date{задано 2014.03.06}
\author{Сергей~Володин, 272 гр.}
\newcommand{\matrixl}{\left|\left|}
\newcommand{\matrixr}{\right|\right|}
% названия автоматов  (rubtsov)
\def\A{{\cal A}}
\def\B{{\cal B}}
\def\C{{\cal C}}

%классы сложности (rubtsov)
\def\PP{{\mathsf{P}}}
\def\NP{{\mathsf{NP}}}
\def\NPc{{\mathsf{NP}}\text{-}{\rm c}}
\def\coNP{{\rm co}\text{-}{\mathsf{NP}}}
\def\DTIME{{\mathsf{DTIME}}}
\def\NTIME{{\mathsf{NTIME}}}
\def\HALT{{\rm{HALT}}}
\def\SAT{{\rm{SAT}}}
\def\3SAT{{\rm{3\text{-}SAT}}}
\def\2SAT{{\rm{2\text{-}SAT}}}
\def\UNSAT{{\rm{UNSAT}}}
\def\LL{{\mathrm{LL}}}
\def\poly{{\rm{poly}}}

\title{Алгоритмы и модели вычислений.\\Задание 4: Сложность вычислений, классы $\PP$, $\NP$ и $\coNP$}

% алгоритмы (Рубцов)
\usepackage{verbatim}
\usepackage{listings}
\usepackage{algorithm2e}

%+= и -=, иначе разъезжаются...
\newcommand{\peq}{\mathrel{+}=}
\newcommand{\meq}{\mathrel{-}=}
\newcommand{\deq}{\mathrel{:}=}
\newcommand{\plpl}{\mathrel{+}+}

% пустое слово  (rubtsov)
\def\eps{\varepsilon}

% регулярные языки  (rubtsov)
\def\REG{{\mathsf{REG}}}
\def\CFL{{\mathsf{CFL}}}
\def\eqdef{\overset{\mbox{\tiny def}}{=}}
\newcommand{\niton}{\not\owns}

%FIRST & FOLLOW (rubtsov)
\def\first{\mathrm{ FIRST} }
\def\follow{\mathrm{ FOLLOW} }

% LL LR (rubtsov)
\def\LL{{\mathrm{LL}}}
\def\LR{{\mathrm{LR}}}

\newcounter{rowItemCount}
\newcounter{subRowItemCount}
\newcommand\rowItem{
    \setcounter{subRowItemCount}{0}
    \arabic{rowItemCount}.\addtocounter{rowItemCount}{1}}
\newcommand\subRowItem{
    \addtocounter{subRowItemCount}{1}
    \addtocounter{rowItemCount}{-1}
    \arabic{rowItemCount}.\arabic{subRowItemCount}.\addtocounter{rowItemCount}{1}}
    
\newcommand{\smalll}[1]{\overline{\overline{#1}}}
\newcommand{\smallo}{\bar{\bar{o}}}

% очень много махания руками, да

\begin{document}
\maketitle
\subsection*{Задача 1}
\begin{enumerate}
\item \label{sat3sat} Докажем, что $\SAT\leqslant_m^p \3SAT$. $\SAT\subset\Sigma^*\ni w=\bigwedge\limits_{i=1}^k\left(\bigvee\limits_{j=1}^{l_i}(x^i_j)^{\sigma^i_j}\right)$. Считаем, что $w$~--- запись формулы. Построим по данной формуле эквивалентную $u\in \3SAT$. Последовательно пройдем по элементам внешней конъюнкции и заменим каждый на эквивалентный (эквивалентные) в смысле выполнимости, содержащие по 3 элемента в дизъюнкции. Обозначаем $y^i_j=(x^i_j)^{\sigma^i_j}$~--- отрицание переменной, либо переменная.\begin{enumerate}
\item Пусть $l_i<3$, в скобке $\Phi_0=y^i_1\vee...\vee y^i_{l_i}$. Повторим какой-либо (для определенности, первый) $y^i_1$ до трех элементов в скобке. Полченную формулу обозначим за $\Phi_1$. Очевидно, полученная функция будет тождественно равна исходной, так как $a\vee a\equiv a$.
\item Пусть $l_i>3$, в скобке $\Phi_0=y_1\vee...\vee y_l$ (для краткости не пишем индексы по $i$, они одни и те же для скобки). Заменим это на $\Phi_1=(y_1\vee y_2\vee z_1)\wedge\underbrace{(\overline{z_1}\vee y_3\vee z_2)\wedge...\wedge(\overline{z_{l-4}}\vee y_{l-2}\vee z_{l-3})}_{\varphi}\wedge (\overline{z_{l-3}}\vee y_{l-1}\vee y_l)$. Элементы $\varphi$ в фигурной скобке строятся следующим образом: в середине $n$-й скобки стоит $y_{i+2}$, первый элемент~--- $\overline{z_{i}}$, последний~--- $z_{i+1}$, т.е. $\varphi=\bigwedge\limits_{n=1}^{l-4}(\overline{z_i}\vee y_{i+1}\vee z_{i+1})$. Докажем, что $$\forall\{x_j\}_{j=1}^m\hookrightarrow\big(\Phi_0(\{x_j\})=1\Leftrightarrow \exists \{z_j\}\colon \Phi_1(\{x_j\},\{z_j\})=1\big)$$\begin{enumerate}
\item Пусть $\Phi_0(\{x_j\}_{j=0}^m)=1$. Поскольку $\Phi_0$~--- дизъюнкция элементов $y_j$, то $\exists j\colon y_j=1$\begin{enumerate}
\item Если $j\in\{1,2\}$, то определим все $z_j=0$. Первая скобка содержит $y_j$, поэтому истинна. В остальных скобках есть отрицание $\overline{z_j}$, поэтому они тоже истинны.
\item $j\in{l-1,l}$. Определим все $z_j=1$. Последняя скобка истинна, так как содержит $y_j$, все предыдущие содержат некоторый $z_j$, поэтому истинны.
\item Оставшиеся случаи ($y_j$ в формуле $\varphi$). Скобка с этим $y_j$: $\overline{z_{j-2}}\vee y_j\vee z_{j-1}$. Определим слева ($w\leqslant j-2$) все $z_w=1$, справа ($w\geqslant j-1$) $z_w=0$. Рассматриваемая скобка истинна, так как содержит $y_j$, скобки слева истинны, так как содержат $z_w=1$, скобки справа истинны, так как содержат $\overline{z_w}=\overline{0}=1$.
\end{enumerate}
\item (контрапозиция) Пусть $\Phi_0(\{x_j\})=0$. Поскольку эта формула~--- дизъюнкция $y_j$, то $y_j=0,\,j\in\overline{1,l}$. Предположим истинность противоположное доказываемому утверждения, т.е. $\exists \{z_j\}\colon \Phi_1(\{x_j\},\{z_j\})=1$. Перепишем формулу с учетом $y_j=0$: $\Phi_1=z_1\wedge\underbrace{(\overline{z_1}\vee z_2)\wedge...\wedge(\overline{z_{l-4}}\vee z_{l-3})}_{\varphi}\wedge \overline{z_{l-3}}$. \underline{Значение равно $1$}, поэтому все конъюнкты истинны, получаем $z_1=1$. Но вторая скобка также истинна, поэтому $z_2=1$. Продолжая (по индукции) получаем, что все $z_j=1$. Но тогда последний конъюнкт $\overline{z_{l-3}}=0$, и \underline{значение формулы~--- $0$}~--- противоречие.
\end{enumerate}
\item Если $l_i=3$, $\Phi_1\eqdef \Phi_0$.
\end{enumerate}
Определим $u=\Phi^i_1$. $u=\Phi^1_1\vee...\vee\Phi^k_1$~--- конъюнкция всех полученных $\Phi^i_1$. Тогда для $u$ выполнено то же свойство, что и для каждого $\Phi_1$:
$$\forall\{x_j\}_{j=1}^m\hookrightarrow\big(w(\{x_j\})=1\Leftrightarrow \exists \{z_j\}\colon u(\{x_j\},\{z_j\})=1\big).$$
Действительно:\begin{enumerate}
\item Пусть $w(\{x_j\})=1$. Тогда выберем $\{z_j\}$ для каждой из формул $\Phi^i_1$ в соответствии с алгоритмом выше. Получим, что все $\Phi^i_1=1$ на полученном наборе.
\item Пусть $\exists \{z_j\}\colon u(\{x_j\},\{z_j\})=1$. Внешняя операция в $u$~--- конъюнкция, поэтому все $\Phi^i_1=1$. Тогда $\Phi^i_0(\{x_j\})=1$ (утверждение для отдельных $\Phi_1$). Но $w$~--- конъюнкция $\Phi^i_0$, поэтому $w(\{x_j\})=1$.
\end{enumerate}
Заметим, что из доказанного свойства следует утверждение: $w$~--- выполнима $\Leftrightarrow$ $u$~--- выполнима.
\begin{enumerate}
\item Оценим длину получившейся формулы. Каждый из элементов $y_j$ добавляет не более одной скобки (ее длина не превышает константы $c_1$). С другой стороны, для исходной формулы $|u|\geqslant c_2\times n$ (каждый $y_j$ имеет ненулевую длину записи), где $n$~--- общее количество $y_j$. Поэтому $|w|\leqslant |u|+c_1n\leqslant|u|+|u|\frac{c_1}{c_2}=O(|u|)=\mbox{poly}(u)$.
\item Определим $f\colon \SAT\subset\Sigma^*\to\Sigma^*\supset\3SAT \colon f(w)=u$ (процедура построения $u$ описана выше). Тогда $f$ вычислима за полиномиальное время. Действительно, алгоритм состоит из $k$ шагов (количество скобок), на каждом шаге скобки модифицируются. Добавляется не более $k$ скобок (в каждой новой скобке есть уникальный $y_j$ из старой скобки). Добавление новой скобки занимает не более, чем $O(|u|)$ (записать строку длины $\leqslant |u|$). Поэтому $T(f(w))=O(k^2|u|)$. Но $|u|=\Omega(k)$, так как каждая скобка имеет непустую запись в исходной формуле, откуда $T(f(w))=O(|u|^3)$
\item Определим $f(w)=($, если $w$~--- не запись формулы. Тогда $f(w)$~--- также не запись формулы (можно проверить за poly).
\item Итак, построена полиномиально-вычислимая функция $f\colon \Sigma^*\to\Sigma^*$, причем $u\in \SAT\Leftrightarrow f(u)\in \3SAT\,\blacksquare$
\end{enumerate}
\item Теорема $\Rightarrow$ $\SAT\in\NPc$, $\ref{sat3sat}\Rightarrow\SAT\leqslant_m^p \3SAT\in \NP$. Поэтому из \ref{abnpc} следует, что $\3SAT\in \NPc$
\end{enumerate}
\subsection*{Задача 2}
Пусть $w\in\Sigma^*\supset\2SAT$. Если $w$~--- не запись формулы в нужном виде (можно проверить за полиномиальное время), останавливаем МТ в не принимающем состоянии. Далее считаем, что $w$~--- запись формулы: $w=(a_1\vee b_1)\wedge(a_2\vee b_2)\wedge...\wedge(a_n\vee b_n)$, где $a_i,\,b_i$~--- переменная $x_j$, либо ее отрицание $\overline{x_j}$. Заметим, что $(a\vee b)\equiv (\overline{a}\Rightarrow b)\wedge (\overline{b}\Rightarrow a)$. Построим граф с вершинами $V=\{x_j\}_{j=1}^m\cup\{\overline{x_j}\}_{j=1}^m$. Есть ребро $(\overline{a},b)\in E\Leftrightarrow $ эквивалентная запись в одной из скобок содержит $\overline{a}\Rightarrow b$. В полученном графе могут существовать пути из $x_i$ в $\overline{x_i}$ и обратно. Докажем утверждение: $w$~--- невыполнима $\Leftrightarrow$ $\exists i\colon x_i\to^*\overline{x_i},\,\overline{x_i}\to^*x_i$ ($a\to^*b$~--- есть путь из $a$ в $b$)\begin{enumerate}
\item $\boxed{\Leftarrow}$ (контрапозиция). Пусть формула выполнима (на наборе $\{x_i\}$), но $\exists i\colon x_i\to^*\overline{x_i},\,\overline{x_i}\to^*x_i$. Пусть $x_i=1$. Тогда все скобки равны $1$, в том числе и те, которые содержат $\overline{x_i}$. Они эквивалентны $(x_i\Rightarrow\cdot)\wedge(\overline{\cdot}\Rightarrow\overline{x_i})$ (см. выше). Обе скобки истинны, поэтому $\cdot=1$. Но это соответствует ребру в графе. Повторяя рассуждение, получаем, что все $y_j$, соответствующие вершинам, достижимым из $x_i$, истинны, в том числе и $\overline{x_i}$~--- противоречие. Аналогично получаем противоречие в случае $x_i=0\,\blacksquare$
\item $\boxed{\Rightarrow}$ (контрапозиция) $\forall i\in\overline{1,m}\hookrightarrow x_i\not\to^*\overline{x_i}\mbox{ или }\overline{x_i}\not\to^*x_i$. Определим в первом случае $x_i=1$, а во втором определим $x_i=0$. Выполним поиск в глубину из $x_i$ или $\overline{x_i}$ соответственно. Устанавливаем значения вершин в $1$. В случае конфликта (установлены $x_k=1$, $\overline{x_k}=1$) отбрасываем найденный путь и <<забываем>> установленные значения\newline
После всех таких обходов будет найдей хотя бы один набор значений входящих в дерево переменных, так как в противном случае (возникают конфликты на каждом пути из каждой вершины) получим $x_i\to^*\overline{x_i},\,\overline{x_i}\to^* x_i$ (поиск запускается из каждой вершины, поэтому каждое ребро может быть пройдено в обе стороны). Тогда на данном наборе формула истинна, так как истинны все следствия в эквивалентных скобках $(\overline{a}\Rightarrow b)\wedge (\overline{b}\Rightarrow a)$
\end{enumerate}
Алгоритм: строим граф, поиском в глубину ищем пути $a\to^*\overline{a}$. Если найдено $a\to^*\overline{a}$ и $\overline{a}\to^* a$, $w\notin \2SAT$, иначе $w\in\2SAT$. Оценим время работы. Длина входа $|w|=\Omega(n)$. Количество вершин не более $2n$, количество ребер не больше $4n^2$. Количество поисков~--- $2n$. Тогда $T(w)=O(2n(|V|+|E|))=O(2n\times 2n+2n\times 4n^2)=O(n^3)=O(|w|^3)$.
\subsection*{(каноническое) Задача 16}
\begin{enumerate}
\item $\begin{Vmatrix}
A & b
\end{Vmatrix}^\Box=\begin{Vmatrix}
\underline{1/1} & 0/1 & 0/1 & 4/1\\
3/1 & 4/1 & 0/1 & 16/1\\
9/1 & 3/1 & 1/1 & 64/1\\
\end{Vmatrix}
\overset{r1 *= \frac{1}{1}}{\sim}
\begin{Vmatrix}
1/1 & 0/1 & 0/1 & 4/1\\
3/1 & 4/1 & 0/1 & 16/1\\
9/1 & 3/1 & 1/1 & 64/1\\
\end{Vmatrix}
\underset{\substack{\frac{16}{1}-\frac{4}{1}\frac{3}{1}=\frac{4}{1}\\\frac{64}{1}-\frac{4}{1}\frac{9}{1}=\frac{28}{1}}}{\overset{s 1 2 \frac{3}{1}, s 1 3 \frac{9}{1}}{\sim}}
\begin{Vmatrix}
1/1 & 0/1 & 0/1 & 4/1\\
0/1 & \underline{4/1} & 0/1 & 4/1\\
0/1 & 3/1 & 1/1 & 28/1\\
\end{Vmatrix}
\overset{r2 *= \frac{1}{4}}{\sim}\newline\sim
\begin{Vmatrix}
1/1 & 0/1 & 0/1 & 4/1\\
0/1 & 1/1 & 0/1 & 1/1\\
0/1 & 3/1 & 1/1 & 28/1\\
\end{Vmatrix}
\underset{\frac{28}{1}-\frac{1}{1}\frac{3}{1}=\frac{25}{1}}{\overset{s 2 3 \frac{3}{1}}{\sim}}
\begin{Vmatrix}
1/1 & 0/1 & 0/1 & 4/1\\
0/1 & 1/1 & 0/1 & 1/1\\
0/1 & 0/1 & 1/1 & 25/1\\
\end{Vmatrix}
$
\begin{enumerate}
\item $r_i *= \frac{c_1}{c_2}$~--- умножение строки $i$ на дробь $\frac{c_1}{c_2}$
\item $s i j \frac{c_1}{c_2}$~--- вычитание $i$-й строки, умноженной на дробь $\frac{c_1}{c_2}$ из $j$-й.
\end{enumerate}
\item $d_{33}^{(2)}=1\times 1\times 1=1=1\times 1\times 1=d_1d_2a_{33}^{(2)}$
\end{enumerate}
\subsection*{(каноническое) Задача 17}
\begin{enumerate}
\item (не дописано) Сертификат~--- решение системы уравнений. Длина полиномиальна (???). Проверочный сертификат: подставляем числа, проверяем равенства. Время проверки полиномиально (???)
\item Неравенства сводятся к равенствам путем увеличения количества переменных. Действительно, $a_1x_1+...+a_nx_n\geqslant b_n$ $\Leftrightarrow a_1x_1+...+a_nx_n=b_n+x$, где $x\geqslant 0$. Дополнительно в сертификат входит список переменных $x$, для которых должно быть выполнено $x>0$.
\item Сертификат для $x$~--- двоичная запись делителя $y$. $R(x,y)=\mbox{,,}y|x\mbox{''}\vee\mbox{,,}y\geqslant 2\mbox{''}\vee \mbox{,,}\mbox{в }x\mbox{ не встречается } 10101\mbox{''}$. $y<x$, поэтому $|y|\leqslant|x|$. Первая часть проверяется за $O(|x|^2)$ (деление в столбик), вторая часть~--- за $O(1)$, третья~--- за $O(|x|)$~--- поиск подстроки.
\end{enumerate}
\subsection*{(каноническое) Задача 18}
\begin{enumerate}
\item Сертификат простоты $3361$. Первообразный корень $g=22$. Делители $p-1=2^3\cdot 3\cdot 5\cdot 7$~--- $\underline{2,3,5,7}$. Далее сертификаты для простых делителей $p-1$\begin{enumerate}
\item $2,3,5$~--- простые (листья рекурсии)
\item Сертификат простоты $7$. Первообразный корень $g=3$. Делители $p-1=2\cdot 3$~--- $\underline{2,3}$
\end{enumerate}
\end{enumerate}
Поэтому сертификат: $(3361, 22, ((2,3),(3,1),(5,1),(7,1))), (7, 3, ((2, 1), (3, 1)))$. Каждая скобка содержит проверяемое число, первообразный корень и разложение $p-1$ на множители. В сертификат входит сертификат для $7$, так как $7|p-1$.
\subsection*{(каноническое) Задача 19}
Пусть $L\in\NP$. $w_i\in L$, $L^*\ni w=w_1...w_n$. Проверочный предикат для $L$~--- $R_0(x,y)$. Определим сертификат $y(w)$. Добавим список позиций $l_i$ начал слов $w_i$ в слове $w$. Количество слов $n\leqslant |w|$, длина записи числа $O(\log |w|)$ (номер позиции не больше, чем длина слова). Тогда суммарная длина $O(n\log|w|)=O(|w|\log|w|)$. Добавим в сертификат $y(w)$ сертификаты $y_i$ для $w_i$. Определим $R(w,y)=R_0(w[1,l_1-1],y_1)\wedge R_0(w[l_1,l_2-1],y_2)\wedge...\wedge R_0(w[l_n,|w|-1],y_n)\wedge (l_i<l_{i+1})$~--- имеется в виду такой предикат, который дает то же значение, но в его записи явным образом не фигурирует $n$ (например, это предикат, построенный по МТ, проверяющей эти условия). Тогда для слов $w\in L^*$ $R(w,y(w))=1$ (по построению $w[l_i,l_{i+1}-1]\in L$, и $y_i$~--- их сертификаты). Пусть $R(w,y)=1$. Тогда задано разбиение слова $1<l_2<l_3<...<l_n<n$ на подслова, причем каждое подслово из $L$, значит, $w\in L^*$.
\subsection*{(каноническое) Задача 20}
Докажем, что $\overline{L}\in\NP$. Сертификат~--- список подразбиений ребер графа $K_{3,3}$ или $K_5$ (и указание, какого именно графа) + описание соответствий вершин полученного графа и входного. Тогда для не планарных графов (слов-описаний графов из $\overline{L}$) такой сертификат существует (теорема Куратовского). Количество подразбиений не больше, чем количество вершин в исходном графе, каждое кодируется константой символов. Соответствие кодируется $O(|V|)$ символами ($V$~--- вершины входного графа). Длина входного слова не меньше, чем $|V|$ и $|E|$. Поэтому длина сертификата $|y(w)|=O(|V|+|V|)=O(|V|)=O(|w|)$. Проверочный предикат: выполняем подразбиения ребер (каждое за константу времени), проверяем соответствие ребер двух графов с заданным соответствием вершин за $O(|E|)$. Время $O(|V|+|E|)=O(|w|^2)$.
\subsection*{Вспомогательные утверждения}
\begin{enumerate}
\item \label{trans} $\leqslant_m^p$~--- транзитивно. Действительно, пусть $\Sigma_1^*\supseteq A\leqslant_m^p B\subseteq \Sigma_2^*$, $B\leqslant_m^p C\subseteq\Sigma_3^*$. Тогда существуют полиномиально-вычислимые функции $f_1\colon \Sigma_1^*\to \Sigma_2^*$, $f_2\colon \Sigma_2^*\to \Sigma_3^*$, причем $\forall x\in\Sigma_1^*\left(x\in A\Leftrightarrow f_1(x)\in B\right)$, $\forall y\in \Sigma_2^*\left(y\in B\Leftrightarrow f_2(y)\in C\right)$\newline
Фиксируем $x\in\Sigma_1^*$, определим $y=f_1(x)$. Тогда $x\in A\Leftrightarrow \underbrace{f_1(x)}_y\in B\Leftrightarrow f_2(f_1(x))\in C$\newline
Функция $g(x)\colon \Sigma_1^*\to\Sigma_3^*$ $g=f_2\circ f_1$ полиномиально-вычислима (как композиция полиномиально-вычислимых). Получаем, что существует полиномиально-вычислимая $g(x)$, такая что $\forall x\in\Sigma_1^*\left(x\in A\Leftrightarrow g(x)\in C\right)$, откуда $A\leqslant_m^p C\,\blacksquare$
\item \label{abnpc} Пусть $A\in\NPc$, и $A\leqslant_m^p B\in \NP$. Тогда $B\in \NPc$. Действительно, $A\in\NPc\Rightarrow\forall C\in \NP\hookrightarrow C\leqslant_m^p A$. Фиксируем $C\in\NP$. $A\leqslant_m^p B$, поэтому из \ref{trans} следует, что $C\leqslant_m^p B$. Поэтому $\forall C\in\NP\hookrightarrow C\leqslant_m^p B$. Значит, $B\in\NPc\,\blacksquare$
\end{enumerate}
\end{document}