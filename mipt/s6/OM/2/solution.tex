\documentclass[a4paper]{article}
\usepackage[a4paper, left=5mm, right=5mm, top=5mm, bottom=5mm]{geometry}
\usepackage[utf8]{inputenc}
\usepackage[english,russian]{babel}
\usepackage{graphicx}
\usepackage{indentfirst}
\usepackage{tikz} %Рисование автоматов
\usetikzlibrary{automata,positioning}
\usepackage{amsmath}
\usepackage{floatflt}
\usepackage{enumerate}
\usepackage{amsfonts}
\usepackage{amssymb}
\newcommand{\matrixr}{\right|\right|}
\newcommand{\matrixl}{\left|\left|}
\def\eqdef{\overset{\mbox{\tiny def}}{=}}
\title{Методы оптимизации. Задание 2}
\date{задано 2016.03.29}
\author{Сергей~Володин, 374 гр.}

\begin{document}
\maketitle
\section*{Задача 1}
(2016.03.29)
Доказать: Пусть $f$~--- $\beta$-гладкая. Тогда $\forall x,y \hookrightarrow f(x)\leqslant f(y)+\nabla^Tf(y)(x-y)+\frac{\beta}{2}||x-y||^2$.

\begin{enumerate}
	\item Имеем $||\nabla f(x)-\nabla f(y)||_*\leqslant \beta ||x-y||$. Тогда 
	\item Обозначим $\mu(t)=f(y+t(x-y))\colon [0,1]\to\mathbb{R}$. Поскольку $f$ дифференцируема, $\mu$ также дифференцируема как композиция дифференцируемых функций. Тогда $\mu(1)=\mu(0)+\int\limits_0^1\mu'(t)dt$. Подставим определение $\mu$, получим формулу Ньютона-Лейбница для $f$ на отрезке $[y,x]\in\mathbb{R}^n$: $f(x)=f(y)+\int\limits_0^1 dt \nabla^Tf(y+t(x-y))(x-y)$.
	\item Рассмотрим величину $\alpha\eqdef f(x)-f(y)-\nabla^Tf(y)(x-y)$ и докажем, что $\alpha\leqslant 0$:
	\item $\alpha=\int\limits_0^1 dt \nabla^Tf(y+t(x-y))(x-y)-\nabla^Tf(y)$. Внесем второе слагаемое под интеграл, получим
	$$\alpha=\int\limits_0^1 dt \left(\nabla^T f(y+t(x-y))-\nabla^T f(y)\right)(x-y)$$
	\item $|\alpha|\leqslant\int\limits_0^1 dt |\underbrace{\left(\nabla^T f(y+t(x-y))-\nabla^T f(y)\right)}_{A}(x-y)|$.
	\item $A$--- оператор, действующий на $x-y$. Поскольку $f$~--- $\beta$-гладкая, т.е. $$\forall x,y\hookrightarrow ||\nabla f(x) - \nabla f(y)||_*\leqslant \beta ||x-y||,$$
	Получаем $||A||_*=\sup\limits_{x\in\mathbb{R}}\frac{|Ax|}{||x||}\leqslant \beta ||y-t(x-y)-y||=\beta t ||x-y||$, откуда $|A(x-y)|\leqslant ||A||_*||x-y||\leqslant \beta t||x-y||^2$
	\item Получаем $|\alpha|\leqslant \int\limits_0^1 \beta t ||x-y||^2 dt = \frac{\beta}{2}||x-y||^2$ $\blacksquare$
\end{enumerate}
\section*{Задача 2}
Определим $\delta_t\eqdef f(x_t)-f(x^*)$, где $x_t$~--- $t$-я точка в алгоритме Frank-Wolfe. Получена оценка
$$\delta_{t+1}\leqslant \frac{\beta R^2}{2}\left(\prod\limits_{k=1}^t(1-\gamma_k)+\sum\limits_{k=1}^t\gamma^2_{t-k}\prod\limits_{j=t-k+1}^t(1-\gamma_j)\right)\boxed{=}$$

Оценить выражение как функцию $\gamma_t$ и выбрать $\gamma_t$ как минимум этой функции.

Переобозначим $k\to t-k$ во втором слагаемом: $\boxed{=}\left((1-\gamma_0)\cdot...\cdot(1-\gamma_t)+\sum\limits_{k=1}^t\gamma_k^2\prod\limits_{j=k+1}^t(1-\gamma_j)\right)\cdot \mu\eqdef \hat{\delta}_{t+1}$, где $\mu=\frac{\beta R^2}{2}$.

Обозначим
$$\Psi_{t+1}\eqdef \frac{\hat{\delta}_{t+1}}{\mu}=(1-\gamma_1)...(1-\gamma_t)+\sum\limits_{k=1}^t\gamma_k^2\prod\limits_{j=k+1}^{t}(1-\gamma_j)$$

Тогда $$\Psi_{t+1}=(1-\gamma_t)\Psi_t+\gamma_t^2$$
причем $\Psi_2=1$ (так как $\delta_2\leqslant \frac{\beta R^2}{2}$, поскольку $f$~--- $\beta$-гладкая) и $\Psi_{t+1}=\Psi_{t+1}(\gamma_1,...,\gamma_t)$

Найдем минимум функции $\Psi_{t+1}(\gamma_1,...,\gamma_t)$.

Поскольку $\Psi_t$ не зависит от $\gamma_t$, получаем:
$$\frac{\partial \Psi_{t+1}}{\partial \gamma_t}=-\Psi_t+2\gamma_t=0$$

Откуда $$\gamma_t=\frac{\Psi_t}{2}$$

Пусть $l<t$. Найдем $$\frac{\partial \Psi_{t+1}}{\partial \gamma_l}=(1-\gamma_t)\frac{\partial \Psi_{t}}{\partial \gamma_l}=...=(1-\gamma_t)\cdot...\cdot(1-\gamma_{l+1})\frac{\partial \Psi_{l+1}}{\partial \gamma_l}=0$$

Считаем, что $\gamma_t\neq 1,...,\gamma_2\neq 1$, откуда

$$\frac{\partial \Psi_{l+1}}{\partial \gamma_l}=0\Rightarrow \gamma_l=\frac{\Psi_l}{2}$$

Пусть $l\in \overline{2,t}$. Подставим $\gamma_l$ в $\Psi_{l+1}$:

$$\Psi_{l+1}=(1-\gamma_l)\Psi_{l}+\gamma_l^2=\Psi_l-\frac{\Psi_l^2}{4}$$

Найдем $$\gamma_{l+1}=\frac{1}{2}\Psi_{l+1}=\frac{1}{2}(\Psi_l-\frac{\Psi_l^2}{4})=\frac{1}{2}(2\gamma_l-\gamma_l^2)=\gamma_l-\frac{\gamma_l}{2}$$

Получаем рекуррентную последовательность:
$$
\begin{cases}
\gamma_1 = 1\\
\gamma_{l+1}=\gamma_l-\frac{\gamma_l^2}{2}\\
\end{cases}
$$

{\em ??? Сюда не подходит $\gamma_t=\frac{2}{t+1}$}
\section*{Задача 3}
$E=(\mathbb{R}^n,||\cdot||)$. Определим $f^*(p)\eqdef\sup\limits_{x\in\mathbb{R}^n}(p^Tx-f(x))$, $p\in E^*$. Найти субдифференциал $\partial f^*(p)$.

Пусть $f\in C(\mathbb{R}^n)$, $f$~--- выпуклая. Тогда $f^{**}(x)=f(x)$.

Пусть $x\in\mathbb{R}^n$. Рассмотрим $p\in\partial f(x)\Leftrightarrow \forall\,x'\in\mathbb{R}^n\hookrightarrow f(x')\geqslant f(x)+p^T(x'-x)\Leftrightarrow \forall\,x'\in\mathbb{R}^n\hookrightarrow f(x')-p^Tx'\geqslant f(x)-p^Tx\Leftrightarrow \forall x'\in\mathbb{R}^n\hookrightarrow p^Tx'-f(x')\leqslant p^Tx-f(x')$.

По определению, $f^*(p)=\sup_{x\in\mathbb{R}^n}(p^Tx-f(x))$

Тогда получаем, что $f^*(p)=p^Tx-f(x)\Leftrightarrow p\in\partial f(x)$. То есть, $$p\in \partial f(x)\Leftrightarrow p^Tx=f(x)+f^*(p)$$

Аналогично, $$x\in\partial f^*(p)\Leftrightarrow x^Tp=f^*(p)+f^{**}(x)$$

Поскольку $f^{**}(x)=f(x)$, получаем

$$x\in\partial f^*(p)\Leftrightarrow p^Tx=f(x)+f^*(p)\Leftrightarrow p\in\partial f(x)\Leftrightarrow x\in (\partial f)^{-1}(p) $$

То есть, $$\partial f^*=(\partial f)^{-1}$$

\end{document}
