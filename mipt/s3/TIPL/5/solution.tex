\documentclass[a4paper]{article}
\usepackage[a4paper, left=5mm, right=5mm, top=5mm, bottom=5mm]{geometry}
\usepackage[utf8]{inputenc}
\usepackage[english,russian]{babel}
\usepackage{indentfirst}
\usepackage{tikz} %Рисование автоматов
\usetikzlibrary{automata,positioning,arrows}
\usepackage{amsmath}
\usepackage{enumerate}
\usepackage{amsfonts}
\usepackage{amssymb}
\usepackage{wasysym}
\title{Теория и реализация языков программирования.\\Задание 5: Регулярные грамматики}
\date{задано 2013.10.02}
\author{Сергей~Володин, 272 гр.}
\newcommand{\matrixl}{\left|\left|}
\newcommand{\matrixr}{\right|\right|}
% названия автоматов
\def\A{{\cal A}}
\def\B{{\cal B}}
\def\C{{\cal C}}

%+= и -=, иначе разъезжаются...
\newcommand{\peq}{\mathrel{+}=}
\newcommand{\meq}{\mathrel{-}=}
\newcommand{\deq}{\mathrel{:}=}
\newcommand{\plpl}{\mathrel{+}+}

% регулярные языки
\def\REG{{\mathsf{REG}}}

\newcommand{\niton}{\not\owns}

\begin{document}
\maketitle
\section*{Задача 1}
\begin{tabular}{l l}
\begin{minipage}{0.6\textwidth}
Нет, предложенный алгоритм может построить грамматику, которая не будет правилинейной регулярной. Например, для автомата $\A$ из условия переход $q_0\overset{\varepsilon}{\longrightarrow}q_1$ по алгоритму должен соответствовать правилу $q_0\longrightarrow\varepsilon q_1$, но это правило не имеет вид $A\longrightarrow xB$ ($\varepsilon=x\notin \Sigma$) или $A\longrightarrow x$ или $A\longrightarrow \varepsilon$.
\end{minipage}
\begin{minipage}{0.3\textwidth}
\begin{tikzpicture}[shorten >=1pt,node distance=2cm,on grid,auto,initial text=]
	\node[state, initial]	(q_0)							{$q_0$};
	\node[state] 		  	(q_1) [above right = of q_0]	{$q_1$};
	\node[state, accepting] 	(q_2) [right = of q_1]			{$q_2$};
	\node[state]			(q_3) [right = 2.5cm of q_0]			{$q_3$};
	\node[state, accepting]	(q_4) [below right = of q_0]	{$q_4$};
	\path[->]
		(q_0)	edge	node {$\varepsilon$}	(q_1)
				edge	node {$\varepsilon$}	(q_4)
				edge	node {$a$} 		(q_3)
		(q_1)	edge	node {$a$}		(q_2)
				edge	node {$b$}		(q_3)
		(q_2)	edge	node {$a$}		(q_3)
		(q_3)	edge	node {$b$}		(q_4)
		(q_4)	edge [bend left]	node {$a$} 		(q_0);
\end{tikzpicture}
\end{minipage}
\end{tabular}\newline
Заметим, что проблему можно решить, преобразовав НКА в ДКА. Тогда $\varepsilon$-переходов не будет. Остается один случай, в котором $q_0\in F$, и в $q_0$ есть переходы из других состояний: $\exists q_1\colon \delta(q_1,\sigma)=q_0$. Соответствующими правилами были бы $q_0\longrightarrow\varepsilon\,q_1\longrightarrow\sigma q_0$, которые не подходят для праволинейной регулярной грамматики (аксиома $q_0$ встречается в правой части при том, что есть переход $q_0\longrightarrow\varepsilon$)

\end{document}