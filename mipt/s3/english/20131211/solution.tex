\documentclass[a4paper]{article}
\usepackage[a4paper]{geometry}
%\geometry{paperwidth=210mm, paperheight=2000pt, left=5pt, top=5pt}
\usepackage[utf8]{inputenc}
\usepackage[english]{babel}
\usepackage{indentfirst}
\usepackage{tikz} %Рисование автоматов
\usetikzlibrary{automata,positioning,arrows,trees}
\usepackage{amsmath}
\usepackage{enumerate}
\usepackage[makeroom]{cancel} % зачеркивание
\usepackage{multicol,multirow} %Несколько колонок
\usepackage{hyperref}
\usepackage{tabularx}
\usepackage{amsfonts}
\usepackage{amssymb}
\usepackage{wasysym}
\title{Opinion essay}
\date{\today}
\author{Sergey Volodin, 272 gr.}

\begin{document}
\maketitle
%The thing is that I don't give a shit about this essay. It is fucking stupid.
By my calculations, I should have paid around \$5000 for films and shows I've watched. But I haven't. Should it be a crime to download films from the Internet without paying?
\\[5pt]
%Introductory
In Russia it is common to download for free music and films. This ,,tradition`` started, I think, in 1990s, when ,,pirate`` video cassetes with foreign films appeared in video stores. Nowadays lots of people watch movies for free in their homes and they don't pay for it. Why is this happening? For me the main reason is that it is not possible to watch American/British TV channels in Russia legally. There are channels that are called the same, but what they do is broadcasting quite old shows often with poor translation. The only way to get an ongoing show with russian subtitles is to download it. So, therefore, I think, such situation should not be regarded as crime, at least for now, because there is no other way. But, in general, I think it should be considered as a crime.
\\[5pt]
Firstly, it is a bad thing to do: making a film does require lots of work and, therefore, its prime cost is high. Therefore, we should pay for content as much as we should pay for other goods. How it is called if you take something expensive and don't pay for it? Right, it is stealing. So, what I and many people do is actually not that different from what do some guys in balaclavas.
\\[5pt]
Futhermore, let's see, what happens, if, like 10000 people don't pay for films they've watched. If they are like me, total amount of money they should have paid is fifty millon dollars, which is almost a budget of The Matrix movie. So, as a result, we could have lost such a movie, which is quite a tragedy. Of course, it is just a made-up example, but here is a fact: many good shows are closing because of lack of viewers, for example, Stargate Universe. But if you count also people, who have downloaded this show for free, this number certanly will be enough.
\\[5pt]
% probably the reason why there are not so many good movies in Russia filmed in last 10 years.
However, the situation with foreign films and shows is not that bad, because there are ,,legal`` viewers. But how many {\em{good}} films maked in Russia in the last $15$ years can you name? This number is probably less than ten. And, by my opinion, downloading films for free is one of the reasons why film companies are not interested in making proper movies. Thus, avoiding paying for films influences number of good films.
\\[5pt]
But why we still do that? The main reason, I think, is because we can: would you spend your money for something, if you can get it totally for free? Probably, not. But, then, why don't you go to the streets and rob some bank? The main difference between robbing banks and downloading films is that in Russia free downloads are not considered as a crime. But as we have seen, both of these things are wrong. It goes even further: some of your friends may look at you strange when you tell them that you have bought some content legally. Therefore, it should be considered as a crime to download films for free.
\\[5pt]
To sum up, I'm certain that it should be a crime do download films illegally after media companies will make it possible to get it legally. I think, it will happen in nearest future.
\end{document}