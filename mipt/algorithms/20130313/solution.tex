\documentclass[a4paper]{article}
\usepackage[utf8]{inputenc}
\usepackage[russian]{babel}
\usepackage{graphicx}
\usepackage{indentfirst}
\usepackage{amsmath}
\usepackage{enumerate}
\usepackage{amsfonts}
\usepackage{amssymb}
\title{Задание от 2013.03.19}
\author{С.~Е.~Володин, 272 гр.}
\date{}
\begin{document}
\maketitle
\begin{enumerate} 
\item Задача 1
\begin{enumerate}
\item Докажем, что в таком массиве (в котором первый и последний элементы различны: $A[0]\neq A[N-1]$) существует искомая подстрока ($01$, либо $10$). Пусть иначе: $\nexists i:A[i]\neq A[i+1]\Leftrightarrow\forall i\hookrightarrow A[i]=A[i+1]$. Тогда, по индукции, $A[i]=A[N-1]$~--- противоречие.\newline
Рассмотрим элемент $A[m]$, $0<m<N-1$. Поскольку $A[0]\neq A[N-1]$, то $A[0]\neq A[m] \vee A[m] \neq A[N-1]$. В первом случае по доказанному выше существует искомая подстрока в $(A[0],\dots,A[m])$, а во втором~--- в $(A[m],\dots,A[N-1])$. Выберем $m=\lfloor\frac{N}{2}\rfloor$. Тогда задача свелась к предыдущей с размерностью $\frac{N}{2}$.\newline
Реализация (двоичный поиск, рекурсия для простоты):\begin{verbatim}void find(l, r)
{
    if(l + 1 == r) ans = l;
    if(l + 1 >= r) return;

    int m = (l + r) / 2;
    if(A[l] != A[m]) find(l, m);
    else find(m, r);
}
\end{verbatim}
Время: $O(\log N)$~--- из Master theorem, так как $T(N)=T(\frac{N}{2})+O(1)$
\end{enumerate}
\item Задача 2 (троичный поиск). Считаем, что последнее значение первой последовательности и первое второй не совпадают (в противном случае максимум достигается на двух элементах). Пусть $i$~--- искомый индекс, т.е. \begin{equation}\begin{array}{lcr}\label{icond}\forall j_1<j_2\in[0,i] & \hookrightarrow & A[j_1]<A[j_2]\\
\forall j_1<j_2\in[i,N-1] & \hookrightarrow & A[j_1]>A[j_2]\\\end{array}.\end{equation} Алгоритм состоит из $O(\log N)$ итераций с константным временем на каждой. Пусть $l,r$~--- границы поиска, изначально $l=0, r=N-1$. На каждой итерации будет сохраняться свойство: \begin{equation}\label{lrcond}i\in [l,r].\end{equation} В начале оно, очевидно, выполнено: $i\in[0,N-1]$. На каждой итерации рассмотрим точки $m_1,m_2:l<m_1<m_2<r$. Подразумевается, что $|r-l|\geq 3$, в противном случае цикл заканчивается.\newline
Рассмотрим три случая, которые могут возникнуть на одной итерации:
\begin{enumerate}
\item $A[m_1]=A[m_2]$. Докажем, что индекс $i\in[m_1,m_2]$. Пусть иначе, рассмотрим случай $i<m_1$. Тогда из (\ref{icond}) получаем противоречие: $m_1<m_2\in[i,N-1]$, но $A[m_1]=A[m_2]$.\newline
Уберем правую часть $(m_2,r]$: $r\leftarrow m_2$.
\item $A[m_1]<A[m_2]$. Докажем, что индекс $i\in[m_1,r]$. Пусть иначе: $i\in[l,m_1)$ Тогда получаем противоречие с (\ref{icond}): $m_1<m_2\in[i,N-1]$, но $A[m_1]<A[m_2]$.\newline
Уберем левую часть $[l,m_1)$: $l\leftarrow m_1$
\item $A[m_1]>A[m_2]$. Докажем, что индекс $i\in[l,m_2]$. Пусть иначе: $i\in(m_2,r]$ Тогда получаем противоречие с (\ref{icond}): $m_1<m_2\in[0,i]$, но $A[m_1]>A[m_2]$.\newline
Уберем правую часть $(m_2,r]$: $r\leftarrow m_2$.
\end{enumerate}
Все возможные случаи разобраны, свойство (\ref{lrcond}) сохранено.\newline
Заметим, что на каждой итерации величина $|r-l|$ уменьшается как минимум на $1$, а на первой итерации равна $N-1$. Поэтому до состояния $|r-l|=3$ будет совершено не более $N$ итераций (алгоритм завершится на любых входных данных).\newline
Будем выбирать точки следующим образом: $m_1=\lfloor\frac{2l+r}{3}\rfloor,m_2=\lfloor\frac{l+2r}{3}\rfloor$. Тогда на каждой итерации величина $|r-l|$ будет уменьшаться в $\frac{3}{2}$ раза. То есть, $N-1=(\frac{3}{2})^k l$, где $l<\frac{3}{2}$, $k$~--- количество итераций, т.е. время работы $O(k)$ будет равно $O(\log N)$.\newline
После заверщения цикла получим числа $l$ и $r$, для которых выполнено свойство (\ref{lrcond}). Перебором 4-х вариантов найдем $i$.\newline
Суммарное время: $O(\log N)$.\newline
Реализация:\begin{verbatim}
int l = 0, r = N - 1;
int m1, m2;
while(l + 3 < r)
{
    m1 = (2 * l +     r) / 3;
    m2 = (    l + 2 * r) / 3;

    if(A[m1] < A[m2])
        l = m1;
    else r = m2;
}

int i = l;
for(int c = l + 1; c <= r; c++)
    if(arr[c] > arr[i]) i = c;\end{verbatim}
%$$m_1=\lfloor\frac{2l+r}{3}\rfloor$ m_2=\lfloor\frac{l+2r}{2}\rfloor$
\end{enumerate}

\end{document}
