 \documentclass[12pt, leqno]{article}
\usepackage[T2A]{fontenc}
\usepackage[utf8]{inputenc}        % Кодировка входного документа;
                                    % при необходимости, вместо cp1251
                                    % можно указать cp866 (Alt-кодировка
                                    % DOS) или koi8-r.

\usepackage[english,russian]{babel} % Включение русификации, русских и
                                    % английских стилей и переносов
%%\usepackage{a4}
%%\usepackage{moreverb}
\usepackage{amsmath,amsfonts,amsthm,amssymb,amsbsy,amstext,amscd,amsxtra,multicol}
\usepackage{verbatim}
\usepackage{listings}
\usepackage{algorithm2e}

\usepackage{tabto} %Табуляция
\usepackage{tikz} %Рисование автоматов
\usetikzlibrary{automata,positioning,trees}
\usepackage{multicol} %Несколько колонок
\usepackage{graphicx}
\usepackage[colorlinks,urlcolor=blue]{hyperref}
\usepackage[stable]{footmisc}
\usepackage{indentfirst}  % абзацный отступ после заголовка
\usepackage{enumitem} %Item tricks


%% \voffset-5mm
%% \def\baselinestretch{1.44}
\renewcommand{\theequation}{\arabic{equation}}
\def\hm#1{#1\nobreak\discretionary{}{\hbox{$#1$}}{}}
\newtheorem{Lemma}{Лемма}
\theoremstyle{definiton}
\newtheorem{Remark}{Замечание}
%%\newtheorem{Def}{Определение}
\newtheorem{Claim}{Утверждение}
\newtheorem{Cor}{Следствие}
\newtheorem{Theorem}{Теорема}
\newtheorem*{known}{Теорема}
\def\proofname{Доказательство}
\theoremstyle{definition}
\newtheorem{Def}{Определение}
\theoremstyle{definition}
\newtheorem{Example}{Пример}

%% \newenvironment{Example} % имя окружения
%% {\par\noindent{\bf Пример.}} % команды для \begin
%% {\hfill$\scriptstyle\qed$} % команды для \end






%\date{22 июня 2011 г.}
\let\leq\leqslant
\let\geq\geqslant
\def\MT{\mathrm{MT}}
%Обозначения ``ажуром''
\def\BB{\mathbb B}
\def\CC{\mathbb C}
\def\RR{\mathbb R}
\def\SS{\mathbb S}
\def\ZZ{\mathbb Z}
\def\NN{\mathbb N}
\def\FF{\mathbb F}
%греческие буквы
\let\epsilon\varepsilon
\let\es\varnothing
\let\eps\varepsilon
\let\al\alpha
\let\sg\sigma
\let\ga\gamma
\let\ph\varphi
\let\om\omega
\let\ld\lambda
\let\Ld\Lambda
\let\vk\varkappa
\let\Om\Omega
\def\first{\mathrm{ FIRST} }
\def\follow{\mathrm{ FOLLOW} }
\let\yield\Rightarrow
\def\abstractname{}

\def\R{{\cal R}}
\def\A{{\cal A}}
\def\B{{\cal B}}
\def\C{{\cal C}}
\def\D{{\cal D}}
\let\w\omega

%классы сложности
\def\REG{{\mathsf{REG}}}
\def\CFL{{\mathsf{CFL}}}
\def\LL{{\mathrm{LL}}}
\newcounter{problem}
\newcounter{uproblem}
\newcounter{subproblem}
\def\pr{\medskip\noindent\stepcounter{problem}{\bf \theproblem .  }\setcounter{subproblem}{0} }
\def\prp{\medskip\noindent\stepcounter{problem}{\bf Задача \theproblem .  }\setcounter{subproblem}{0} }
\def\prstar{\medskip\noindent\stepcounter{problem}{\bf Задача $\bf{\theproblem^*}\negmedspace .$  }\setcounter{subproblem}{0} }
\def\prdag{\medskip\noindent\stepcounter{problem}{\bf Задача $\theproblem^\dagger$ .  }\setcounter{subproblem}{0} }
\def\upr{\medskip\noindent\stepcounter{uproblem}{\bf Упражнение \theuproblem .  }\setcounter{subproblem}{0} }
%\def\prp{\vspace{5pt}\stepcounter{problem}{\bf Задача \theproblem .  } }
%\def\prs{\vspace{5pt}\stepcounter{problem}{\bf \theproblem .*   }
\def\prsub{\medskip\noindent\stepcounter{subproblem}{\rm \thesubproblem .  } }
%прочее
\def\quotient{\backslash\negthickspace\sim}

\begin{document}
\centerline{\LARGE Задание 5}

\medskip

\begin{center}
	{\Large Теория чисел }
\end{center}

\bigskip


\section{Что стоит вспомнить}
	Не смотря на то, что вы изучали теорию чисел в рамках курса алгебры, скорее всего вам стоит освежить эти знания. Для этого вполне подойдёт Кормен. В этом разделе я напишу на что стоит обратить особое внимание, но это не означает, что остальной материал по теории чисел из Кормена читать не надо.
	\begin{itemize}
		\item Аддитивная группа $\ZZ_n$ и мультипликативная группа $\ZZ^*_{n}$
		\item Функция Эйлера $\phi(n)$ (число взаимнопростых чисел с $n$).
		\item Формула Эйлера $$a^{\phi(n)} = 1 \mod n$$
		\item Первообразный корень (генератор).
		\item Дискретный логарифм.
		\item Алгоритм Евклида, \textbf{обобщённый алгоритм Евклида}.
		\item Разрешимость уравнений вида $$ax = b \mod n$$.
		\item Китайская теорема об остатках.
		
	\end{itemize}
	
Аддитивной группой $\ZZ_n$ называется группа с элементами $\{0, 1, \ldots, n\}$ с операцией сложения по модулю $n$. Вообще говоря, можно считать, что элементы $\ZZ_n$ есть классы эквивалентности вида $[a]_n = \{ a+kn\,|\, k \in \ZZ \}$.

Мультипликативной группой $\ZZ_n^*$ называется группа элементами которой являются числа, взаимнопростые с $n$. Операция определена как умножение по модулю $n$. Мощность группы выражается через формулу Эйлера: $|\ZZ_n| = \phi(n) $.

$$ \phi(n) = n\prod_{p | n}(1 - \frac1p) $$
где $p$ -- простые числа.

Первообразным корнем или генератором группы $\ZZ_n^*$ называется такой её элемент $g$, что любой другой элемент $a \in \ZZ_n^*$ является степенью $g$. Число $x$ называется индексом числа $a$ по основанию $g$. Решение уравнения $g^x = a$ называется дискретным логарифмом элемента $a$ по основанию $g$.

Алгоритм Евклида позволяет вычислить наибольший общий делитель ${\rm gcd}(a,b)$ чисел $a$ и $b$. Пусть ${\rm gcd}(a,b) = d$. Тогда найдутся такие целые числа $x$ и $y$, что $d = ax + by$. Их позволяет вычилить обобщённый алгоритм Евклида. Также с его помощью можно решать уравнения вида $ax  = b \mod n$.




\section{Домашнее задание}

Задачи из задания №41-45.




\end{document}