\documentclass[a4paper]{article}
\usepackage[a4paper, left=5mm, right=5mm, top=5mm, bottom=5mm]{geometry}
%\geometry{paperwidth=210mm, paperheight=2000pt, left=5pt, top=5pt}
\usepackage[utf8]{inputenc}
\usepackage[english,russian]{babel}
\usepackage{indentfirst}
\usepackage{tikz} %Рисование автоматов
\usetikzlibrary{automata,positioning,arrows,trees}
\usepackage{amsmath}
\usepackage{enumerate}
\usepackage[makeroom]{cancel} % зачеркивание
\usepackage{multicol} %Несколько колонок
\usepackage{hyperref}
\usepackage{tabularx}
\usepackage{amsfonts}
\usepackage{amssymb}
\DeclareMathOperator*{\argmin}{arg\,min}
\usepackage{wasysym}
\title{Теория и реализация языков программирования.\\Задание 10: LL-анализ}
\date{задано 2013.11.13}
\author{Сергей~Володин, 272 гр.}
\newcommand{\matrixl}{\left|\left|}
\newcommand{\matrixr}{\right|\right|}
% названия автоматов  (rubtsov)
\def\A{{\cal A}}
\def\B{{\cal B}}
\def\C{{\cal C}}

%+= и -=, иначе разъезжаются...
\newcommand{\peq}{\mathrel{+}=}
\newcommand{\meq}{\mathrel{-}=}
\newcommand{\deq}{\mathrel{:}=}
\newcommand{\plpl}{\mathrel{+}+}

% пустое слово  (rubtsov)
\def\eps{\varepsilon}

% регулярные языки  (rubtsov)
\def\REG{{\mathsf{REG}}}
\def\CFL{{\mathsf{CFL}}}
\def\eqdef{\overset{\mbox{\tiny def}}{=}}
\newcommand{\niton}{\not\owns}

%FIRST & FOLLOW (rubtsov)
\def\first{\mathrm{ FIRST} }
\def\follow{\mathrm{ FOLLOW} }

% LL LR (rubtsov)
\def\LL{{\mathrm{LL}}}
\def\LR{{\mathrm{LR}}}

\newcounter{rowItemCount}
\newcounter{subRowItemCount}
\newcommand\rowItem{
    \setcounter{subRowItemCount}{0}
    \arabic{rowItemCount}.\addtocounter{rowItemCount}{1}}
\newcommand\subRowItem{
    \addtocounter{subRowItemCount}{1}
    \addtocounter{rowItemCount}{-1}
    \arabic{rowItemCount}.\arabic{subRowItemCount}.\addtocounter{rowItemCount}{1}}

\begin{document}
\maketitle
\subsection*{Упражнение 1}
Пусть $G=(N,T,P,S)$. Занумеруем правила из $P$: $P=\{P_1,...,P_n\}$.\newline
Определим синтаксический перевод $T_l=(N, T, T', R, S)$:\begin{enumerate}
\item $T'=\{1,...,n\}$
\item $R$ определяется через $P$: каждому правилу $P\ni P_i=(X,Y_1...Y_n)$ сопоставим правила в $R$: пусть $Y_{j_1}...Y_{j_l}$~--- максимальная подпоследовательность из нетерминалов из слова $Y_1...Y_n$. Тогда $X\longrightarrow Y_1...Y_n,iY_{j_1}...Y_{j_l})\in P'$.\newline
По построению нетерминалы, входящие в $\alpha\equiv Y_1...Y_n$ входят также в $\beta\equiv Y_{j_1}...Y_{j_l}$, причем с той же кратностью.
\end{enumerate}
Докажем, что слово $w\in L(G)$ переводится в левый вывод $w$. {\bf TODO}
\subsection*{Упражнение 2}
$w=a*(a+a)$. Построим правый вывод по дереву вывода (из задания):\newline
\begin{tabular}{ll}
\begin{minipage}{0.4\textwidth}
\begin{tikzpicture}
	\node {$E$} %root
		child{ node{$2$} }
		child{ node   {$T$}
			child{ node{$3$} }
		 	child{ node {$T$}
				child{ node{$4$} }				
				child[sibling angle=90,clockwise from=0]{ node{$F$}
					child[sibling angle=0,clockwise from=270]{ node {$6$} }
				}
			}
			child{ node {$F$}
				child{ node {$5$}  }
				child{ node {$E$} 
					child{ node{$1$} } 
					child{ node{$E$}
						child{ node{$2$} }
						child[sibling angle=90,clockwise from=0]{ node{$T$}
							child[sibling angle=0,clockwise from=270]{ node{$4$} }
							child[level distance = 1.75cm, sibling angle=0,clockwise from=300]{ node{$F$}
								child[sibling angle=0,clockwise from=270]{ node {$6$} }
							}
						}
					}
					child{ node{$T$}
						child{ node{$4$} }
						child{ node{$F$}
							child{ node {$6$} }
						}
					}
				}
			}
	};
\end{tikzpicture}
\end{minipage} &
\begin{minipage}{0.5\textwidth}
Чтобы получить правый вывод, обойдем дерево разбора в $G'$ следующим образом:
\begin{enumerate}
\item Выпишем самого левого потомка (по структуре правил, это всегда будет номер правила из $G$)
\item Выполним разбор оставшихся потомков справа налево.
\end{enumerate}
\end{minipage}\\
\end{tabular}
\newline
Получаем последовательность правил правого вывода $w$ в $G$: $P_r=23514624646$.\newline
Правый вывод {\em (выделен раскрываемый нетерминал)}: $\underline{E}\overset{2}{\Rightarrow}\underline{T}\overset{3}{\Rightarrow}T*\underline{F}\overset{5}{\Rightarrow}T*(\underline{E})\overset{1}{\Rightarrow}T*(E+\underline{T})\overset{4}{\Rightarrow}T*(E+\underline{F})\overset{6}{\Rightarrow}T*(\underline{E}+a)\overset{2}{\Rightarrow}T*(\underline{T}+a)\overset{4}{\Rightarrow}T*(\underline{F}+a)\overset{6}{\Rightarrow}\underline{T}*(a+a)\overset{4}{\Rightarrow}\underline{F}*(a+a)\overset{6}{\Rightarrow}a*(a+a)=w$.\newline
По определению, правый разбор~--- примененнные при правом выводе правила в обратном порядке: $(P_r)^R=64642641532$.
\subsection*{Упражнение 3}
\subsection*{Упражнение 4}
\subsection*{Упражнение 5}
\subsection*{Упражнение 6}
\newpage
\subsection*{Задача 1}
$w=((a))\in L(G)$: $\underline{E}\overset{2}{\Rightarrow}\underline{T}\overset{4}{\Rightarrow}\underline{F}\overset{5}{\Rightarrow}(\underline{E})\overset{2}{\Rightarrow}(\underline{T})\overset{4}{\Rightarrow}(\underline{F})\overset{5}{\Rightarrow}((E))\overset{2}{\Rightarrow}((\underline{T}))\overset{4}{\Rightarrow}((\underline{F}))\overset{6}{\Rightarrow}((a))$.\begin{enumerate}
\item Построим дерево вывода $w$ в $G$ и соответствующее дерево в $G'$:
	\begin{multicols}{2}
		\begin{flushleft}
			
		\begin{tikzpicture}
			\node {$E$} %root
				child
				{
					node {$T$}
					child
					{ 
						node {$F$}
						child
						{
							node {$($}
						}
						child
						{
							node {$E$}	
							child
							{
								node{$T$}
								child
								{
									node{$F$}
									child
									{
										node {$($}
									}
									child
									{
										node {$E$}
										child
										{
											node {$T$}
											child
											{
												node {$F$}
												child
												{
													node {$a$}
												}
											}
										}
									}
									child
									{
										node {$)$}
									}
								}
							}
						}
						child
						{
							node {$)$}
						}
					}
				};
		\end{tikzpicture}
		\end{flushleft}
		
		\begin{flushright}
			
		\begin{tikzpicture}
%						child[sibling angle=90,clockwise from=0]{ node{$fF$}
			\node {$E$} %root
				child {node {$2$}}
				child
				{
					node {$T$}
					child {node {$4$}}
					child
					{ 
						node {$F$}
						child {node {$5$}}
						child
						{
							node {$E$}	
							child {node {$2$}}
							child
							{
								node{$T$}
								child {node {$4$}}
								child
								{
									node{$F$}
									child {node {$5$}}
									child
									{
										node {$E$}
										child {node {$2$}}
										child
										{
											node {$T$}
											child {node {$4$}}
											child
											{
												node {$F$}
												child {node {$6$}}
											}
										}
									}
								}
							}
						}
					}
				};
		\end{tikzpicture}
		\end{flushright}	
	\end{multicols}
\item Левый разбор: обойдем второе дерево в глубину, всегда выбирая самого левого непосещенного потомка: $P_l=245245246$.
\item Правый разбор: обойдем второе дерево в глубину, как указано в решении упражнения 2: $(P_r)^R=245245246\Rightarrow P_r=642542542$.
\end{enumerate}
\subsection*{Задача 2}
\begin{enumerate}
\item $\Sigma'=\{0,1, \$\}$, $N'=\{S',S\}$. Пополненная грамматика $G'=(N',\Sigma',P',S')$, $P=\big\{\overbrace{S'\to S\$}^{(0)},\overbrace{S\to 0S}^{(1)},\overbrace{S\to 1S}^{(2)},\overbrace{S\to \varepsilon}^{(3)}\big\}$.
\item Вычислим $\first$:\newline
\begin{tabular}{rl|c|c|c|c|c|}
& & $F_i(0)$ & $F_i(1)$ & $F_i(\$)$ & $F_i(S)$ & $F_i(S')$\\ \hline
\rowItem & Определим $F_0$: & $\varnothing$ & $\varnothing$ & $\varnothing$ & $\varnothing$ & $\varnothing$\\
\subRowItem & Терминалы: $F_0(0)\eqdef \{0\}$, $F_0(1)\eqdef \{1\}$, $F_0(\$)\eqdef\{\$\}$. & $\{0\}$ & $\{1\}$ & $\{\$\}$ & $\varnothing$ & $\varnothing$\\
\subRowItem & Есть правило $S\overset{(3)}{\to}\varepsilon\Rightarrow F_0(S)\eqdef\{\varepsilon\}$ & $\{0\}$ & $\{1\}$ & $\{\$\}$ & $\{\varepsilon\}$ & $\varnothing$\\
\subRowItem & Нет правила $S'\to\varepsilon\Rightarrow F_0(S')\eqdef\varnothing$ & $\{0\}$ & $\{1\}$ & $\{\$\}$ & $\{\varepsilon\}$ & $\varnothing$\\\hline
\rowItem & Определим $F_1=F_0$ & $\{0\}$ & $\{1\}$ & $\{\$\}$ & $\{\varepsilon\}$ & $\varnothing$\\
\subRowItem & \begin{minipage}{0.45\textwidth}Рассмотрим символы правой части правила $S'\overset{(0)}{\to}S\$$.
\begin{minipage}{0.9\textwidth}
\begin{enumerate}
\item[1. $\underline{S}\$$] $F_0(\underline{S})=\{\varepsilon\}\ni\varepsilon$. $F_0(\underline{S})\setminus\{\varepsilon\}=\varnothing\rightarrow F_1(S')$.
\item[2. $S\underline{\$}$] $F_0(\underline{\$})=\{\$\}\niton\varepsilon$. $F_0(\underline{\$})\setminus\{\varepsilon\}=\{\$\}\rightarrow F_1(S')$.
\end{enumerate}
\end{minipage}
\end{minipage} & $\{0\}$ & $\{1\}$ & $\{\$\}$ & $\{\varepsilon\}$ & $\{\$\}$\\
\subRowItem & Рассмотрим правило $S\overset{(1)}{\to}\underline{0}S$. $F_0(\underline{0})=\{0\}\niton\varepsilon\Rightarrow F_1(S)\leftarrow \{0\}$ & $\{0\}$ & $\{1\}$ & $\{\$\}$ & $\{\varepsilon,0\}$ & $\{\$\}$ \\
\subRowItem & Рассмотрим правило $S\overset{(2)}{\to}\underline{1}S$. $F_0(\underline{1})=\{1\}\niton\varepsilon\Rightarrow F_1(S)\leftarrow \{1\}$ & $\{0\}$ & $\{1\}$ & $\{\$\}$ & $\{\varepsilon,0,1\}$ & $\{\$\}$ \\
\subRowItem & Рассмотрим правило $S\overset{(3)}{\to}\underline{\varepsilon}$. $|\underline{\varepsilon}|=0\Rightarrow$ не изменяем $F_1$ & $\{0\}$ & $\{1\}$ & $\{\$\}$ & $\{\varepsilon,0,1\}$ & $\{\$\}$ \\\hline


\rowItem & Определим $F_2=F_1$: & $\{0\}$ & $\{1\}$ & $\{\$\}$ & $\{\varepsilon,0,1\}$ & $\{\$\}$\\
\subRowItem & \begin{minipage}{0.45\textwidth}Рассмотрим символы правой части правила $S'\overset{(0)}{\to}S\$$.
\begin{minipage}{\textwidth}
\begin{enumerate}
\item[1. $\underline{S}\$$] $F_1(\underline{S})=\{\varepsilon,0,1\}\ni\varepsilon$. $F_1(\underline{S})\setminus\{\varepsilon\}=\{0,1\}\rightarrow F_2(S')$.
\item[2. $S\underline{\$}$] $F_1(\underline{\$})=\{\$\}\niton\varepsilon$. $F_0(\underline{\$})\setminus\{\varepsilon\}=\{\$\}\rightarrow F_2(S')$.
\end{enumerate}
\end{minipage}
\end{minipage} & $\{0\}$ & $\{1\}$ & $\{\$\}$ & $\{\varepsilon,0,1\}$ & $\{\$,0,1\}$ \\
\subRowItem & Рассмотрим правило $S\overset{(1)}{\to}\underline{0}S$. $F_1(\underline{0})=\{0\}\niton\varepsilon\Rightarrow F_2(S)\leftarrow \{0\}$ & $\{0\}$ & $\{1\}$ & $\{\$\}$ & $\{\varepsilon,0,1\}$ & $\{\$,0,1\}$\\
\subRowItem & Рассмотрим правило $S\overset{(2)}{\to}\underline{1}S$. $F_1(\underline{1})=\{1\}\niton\varepsilon\Rightarrow F_2(S)\leftarrow \{1\}$ & $\{0\}$ & $\{1\}$ & $\{\$\}$ & $\{\varepsilon,0,1\}$ & $\{\$,0,1\}$\\
\subRowItem & Рассмотрим правило $S\overset{(3)}{\to}\underline{\varepsilon}$. $|\underline{\varepsilon}|=0\Rightarrow$ не изменяем $F_2$ & $\{0\}$ & $\{1\}$ & $\{\$\}$ & $\{\varepsilon,0,1\}$ & $\{\$,0,1\}$\\\hline
\rowItem & Определим $F_3=F_2$: & $\{0\}$ & $\{1\}$ & $\{\$\}$ & $\{\varepsilon,0,1\}$ & $\{\$,0,1\}$\\
\subRowItem & \begin{minipage}{0.45\textwidth}Рассмотрим символы правой части правила $S'\overset{(0)}{\to}S\$$.
\begin{minipage}{\textwidth}
\begin{enumerate}
\item[1. $\underline{S}\$$] $F_2(\underline{S})=\{\varepsilon,0,1\}\ni\varepsilon$. $F_2(\underline{S})\setminus\{\varepsilon\}=\{0,1\}\rightarrow F_3(S')$.
\item[2. $S\underline{\$}$] $F_2(\underline{\$})=\{\$\}\niton\varepsilon$. $F_2(\underline{\$})\setminus\{\varepsilon\}=\{\$\}\rightarrow F_3(S')$.
\end{enumerate}
\end{minipage}
\end{minipage} & $\{0\}$ & $\{1\}$ & $\{\$\}$ & $\{\varepsilon,0,1\}$ & $\{\$,0,1\}$\\
\subRowItem & Рассмотрим правило $S\overset{(1)}{\to}\underline{0}S$. $F_2(\underline{0})=\{0\}\niton\varepsilon\Rightarrow F_3(S)\leftarrow \{0\}$ & $\{0\}$ & $\{1\}$ & $\{\$\}$ & $\{\varepsilon,0,1\}$ & $\{\$,0,1\}$\\
\subRowItem & Рассмотрим правило $S\overset{(2)}{\to}\underline{1}S$. $F_2(\underline{1})=\{1\}\niton\varepsilon\Rightarrow F_3(S)\leftarrow \{1\}$ & $\{0\}$ & $\{1\}$ & $\{\$\}$ & $\{\varepsilon,0,1\}$ & $\{\$,0,1\}$\\
\subRowItem & Рассмотрим правило $S\overset{(3)}{\to}\underline{\varepsilon}$. $|\underline{\varepsilon}|=0\Rightarrow$ не изменяем $F_3$ & $\{0\}$ & $\{1\}$ & $\{\$\}$ & $\{\varepsilon,0,1\}$ & $\{\$,0,1\}$\\\hline
\subRowItem & Имеем $F_3=F_2$ $\Rightarrow$ выход & $\{0\}$ & $\{1\}$ & $\{\$\}$ & $\{\varepsilon,0,1\}$ & $\{\$,0,1\}$\\\hline
\end{tabular}
\item Вычислим $\follow$:\newline
\setcounter{subRowItemCount}{0}
\setcounter{rowItemCount}{0}
\begin{tabular}{rl|c|c|}
& & $F_i(S)$ & $F_i(S')$\\ \hline
\rowItem & Определим $F_0$: & $\varnothing$ & $\varnothing$\\
\rowItem & Определим $F_1=F_0$: & $\varnothing$ & $\varnothing$\\\hline
\subRowItem & Рассмотрим правило $\underbrace{S'}_A\overset{(0)}{\to}\underbrace{\varepsilon}_\alpha\underbrace{S}_X\underbrace{\$}_{\beta}$
\begin{minipage}{0.4\textwidth}\begin{enumerate}
\item $\first(\beta)=\{\$\}\Rightarrow\first(\beta)\setminus\{\varepsilon\}=\{\$\}\to F_1(S)$.
\item $\varepsilon\notin\first(\beta)$.
\end{enumerate}\end{minipage} & $\{\$\}$ & $\varnothing$\\\hline
\subRowItem & Рассмотрим правило $\underbrace{S}_A\overset{(1)}{\to}\underbrace{0}_\alpha\underbrace{S}_X\underbrace{\varepsilon}_{\beta}$
\begin{minipage}{0.4\textwidth}\begin{enumerate}
\item $\first(\beta)=\{\varepsilon\}\Rightarrow\first(\beta)\setminus\{\varepsilon\}=\varnothing\to F_1(S)$.
\item $\varepsilon\in\first(\beta)$, поэтому $F_1(S)\leftarrow F_0(S)=\varnothing$
\end{enumerate}\end{minipage} & $\{\$\}$ & $\varnothing$\\\hline
\subRowItem & Рассмотрим правило $\underbrace{S}_A\overset{(2)}{\to}\underbrace{1}_\alpha\underbrace{S}_X\underbrace{\varepsilon}_{\beta}$
\begin{minipage}{0.4\textwidth}\begin{enumerate}
\item $\first(\beta)=\{\varepsilon\}\Rightarrow\first(\beta)\setminus\{\varepsilon\}=\varnothing\to F_1(S)$.
\item $\varepsilon\in\first(\beta)$, поэтому $F_1(S)\leftarrow F_0(S)=\varnothing$
\end{enumerate}\end{minipage} & $\{\$\}$ & $\varnothing$\\\hline
\subRowItem & Рассмотрим правило $\underbrace{S}_A\overset{(3)}{\to}\varepsilon$. Оно не имеет вид $A\to \alpha X\beta$, не изменяем $F_1$ & $\{\$\}$ & $\varnothing$\\\hline
\rowItem & Определим $F_2=F_1$: & $\{\$\}$ & $\varnothing$\\\hline
\subRowItem & Рассмотрим правило $\underbrace{S'}_A\overset{(0)}{\to}\underbrace{\varepsilon}_\alpha\underbrace{S}_X\underbrace{\$}_{\beta}$
\begin{minipage}{0.4\textwidth}\begin{enumerate}
\item $\first(\beta)=\{\$\}\Rightarrow\first(\beta)\setminus\{\varepsilon\}=\{\$\}\to F_2(S)$.
\item $\varepsilon\notin\first(\beta)$.
\end{enumerate}\end{minipage} & $\{\$\}$ & $\varnothing$\\\hline
\subRowItem & Рассмотрим правило $\underbrace{S}_A\overset{(1)}{\to}\underbrace{0}_\alpha\underbrace{S}_X\underbrace{\varepsilon}_{\beta}$
\begin{minipage}{0.4\textwidth}\begin{enumerate}
\item $\first(\beta)=\{\varepsilon\}\Rightarrow\first(\beta)\setminus\{\varepsilon\}=\varnothing\to F_2(S)$.
\item $\varepsilon\in\first(\beta)$, поэтому $F_3(S)\leftarrow F_1(S)=\{\$\}$
\end{enumerate}\end{minipage} & $\{\$\}$ & $\varnothing$\\\hline
\subRowItem & Рассмотрим правило $\underbrace{S}_A\overset{(2)}{\to}\underbrace{1}_\alpha\underbrace{S}_X\underbrace{\varepsilon}_{\beta}$
\begin{minipage}{0.4\textwidth}\begin{enumerate}
\item $\first(\beta)=\{\varepsilon\}\Rightarrow\first(\beta)\setminus\{\varepsilon\}=\varnothing\to F_2(S)$.
\item $\varepsilon\in\first(\beta)$, поэтому $F_3(S)\leftarrow F_2(S)=\{\$\}$
\end{enumerate}\end{minipage} & $\{\$\}$ & $\varnothing$\\\hline
\subRowItem & Рассмотрим правило $\underbrace{S}_A\overset{(3)}{\to}\varepsilon$. Оно не имеет вид $A\to \alpha X\beta$, не изменяем $F_1$ & $\{\$\}$ & $\varnothing$\\\hline
\subRowItem & Имеем $F_2=F_1$ $\Rightarrow$ выход & $\{\$\}$ & $\varnothing$\\\hline
\end{tabular}
\item Таблица переходов для $LL(1)$-анализатора:\newline
\begin{tabular}{cc}
\begin{minipage}{0.29\textwidth}
\begin{tabular}{|l|l|l|l|}
\hline
& $0$ & $1$ & $\$$\\\hline
$S'$ & $S'\overset{(0)}{\to}S\$$ & $S'\overset{(0)}{\to}S\$$ & $S'\overset{(0)}{\to}S\$$ \\\hline
$S$ & $S\overset{(1)}{\to}0S$ & $S\overset{(2)}{\to}1S$ & $S\overset{(3)}{\to}\varepsilon$ \\\hline
$0$ & $\varepsilon$ & {\bf Err.} & {\bf Err.}\\\hline
$1$ & {\bf Err.} & $\varepsilon$ & {\bf Err.}\\\hline
$\$$ & {\bf Err.} & {\bf Err.} & {\bf Acc.} \\\hline
\end{tabular}
\end{minipage} &
\begin{minipage}{0.7\textwidth}
\begin{enumerate}
\item $(S',0)$: правило $S'\overset{(0)}{\to}S\$$: $\first(S\$)=\first(S)\oplus\first(\$)=\{0,1,\$\}\ni 0$
\item $(S',1)$: правило $S'\overset{(0)}{\to}S\$$: $\first(S\$)=\first(S)\oplus\first(\$)=\{0,1,\$\}\ni 1$
\item $(S',\$)$: правило $S'\overset{(0)}{\to}S\$$: $\first(S\$)=\first(S)\oplus\first(\$)=\{0,1,\$\}\ni \$$
\item $(S,0)$: правило $S\overset{(1)}{\to}0S$: $\first(0S)=\{0\}\ni 0$
\item $(S,1)$: правило $S\overset{(2)}{\to}1S$: $\first(1S)=\{1\}\ni 1$
\item $(S,\$)$: правило $S\overset{(3)}{\to}\varepsilon$: $\follow(S)=\{\$\}\ni \$$
\end{enumerate}
\end{minipage}
\end{tabular}
\end{enumerate}
\subsection*{Задача 3}
\subsection*{Задача 4}
\begin{enumerate}
\item $\Sigma'\eqdef\{0,1,\$\}$, $N'\eqdef(S',S,A)$, пополненная грамматика $G'=(N',\Sigma',P',S)$.\newline $P'\eqdef\{S'\overset{(0)}{\to}S\$,S\overset{(1)}{\to}aAaa,S\overset{(2)}{\to}bAba,A\overset{(3)}{\to}b,A\overset{(4)}{\to}\varepsilon\}$
\item Найдем $\first_1$:\newline
\begin{tabular}{|r|c|c|c|c|c|c|}
\hline
$i$ & $F_i(a)$ & $F_i(b)$ & $F_i(\$)$ & $F_i(S)$ & $F_i(S')$ & $F_i(A)$\\\hline
$0$ & $\{a\}$ & $\{b\}$ & $\{\$\}$ & $\varnothing$ & $\varnothing$ & $\{\varepsilon\}$\\\hline
$1$ & $\{a\}$ & $\{b\}$ & $\{\$\}$ & $\{a,b\}$ & $\varnothing$ & $\{b,\varepsilon\}$\\\hline
$2$ & $\{a\}$ & $\{b\}$ & $\{\$\}$ & $\{a,b\}$ & $\{a,b\}$ & $\{b,\varepsilon\}$\\\hline
$3$ & $\{a\}$ & $\{b\}$ & $\{\$\}$ & $\{a,b\}$ & $\{a,b\}$ & $\{b,\varepsilon\}$\\\hline
\end{tabular}
\item Возьмем $\alpha=ba,\,w=b,\,\beta=b,\,\gamma=\varepsilon$, нетерминал $A$. Тогда $A\overset{(3)}{\to}b\equiv\beta,\,A\overset{(4)}{\to}\varepsilon\equiv\gamma,\,S'\overset{(0)}{\Rightarrow}\underline{S}\$\overset{(2)}{\Rightarrow}\underbrace{b}_wA\underbrace{ba\$}_\alpha$.\newline
Имеем $F\eqdef\first(\beta\alpha)\equiv\first(ba\$)=\{b\}$ и $G\eqdef\first(\gamma\alpha)=\first(bba\$)=\{b\}$ и $F\cap G=\{b\}\neq\varnothing$.\newline
Получаем $\exists A\exists\alpha,\beta\colon A\to\alpha,A\to\beta\in P',\,\exists w\,\exists \alpha\colon S'\Rightarrow_l^* wA\alpha,\,\first(\beta\alpha)\cap\first(\gamma\alpha)\neq\varnothing$~--- формальное отрицание утверждения Теоремы 1 из задания. Получаем, что $G'$~--- не $\LL(1)$-грамматика.
\item Найдем $\first_2$:\newline
\begin{tabular}{|r|c|c|c|c|c|c|}
\hline
$i$ & $F_i(a)$ & $F_i(b)$ & $F_i(\$)$ & $F_i(S)$ & $F_i(S')$ & $F_i(A)$\\\hline
$0$ & $\{a\}$ & $\{b\}$ & $\{\$\}$ & $\varnothing$ & $\varnothing$ & $\varnothing$\\\hline
$1$ & $\{a\}$ & $\{b\}$ & $\{\$\}$ & $\{ab,aa,bb\}$ & $\varnothing$ & $\{b,\varepsilon\}$\\\hline
$2$ & $\{a\}$ & $\{b\}$ & $\{\$\}$ & $\{ab,aa,bb\}$ & $\{ab,aa,bb\}$ & $\{b,\varepsilon\}$\\\hline
$3$ & $\{a\}$ & $\{b\}$ & $\{\$\}$ & $\{ab,aa,bb\}$ & $\{ab,aa,bb\}$ & $\{b,\varepsilon\}$\\\hline
\end{tabular}
\item Докажем, что $G'$~--- $\LL(2)$-грамматика, пользуясь Теоремой 1. Рассмотрим пары правил $X\to \beta,\,X\to\gamma$:\begin{enumerate}
\item $S\overset{(1)}{\to}\underbrace{aAaa}_\beta,\,S\overset{(2)}{\to}\underbrace{bAba}_\gamma$. Тогда $\forall\alpha\hookrightarrow$ слова из $F\eqdef\first_2(\beta\alpha)$ начинаются с $a$, слова из $G\eqdef\first_2(\gamma\alpha)$ начинаются с $b$, поэтому $F\cap G=\varnothing$
\item $A\overset{(3)}{\to}\underbrace{b}_\beta,\,A\overset{(4)}{\to}\underbrace{\varepsilon}_\gamma$. Пусть $S\Rightarrow_l^*wA\alpha$. Тогда $\alpha[1,2]\in\{aa,ba\}$~--- действительно, нетерминал $A$ может появиться только после применения $(1)$ или $(2)$. Рассмотрим эти два случая:\begin{enumerate}
\item $\alpha[1,2]=aa$. Тогда $F\eqdef\first_2(\beta\alpha)=\first_2(baa)=\{ba\}$, $G\eqdef\first_2(\gamma\alpha)=\first_2(aa)=\{aa\}$. Поэтому $F\cap G=\varnothing$
\item $\alpha[1,2]=ba$. Тогда $F\eqdef\first_2(\beta\alpha)=\first_2(bba)=\{bb\}$, $G\eqdef\first_2(\gamma\alpha)=\first_2(ba)=\{ba\}$. Поэтому $F\cap G=\varnothing$
\end{enumerate}
\end{enumerate}
\item Найдем $\follow_2$:\newline
\begin{tabular}{|r|c|c|c|}
\hline
$i$ & $F_i(S)$ & $F_i(S')$ & $F_i(A)$\\\hline
$0$ & $\varnothing$ & $\varnothing$ & $\varnothing$\\\hline
$1$ & $\{\$\}$ & $\varnothing$ & $\{aa,ba\}$\\\hline
$2$ & $\{\$\}$ & $\varnothing$ & $\{aa,ba\}$\\\hline
\end{tabular}
\end{enumerate}
\subsection*{Задача 5}
\begin{enumerate}
\item $N\eqdef\{S,A\}$, $T\eqdef\{a,b\}$, $G\eqdef\{N,T,P,S\}$. $P=\{S\to ba|A,\,A\to a|Aab|Ab\}$.
\item Удалим непосредственную левую рекурсию: $N'\eqdef\{S,A,A'\}$, $P'\eqdef\{S\to bA|A,\,A\to aA',\,A'\to abA'|bA'|\varepsilon\}$.\newline
$G'\eqdef\{N',T,P',S\}$.
\item $L(G')=L(G)$~--- так как применен алгоритм
\item $G''$~--- пополненная грамматика: $T''\eqdef\{a,b,\$\}$, $N''\eqdef\{S,S',A,A'\}$,\newline
$P''\eqdef\{S'\overset{(0)}{\to}S\$,S\overset{(1)}{\to}ba,S\overset{(2)}{\to}A,A\overset{(3)}{\to}aA',A'\overset{(4)}{\to}abA',A'\overset{(5)}{\to}bA',A'\overset{(6)}{\to}\varepsilon\}$
\item Найдем $\first$:\newline
\begin{tabular}{|r|c|c|c|c|c|c|c|}
\hline
$i$ & $F_i(a)$ & $F_i(b)$ & $F_i(\$)$ & $F_i(S)$ & $F_i(S')$ & $F_i(A)$ & $F_i(A')$\\\hline
$0$ & $\{a\}$ & $\{b\}$ & $\{\$\}$ & $\varnothing$ & $\varnothing$ & $\varnothing$ & $\{\varepsilon\}$\\\hline
$1$ & $\{a\}$ & $\{b\}$ & $\{\$\}$ & $\{b\}$ & $\varnothing$ & $\{a\}$ & $\{a,b,\varepsilon\}$\\\hline
$2$ & $\{a\}$ & $\{b\}$ & $\{\$\}$ & $\{b,a\}$ & $\{b\}$ & $\{a\}$ & $\{a,b,\varepsilon\}$\\\hline
$3$ & $\{a\}$ & $\{b\}$ & $\{\$\}$ & $\{b,a\}$ & $\{b,a\}$ & $\{a\}$ & $\{a,b,\varepsilon\}$\\\hline
$4$ & $\{a\}$ & $\{b\}$ & $\{\$\}$ & $\{b,a\}$ & $\{b,a\}$ & $\{a\}$ & $\{a,b,\varepsilon\}$\\\hline
\end{tabular}
\item Докажем, что $G'$~--- $\LL(1)$-грамматика. Рассмотрим пары правил $X\to\beta,\,X\to\gamma$:\begin{enumerate}
\item $S\overset{(1)}{\to}\underbrace{ba}_\beta$, $S\overset{(2)}{\to}\underbrace{A}_\gamma$. Тогда $\forall\alpha\hookrightarrow F\eqdef\first(\beta\alpha)=\first(b)=\{b\},\,G\eqdef\first(\gamma\alpha)=\{a\}\Rightarrow F\cap G=\varnothing$
\item $A'\overset{(4)}{\to}\underline{a}bA'$, $A'\overset{(5)}{\to}\underline{b}A'$. Аналогично $F\cap G=\varnothing$.
\item $A'\overset{(4)}{\to}\underbrace{abA'}_\beta$, $A'\overset{(6)}{\to}\underbrace{\varepsilon}_\gamma$. Пусть $S'\Rightarrow_l^*wA\alpha$. Тогда $\alpha=\$$, так правила $(4),(5),(6)$ оставляют $A'$ последним символом слова. Тогда $F\eqdef\first(\beta\alpha)=\{a\}$, $G\eqdef\first(\gamma\alpha)=\{\$\}$, поэтому $F\cap G=\varnothing$.
\item $A'\overset{(5)}{\to}\underbrace{bA'}_\beta$, $A'\overset{(6)}{\to}\underbrace{\varepsilon}_\gamma$. Пусть $S'\Rightarrow_l^*wA\alpha$. Аналогично $\alpha=\$$. Тогда $F\eqdef\first(\beta\alpha)=\{b\}$, $G\eqdef\first(\gamma\alpha)=\{\$\}$, поэтому $F\cap G=\varnothing$.
\end{enumerate}
\item Найдем $\follow$:\newline
\begin{tabular}{|r|c|c|c|c|c|c|c|}
\hline
$i$ & $F_i(S')$ & $F_i(S)$ & $F_i(A)$ & $F_i(A')$\\\hline
$0$ & $\varnothing$ & $\varnothing$ & $\varnothing$ & $\varnothing$\\\hline
$1$ & $\varnothing$ & $\{\$\}$ & $\varnothing$ & $\varnothing$\\\hline
$2$ & $\varnothing$ & $\{\$\}$ & $\{\$\}$ & $\varnothing$\\\hline
$3$ & $\varnothing$ & $\{\$\}$ & $\{\$\}$ & $\{\$\}$\\\hline
$4$ & $\varnothing$ & $\{\$\}$ & $\{\$\}$ & $\{\$\}$\\\hline
\end{tabular}
\item Построим $\LL$-анализатор:\newline
\begin{tabular}{|l|c|c|c|}
\hline
& $a$ & $b$ & $\$$\\\hline
$S'$ & $S'\overset{(0)}{\to}S\$$ & $S'\overset{(0)}{\to}S\$$ & {\bf Err.} \\\hline
$S$ & $S\overset{(2)}{\to}A$ & $S\overset{(2)}{\to}A$ & {\bf Err.} \\\hline
$A$ & $A\overset{(3)}{\to}aA'$ & {\bf Err.} & {\bf Err.} \\\hline
$A'$ & $A'\overset{(4)}{\to}abA'$ & $A'\overset{(5)}{\to}bA'$ & $A'\overset{(6)}{\to}\varepsilon$ \\\hline
$a$ & $\varepsilon$ & {\bf Err.} & {\bf Err.}\\\hline
$b$ & {\bf Err.} & $\varepsilon$ & {\bf Err.} \\\hline
$\$$ & {\bf Err.} & {\bf Err.} & {\bf Acc.} \\\hline
\end{tabular}
\end{enumerate}
\subsection*{Задача 6}
Предположим, что $L\eqdef a^*\cup a^nb^n$~--- LL-язык. Тогда $\exists k\exists G\colon L(G)=L$ и $G$~--- $\LL(k)$-грамматика. Тогда $\exists \A$~--- детерминированный МП-автомат с выходом.
\begin{enumerate}
\item Фиксируем $n$. $\varangle a^{nk}\in L$, $(a^(nk)\$,S\$,\varepsilon)\vdash^*(a^k,Y_1...Y_l\$,\cdot)$. Рассмотрим
\end{enumerate}
\end{document}