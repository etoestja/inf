\documentclass[a4paper]{article}
\usepackage[a4paper, left=5mm, right=5mm, top=10mm, bottom=20mm]{geometry}
\usepackage[utf8]{inputenc}
\usepackage[russian]{babel}
\usepackage{graphicx}
\usepackage{indentfirst}
\usepackage{amsmath}
\usepackage{enumerate}
\usepackage{amsfonts}
\usepackage{amssymb}
\title{Задание от 2013.04.05}
\author{С.~Е.~Володин, 272 гр.}
\date{}
\newcommand{\matrixl}{\left|\left|}
\newcommand{\matrixr}{\right|\right|}
\begin{document}
\maketitle
\begin{enumerate}
\item Пусть $\underline{x}=\matrixl
\begin{array}{ccc}
x_1 \\
x_2 \\
x_3
\end{array}
\matrixr$~--- количества материалов (в м$^3$). Определим
\begin{equation} \underline{c}=\matrixl
\begin{array}{c}
1000 \\
1200 \\
12000
\end{array}
\matrixr
,\ 
A=\matrixl
\begin{array}{lcr}
1 & 1 & 1 \\
2 & 1 & 3 \\
1 & 0 & 0 \\
0 & 1 & 0 \\
0 & 0 & 1 \\
\end{array}
\matrixr
,\ 
\underline{b}=\matrixl
\begin{array}{c}{1}
60 \\
100 \\
40 \\
30 \\
20
\end{array}
\matrixr.
\end{equation}
Тогда $A\underline{x}\leqslant\underline{b}$~--- система неравенств, равносильная условию (первые два неравенства~--- ограничение массы и объема, последние три~--- ограничения количеств материалов). Условие $\underline{x}\geqslant 0$ очевидно (количества $\geqslant 0$). Число $\underline{c}^Tx$~--- суммарная стоимость.\newline
Перебором найдем решение $\underline{x}=\matrixl
\begin{array}{ccc}
5 \\
30 \\
20
\end{array}
\matrixr$, $\underline{c}^Tx=281000$.\newline
Двойственная задача
$\min\underline{y}^T\underline{b}$, где $\underline{y}\geqslant 0$ и $\underline{y}^TA\geqslant\underline{c}^T$ имеет решение
$\underline{y}^T=\matrixl
\begin{array}{ccccc}
0 & 500 & 0 & 700 & 10500
\end{array}
\matrixr$, $\underline{y}^T\underline{b}=281000$. Если назвать то, что соответствует строкам $A$ ресурсами (т.е. количество м$^3$ в самолете, суммарный вес, количества товаров), то двойственная задача имеет следующую интерпретацию:\newline
Какие минимальные цены за ресурсы нужно установить, чтобы продать их все за ту же стоимость $281000$, причем, чтобы для покупателя выполнить тот же рейс было бы невыгодно (условие $\underline{y}^TA\geqslant \underline{c}^T$~--- цена за каждый материала для покупателя получается не меньше, чем <<реальная>> цена), а именно, $\underline{y}^TA=
\matrixl
\begin{array}{c}
1000 \\
1200 \\
12000
\end{array}
\matrixr
=\underline{c}.$
\newpage
\item Для сети $G(V,E)$ с пропускной способностью $c$ и пропускной способностью вершин $d$ создадим новую сеть $G'(V', E')$, такую что для каждой вершины $v$ исходной сети есть две вершины $v^1$ и $v^2$, ребрам, для которых $v$~---конечная вершина, соответствуют ребра инцидентные $v^1$, а тем, для которых $v$~--- начальная, соответствую ребра, инцидентные $v^2$. Также есть ребра между $v^1$ и $v^2$.
То есть, если $V=\{v_1,\dots,v_n\}$ и $v_1$~--- исток, а $v_n$~--- сток, то $V'=\{v^1_1,v^2_1,\dots v^1_n,v^2_n\}$,
$v_1^1$~--- исток, $v_n^2$~--- сток; если $E=\{(a_1,b_1),\dots,(a_m,b_m)\}$, то
$E'=\{(a^2_1,b^1_1),\dots,(a^2_m,b^1_m)\}\bigcup\{(v^1_1,v^2_1),\dots,(v^1_n,v^2_n)\}$.\newline
Пропускную способность определим следующим образом ($x\neq y\in G$):\newline
$c'(u,v)=
\left\{\begin{array}{lr}
d(u) & u=x^1,v=x^2 \\
0 & u=x^2,v=x^1 \\
c(x,y) & u=x^2,v=y^1 \\
0 & u=x^1,v=y^2
\end{array}
\right.
$.\newline
Пусть найден максимальный поток $f'$ в $G'(V',E')$. Определим в исходной сети
$f(u,v)=
\left\{\begin{array}{lr}
f'(u^2,v^1), & (u,v)\in E \\
f'(v^1,u^2), & (v,u)\in E \\
0 & else
\end{array}.
\right.$ Тогда он является потоком в $G(V,E)$, так как
\begin{enumerate}
\item Для $(u,v)\in E$
\begin{enumerate}
\item $f(u,v)=f'(u^2,v^1)\leqslant c'(u^2,v^1)=c(u,v)$.
\item \label{a} $f(u,v)=f'(u^2,v^1)=-f'(v^1,u^2)=-f(v,u)$.
\item $f(v,u)=f'(u^1,v^2)\leqslant c'(u^1,v^2)=0=c(v,u)$.
\item $f(v,u)=-f(u,v)$~--- см. \ref{a}.
\end{enumerate}
\item Для $(u,v), (v,u)\notin E$
\begin{enumerate}
\item $f(u,v)=0\leqslant c(u,v)$.
\item $f(u,v)=0=-0=-f(v,u)$.
\end{enumerate}
\item Для $u\in V$, кроме истока и стока:\newline
${\sum\limits_{v\in V}f(u,v)}={\sum\limits_{\substack{v\in V,\\ (u,v)\in E,\\v\neq u}}f(u,v)}+{\sum\limits_{\substack{v\in V,\\ (v,u)\in E,\\v\neq u}}f(u,v)}=
{\sum\limits_{\substack{v^1\in V',\\(u^2,v^1)\in E',\\v^1\neq u^1}}f'(u^2,v^1)}+{\sum\limits_{\substack{v^1\in V',\\(v^2,u^1)\in E',\\v^1\neq u^1}}f'(v^1,u^2)}=\newline
={\sum\limits_{\substack{v^1\in V',\\(u^2,v^1)\in E',\\v^1\neq u^1}}f'(u^2,v^1)}-{\sum\limits_{\substack{v^1\in V',\\(v^2,u^1)\in E',\\v^1\neq u^1}}f'(u^2,v^1)}=
\underbrace{\sum\limits_{v^1\in V'}f'(u^2,v^1)}_{=0,\ \text{flow\ conservation}}\underbrace{+f'(u^2,u^1)-f'(u^2,u^1)}_{=0}=0.$
\end{enumerate}
Условие задачи выполнено, так как\newline
$f(v)={\sum\limits_{\substack{u\in V,\\(u,v)\in E}}f(u,v)}={\sum\limits_{\substack{u^2\in V',\\(u^2,v^1)\in E'}}f'(u^2,v^1)}=-f'(v^2,v^1)=
f'(v^1,v^2)\leqslant c'(v^1,v^2)=g(v)$\newline
Докажем, что $|f|=|f'|$:\newline
$|f|={\sum\limits_{v\in V}f(v_1,v)}={\sum\limits_{\substack{v\in V\\(v_1,v)\in E}}f(v_1,v)}=
{\sum\limits_{\substack{v^1\in V'\\(v^2_1,v^1)\in E}}f'(v^2_1,v^1)}=-f'(v^2_1,v^1_1)=f'(v^1_1,v^2_1)={\sum\limits_{v\in V'}f'(v^1_1,v)}=|f'|$\newline
Пусть $f$~--- не максимальный. Тогда существует поток $g$, величина которого больше. Определим аналогично поток $g'$ в $G'$ (это будет поток, т.к. определение аналогичное). Его величина будет больше, чем величина $f'$, так как $|f|=|f'|$, $|g|=|g'|$~--- противоречие с тем, что $f'$~--- максимальный.\newline
То есть, найден максимальный поток в $G$, удовлетворяющий условию.\newline
Время полиномиальное, так как размеры входных данных~--- полиномы от размеров исходных данных, и алгоритм поиска максимального потока~--- полиномиальный (алгоритм Диница):
$n'=2n,m'=m+n$$\blacksquare$
\end{enumerate}
\end{document}