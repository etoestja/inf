\documentclass[a4paper]{article}
\usepackage[a4paper, left=5mm, right=5mm, top=5mm, bottom=5mm]{geometry}
\usepackage[utf8]{inputenc}
\usepackage[english,russian]{babel}
\usepackage{indentfirst}
\usepackage{tikz} %Рисование автоматов
\usetikzlibrary{automata,positioning,arrows}
\usepackage{amsmath}
\usepackage{enumerate}
\usepackage{amsfonts}
\usepackage{amssymb}
\usepackage{wasysym}
\title{Теория и реализация языков программирования.\\Задание 4: Замкнутость регулярных языков, теорема Майхилла-Нероуда и минимальные автоматы}
\date{задано 2013.09.25}
\author{Сергей~Володин, 272 гр.}
\newcommand{\matrixl}{\left|\left|}
\newcommand{\matrixr}{\right|\right|}
% названия автоматов
\def\A{{\cal A}}
\def\B{{\cal B}}
\def\C{{\cal C}}

% регулярные языки
\def\REG{{\mathsf{REG}}}

\newcommand{\niton}{\not\owns}

\begin{document}
\maketitle
\section*{Упражнение 1}
\section*{Задача 1}
Будем <<искать>> представителей классов. Сначала найден $\varepsilon\in C_1$. Если $\varepsilon\sigma\equiv\sigma\notin C(\varepsilon)\Leftrightarrow f(\sigma,\varepsilon)=0$, найден представитель нового класса. Данную процедуру повторяем для всех найденных классов $\sim n^2$ операций, для них же на каждом шаге определяем $\delta(C_i,\sigma)=C_j$, где $x_i\in C_i$~--- найден, $j\colon x_i\sigma\in C_j$. Так будут найдены все классы, потому что на каждом шаге определяются переходы для какого-то состояния ДКА. Состояний конечное число, а когда автомат будет полным, алгоритм можно считать законченным. Корректность следует из построения: $\delta(C_i,\sigma)=C_j\Leftrightarrow x_i\sigma\in C_j$~--- см. доказательство теоремы Майхилла-Нероуда.\newline
Более формально: $L\subset\Sigma^*\in\REG,\Sigma^*/\sim_L=\{C_i\}\equiv\{C_1,...,C_n\}$ ($n$ неизвестно, $C_i$ попарно различны). $f\colon \Sigma^*\times\Sigma^*\longrightarrow\{0,1\}$~--- задана, $f(x,y)=1\Leftrightarrow x\sim_Ly$.
Построим ДКА $\A\colon L(\A)=L$.\newline
$Q\overset{\mbox{\tiny def}}{=}\{C_i\}, q_0\overset{\mbox{\tiny def}}{=}C(\varepsilon)$. Докажем, что на $n$-м шаге нижеописанного алгоритма выполняется\newline
$P(n)=[\forall i\in\overline{1,n}\hookrightarrow \mbox{найдены }x_i\in C_i, \forall\sigma\in\Sigma\hookrightarrow\mbox{определены }\delta(C_i,\sigma)=C_j\Leftrightarrow C_i\sigma\in C_j]$.
\begin{enumerate}
\item ($n=1$). $\Sigma^*\ni\varepsilon$ принадлежит какому-то классу. Без ограничения общности $\varepsilon\in C_1$. Рассмотрим все $\sigma_k\in\Sigma$. Если $f(\varepsilon,\sigma_k)=1$, то $x$
\end{enumerate}
\section*{Задача 2}
\section*{Задача 3}
\section*{Задача 4}
\begin{enumerate}
\item $\Sigma=\{0,1\}$. Докажем, что $L(\A)=L$, $L=\{w\,\big|\,|w|_1=2t,t\in{\mathbb Z}\}$, ДКА $\A:$
\begin{center}
\begin{center}
\begin{tikzpicture}[shorten >=1pt,node distance=2cm,on grid,auto,initial text=]
	  \node[state, initial, accepting]  (q_0)   {$q_0$};
  	  \node[state] (q_1) [right = of q_0] {$q_1$};
  	  \path[->] 
			(q_0)	edge[bend right=40]		node	{$1$}	(q_1)
			(q_1)	edge[bend right=40]		node	{$1$}	(q_0)
			(q_0)	edge[loop below]		node	{$0$}	(q_0)
			(q_1)	edge[loop below]		node	{$0$}	(q_1);
\end{tikzpicture}
\end{center}
\end{center}
Докажем утверждение $P(n)=\big[\forall w\in\Sigma^*\colon |w|=n\hookrightarrow \big(q_0\overset{w}{\longrightarrow}q_i\Rightarrow i=|w|_1\mod 2\big)\big]$.
\begin{enumerate}
\item Докажем $P(0)$. Поскольку $|w|=0\Rightarrow w=\varepsilon$, $P(0)=\big[q_0\overset{\varepsilon}{\longrightarrow}q_i\Rightarrow i=|\varepsilon|_1\mod 2\big]$. Поскольку $\delta(q_0,\varepsilon)=q_{\underline{0}}$, и $\underline{0}=|\varepsilon|_1$, получаем $P(0)\,\blacksquare$
\item Пусть доказано $P(n)$, докажем $P(n+1)$. $P(n)=\big[\forall w\in\Sigma^*\colon |w|=n\hookrightarrow \big(q_0\overset{w}{\longrightarrow}q_i\Rightarrow i=|w|_1\mod 2\big)\big]$. Фиксируем $w\in\Sigma^*,|w|=n+1,w=w_0\sigma,|w_0|=n,|\sigma|=1$. $\A$~--- полный $\Rightarrow(q_0,w)\equiv(q_0,w_0\sigma)\vdash^*(q_i,\sigma)\vdash(q_j,\varepsilon)$. $|w_0|=n\overset{P(n)}{\Rightarrow}i=|w_0|_1\mod 2$. $i\in\{0,1\}\,,\sigma\in\{0,1\}\Rightarrow$ рассмотрим четыре случая:
\begin{enumerate}[a.]
\item ($i=0,\sigma=0$). $(q_0,w_00)\vdash^*(q_0,0)\vdash(q_0,\varepsilon)\Rightarrow q_0\overset{w}{\longrightarrow}q_0\Rightarrow j=0$. $|w|_1\mod 2=|w_0|_1\mod 2+|0|_1\mod 2=0+0=0\Rightarrow 0=j=|w|_1\mod 2=0$.
\item ($i=0,\sigma=1$). $(q_0,w_01)\vdash^*(q_0,1)\vdash(q_1,\varepsilon)\Rightarrow q_0\overset{w}{\longrightarrow}q_1\Rightarrow j=1$. $|w|_1\mod 2=|w_0|_1\mod 2+|1|_1\mod 2=0+1=1\Rightarrow 1=j=|w|_1\mod 2=1$.
\item ($i=1,\sigma=0$). $(q_0,w_00)\vdash^*(q_1,0)\vdash(q_1,\varepsilon)\Rightarrow q_0\overset{w}{\longrightarrow}q_1\Rightarrow j=1$. $|w|_1\mod 2=|w_0|_1\mod 2+|0|_1\mod 2=1+0=1\Rightarrow 1=j=|w|_1\mod 2=1$.
\item ($i=1,\sigma=1$). $(q_0,w_01)\vdash^*(q_1,1)\vdash(q_0,\varepsilon)\Rightarrow q_0\overset{w}{\longrightarrow}q_0\Rightarrow j=0$. $|w|_1\mod 2=|w_0|_1\mod 2+|1|_1\mod 2=(1+1)\mod 2=0\Rightarrow 0=j=|w|_1\mod 2=0$.
\end{enumerate}
\end{enumerate}
Таким образом, $\forall n\in{\mathbb N}\cup\{0\}\hookrightarrow P(n)\Rightarrow \forall n\in{\mathbb N}\cup\{0\}\hookrightarrow \big[\forall w\in\Sigma^*\colon |w|=n\hookrightarrow \big(q_0\overset{w}{\longrightarrow}q_i\Rightarrow i=|w|_1\mod 2\big)\big]\Rightarrow\newline\forall w\in\Sigma^*\hookrightarrow q_0\overset{w}{\longrightarrow}q_{|w|_1\mod 2}$.
Пусть $w\in L\Leftrightarrow |w|_1\mod 2=0\Leftrightarrow q_0\overset{w}{\longrightarrow}q_0\Leftrightarrow w\in L(\A)\,\blacksquare$
\end{enumerate}
\section*{Задача 5}
\begin{tabular}{l l}
Исходный автомат $\A$: & 
\begin{minipage}{0.4\textwidth}\begin{center}
% копипаста из задания
\begin{tikzpicture}[shorten >=1pt,node distance=2cm,on grid,auto,initial text=]
  %\draw[help lines] (0,0) grid (3,2);
  \node[state ,initial] (q_0) 					 {$q_0$};
  \node[state, accepting]		    (q_1) [ right = of q_0 ] {$q_1$};
  \node[state]          (q_2) [ right = of q_1] {$q_2$};
  \node[state, accepting]           (q_3) [ right = of q_2] {$q_3$};

  \path[->] 
		(q_0)	edge		node	{$a$}	(q_1)
				edge [bend left] node {$b$}	(q_3)
		(q_1)
				edge		node	{$b$}	(q_2)
		(q_2)
				edge		node	{$b$}	(q_3)
				edge [bend left] node {$a$}	(q_1);
\end{tikzpicture}
\end{center}\end{minipage}\\
\begin{minipage}{0.4\textwidth}
Пополним автомат $\A$ до $\A'$ и удалим недостижимые из $q_0$ состояния: добавим $q_4\in Q',\,q_4\notin F'$, в него направим недостающие переходы:
\end{minipage} &
\begin{minipage}{0.4\textwidth}
\begin{center}
\begin{tikzpicture}[shorten >=1pt,node distance=2cm,on grid,auto,initial text=]
  %\draw[help lines] (0,0) grid (3,2);
  \node[state ,initial] (q_0) 					 {$q_0$};
  \node[state, accepting]		    (q_1) [ right = of q_0 ] {$q_1$};
  \node[state]          (q_2) [ right = of q_1] {$q_2$};
  \node[state, accepting]           (q_3) [ right = of q_2] {$q_3$};
  \node[state]           (q_4) [ below = of q_2] {$q_4$};

  \path[->] 
		(q_0)	edge		node	{$a$}	(q_1)
				edge [bend left] node {$b$}	(q_3)
		(q_1)
				edge		node	{$b$}	(q_2)
				edge [bend right]		node	{$a$}	(q_4)
		(q_2)
				edge		node	{$b$}	(q_3)
				edge [bend left] node {$a$}	(q_1)
		(q_3)
				edge [bend left]		node	{$b$}	(q_4)
				edge [bend right]		node	{$a$}	(q_4)
		(q_4)
				edge [loop below]		node	{$a,b$}	(q_4);
\end{tikzpicture}
\end{center}
\end{minipage}\\
\end{tabular}\newline
$L(\A')=L(\A)$, так как $x\in L(\A)\Rightarrow x\in L(\A')$, потому что $Q\subset Q',\,F=F',\delta\subset\delta'$. $x\notin L(\A)\Rightarrow$ либо $q_0\overset{x}{\longrightarrow}q\notin F$, но тогда $q_0\overset{x}{\longrightarrow}q\notin F'\Rightarrow x\notin L(\A')$, либо $\delta(q_0,x)=\varnothing$, тогда $\delta'(q_0,x)=q_4$, потому что был выполнен переход в $q_4$, которого не было в $\A$ (по построению, добавлены переходы только в $q_4$), и при обработке последующих символов $\A'$ остается в $q_4$.
\\[5pt]
\begin{tabular}{l l}
\begin{minipage}{0.4\textwidth}
Построим $A''\colon L(\A'')=\overline{L(\A')}\equiv\overline{L(\A)}$ по полному автомату $\A'$, определив $F''\overset{\mbox{\tiny def}}{=}Q'\setminus F'$:
\end{minipage} &
\begin{minipage}{0.4\textwidth}
\begin{center}
\begin{tikzpicture}[shorten >=1pt,node distance=2cm,on grid,auto,initial text=]
  %\draw[help lines] (0,0) grid (3,2);
  \node[state, initial, accepting] (q_0) 					 {$q_0$};
  \node[state]		    (q_1) [ right = of q_0 ] {$q_1$};
  \node[state, accepting]          (q_2) [ right = of q_1] {$q_2$};
  \node[state]           (q_3) [ right = of q_2] {$q_3$};
  \node[state, accepting]           (q_4) [ below = of q_2] {$q_4$};

  \path[->] 
		(q_0)	edge		node	{$a$}	(q_1)
				edge [bend left] node {$b$}	(q_3)
		(q_1)
				edge		node	{$b$}	(q_2)
				edge [bend right]		node	{$a$}	(q_4)
		(q_2)
				edge		node	{$b$}	(q_3)
				edge [bend left] node {$a$}	(q_1)
		(q_3)
				edge [bend left]		node	{$b$}	(q_4)
				edge [bend right]		node	{$a$}	(q_4)
		(q_4)
				edge [loop below]		node	{$a,b$}	(q_4);
\end{tikzpicture}
\end{center}
\end{minipage}\\
\end{tabular}
\newline
Далее построим по $\A''$ минимальный $\A'''$ по алгоритму:
\newline
\begin{tabular}{c c}
\begin{minipage}{0.4\textwidth}
\begin{enumerate}[1.]
\item \begin{tikzpicture}[baseline=-0.5ex,node distance=10mm,on grid,auto]
\node (d_b) {$\big\{$};
\node (q_0) [node distance=3mm, right of=d_b] {$0$};
\node (q_2) [right of=q_0] {$2$};
\node (q_4) [right of=q_2] {$4$};
\node (d_41) [node distance=5mm, right of=q_4] {$\big|$};
\node (q_1) [right of=q_4] {$1$};
\node (q_3) [right of=q_1] {$3$};
\node (d_f) [node distance=3mm,right of=q_3] {$\big\}$};
\draw[->, bend left] (q_0) to node {$a$} (q_1);
\draw[->, bend left] (q_2) to node {$a$} (q_1);
\draw[->, loop below] (q_4) to node {$a$} (q_4);
\draw[->, bend left] (q_1) to node {$a$} (q_4);
\draw[->, bend right=50] (q_3) to node {$a$} (q_4);
\end{tikzpicture}
\item \begin{tikzpicture}[baseline=-0.5ex,node distance=10mm,on grid,auto]
\node (d_b) {$\big\{$};
\node (q_0) [node distance=3mm, right of=d_b] {$0$};
\node (q_2) [right of=q_0] {$2$};
\node (d_02) [node distance=5mm, right of=q_2] {$\big|$};
\node (q_4) [right of=q_2] {$4$};
\node (d_41) [node distance=5mm, right of=q_4] {$\big|$};
\node (q_1) [right of=q_4] {$1$};
\node (d_13) [node distance=5mm, right of=q_1] {$\big|$};
\node (q_3) [right of=q_1] {$3$};
\node (d_f) [node distance=3mm,right of=q_3] {$\big\}$};
\draw[->, bend right] (q_0) to node {$b$} (q_3);
\draw[->, bend right] (q_2) to node [swap] {$b$} (q_3);
\draw[->, loop below] (q_4) to node {$b$} (q_4);
\draw[->, bend right] (q_1) to node [swap] {$b$} (q_2);
\draw[->, bend right=50] (q_3) to node [swap] {$b$} (q_4);
\end{tikzpicture}~--- OK
\item \begin{tikzpicture}[baseline=-0.5ex,node distance=10mm,on grid,auto]
\node (d_b) {$\big\{$};
\node (q_0) [node distance=3mm, right of=d_b] {$0$};
\node (q_2) [right of=q_0] {$2$};
\node (d_02) [node distance=5mm, right of=q_2] {$\big|$};
\node (q_4) [right of=q_2] {$4$};
\node (d_41) [node distance=5mm, right of=q_4] {$\big|$};
\node (q_1) [right of=q_4] {$1$};
\node (d_13) [node distance=5mm, right of=q_1] {$\big|$};
\node (q_3) [right of=q_1] {$3$};
\node (d_f) [node distance=3mm,right of=q_3] {$\big\}$};
\draw[->, bend left] (q_0) to node {$a$} (q_1);
\draw[->, bend left] (q_2) to node {$a$} (q_1);
\draw[->, loop below] (q_4) to node {$a$} (q_4);
\draw[->, bend left] (q_1) to node {$a$} (q_4);
\draw[->, bend right=50] (q_3) to node {$a$} (q_4);
\end{tikzpicture}~--- OK
\end{enumerate}
\end{minipage} &
\begin{minipage}{0.3\textwidth}
\begin{center}
\begin{tikzpicture}[shorten >=1pt,node distance=2cm,on grid,auto,initial text=]
  %\draw[help lines] (0,0) grid (3,2);
  \node[state, initial, accepting] (q_0) 					 {$q_0$};
  \node[state]		    (q_1) [ right = of q_0 ] {$q_1$};
  \node[state, accepting] (q_4) [ right = of q_1] {$q_4$};
  \node[state]           (q_3) [ right = of q_4] {$q_3$};

  \path[->] 
		(q_0)	edge [bend right]		node	{$a$}	(q_1)
				edge [bend left] node {$b$}	(q_3)
		(q_1)
				edge [bend right]		node	{$b$}	(q_0)
				edge 		node	{$a$}	(q_4)
		(q_3)
				edge [bend left]		node	{$b$}	(q_4)
				edge [bend right]		node	{$a$}	(q_4)
		(q_4)
				edge [loop below]		node	{$a,b$}	(q_4);
\end{tikzpicture}
\end{center}
\end{minipage}\\
\end{tabular}
\section*{Задача 6}
\end{document}
