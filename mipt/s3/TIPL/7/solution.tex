\documentclass[a4paper]{article}
\usepackage[a4paper, left=5mm, right=5mm, top=5mm, bottom=5mm]{geometry}
%\geometry{paperwidth=210mm, paperheight=2000pt, left=5pt, top=5pt}
\usepackage[utf8]{inputenc}
\usepackage[english,russian]{babel}
\usepackage{indentfirst}
\usepackage{tikz} %Рисование автоматов
\usetikzlibrary{automata,positioning,arrows}
\usepackage{amsmath}
\usepackage{enumerate}
\usepackage{hyperref}
\usepackage{amsfonts}
\usepackage{amssymb}
\DeclareMathOperator*{\argmin}{arg\,min}
\usepackage{wasysym}
\title{Теория и реализация языков программирования.\\Задание 7: контекстно-свободные языки и магазинные автоматы}
\date{задано 2013.10.16}
\author{Сергей~Володин, 272 гр.}
\newcommand{\matrixl}{\left|\left|}
\newcommand{\matrixr}{\right|\right|}
% названия автоматов
\def\A{{\cal A}}
\def\B{{\cal B}}
\def\C{{\cal C}}

%+= и -=, иначе разъезжаются...
\newcommand{\peq}{\mathrel{+}=}
\newcommand{\meq}{\mathrel{-}=}
\newcommand{\deq}{\mathrel{:}=}
\newcommand{\plpl}{\mathrel{+}+}

% регулярные языки
\def\REG{{\mathsf{REG}}}
\def\eqdef{\overset{\mbox{\tiny def}}{=}}
\newcommand{\niton}{\not\owns}

\begin{document}
\maketitle
\section*{Упражнение 1}
\section*{Упражнение 2}
\section*{Упражнение 3}
\begin{enumerate}[1.]
\item Грамматика $\Gamma=(\{S\},\Sigma_n\cup\overline{\Sigma}_n,P,S)$. $P=\big\{S\longrightarrow \sigma_i\overline{\sigma}_i|\sigma_iS\overline{\sigma}_i|SS\big\}$. $D_n=L(\Gamma)$.
\item Исходное утверждение: $\forall w\left(\underbrace{w\in D_n}_{A}\Rightarrow\underbrace{\forall i\leqslant n\,\forall k\leqslant |w|\hookrightarrow ||w[1,k]||_i\geqslant 0,\,||w||_i=0}_{B}\right)$
\item Отрицание обратного утверждения: $\exists w\colon \left(B\,\wedge\,\urcorner A\right)$. Пусть $w=\varepsilon$.\begin{enumerate}[a.]
\item Тогда $k\leqslant |w|\Rightarrow k=0$, поэтому $\forall i\leqslant n\hookrightarrow ||w[1,k]||_i\equiv|\varepsilon|_{\sigma_i}-|\varepsilon|_{\overline{\sigma_i}}=0$ и $\forall i\leqslant n\hookrightarrow||w||_i=0$. Получаем $B$.
\item Но $w=\varepsilon$ не порождается грамматикой $\Gamma$: первые два правила добавляют нетерминалов, поэтому не могут быть применены, и применение третьего правила не уменьшает количества нетерминалов. Получаем $\urcorner A\,\blacksquare$
\end{enumerate}
\end{enumerate}
\section*{Задача 1}
\section*{Задача 2}
\section*{Задача 3}
\end{document}
