\documentclass[a4paper]{article}
\usepackage[a4paper, left=5mm, right=5mm, top=5mm, bottom=5mm]{geometry}
%\geometry{paperwidth=210mm, paperheight=2000pt, left=5pt, top=5pt}
\usepackage[utf8]{inputenc}
\usepackage[english,russian]{babel}
\usepackage{indentfirst}
\usepackage{tikz} %Рисование автоматов
\usetikzlibrary{automata,positioning,arrows}
\usepackage{amsmath}
\usepackage{enumerate}
\usepackage{hyperref}
\usepackage{amsfonts}
\usepackage{amssymb}
\DeclareMathOperator*{\argmin}{arg\,min}
\usepackage{wasysym}
\title{Теория и реализация языков программирования.\\Задание 7: контекстно-свободные языки и магазинные автоматы}
\date{задано 2013.10.16}
\author{Сергей~Володин, 272 гр.}
\newcommand{\matrixl}{\left|\left|}
\newcommand{\matrixr}{\right|\right|}
% названия автоматов
\def\A{{\cal A}}
\def\B{{\cal B}}
\def\C{{\cal C}}

%+= и -=, иначе разъезжаются...
\newcommand{\peq}{\mathrel{+}=}
\newcommand{\meq}{\mathrel{-}=}
\newcommand{\deq}{\mathrel{:}=}
\newcommand{\plpl}{\mathrel{+}+}

% пустое слово
\def\eps{\varepsilon}

% регулярные языки
\def\REG{{\mathsf{REG}}}
\def\eqdef{\overset{\mbox{\tiny def}}{=}}
\newcommand{\niton}{\not\owns}

\begin{document}
\maketitle
\section*{Упражнение 1}
\section*{Упражнение 2}
\section*{Упражнение 3}
\begin{enumerate}[1.]
\item Грамматика $\Gamma=(\{S\},\Sigma_n\cup\overline{\Sigma}_n,P,S)$. $P=\big\{S\longrightarrow \sigma_i\overline{\sigma}_i|\sigma_iS\overline{\sigma}_i|SS\big\}$. $D_n=L(\Gamma)$.
\item Исходное утверждение: $\forall w\left(\underbrace{w\in D_n}_{A}\Rightarrow\underbrace{\forall i\leqslant n\,\forall k\leqslant |w|\hookrightarrow ||w[1,k]||_i\geqslant 0,\,||w||_i=0}_{B}\right)$
\item Отрицание обратного утверждения: $\exists w\colon \left(B\,\wedge\,\urcorner A\right)$. Пусть $w=\varepsilon$.\begin{enumerate}[a.]
\item Тогда $k\leqslant |w|\Rightarrow k=0$, поэтому $\forall i\leqslant n\hookrightarrow ||w[1,k]||_i\equiv|\varepsilon|_{\sigma_i}-|\varepsilon|_{\overline{\sigma_i}}=0$ и $\forall i\leqslant n\hookrightarrow||w||_i=0$. Получаем $B$.
\item Но $w=\varepsilon$ не порождается грамматикой $\Gamma$: первые два правила добавляют нетерминалов, поэтому не могут быть применены, и применение третьего правила не уменьшает количества нетерминалов. Получаем $\urcorner A\,\blacksquare$
\end{enumerate}
\end{enumerate}
\section*{Задача 1}
Определим МП-автомат $\A=(\Sigma,\Gamma,Q,q_0,Z_0,\delta,F)$, допускающий по пустому стеку.\newline
\begin{tabular}{cc}
\begin{minipage}{0.46\textwidth}
\begin{enumerate}
\item $n\eqdef2$
\item $\Sigma_n\eqdef\{[_1,...,[_n\}\equiv\{[_1,[_2\},\,\overline{\Sigma}_n\eqdef\{]_1,...,]_n\}\equiv\{]_1,]_2\}$.
\item $\Sigma\eqdef\Sigma_n\cup\overline{\Sigma}_n\equiv\{[_1,]_1,[_2,]_2\}$
\item $\Gamma\eqdef\{Z_0\}\cup\overline{1,n}\equiv\{Z_0,1,2\}$.
\item $Q\eqdef\{q_0,q_1\}$
\item $\delta$ изображена справа
\item $F\eqdef\varnothing$ ($N$-автомат)
\end{enumerate}
\end{minipage}
&
\begin{minipage}{0.46\textwidth}
\begin{tikzpicture}[shorten >=1pt,node distance=2cm,on grid,auto,every node/.style={text centered},initial text=]
	\node [state,initial] (q_0)	{$q_0$};
	\node [state] (q_1) [right = 4cm of q_0 ] {$q_1$};
	\path[->]
		(q_0) edge [out=30,in=150,loop] node[swap] {$[_1,Z_0/1Z_0$} (q_0)
			  edge [in=210,out=330,loop] node {$[_2,Z_0/2Z_0$} (q_0)
			  edge [bend left] node {$]_1,1/\varepsilon$} (q_1)
			  edge [bend right] node [swap] {$]_2,2/\varepsilon$} (q_1)
		(q_1) edge [out=30,in=150,loop] node [swap] {$]_1,1/\varepsilon$} (q_0)
			  edge [in=210,out=330,loop] node {$]_2,2/\varepsilon$} (q_1)
			  edge [in=1,out=179] node [swap] {$[_1,1/\varepsilon$} (q_0)
			  edge [in=-1,out=181] node {$]_2,2/\varepsilon$} (q_0)
			  edge [loop right] node {$\varepsilon,Z_0/\varepsilon$} (q_1);
\end{tikzpicture}
\end{minipage}
\end{tabular}
\section*{Задача 2}
\section*{Задача 3}
\end{document}
