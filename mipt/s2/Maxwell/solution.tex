\documentclass[a4paper]{article}
\usepackage[a4paper, left=5mm, right=5mm, top=10mm, bottom=20mm]{geometry}
\usepackage[utf8]{inputenc}
\usepackage[english,russian]{babel}
\usepackage{graphicx}
\usepackage{indentfirst}
\usepackage{amsmath}
\usepackage{enumerate}
\usepackage{amsfonts}
\usepackage{amssymb}
\usepackage{fixltx2e}
\title{Максвелловское распределение и {$\frac{nv}{4}$}}
\author{С.~Е.~Володин, 272 гр.}
\date{}
\newcommand{\matrixl}{\left|\left|}
\newcommand{\matrixr}{\right|\right|}
\begin{document}
\maketitle
Рассмотрим газ, вытекающий через отверстие площади $S$. Количество {\em вылетевших} молекул $d\Phi$ в единицу времени, имеющих скорость от $v$ до $v+dv$ записывается как
$$
d\Phi=S\frac{v}{4}dn=S\frac{v}{4}4\pi v^2 \left(\frac{m}{2\pi k T}\right)^\frac{3}{2}e^{-\frac{mv^2}{2 k T}}dv.
$$
Обозначим $g(v)=S\frac{v}{4}dn$. Тогда $d\Phi=g(v)dv.$\newline
Найдем среднюю энергию в потоке:
$$
\langle\frac{mv^2}{2}\rangle=\frac{\int\limits_{0}^{+\infty} \frac{mv^2}{2}g(v)dv}{\int\limits_{0}^{+\infty} g(v)dv}=2kT,
$$
Отсюда среднеквадратичная скорость $v^{\text{скв}}=\sqrt{\frac{4kT}{m}}$.\newline
Найдем среднюю скорость в потоке:
$$
\langle v\rangle=\frac{\int\limits_{0}^{+\infty} vg(v)dv}{\int\limits_{0}^{+\infty} g(v)dv}=\frac{3}{2}\sqrt{\frac{\pi k T}{2 m}}.
$$
Найдем отношение
$$
\frac{\langle v\rangle}{v^{\text{скв}}}=\frac{3}{4}\sqrt{\frac{\pi}{2}}.
$$
Для Максвелловского распределения должно выполняться
$$
\frac{\langle v\rangle}{v^{\text{скв}}}=\frac{\sqrt{\frac{8kT}{\pi m}}}{\sqrt{\frac{3kT}{m}}}=2\sqrt{\frac{2}{3 \pi}}.
$$
Таким образом, распределение молекул по скоростям после прохождения через отверстие (или через пористую перегородку) \textbf{не является максвелловским}.
\end{document}
