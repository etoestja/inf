\documentclass[a4paper]{article}
\usepackage[a4paper, left=5mm, right=5mm, top=5mm, bottom=5mm]{geometry}
\usepackage[utf8]{inputenc}
\usepackage[english,russian]{babel}
\usepackage{graphicx}
\usepackage{indentfirst}
\usepackage{tikz} %Рисование автоматов
\usetikzlibrary{automata,positioning}
\usepackage{amsmath}
\usepackage{floatflt}
\usepackage{enumerate}
\usepackage{amsfonts}
\usepackage{amssymb}
\newcommand{\matrixr}{\right|\right|}
\newcommand{\matrixl}{\left|\left|}
\def\eqdef{\overset{\mbox{\tiny def}}{=}}
\title{Методы оптимизации.\\Задание 1: Субградиентный спуск}
\date{задано 2016.02.09}
\author{Сергей~Володин, 374 гр.}

\begin{document}
\maketitle
\section*{Задача 1}
Делаем проекцию на итерациях с номерами из $K$. Пусть $z_{k+1}\eqdef x_k-\alpha_kg^k, x_{k+1}\eqdef\begin{cases}
\pi_Q(z_{k+1}), k\in K\\
z_{k+1}, else
\end{cases}$.

Заметим, что $||x_k-x^*||\leqslant||z_k-x^*||$, $x^*\in Q$. В одном случае неравенство очевидно ($z_k\equiv x_k$), в другом $||x_k-x^*||\equiv||\pi_Q(z_k)-x^*||\leqslant ||z_k-x^*||$
Рассмотрим последовательность неравенств $$\begin{cases}
||z_{k+1}-x^*||_2^2\leqslant||z_k-x^*||_2^2-2\alpha_k(f(z_k)-f^*)+\alpha_k^2||g^k||_2^2, k\in\overline{0,N} \\
\end{cases}$$
Эти неравенства верны как базовые неравенства для метода субградиентного спуска.
Получим аналогичные неравенства для $x_{k+1}$ (для $k\in\{0,1,2,...\}$). При $k=0$ это очевидно ($x_0\equiv z_0$, $||x_1-x^*||\leqslant||z_1-x^*||$):

$$||x_1-x^*||_2^2\leqslant||x_0-x^*||_2^2-2\alpha_0(f(x_0)-f^*)+\alpha_k^2||g^k||^2$$. Пусть верно для $k$
\section*{Задача 2}
Ответ: да, верно, да, может. Приведем пример $f\colon \mathbb{R}^n\to\mathbb{R}$~--- выпуклая, $x_0\in\mathbb{R}^n, a\in\partial f(x_0)$, $x_0$~--- не точка минимума $f$, $-a$~--- не направление убывания $f$.

$f(\matrixl
\begin{array}{c}
x_1\\
x_2
\end{array}
\matrixr)\eqdef |x_1|+|x_2|\colon \mathbb{R}^2\to\mathbb{R}$. Точка $x_0\eqdef \matrixl
\begin{array}{cc}
1 & 0
\end{array}
\matrixr^T$ Тогда\begin{enumerate}
\item $f$~--- выпуклая: пусть $t_1,t_2\in\mathbb{R}_+$, $x,y\in\mathbb{R}^2$. $f(t_1x+t_2y)=|t_1x_1+t_2y_1|+|t_1x_2+t_2y_2|\leqslant t_1|x_1|+t_2|y_1|+t_1|x_2|+t_2|y_2|=t_1(|x_1|+|x_2|)+t_2(|y_1|+|y_2|)=t_1f(x)+t_2f(y)$. Возьмем $t_1\in[0,1]$, $t_2=1-t_1$, получим определение выпуклой функции.
\item Пусть $a=\matrixl
\begin{array}{cc}
1 & 1
\end{array}
\matrixr^T$. Докажем, что $a\in\partial f(x_0)$. Фиксируем $x\in\mathbb{R}^2$. $f(x)-f(x_0)=|x_1|+|x_2|-1=1\cdot(|x_1|-1)+1\cdot(|x_2|)\equiv(a, x-x_0)$. То есть, верно:
$$\forall x\in\mathbb{R}^2\hookrightarrow f(x)-f(x_0)\geqslant (a,x-x_0)$$
То есть, $a$~--- субградиент.
\item $-a$~--- не направление убывания в $x_0$. Пусть $t\in(0,1)$. Рассмотрим $f(x_0-ta)=|1-t|+|-t|=1-t+t=1$. Получаем $\forall t\in(0,1)\hookrightarrow f(x_0-ta)=f(x_0)$. Получаем, $$\forall t_0>0\,\exists t\eqdef\min\{1/2,t_0/2\}<t_0\colon f(x_0-ta)\geqslant f(x_0)$$
Это отрицания определения направления убывания.
\item $x_0$~--- не точка минимума $f$: $f(x_0)=|1|+|0|=1$, $f(0)=0<1=f(x_0)$.
\end{enumerate}
\end{document}
