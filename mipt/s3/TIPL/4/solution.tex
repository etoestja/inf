\documentclass[a4paper]{article}
\usepackage[a4paper, left=5mm, right=5mm, top=5mm, bottom=5mm]{geometry}
\usepackage[utf8]{inputenc}
\usepackage[english,russian]{babel}
\usepackage{indentfirst}
\usepackage{tikz} %Рисование автоматов
\usetikzlibrary{automata,positioning}
\usepackage{amsmath}
\usepackage{enumerate}
\usepackage{amsfonts}
\usepackage{amssymb}
\usepackage{wasysym}
\title{Теория и реализация языков программирования.\\Задание 4: Замкнутость регулярных языков, теорема Майхилла-Нероуда и минимальные автоматы}
\date{задано 2013.09.25}
\author{Сергей~Володин, 272 гр.}
\newcommand{\matrixl}{\left|\left|}
\newcommand{\matrixr}{\right|\right|}
% названия автоматов
\def\A{{\cal A}}
\def\B{{\cal B}}
\def\C{{\cal C}}

% регулярные языки
\def\REG{{\mathsf{REG}}}

\newcommand{\niton}{\not\owns}

\begin{document}
\maketitle
\section*{Упражнение 1}
\section*{Задача 1}
\section*{Задача 2}
\section*{Задача 3}
\section*{Задача 4}
\begin{enumerate}
\item $\Sigma=\{0,1\}$. Докажем, что $L(\A)=L$, $L=\{w\,\big|\,|w|_1=2t,t\in{\mathbb Z}\}$, ДКА $\A:$
\begin{center}
\begin{center}
\begin{tikzpicture}[shorten >=1pt,node distance=2cm,on grid,auto,initial text=]
	  \node[state, initial, accepting]  (q_0)   {$q_0$};
  	  \node[state] (q_1) [right = of q_0] {$q_1$};
  	  \path[->] 
			(q_0)	edge[bend right=40]		node	{$1$}	(q_1)
			(q_1)	edge[bend right=40]		node	{$1$}	(q_0)
			(q_0)	edge[loop below]		node	{$0$}	(q_0)
			(q_1)	edge[loop below]		node	{$0$}	(q_1);
\end{tikzpicture}
\end{center}
\end{center}
Докажем утверждение $P(n)=\big[\forall w\in\Sigma^*\colon |w|=n\hookrightarrow \big(q_0\overset{w}{\longrightarrow}q_i\Rightarrow i=|w|_1\mod 2\big)\big]$.
\begin{enumerate}
\item Докажем $P(0)$. Поскольку $|w|=0\Rightarrow w=\varepsilon$, $P(0)=\big[q_0\overset{\varepsilon}{\longrightarrow}q_i\Rightarrow i=|\varepsilon|_1\mod 2\big]$. Поскольку $\delta(q_0,\varepsilon)=q_{\underline{0}}$, и $\underline{0}=|\varepsilon|_1$, получаем $P(0)\,\blacksquare$
\item Пусть доказано $P(n)$, докажем $P(n+1)$. $P(n)=\big[\forall w\in\Sigma^*\colon |w|=n\hookrightarrow \big(q_0\overset{w}{\longrightarrow}q_i\Rightarrow i=|w|_1\mod 2\big)\big]$. Фиксируем $w\in\Sigma^*,|w|=n+1,w=w_0\sigma,|w_0|=n,|\sigma|=1$. $\A$~--- полный $\Rightarrow(q_0,w)\equiv(q_0,w_0\sigma)\vdash^*(q_i,\sigma)\vdash(q_j,\varepsilon)$. $|w_0|=n\overset{P(n)}{\Rightarrow}i=|w_0|_1\mod 2$. $i\in\{0,1\}\,,\sigma\in\{0,1\}\Rightarrow$ рассмотрим четыре случая:
\begin{enumerate}[a.]
\item ($i=0,\sigma=0$). $(q_0,w_00)\vdash^*(q_0,0)\vdash(q_0,\varepsilon)\Rightarrow q_0\overset{w}{\longrightarrow}q_0\Rightarrow j=0$. $|w|_1\mod 2=|w_0|_1\mod 2+|0|_1\mod 2=0+0=0\Rightarrow 0=j=|w|_1\mod 2=0$.
\item ($i=0,\sigma=1$). $(q_0,w_01)\vdash^*(q_0,1)\vdash(q_1,\varepsilon)\Rightarrow q_0\overset{w}{\longrightarrow}q_1\Rightarrow j=1$. $|w|_1\mod 2=|w_0|_1\mod 2+|1|_1\mod 2=0+1=1\Rightarrow 1=j=|w|_1\mod 2=1$.
\item ($i=1,\sigma=0$). $(q_0,w_00)\vdash^*(q_1,0)\vdash(q_1,\varepsilon)\Rightarrow q_0\overset{w}{\longrightarrow}q_1\Rightarrow j=1$. $|w|_1\mod 2=|w_0|_1\mod 2+|0|_1\mod 2=1+0=1\Rightarrow 1=j=|w|_1\mod 2=1$.
\item ($i=1,\sigma=1$). $(q_0,w_01)\vdash^*(q_1,1)\vdash(q_0,\varepsilon)\Rightarrow q_0\overset{w}{\longrightarrow}q_0\Rightarrow j=0$. $|w|_1\mod 2=|w_0|_1\mod 2+|1|_1\mod 2=(1+1)\mod 2=0\Rightarrow 0=j=|w|_1\mod 2=0$.
\end{enumerate}
\end{enumerate}
Таким образом, $\forall n\in{\mathbb N}\cup\{0\}\hookrightarrow P(n)\Rightarrow \forall n\in{\mathbb N}\cup\{0\}\hookrightarrow \big[\forall w\in\Sigma^*\colon |w|=n\hookrightarrow \big(q_0\overset{w}{\longrightarrow}q_i\Rightarrow i=|w|_1\mod 2\big)\big]\Rightarrow\newline\forall w\in\Sigma^*\hookrightarrow q_0\overset{w}{\longrightarrow}q_{|w|_1\mod 2}$.
Пусть $w\in L\Leftrightarrow |w|_1\mod 2=0\Leftrightarrow q_0\overset{w}{\longrightarrow}q_0\Leftrightarrow w\in L(\A)\,\blacksquare$
\end{enumerate}
\section*{Задача 5}
\section*{Задача 6}
\end{document}
