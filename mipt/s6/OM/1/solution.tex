\documentclass[a4paper]{article}
\usepackage[a4paper, left=5mm, right=5mm, top=5mm, bottom=5mm]{geometry}
\usepackage[utf8]{inputenc}
\usepackage[english,russian]{babel}
\usepackage{graphicx}
\usepackage{indentfirst}
\usepackage{tikz} %Рисование автоматов
\usetikzlibrary{automata,positioning}
\usepackage{amsmath}
\usepackage{floatflt}
\usepackage{enumerate}
\usepackage{amsfonts}
\usepackage{amssymb}
\newcommand{\matrixr}{\right|\right|}
\newcommand{\matrixl}{\left|\left|}
\def\eqdef{\overset{\mbox{\tiny def}}{=}}
\title{Методы оптимизации. Задание 1: Субградиентный спуск}
\date{задано 2016.02.09}
\author{Сергей~Володин, 374 гр.}

\begin{document}
\maketitle
\section*{Задача 1}
Делаем проекцию при $k\in K$. $Q\ni x^*=\underset{x\in Q}{\arg\min} f(x)$. Рассмотрим $k+1$-ю итерацию:
\begin{enumerate}
\item Пусть $k+1\in K$. Тогда $x_{k+1}\eqdef \pi_Q(x_k-\alpha_k g^k)$
\item Иначе $x_{k+1}=x_k-\alpha_k g^k$.
\end{enumerate}

Здесь $g^k\in\partial f(x_k)$. В первом случае $||x_{k+1}-x^*||=||\pi_Q(x_k-\alpha_k g^k)-x^*||\leqslant||x_k-\alpha_k g^k-x^*||$ по свойству проекции (расстояние до $x^*\in Q$ не увеличивается при проектировании на $Q$).

Тогда в обоих случаях $||x_{k+1}-x^*||^2\leqslant ||x_k-\alpha_k g^k-x^*||^2=||x_k^2-x^*||^2+\alpha_k^2||g^k||^2-2\alpha_k(g^k, x_k-x^*)\overset{g\in\partial f(x_k)}{\leqslant}||x_k-x^*||^2+\alpha_k^2||g^k||^2-2\alpha_k(f(x_k)-f(x^*))$.

Последовательно подставим (индукция) оценку разности для $x_k$, ..., $x_1$ в $x_{k+1}$, получим такую же оценку, как и для обычного метода градиентного спуска, т.е. $f_{\mbox{\small{best}}}-f_*\leqslant \frac{RL}{\sqrt{k}}$

Оценка в худшем случае (равенство) не изменится.

\section*{Задача 2}
Ответ: да, верно, да, может. Приведем пример $f\colon \mathbb{R}^n\to\mathbb{R}$~--- выпуклая,
$x_0\in\mathbb{R}^n, a\in\partial f(x_0)$, $x_0$~--- не точка минимума $f$, $-a$~--- не направление убывания $f$.

$f(\matrixl
\begin{array}{c}
x_1\\
x_2
\end{array}
\matrixr)\eqdef |x_1|+|x_2|\colon \mathbb{R}^2\to\mathbb{R}$. Точка $x_0\eqdef \matrixl
\begin{array}{cc}
1 & 0
\end{array}
\matrixr^T$ Тогда\begin{enumerate}
\item $f$~--- выпуклая: пусть $t_1,t_2\in\mathbb{R}_+$, $x,y\in\mathbb{R}^2$. $f(t_1x+t_2y)=|t_1x_1+t_2y_1|+|t_1x_2+t_2y_2|\leqslant t_1|x_1|+t_2|y_1|+t_1|x_2|+t_2|y_2|=t_1(|x_1|+|x_2|)+t_2(|y_1|+|y_2|)=t_1f(x)+t_2f(y)$. Возьмем $t_1\in[0,1]$, $t_2=1-t_1$, получим определение выпуклой функции.
\item Пусть $a=\matrixl
\begin{array}{cc}
1 & 1
\end{array}
\matrixr^T$. Докажем, что $a\in\partial f(x_0)$. Фиксируем $x\in\mathbb{R}^2$. $f(x)-f(x_0)=|x_1|+|x_2|-1=1\cdot(|x_1|-1)+1\cdot(|x_2|)\equiv(a, x-x_0)$. То есть, верно:
$$\forall x\in\mathbb{R}^2\hookrightarrow f(x)-f(x_0)\geqslant (a,x-x_0)$$
То есть, $a$~--- субградиент.
\item $-a$~--- не направление убывания в $x_0$. Пусть $t\in(0,1)$. Рассмотрим $f(x_0-ta)=|1-t|+|-t|=1-t+t=1$. Получаем $\forall t\in(0,1)\hookrightarrow f(x_0-ta)=f(x_0)$. Получаем, $$\forall t_0>0\,\exists t\eqdef\min\{1/2,t_0/2\}<t_0\colon f(x_0-ta)\geqslant f(x_0)$$
Это отрицания определения направления убывания.
\item $x_0$~--- не точка минимума $f$: $f(x_0)=|1|+|0|=1$, $f(0)=0<1=f(x_0)$.
\end{enumerate}
\section*{Задача 3}
Пусть $f(x_1,x_2)\colon \mathbb{R}^{n_1+n_2}\to\mathbb{R}$, $g\colon \mathbb{R}^{n_1}\to\mathbb{R}$,  $h\colon \mathbb{R}^{n_2}\to\mathbb{R}$, $\forall x_1,x_2\hookrightarrow f(x_1,x_2)=g(x_1)+h(x_2)$, $f,\,g,\,h$~--- выпуклые и непрерывно дифференцируемые.

Пусть $g$ достигает минимума в $x_1^*$, $h$~--- в $x_2^*$. Тогда $f(x_1,x_2)=g(x_1)+h(x_2)\geqslant g(x_1^*)+h(x_1^*)=f(x_1^*,x_2^*)$, т.е. $(x_1^*,x_2^*)$~--- также минимум $f$.

Пусть $(x_1^0,x_2^0)$~--- точка старта алгоритма.

Пусть $||g'_{x_1}||\leqslant R_1$, $||h'_{x_2}||\leqslant R_2$, $||x_1^0-x_1^*||\leqslant L_1$, $||x_2^0-x_2^*||\leqslant L_2$.

Тогда $||f'_{x_1,x_2}||=||(g'(x_1),h'(x_2))||=\sqrt{R_1^2+R_2^2}$, $||x^0-x^*||=||(x_1^0-x_1^*,x_2^0-x_2^*))||=\sqrt{L_1^2+L_2^2}$.

Алгоритм, минимизирующий $f$ целиком, сделает (в худшем случае) $k=\frac{R^2L^2}{\varepsilon^2}$ шагов. На каждом шаге он вычисляет $g, h, g', h'$, т.е. он сделает $a=4k=4\frac{R^2L^2}{\varepsilon^2}$ операций

Алгоритм, минимизирующий только $g$, сделает (в худшем случае) $k_1=\frac{R_1^2L_1^2}{\varepsilon^2/4}$ шагов (берем $\varepsilon/2$, чтобы в сумме с такой же погрешностью $h$ дало $\varepsilon$). На каждой итерации вычисляется $g, g'$, т.е. каждый шаг требует двух операций: $a_1=2\frac{R_1^2L_1^2}{\varepsilon^2/4}$.

Аналогично $a_2=2\frac{R_2^2L_2^2}{\varepsilon^2/4}$

Вычитаем, получаем выигрыш: $a-a_1-a_2=4\frac{(R_2^2-R_1^2)(L_1^2-L_2^2)}{\varepsilon^2}>0$ при $\begin{cases}
L_1>L_2\\
R_2>R_1
\end{cases}$ или $\begin{cases}
L_1<L_2\\
R_2<R_1
\end{cases}$
\end{document}
