\documentclass[a4paper]{article}
\usepackage[a4paper, left=5mm, right=5mm, top=5mm, bottom=5mm]{geometry}
\usepackage[utf8]{inputenc}
\usepackage[english,russian]{babel}
\usepackage{indentfirst}
\usepackage{tikz} %Рисование автоматов
\usetikzlibrary{automata,positioning}
\usepackage{amsmath}
\usepackage{enumerate}
\usepackage{amsfonts}
\usepackage{amssymb}
\usepackage{wasysym}
\title{Теория и реализация языков программирования.\\Задание 3: Вычислительные возможности конечных автоматов}
\date{задано 2013.09.18}
\author{Сергей~Володин, 272 гр.}
\newcommand{\matrixl}{\left|\left|}
\newcommand{\matrixr}{\right|\right|}
% названия автоматов
\def\A{{\cal A}}
\def\B{{\cal B}}
\def\C{{\cal C}}

% регулярные языки
\def\REG{{\mathsf{REG}}}

\newcommand{\niton}{\not\owns}

\begin{document}
\maketitle
\section*{Упражнение 1}
Пусть $\sim\,\subset X\times X$. $C(x)=\{z\in X|x\sim z\}$, $C(y)=\{w\in X|y\sim z\}$.
\\[1pt]
Пусть $\exists z\in C(x)\cap C(y)$. Тогда $x\sim z,y\sim z$, и $w\in C(x)\overset{\mathrm{def}}{\Leftrightarrow}x\sim w\underset{\substack{\mbox{\tiny транз.} \\ \mbox{\tiny симм.}}}{\overset{z\sim x}{\Leftrightarrow}}z\sim w\overset{y\sim z}{\Leftrightarrow}y\sim w\overset{\mathrm{def}}{\Leftrightarrow}w \in C(y)$, то есть, $C(x)=C(y)$.\newline
В противном случае $\rceil (\exists z\in C(x)\cap C(y))\Leftrightarrow C(x)\cap C(y)=\varnothing$.
Получаем, что возможны два случая:
\begin{enumerate}[1.]
\item $C(x)\cap C(y)=\varnothing$ {\em (не пересекаются)}
\item $C(x)=C(y)$ {\em (совпадают)}
\end{enumerate}
\section*{Упражнение 2}
Пусть $\varphi\colon\Sigma^*\supseteq X\longrightarrow\Delta^*$. $\varphi(\sigma_i)\overset{\mbox{\tiny def}}{=}\delta_i\in\Delta^*$, $|\sigma_i|=1$.\newline
\begin{enumerate}
\item {\em (единственность)} Предположим, что существует такое $\varphi$~--- морфизм. Тогда $\forall w=w_1...w_n\in X, |w_i|=1\hookrightarrow\varphi(w)\equiv \varphi(w_1...w_n)=\varphi(w_1)\cdot\varphi(w_2...w_n)=...=\varphi(w_1)\cdot...\cdot\varphi(w_n)\in\Delta^*$. Для $w=\varepsilon$ получаем $\varphi(\varepsilon)=\varepsilon$, так как $\varphi$~--- морфизм: $w_0\overset{\mathrm{def}}{=}\varphi(\varepsilon)=\varepsilon$. $\varphi(\varepsilon)\equiv\varphi(\varepsilon\varepsilon)=\varphi(\varepsilon)\varphi(\varepsilon)=w_0w_0\Rightarrow w_0=w_0w_0\Rightarrow |w_0|=|w_0||w_0|\Rightarrow w_0=\varphi(\varepsilon)=\varepsilon$.
\\[5pt]
Таким образом, получаем, что такой морфизм единственный (если существует).
\item {\em (существование)} Докажем, что определенное выше отображение $\varphi$~--- морфизм: пусть $x,y\in X$. Рассмотрим случаи:
\begin{enumerate}[a.]
\item $|x|=0,|y|=0\Rightarrow\varphi(xy)=\varphi(\varepsilon\varepsilon)=\varphi(\varepsilon)=\varepsilon=\varepsilon\varepsilon=\varphi(\varepsilon)\varphi(\varepsilon)$
\item $|x|=0,|y|>0\Rightarrow\varphi(xy)=\varphi(y)=\varepsilon\varphi(y)=\varphi(x)\varphi(y)$
\item $|x|>0,|y|=0\Rightarrow\varphi(xy)=\varphi(x)=\varphi(x)\varepsilon=\varphi(x)\varphi(y)$
\item $|x|>0,|y|>0\Rightarrow\varphi(xy)=\varphi(x_1...x_my_1...y_n)=\underbrace{\varphi(x_1)...\varphi(x_m)}\underbrace{\varphi(y_1)...\varphi(y_n)}=\varphi(x)\varphi(y)$.
\end{enumerate}
\end{enumerate}
Таким образом, если заданы значения $\varphi(\sigma_i),\sigma_i\in X\subset\Sigma^*$, то морфизм $\varphi\colon \Sigma^*\supseteq X\longrightarrow\Delta^*$ с этими значениями существует и единственнен.
\section*{Задача 1}
Определим $R_3:\REG\ni X\longrightarrow \mathbb{N}\cup\{0\}$~--- количество применений правила $3$ из определения регулярности $X$. В случае $X=AB$ или $X=A|B, A,B\in\REG$ $R_3(X)\overset{\mathrm{def}}{=}1+R_3(A)+R_3(B)$. В случае $X=A^*, A\in\REG$, определим $R_3(X)\overset{\mathrm{def}}{=}1+R_3(A)$. В случае $X=\varnothing$ или $X=\{\sigma\}$ определим $R_3(X)\overset{\mathrm{def}}{=}0$. Функция $R_3(X)$ определена корректно, так как определение регулярного языка корректное.\newline
Пусть $\varphi\colon\Sigma^*\supset X\longrightarrow Y\subset\Delta^*$~--- морфизм, $X\in\REG$. Докажем, что $Y\equiv\varphi(X)\in\REG$ индукцией по $R_3(X)$:
\\[5pt]
$P(i)=(\forall X\in\REG\colon R_3(X)\leqslant i\,\,\forall\varphi\mbox{~--- морфизм}\hookrightarrow \varphi(X)\in\REG)$.
\\[1pt]
\begin{enumerate}[1.]
\item Докажем $P(0)$: пусть $X\in\REG\colon R_3(X)=0$. Тогда $X$ получен без применения третьего правила. Значит, $\forall\varphi$~--- морфизм либо $X=\varnothing\Rightarrow\varphi(X)=\varnothing$, либо $X=\{\sigma\}\Rightarrow\varphi(X)=\{\varphi(\sigma)\}=\{w\}, w\in \Delta^*$.\newline
\\[5pt]
Докажем, что $\Delta^*\supset\{w\}\in\REG$. $\{w\}\equiv\{\sigma_1...\sigma_n\}\equiv\{\sigma_1\}\cdot...\cdot\{\sigma_n\}$. Поскольку $\{\sigma_i\}\in\REG$, и регулярные языки замкнуты относительно конкатенации (по определению), получаем требуемое.
\\[5pt]
Итак, $\varphi(X)\in\REG\,\blacksquare$
\item Пусть $P(n)$. Докажем $P(n+1)$. Пусть $\REG\ni X\colon R_3(X)\leqslant n+1$. Если $R_3(X)<n+1$, $P(n)\Rightarrow X\in\REG$.\newline
$\varangle X\colon R_3(X)=n+1$. Возможны случаи:
\begin{enumerate}[a.]
\item $X=WZ$, $W,Z\in\REG$. Тогда $\varphi(X)\equiv\varphi(WZ)=\{\varphi(wz)|w\in W,z\in Z\}=\{\varphi(w)\varphi(z)|w\in W,z\in Z\}=\\\{\varphi(w)|w\in W\}\cdot\{\varphi(z)|z\in Z\}=\varphi(W)\varphi(Z)$. $R_3(X)=1+R_3(W)+R_3(Z)=n+1\Rightarrow R_3(W),R_3(Z)\leqslant n\overset{P(n)}{\Rightarrow} \varphi(W),\varphi(Z)\in\REG\Rightarrow \varphi(X)=\varphi(W)\varphi(Z)\in\REG$.
\item $X=W|Z$, $W,Z\in\REG$. Тогда $\varphi(X)\equiv\varphi(W|Z)\equiv\varphi(W)|\varphi(Z)$. Аналогично $R_3(W),R_3(Z)\leqslant n\overset{P(n)}{\Rightarrow}\varphi(W),\varphi(Z)\in\REG\Rightarrow\varphi(X)=\varphi(W)|\varphi(Z)\in\REG$.
\item $X=W^*, W\in\REG$. Тогда $R_3(X)=1+R_3(W)=n+1\Rightarrow R_3(W)=n\overset{P(n)}{\Rightarrow}\varphi(W)\in\REG\Rightarrow \varphi(W^*)=\varphi(\varepsilon|W|WW|...)=\varphi(\varepsilon)|\varphi(W)|\varphi(WW)...\overset{\varphi(\varepsilon)=\varepsilon}{=}\varepsilon|\varphi(W)|\varphi(WW)...=\varphi(W)^*\in\REG$.
\end{enumerate}
\end{enumerate}
Получаем $\forall i\geqslant 0\hookrightarrow P(i)\Rightarrow\forall X\in\REG\,\forall\varphi\mbox{~--- морфизм}\hookrightarrow\varphi(X)\in\REG\,\blacksquare$
\section*{Задача 2}
\begin{enumerate}[1.]
%Почему \varphi^-1 - морфизм, если \varphi - морфизм?
\item Нет. Пусть $\Sigma=\{0,1\}$, $L=\Sigma^*$. Определим $\varphi\colon L\longrightarrow L\colon \forall w\in L\hookrightarrow\varphi(w)=\varepsilon$. В этом случае $\varphi$~--- морфизм, так как $\forall x\in L\,\forall y\in L\hookrightarrow\varphi(xy)=\varepsilon=\varepsilon\varepsilon=\varphi(x)\varphi(y)$. Тогда $\forall \varnothing\neq X\subset L\hookrightarrow\varphi(X)=\{\varepsilon\}$, так как $\forall w\in L\hookrightarrow\varphi(w)=\varepsilon$. Поскольку $\varphi(\varepsilon)=\varepsilon\in L$, $\varphi^{-1}(\varepsilon)\ni\varepsilon\Rightarrow \varphi^{-1}(L)\supset\{\varepsilon\}\neq\varnothing\Rightarrow\varphi^{-1}(L)\neq\varnothing\Rightarrow\varphi(\varphi^{-1}(L))=\{\varepsilon\}\neq L$.\newline
Таким образом, $\exists L\subseteq\Sigma^*\,\exists\varphi\mbox{~--- морфизм}\colon \varphi(\varphi^{-1}(L))\neq L$.
\item Нет. Пусть $\Sigma=\{a,b\}$, $L=\{b\}^*$, $\varphi(a)\overset{\mathrm{def}}{=}\varphi(b)\overset{\mathrm{def}}{=}a$. Доопределим $\varphi$ так, чтобы оно было морфизмом (это возможно, см. упражнение 2). Тогда $\varphi(L)\equiv\varphi(\{b^*\})\ni\varphi(b)=a\Rightarrow\varphi^{-1}(\varphi(L))\supset\varphi^{-1}(a)\ni a\notin L\Rightarrow \varphi^{-1}(\varphi(L))\nsubseteq L\Rightarrow\varphi^{-1}(\varphi(L))\neq L$.\newline
Таким образом, $\exists L\subseteq\Sigma^*\,\exists\varphi\mbox{~--- морфизм}\colon \varphi^{-1}(\varphi(L))\neq L$.
\item Нет. Пусть $\Sigma=\{a,b\}$, $L=\{ab\}$, морфизм $\varphi\colon\Sigma^*\longrightarrow\Sigma^*$~--- из предыдущего пункта. Тогда $\varphi(L)=\{\varphi(ab)\}=\{\varphi(a)\varphi(b)\}=\{aa\}$, $\varphi^{-1}(L)=\{x\in\Sigma^*|\varphi(x)\in\{ab\}\}=\{x\in\Sigma^*|\varphi(x)=ab\}=\varnothing$, так как $\varphi(\Sigma^*)=\varphi((a|b)^*)\overset{1.2.c}{=}(\varphi(a|b))^*=\{\varphi(a),\varphi(b)\}^*=\{a\}^*=a^*\niton ab$. Тогда $\varphi(\varphi^{-1}(L))=\varphi(\varnothing)=\varnothing\niton aa \in\varphi^{-1}(aa)=\varphi^{-1}(\varphi(L))$.\newline
Таким образом, $\exists L\subseteq\Sigma^*\,\exists\varphi\mbox{~--- морфизм}\colon \varphi(\varphi^{-1}(L))\neq \varphi^{-1}(\varphi(L))$.
\section*{Упражнение}
Докажем, что не всякий обратный морфизм~--- морфизм, то есть $\exists\Sigma\,\exists\Delta\,\exists\varphi\colon\Sigma^*\longrightarrow\Delta^*\colon\exists w_1\in\Delta^*\,\exists w_2\in\Delta^*\colon \varphi^{-1}(w_1w_2)\neq\varphi^{-1}(w_1)\cdot\varphi^{-1}(w_2)$ {\em (здесь немного модифицировано определение морфизма для $\varphi^{-1}$, так как множество значений $\varphi^{-1}$~--- это $2^{\Sigma^*}$, а не $\Sigma^*$)}.\newline
Пусть $\Sigma=\Delta=\{a,b\}$, $\varphi(a)\overset{\mbox{\tiny def}}{=}\varphi(b)\overset{\mbox{\tiny def}}{=}ab$. Доопределим $\varphi$ так, чтобы оно было морфизмом (это возможно, см. упражнение 2). Тогда $\varphi^{-1}(a)=\varphi^{-1}(b)=\varnothing$, так как $\forall \varepsilon\neq w\in\Sigma^*\hookrightarrow|\varphi(w)|\geqslant 2$ и $|\varphi(\varepsilon)|=|\varepsilon|=0$, то есть, значение $|\varphi(w)|=1$ не достигается. Отсюда $\varphi^{-1}(a)\cdot\varphi^{-1}(b)=\varnothing$, но $\varphi^{-1}(ab)\supset\{a,b\}\Rightarrow\varphi^{-1}(ab)\neq\varnothing$. Поэтому $\varphi^{-1}(ab)\neq\varphi^{-1}(a)\varphi^{-1}(b)\,\blacksquare$
\section*{Задача 3}
$\Delta^*\supset L\in\REG\Rightarrow\exists\A=(Q,\Delta,q_0,\gamma,F),Q=\{q_0,...,q_n\}$~--- ДКА: $L(\A)=L$. Построим НКА $\A'=(Q',\Sigma,q_0,\gamma',F)$ для $L^{-1}\overset{\mbox{\tiny def}}{=}\varphi^{-1}(L)$.
\begin{enumerate}
\item $Q'\overset{\mbox{\tiny def}}{=}Q\cup\bigcup\limits_{\substack{i,j\in\overline{0,n},\delta\in\Delta\\\gamma(q_i,\delta)=q_j}}\{q^{ijl}_{\delta}|l\in\overline{1,\varphi(\delta)-1}\}$~--- для каждого перехода добавляются $\varphi(w)-1$ состояний.
\item $\gamma'(q,\sigma)=\begin{cases}

\end{cases}$
\end{enumerate}
%Дописать!
\section*{Задача 4}
Пусть языки $\Sigma^*\supset X,Y\in\REG$. Докажем, что
\begin{enumerate}[1.]
\item $X\cup Y\in\REG$: из определения регулярности $\forall X,Y\in\REG\hookrightarrow X\cup Y\in\REG\,\blacksquare$
\item $\overline{X}\overset{\mbox{\tiny def}}{=}\Sigma^*\backslash X\in\REG$: $X\in\REG\Rightarrow\exists\mbox{ полный ДКА }\A\colon L(\A)=X$. $F'\overset{\mbox{\tiny def}}{=}Q\backslash F$, $\A'$~--- автомат $\A$ с множеством принимающих состояний $F'$. Докажем, что $L(\A')=\Sigma^*\backslash X$: $w\in \Sigma^*$, $(q_0,w)\vdash^*(q_w,\varepsilon)$ {\em (здесь используется полнота)}. $w\in X\Leftrightarrow w\in L(\A)\Leftrightarrow q_w\in F\Leftrightarrow\urcorner(q_w\in Q\backslash F)\Leftrightarrow\urcorner(q_w\in F')\Leftrightarrow \urcorner(w\in L(\A'))$. Но $w\in X\Leftrightarrow\urcorner(w\in\Sigma^*\backslash X)$, откуда $\urcorner(w\in\Sigma^*\backslash X)\Leftrightarrow\urcorner(w\in L(\A'))$ и Получаем ДКА $\A'\colon L(\A')=\Sigma^*\backslash X\overset{\mbox{\tiny на}}{\underset{\mbox{\tiny семинаре}}{\Rightarrow}}\Sigma^*\backslash X\in\REG\,\blacksquare$
\item $X\cap Y\in\REG$: $X\cap Y=\overline{\overline{X}\cup\overline{Y}}$. $X,Y\in\REG\overset{(2)}{\Rightarrow}\overline{X},\overline{Y}\in\REG\overset{(1)}{\Rightarrow}\overline{X}\cup\overline{Y}\in\REG\overset{(2)}{\Rightarrow}\overline{\overline{X}\cup\overline{Y}}\in\REG\,\blacksquare$
\\[5pt]
$w\in X\cap Y\Leftrightarrow \begin{cases}w\in X\\w\in Y\end{cases}\Leftrightarrow\begin{cases}\urcorner(w\in\overline{X})\\\urcorner(w\in\overline{Y})\end{cases}\Leftrightarrow
\urcorner{\left[
\begin{array}{l}
w\in\overline{X}\\
w\in\overline{Y}
\end{array}
\right.
}\Leftrightarrow\urcorner(w\in\overline{X}\cup\overline{Y})\Leftrightarrow w\in\overline{\overline{X}\cup\overline{Y}}
$ (подразумевается $w\in\Sigma^*$) $\blacksquare$
\\[5pt]
\item $X\backslash Y\in\REG$: $X\backslash Y=X\cap\overline{Y}$. $Y\in\REG\overset{(2)}{\Rightarrow}\overline{Y}\in\REG\overset{(3)}{\Rightarrow}X\cap\overline{Y}\in\REG\,\blacksquare$
\\[5pt]
$w\in X\cap\overline{Y}\Leftrightarrow\begin{cases}
w\in X\\
w\in \overline{Y}
\end{cases}\Leftrightarrow\begin{cases}
w\in X\\
\urcorner(w\in Y)
\end{cases}\Leftrightarrow w\in X\backslash Y$  (подразумевается $w\in\Sigma^*$) $\blacksquare$
\\[5pt]
\end{enumerate}
\end{enumerate}
\section*{Задача 5}
$\Sigma\overset{\mbox{\tiny def}}{=}\{a\}$. Предположим, что $\Sigma^*\subset L=\{a^{2^n}|n\geqslant 0\}\in\REG\overset{\mbox{\tiny по лемме}}{\underset{\mbox{\tiny о накачке}}{\Rightarrow}}\exists p=p_0\geqslant 1\colon \forall w\in L\hookrightarrow(w=xyz,|y|\geqslant 1,|xy|\leqslant p, (\forall i\geqslant 0\hookrightarrow xy^iz\in L))$. Фиксируем $n=n_0=p, w_0=a^{2^{p}}\in L$. Получаем $w_0=x_0y_0z_0, |y0|\geqslant 1, |x_0y_0|\leqslant p$. Поскольку $L\subset a^*, y\in a^*$, откуда $y=a^j,\,j\geqslant 1$. Аналогично $x=a^i,z=a^k\Rightarrow w_0=a^{2^p}=xyz=a^{i+j+k}\Rightarrow i+j+k=2^p$. По лемме должно выполняться $xy^2z=a^{i+2j+k}\in L\Rightarrow a^{i+2j+k}=a^{2^q}$, откуда $i+2j+k=2^{q}\Rightarrow j=2^q-2^p\geqslant 2^{p+1}-2^p=2^p(2-1)=2^p$. Но $|x_0y_0|\leqslant p\Rightarrow |y_0|\leqslant p$. Получаем $p\geqslant |y_0|=j\geqslant 2^p$~--- противоречие, т.к. $\forall p\geqslant 1\hookrightarrow p<2^p$.
\\[3pt]
Значит, предположение неверно, и $L\not\in\REG\,\blacksquare$
\section*{Задача 6}
\begin{enumerate}[1.]
\item Да. $L_1=\{a^{2013n+5}|n\in{\mathbb{N}}\cup\{0\}\}\cap\{a^{509k+29}|k\in{\mathbb N},k\geqslant 401\}$. $w\in L_1\Leftrightarrow\exists n_w\in{\mathbb{N}}\cup\{0\},401\leqslant k_w\in{\mathbb N}\colon w=a^{2013n_w+5}=a^{509k_w+29}$.
\\[4pt]
Решим в целых числах $2013n+5=509k+29\Leftrightarrow 2013n-509k=24\Leftrightarrow(*)$~--- линейное диофантово уравнение, $24\,\vdots\,1=gcd(2013,509)\Rightarrow$ решение существует, и $(*)\Leftrightarrow
\matrixl
\begin{array}{c}
n\\
k
\end{array}
\matrixr=\matrixl
\begin{array}{c}
x_0\\
y_0
\end{array}
\matrixr+t
\matrixl
\begin{array}{c}
x\\
y
\end{array}
\matrixr;x_0,y_0,\,x,y,\,t\in{\mathbb Z}$, $x_0,y_0,\,x,y$~--- фиксированные, $t\geqslant t_0$~--- параметр. Тогда $2013n+5=509k+29=2013(x_0+xt)+5=pt+q$, $p,q\in{\mathbb Z}$~--- фиксированные, ${\mathbb Z}\ni t\geqslant 0$~--- параметр.
\\[4pt]
Получаем $L_1=\{a^{pt+q}|{\mathbb Z}\ni t\geqslant 0\}=\{a^{pt}|{\mathbb Z}\ni t\geqslant 0\}\cdot\{a^q\}=\{{(a^p)}^t|{\mathbb Z}\ni t\geqslant 0\}\cdot\{a^q\}=(a^p)^*a^q\equiv (\underbrace{a...a}_p)^*\underbrace{a...a}_q$~--- задается регулярным выражением $\blacksquare$
\item Нет. Предположим, что $L_2=\{a^{200n^2+1}|{\mathbb Z}\ni n\geqslant 1000\}\in\REG\overset{\mbox{\tiny по лемме}}{\underset{\mbox{\tiny о накачке}}{\Rightarrow}}\exists p\geqslant 1\colon \forall w\in L_2\hookrightarrow(w=xyz,|y|\geqslant 1,|xy|\leqslant p, (\forall i\geqslant 0\hookrightarrow xy^iz\in L_2))$. Выберем ${\mathbb Z}\ni n=\max\{p,1000\}\geqslant 1000\Rightarrow w\overset{\mbox{\tiny def}}{=}a^{200n^2+1}\in L_2$. Получаем $\exists x,y,z\colon |y|\geqslant 1,|xy|\leqslant p\colon w=xyz$, откуда $x=a^i,y=a^j,z=a^k, i+j+k=200n^2+1$. Также получаем $xy^2z\in L_2$. Но $xy^2z=a^{i+2j+k}=a^{200m^2+1}\Rightarrow i+2j+k=200m^2+1\geqslant 200(n+1)^2+1\Rightarrow j\geqslant [200(n+1)^2+1]-[200n^2+1]=200+400n\geqslant 200+400p$. С другой стороны, $|xy|\leqslant p\Rightarrow j=|y|\leqslant p\Rightarrow p\geqslant j \geqslant 200+400p\Rightarrow 399p+200\leqslant 0$ при $p\geqslant 1$~--- противоречие.
\\[3pt]
Значит, предположение неверно, и $L_2\not\in\REG\,\blacksquare$
%Критерий Коши, блин :)
\item[3,4.] {\bf Можно не читать, доказательство не закончено} Да. $L_3=\{\mbox{itoa}_2(x)|0\leqslant x\mod 3=2\}$, где $\mbox{itoa}_2(x)$~--- запись числа $x$ в двоичной системе счисления, начиная со старших разрядов. Построим ДКА $\A\colon L(\A)=L_3$, чем докажем $L_3\in\REG$.\newline
Реализуем алгоритм деления в столбик на конечном автомате $\A$.\newline
Формализуем деление на 3 в столбик с остатком в двоичной системе счисления. Пусть $|\mbox{itoa}_2(x)|=n$. Функции $p,r\colon\overline{0,n}\longrightarrow{\mathbb N}\cup\{0\}$. $P(i)\overset{\mbox{\tiny def}}{=}\{x=3p(i)+r(i),r(i)<3\cdot 2^{n-i}\}$. Определим эти функции индуктивно и докажем $\forall i\in\overline{0,n}\hookrightarrow P(i)$.
\begin{enumerate}
\item Определение: $p(0)\overset{\mbox{\tiny def}}{=}0,r(0)\overset{\mbox{\tiny def}}{=}x$. Доказательство $P(0)$: $x=3\cdot 0+x$, $r(0)=x=\sum\limits_{k=0}^{n-1}2^kx_{n-k}<2^n<3\cdot 2^n$.
\item Пусть $p(k),r(k)$ определены для $k\in\overline{0,i},\forall k\in\overline{0,i}\hookrightarrow P(k)$. Определим $p(i+1),r(i+1)$ и докажем $P(i+1)$
\begin{enumerate}[1.]
\item Если $r(i)<3\cdot 2^{n-i-1}$, то $p(i+1)\overset{\mbox{\tiny def}}{=}p(i)$, $r(i+1)\overset{\mbox{\tiny def}}{=}r(i)$. $x\overset{P(i)}{=}3p(i)+r(i)\equiv 3p(i+1)+r(i+1)$, $r(i+1)\equiv r(i)\overset{\mbox{\tiny случай}}{<}3\cdot 2^{n-i-1}\Rightarrow P(i+1)\,\blacksquare$
\item Иначе, если $r(i)\geqslant 3\cdot 2^{n-i-1}$, $p(i+1)\overset{\mbox{\tiny def}}{=}p(i)+2^{n-i-1}$, $r(i+1)\overset{\mbox{\tiny def}}{=}r(i)-3\cdot 2^{n-i-1}\overset{\mbox{\tiny случай}}{\geqslant}0\Rightarrow 3p(i+1)+r(i+1)=3p(i)+3\cdot 2^{n-i+1}+r(i)-3\cdot 2^{n-i+1}=3p(i)+r(i)=x$. $r(i)\overset{P(i)}{<}3\cdot 2^{n-i}\Leftrightarrow r(i)<3\cdot 2^{n-i-1}+3\cdot 2^{n-i-1}\Leftrightarrow r(i)-3\cdot 2^{n-i-1}<3\cdot 2^{n-i-1}\Leftrightarrow r(i+1)<3\cdot 2^{n-i-1}$. Получаем $P(i+1)\,\blacksquare$
\end{enumerate}
\end{enumerate}
Получаем $P(n)\Rightarrow x=3p(n)+r(n),r(n)<3\cdot 2^{n-n}=3\Rightarrow$ $p(n)$~--- частное, $r(n)$~--- остаток.
\\[5pt]
%Пусть $\A=(Q,\Sigma,q_0,\delta,F), \Sigma=\{0,1\}$.
{\em Дальше была идея формально доказать, что строку можно разбить на куски длиной не более, чем 3 символа $\{0,11,101,100\}$ (<<пропусков>> вида (b.1) не больше трех подряд не в конце и не в начале слова) и что после прочтения каждого куска можно только хранить одно из трех чисел $\{0,1,2\}$ в состоянии автомата, а в конце это число и будет остатком...}
%Будем последовательно рассматривать символы $\mbox{itoa}_2(x)$. $p_i,r_i\geqslant 0$~--- некоторые значения при рассмотрении $i$-го символа, причем $\forall x=3p+r\colon p,r\geqslant 0$. Не более, чем через шаге $p$ увеличивается, а $r$~--- уменьшается. Доказав это, будет доказана конечность. Если в некоторый момент $r\leq 3$, получим частное и остаток по определению делимости. Доказательство проведем по индукции по количеству рассмотренных символов.
%Состояния будут называться $q_t^n$, где $0\leqslant r<2$~--- текущий остаток при делении числа, $n$~--- номер состояния. 
%\\[1pt]
%На $i$-м шаге рассматрива
\end{enumerate}
\end{document}
