\documentclass[a4paper]{article}
\usepackage[a4paper, left=5mm, right=5mm, top=5mm, bottom=5mm]{geometry}
%\geometry{paperwidth=210mm, paperheight=2000pt, left=5pt, top=5pt}
\usepackage[utf8]{inputenc}
\usepackage[english,russian]{babel}
\usepackage{indentfirst}
\usepackage{tikz} %Рисование автоматов
\usetikzlibrary{automata,positioning,arrows}
\usepackage{amsmath}
\usepackage{enumerate}
\usepackage{hyperref}
\usepackage{amsfonts}
\usepackage{amssymb}
\DeclareMathOperator*{\argmin}{arg\,min}
\usepackage{wasysym}
\title{Теория и реализация языков программирования.\\Задание 8: контекстно-свободные языки и магазинные автоматы II}
\date{задано 2013.10.23}
\author{Сергей~Володин, 272 гр.}
\newcommand{\matrixl}{\left|\left|}
\newcommand{\matrixr}{\right|\right|}
% названия автоматов
\def\A{{\cal A}}
\def\B{{\cal B}}
\def\C{{\cal C}}

%+= и -=, иначе разъезжаются...
\newcommand{\peq}{\mathrel{+}=}
\newcommand{\meq}{\mathrel{-}=}
\newcommand{\deq}{\mathrel{:}=}
\newcommand{\plpl}{\mathrel{+}+}

% пустое слово
\def\eps{\varepsilon}

% регулярные языки
\def\REG{{\mathsf{REG}}}
\def\eqdef{\overset{\mbox{\tiny def}}{=}}
\newcommand{\niton}{\not\owns}

\begin{document}
\maketitle
\section*{Упражнение 1}
\begin{enumerate}
\item Левый вывод однозначно задает дерево вывода, так как вывод~--- последовательность слов из $(N\cup T)^*$ с указанием нетерминала для каждого слова и правила. Будем определять дерево так: если $A\longrightarrow x_1...x_n$. добавим к вершине $A$ потомков $x_1$,...,$x_n$ в таком порядке. Если два левых вывода различны, то построенные таким образом деревья также получатся различными.
\item Если дано дерево вывода, то левый вывод можно получить его обходом в глубину, выбирая на каждом шаге самого левого из непросмотренных потомков. Если два дерева вывода различны, то так построенные выводы также получатся различными.
\item Пусть существует единственное дерево вывода. Предположим, что существуют два различных левых вывода. Построим по ним деревья вывода. Они получатся различными~--- противоречие.
\item Аналогично, если существует единственный левый вывод, то существует единственное дерево вывода.
\end{enumerate}
\section*{Упражнение 2}
\section*{Задача 1}
\begin{enumerate}
\item Определим МП-автомат $\A=(\Sigma,\Gamma,Q,q_0,Z,\delta,F)$, допускающий по принимающему состоянию.\newline
\begin{tabular}{cc}
\begin{minipage}{0.4\textwidth}
\begin{enumerate}
\item $\Sigma\eqdef\{a,b,c,f\}$
\item $\Gamma\eqdef\{F,Z\}$
\item $Q\eqdef\{q_0,q_1,q_2,q_3\}$
\item $\delta$ изображена справа
\item $F\eqdef\{q_0,q_3\}$
\end{enumerate}
\end{minipage}
&
\begin{minipage}{0.6\textwidth}

\begin{tikzpicture}[shorten >=1pt,node distance=2cm,on grid,auto,every node/.style={text centered},initial text=]
	\node [state,initial,accepting] (q_0)	{$q_0$};
	\node [state] (q_1) [right = 2.5cm of q_0 ] {$q_1$};
	\node [state] (q_2) [right = 2.5cm of q_1 ] {$q_2$};
	\node [state,accepting] (q_3) [right = 2.5cm of q_2 ] {$q_3$};
	\path[->]
		(q_0) edge [in=225,out=-45,loop] node {$f,Z/F$} (q_0)
			  edge [in=135,out=45,loop] node[swap] {$f,F/F$} (q_0)
			  edge node {$\substack{a,F/aF\\a,Z/aZ}$} (q_1)
			  edge [out=-25.5,in=205.5] node[swap] {$c,Z/Z$} (q_3)
		(q_1) edge [in=135,out=45,loop] node[swap] {$a,a/aa$} (q_1)
			  edge node {$b,a/\varepsilon$} (q_2)
		(q_2) edge [in=135,out=45,loop] node[swap] {$b,a/\varepsilon$} (q_2)
			  edge node {$\substack{\varepsilon,Z/Z\\\varepsilon,F/F}$} (q_3)
		(q_3) edge [in=135,out=45,loop] node[swap] {$c,Z/Z$} (q_3)
		;
\end{tikzpicture}
\end{minipage}
\end{tabular}
%проверить однозначность!
\item Определим грамматику $G=(\Sigma,T,P,S)$:\begin{enumerate}[1.]
\item $T\eqdef\{S,X,C,F\}$
\item P:\begin{enumerate}
\item $S\longrightarrow XC|FX|X$
\item $X\longrightarrow aXb|\varepsilon$
\item $C\longrightarrow cC|c$
\item $F\longrightarrow fF|f$
\end{enumerate}
\end{enumerate}
\item Автомат является детерминированным по определению ($|\delta(q,\sigma,\gamma)|\leqslant 1$, и если $\delta(q,\sigma,\gamma)\neq\varnothing$, то $\delta(q,\varepsilon,\gamma)=\varnothing$).
\item Докажем, что грамматика однозначная: действительно, для каждого нетерминала $X,C,F$ можно применить только два правила, причем одно из них уменьшает количество $X,C,F$, а второе сохраняет. Поэтому после применения второго нельзя применить первые. Правила для $X$ либо добавляют $a,b$, либо нет, причем другие правила не добавляют $a,b$. Значит, количество применений правила $X\longrightarrow aXb$ фиксировано для каждого слова. Аналогично получаем, что к $C,F$ можно применить только одно правило при фиксированном порожденном слове.\newline
$XC$, $FX$ и $X$ не имеют пересечений в множествах порождаемых слов из терминалов: действительно, $C$ добавляет $c$, $F$ добавляет $f$, $X$ не порождает $c$ или $f$. Поэтому при фиксированном слове $S$ может перейти только в одно из этих слов.
\item Докажем, что $L(G)=L$:\begin{enumerate}
\item $L\subset L(G)$. Пусть $w\in L$. Построим вывод:\begin{enumerate}
\item Если $w=f^na^mb^m$, $n>0$,\newline
$\underline{S}\rightarrow \underline{F}X\rightarrow f\underline{F}X\rightarrow...\rightarrow f^n\underline{X}\rightarrow f^na\underline{X}b\rightarrow...\rightarrow f^na^m\underline{X}b^m\rightarrow f^na^mb^m$.
\item Если $w=a^nb^nc^m$, $m>0$,\newline
$\underline{S}\rightarrow \underline{X}C\rightarrow a\underline{X}bC\rightarrow...\rightarrow a^n\underline{X}b^nC\rightarrow a^nb^n\underline{C}\rightarrow a^nb^nc\underline{C}\rightarrow...\rightarrow a^nb^nc^m$.
\item Если $w=a^nb^n$,\newline
$\underline{S}\rightarrow \underline{X}\rightarrow a\underline{X}b\rightarrow...\rightarrow a^n\underline{X}b^n\rightarrow a^nb^n$.
\end{enumerate}
\item $L(G)\subset L$. Очевидно, $C$ порождает $c^k$, $k>0$; $F$ порождает $f^l$, $l>0$; $X$ порождает $a^nb^n$, $n\geqslant 0$. Также заметим.\newline
Пусть $w\in L(G)$. Рассмотрим вывод. Рассмотрим первое правило:\begin{enumerate}
\item $S\rightarrow X$. Тогда $w=a^nb^n\in L$
\item $S\rightarrow XC$. Тогда $w=a^nb^nc^n\in L$
\item $S\rightarrow FX$. Тогда $w=f^na^mb^m\in L$.
\end{enumerate}
\end{enumerate}
\item Докажем, что $L(\A)=L$:\begin{enumerate}
\item Пусть $w\in L$.\begin{enumerate}
\item \label{bb} $w=a^nb^nc^m,n>0,m\geqslant 0$. Тогда $(q_0,a^nb^nc^m,Z)\vdash(q_1,a^{n-1}b^nc^m,aZ)\vdash(q_1,a^{n-2}b^nc^m,aaZ)\vdash...\vdash(q_1,b^nc^m,a^nZ)\vdash(q_2,b^{n-1}c^m,a^{n-1}Z)\vdash(q_2,b^{n-2}c^m,a^{n-2}Z)\vdash...\vdash(q_2,c^m,Z)\vdash(q_3,c^m,Z)\vdash...\vdash(q_3,\varepsilon,Z)$. $q_3\in F\Rightarrow w\in L(\A)$.
\item $w=c^m,m\geqslant 0$. Тогда $(q_0,c^m,Z)\vdash(q_3,c^m,Z)\vdash...\vdash(q_3,\varepsilon,Z)$. $q_3\in F\Rightarrow w\in L(\A)$
\item \label{aa} $w=f^n,n\geqslant 0$. Тогда $(q_0,f^n,Z)\vdash(q_0,f^{n-1},F)\vdash...\vdash(q_0,\varepsilon,F)$. $q_0\in F\Rightarrow w\in L(\A)$
\item $w=f^na^mb^m, n\geqslant 0, m>0$. Тогда $(q_0,f^na^mb^m)\overset{\ref{aa}}{\vdash^*}(q_0,a^mb^m,F)\overset{\ref{bb}}{\vdash^*}(q_2,\varepsilon,F)\vdash(q_3,\varepsilon,F)$. $q_3\in F\Rightarrow w\in L(\A)$.
\end{enumerate}
\item Пусть $w\in L(\A)\Rightarrow (q_0,w,Z)\vdash^*(q,\varepsilon,\gamma),\,q\in F\equiv\{q_0,q_3\}$.\begin{enumerate}
\item $q=q_0$. Переходы в $q_0$ только из $q_0$ по $f$, поэтому $w=f^m\in L$
\item $q=q_3$. Переходов в $q_3$ три:\begin{enumerate}
\item Был совершен переход $q_0\overset{c,Z/Z}{\longrightarrow}q_3$. Как было отмечено, в $q_0$ переходы только по $f$. Но эти переходы заменяют $Z$ на дне стека на $F$, поэтому переход из данного случая не мог быть произведен. Значит, $w=cw_1$, и цепочка конфигураций имеет вид $(q_0,cw_1,Z)\vdash(q_3,w_1,Z)$. Но из $q_3$ переходы только по $c$, значит, $w=c^l\in L$
\item Был совершен переход $q_2\overset{\varepsilon,Z/Z}{\longrightarrow}q_3$. Поскольку на дне стека $Z$, а не $F$, то символы $f$ не были прочитаны (они могут быть прочитаны только в $q_0$, когда $Z$ на дне стека, и прочтение заменит $Z$ на $f$). Рассмотрим последний переход в $q_2$. Это был переход вида $\overset{b,a/\varepsilon}{\longrightarrow}q_2$ (других нет), значит, в стеке был символ $a$. В $q_2$ есть переходы только из $q_1$, поэтому перед попаданием в $q_2$ (буквы $a$ удаляются из стека при прочтении $b$) конфигурация была $(q_1,b^nx,a^n\gamma)$. Но буквы $a$ могли быть добавлены только при прочтении $a$, другие символы не могли быть прочитаны, поэтому $w=a^nb^nx$. Из $q_3$ есть переходы только по $c$, значит, $w=a^nb^nc^m\in L$
\item Был совершен переход $q_2\overset{\varepsilon,F/F}{\longrightarrow}q_3$. Значит, были прочтены $f$. Аналогично получаем, что $w=f^na^mb^mx$. Но из $q_3$ нет переходов при $F$ на верхушке стека, значит, $x=\varepsilon$, и $w=f^na^mb^m\in L$.
\end{enumerate}
\end{enumerate}
\end{enumerate}
\end{enumerate}
\section*{Задача 2}
\section*{Задача 3}
\begin{enumerate}
\item $M'=(\Sigma,\Gamma,Q',q_0,Z,\delta',F)$~--- расширенный МП-автомат, допускающий по принимающему состоянию. Построим {\em обычный} МП-автомат $M=(\Sigma,\Gamma,Q,q_0,Z,\delta,F)$, допускающий по принимающему состоянию:\begin{enumerate}
\item $Q=Q'\cup\bigcup\limits_{i=1}^{|\delta'|}Q_i$, где для каждого перехода $i$ в исходном автомате $\delta(q,\sigma,\alpha)\ni(s,\beta)$ определено\newline
$Q_i\eqdef\begin{cases}
\varnothing, & |\alpha|=1\\
\{s^1_i,...,s^{|\alpha|-1}_i\} & |\alpha|\geqslant 2
\end{cases}$~--- $|\alpha|-1$ новых состояний.
\item $\delta=\bigcup\limits_{i=1}^{|\delta'|}\delta_i$, где для каждого перехода $i$ в исходном автомате $\delta(q,\sigma,\alpha)\ni(s,\beta)$ определено\newline
$\delta_i\eqdef\begin{cases}
q\overset{\sigma,\alpha/\beta}{\longrightarrow} s, & |\alpha|=1\\
q\overset{\sigma,\alpha_1/\varepsilon}{\longrightarrow}s_i^1\cup s_i^1\overset{\varepsilon,\alpha_2/\varepsilon}{\longrightarrow}s_i^2\cup...\cup s_i^{n-1}\overset{\varepsilon,\alpha_n/\beta}{\longrightarrow}s & |\alpha|\geqslant 1
\end{cases}$, где $\alpha=\alpha_1...\alpha_n,\,\forall i\in\overline{1,n}\hookrightarrow \alpha_i\in \Gamma$ во втором случае.
\end{enumerate}
Иными словами, переходы, удаляющие один символ не изменяются, а остальные переходы вида
%\begin{tabular}{ccc}
\begin{center}
\begin{tikzpicture}[shorten >=1pt,node distance=2cm,on grid,auto,every node/.style={text centered},initial text=]
	\node [state] (q_0)	{$p$};
	\node [state] (q_1) [right = 2cm of q_0 ] {$q$};
	\path[->]
		(q_0) edge node {$\sigma,\alpha/\beta$} (q_1);
\end{tikzpicture}
\end{center}
%&

заменяются на переходы
%&
\begin{center}
\begin{tikzpicture}[shorten >=1pt,node distance=2cm,on grid,auto,every node/.style={text centered},initial text=]
	\node [state] (p)	{$p$};
	\node [state] (q) [right = 8cm of p ] {$q$};
	\node [state] (q_1) [below right = 2cm of p] {$s_1$};
	\node [state] (q_2) [right = 2cm of q_1] {$s_2$};
	\node (dots)  [right = 1.5cm of q_2] {$...$};
	\node [state] (q_n1) [below left = 2cm of q] {$s_{n-1}$};
	\path[->]
		(p) edge node {$\sigma,\alpha_1/\varepsilon$} (q_1)
		(q_1) edge node {$\varepsilon,\alpha_2/\varepsilon$} (q_2)
		(q_2) edge node {$ $} (dots)
		(dots) edge node {$ $} (q_n1)
		(q_n1) edge node {$\varepsilon,\alpha_n/\beta$} (q)
		;
\end{tikzpicture}
%\end{tabular}
\end{center}
\begin{enumerate}[1.]
\item ($L(M')\subseteq L(M)$) Пусть $w\in L(M')$. Рассмотрим цепочку конфигураций в исходном автомате. Каждый переход, удаляющий один символ из стека может быть совершен в новом автомате, т.к. он содержится в $\delta$. Переходы, удаляющие больше одного символы также могут быть совершены (по построению). Получаем, что $M$ оказывается в том же состоянии $q$, что и $M'$. $q\in F\Rightarrow w\in L(M)$
\item ($L(M)\subseteq L(M')$). Пусть $w\in L(M)\Rightarrow q_0\overset{w}{\longrightarrow}q\in F$. Рассмотрим цепочку конфигураций. Переходы по состояниям $s_i^1...s_i^n$ можно <<свернуть>> в один переход из исходного по построению (степени исхода и захода у созданных вершин равны 1, поэтому, путь из $s^i_j$ может быть только в соответствующее состояние $q$). Таким образом, исходный автомат также может оказаться в состоянии $q\in F\Rightarrow w\in L(M')$
\end{enumerate}
\item {\em{В задании не дано определение детерминированного расширенного МП-автомата, поэтому используется определение для обычных МП-автоматов (одно правило по первому символу, который удаляется из стека)}}\newline
Пусть $M'$~--- детерминированный, докажем, что $M$~--- детерминированный. Действительно, из созданных состояний $s_i^j$ переход один по построению, и в другое состояние.\newline
По определению детерминированности $M'$ имеем\newline
$\forall q\in Q'\,\forall\sigma\in\Sigma\,\forall\gamma\in\Gamma\hookrightarrow|\delta'(q,\sigma,\gamma)|\leqslant 1$, и\newline $\forall\sigma\in\Sigma\,\forall\gamma\in\Gamma\hookrightarrow(\delta'(q,\varnothing,\gamma)\neq\varnothing\Rightarrow\delta'(q,\sigma,\gamma)=\varnothing)$
Поскольку для каждого измененного перехода из $q$ было изменено только состояние, куда совершается переход и символ, удаляемый из стека (вместо слова), получаем это же утверждение для $M$ $\Rightarrow$ $M$~--- детерминированный.
\end{enumerate}
\section*{Задача 4}
\section*{Задача 5}
\end{document}