\documentclass[a4paper]{article}
\usepackage[a4paper, left=5mm, right=5mm, top=5mm, bottom=5mm]{geometry}
%\geometry{paperwidth=210mm, paperheight=2000pt, left=5pt, top=5pt}
\usepackage[utf8]{inputenc}
\usepackage[english,russian]{babel}
\usepackage{indentfirst}
\usepackage{tikz} %Рисование автоматов
\usetikzlibrary{automata,positioning,arrows,trees}
\usepackage{amsmath}
\usepackage[makeroom]{cancel} % зачеркивание
\usepackage{multicol,multirow} %Несколько колонок
\usepackage{hyperref}
\usepackage{tabularx}
\usepackage{amsfonts}
\usepackage{amssymb}
\DeclareMathOperator*{\argmin}{arg\,min}
\usepackage{wasysym}
\title{Алгоритмы и модели вычислений.\\Задание 2: Арифметические операции и линейные рекуррентные последовательности}
\date{задано 2014.02.20}
\author{Сергей~Володин, 272 гр.}
\newcommand{\matrixl}{\left|\left|}
\newcommand{\matrixr}{\right|\right|}
% названия автоматов  (rubtsov)
\def\A{{\cal A}}
\def\B{{\cal B}}
\def\C{{\cal C}}

% алгоритмы (Рубцов)
\usepackage{verbatim}
\usepackage{listings}
\usepackage{algorithm2e}

%+= и -=, иначе разъезжаются...
\newcommand{\peq}{\mathrel{+}=}
\newcommand{\meq}{\mathrel{-}=}
\newcommand{\deq}{\mathrel{:}=}
\newcommand{\plpl}{\mathrel{+}+}

% пустое слово  (rubtsov)
\def\eps{\varepsilon}

% регулярные языки  (rubtsov)
\def\REG{{\mathsf{REG}}}
\def\CFL{{\mathsf{CFL}}}
\def\eqdef{\overset{\mbox{\tiny def}}{=}}
\newcommand{\niton}{\not\owns}

%FIRST & FOLLOW (rubtsov)
\def\first{\mathrm{ FIRST} }
\def\follow{\mathrm{ FOLLOW} }

% LL LR (rubtsov)
\def\LL{{\mathrm{LL}}}
\def\LR{{\mathrm{LR}}}

\newcounter{rowItemCount}
\newcounter{subRowItemCount}
\newcommand\rowItem{
    \setcounter{subRowItemCount}{0}
    \arabic{rowItemCount}.\addtocounter{rowItemCount}{1}}
\newcommand\subRowItem{
    \addtocounter{subRowItemCount}{1}
    \addtocounter{rowItemCount}{-1}
    \arabic{rowItemCount}.\arabic{subRowItemCount}.\addtocounter{rowItemCount}{1}}
    
\newcommand{\smalll}[1]{\overline{\overline{#1}}}
\newcommand{\smallo}{\bar{\bar{o}}}

\begin{document}
\maketitle
\subsection*{Упражнение 3}
Определим $A_d\eqdef\matrixl
\begin{array}{ccccc}
c_1 & c_2 & ... & c_{d-1} & c_d \\
1   & 0   & ... & 0 & 0\\
0 & 1 & ... & 0 & 0\\
\multicolumn{5}{c}{...}\\
0 & 0 & ... & 1 & 0
\end{array}
\matrixr$

Докажем по индукции $P(d)\eqdef \left[\det(A_d-\lambda E)=(-1)^d(\lambda^d-c_1\lambda^{d-1}-c_2\lambda^{d-2}-...-c_{d-1}\lambda-c_d)\right]$
\begin{enumerate}
\item База. $d=3\Rightarrow$ $\det(A_3-\lambda E)=\left|
\begin{array}{ccc}
(c_1-\lambda) & c_2 & c_3\\
1 & -\lambda & 0\\
0 & 1 & -\lambda\\
\end{array}
\right|=c_1\lambda^2-\lambda^3+c_3+c_2\lambda=(-1)^3(\lambda^3-c_1\lambda^2-c_2\lambda-c_3)\Rightarrow P(3)\,\blacksquare$
\item Пусть $\underline{P(d-1)}$. Рассмотрим $\det(A_d-\lambda E)=\left|
\begin{array}{ccccc}
(c_1-\lambda) & c_2 & ... & c_{d-1} & c_d \\
1   & -\lambda   & ... & 0 & 0\\
0 & 1 & ... & 0 & 0\\
\multicolumn{5}{c}{...}\\
0 & 0 & ... & 1 & -\lambda\\
\end{array}
\right|\boxed{=}$.\newline
Разложим по последнему столбцу: $\boxed{=}-\lambda\underbrace{\left|
\begin{array}{cccc}
(c_1-\lambda) & c_2 & ... & c_{d-1}\\
1 & -\lambda & ... & 0\\
\multicolumn{4}{c}{...}\\
0 & 0 & ... & -\lambda\\
\end{array}
\right|}_{\det(A_{d-1}-\lambda E)}+(-1)^{d+1}c_d\underbrace{\left|
\begin{array}{ccccc}
1 & -\lambda & 0 & ... & 0\\
0 & 1 & -\lambda & ... & 0\\
\multicolumn{5}{c}{...}\\
0 & 0 & 0 & ... & 1
\end{array}
\right|}_{=1\mbox{\tiny, верхн.-треуг.}}\overset{P(d-1)}{=}\newline\overset{P(d-1)}{=}
-\lambda(-1)^{d-1}(\lambda^{d-1}-c_1\lambda^{d-2}-...-c_{d-2}\lambda-c_{d-1})-(-1)^dc_d=(-1)^d(\lambda^d-c_1\lambda^{d-1}-...-c_{d-1}\lambda-c_d)$. Получаем $\underline{P(d)}\,\blacksquare$
\end{enumerate}
\subsection*{Задача 1*}
Последовательность $\{a_n\}_{n=1}^\infty$~--- ЛРП порядка $d$: $a_n=\sum\limits_{i=1}^dc_ia_{n-i}$. Выпишем матрицу $A=\matrixl
\begin{array}{ccccc}
c_1 & c_2 & ... & c_{d-1} & c_d\\
1 & 0 & ... & 0 & 0\\
0 & 1 & ... & 0 & 0\\
\multicolumn{5}{c}{...}\\
0 & 0 & ... & 1 & 0
\end{array}
\matrixr$. Определим $\vec{a_n}=\matrixl
\begin{array}{c}
a_n\\
a_{n-1}\\
...\\
a_{n-d+1}
\end{array}
\matrixr$. Тогда $\vec{a_n}=A^{n-d}\vec{g}_d$. Обозначим $\vec{a}=\vec{g}_d$. По условию существуют $d$ различных корней $\lambda\in\{\lambda_1,...,\lambda_d\}$ многочлена $\det(A-\lambda\cdot E)=0$. Значит, существует матрица $S=\matrixl
\begin{array}{ccc}
s_{11} & ... & s_{1d}\\
... &  & ...\\
s_{d1} & ... & s_{dd}\\
\end{array}
\matrixr$, такая что ее $i$-й столбец является собственным вектором $\vec{h}_i$ матрицы $A$, соответствующим собственному значению $\lambda_i$, и $A'=S^{-1}AS=\mbox{diag}(\lambda_1,...,\lambda_d)$. $S^{-1}$ существует, так как $\vec{h}_i$~--- линейно независимы. Выразим $A=SA'S^{-1}$,\newline рассмотрим $A^n=\underbrace{SA'\cancelto{0}{S^{-1}\cdot S}A'S^{-1}\cdot ... \cdot SA'\cancelto{0}{S^{-1}\cdot S}A'S^{-1}}_n=SA'^nS^{-1}$. Определим $\vec{\xi}\eqdef \vec{e}_1=\matrixl
\begin{array}{c}
1\\
0\\
..\\
0
\end{array}
\matrixr\in\mathbb{R}^d$. Заметим, что $\vec{a}_n=\vec{\xi}^T\vec{g}_n$, откуда $a_n=\vec{\xi}^TSA'^{n-d}S^{-1}\vec{a}$. Найдем $\vec{\xi}^TS=\matrixl
\begin{array}{cccc}
1 & 0 & ... & 0
\end{array}
\matrixr
\matrixl
\begin{array}{ccc}
s_{11} & ... & s_{1d}\\
... &  & ...\\
s_{d1} & ... & s_{dd}\\
\end{array}
\matrixr=\matrixl
\begin{array}{cccc}
s_{11} & s_{12} & ... & s_{1d}
\end{array}
\matrixr$, строка $\vec{\xi}^TSA'^{n-d}=\matrixl
\begin{array}{cccc}
s_{11} & s_{12} & ... & s_{1d}
\end{array}
\matrixr\matrixl
\begin{array}{cccc}
\lambda_1^{n-d} & 0 & ... & 0\\
0 & \lambda_2^{n-d} & ... & 0\\
\multicolumn{4}{c}{...}\\
0 & 0 & ... & \lambda_d^{n-d}
\end{array}
\matrixr=\matrixl
\begin{array}{ccc}
\lambda_1^{n-d}s_{11} & ... & \lambda_d^{n-d}s_{1d}
\end{array}
\matrixr$,\newline
$i$-й элемент этой строки $(\vec{\xi}^TSA'^{n-d})_i=\lambda_i^{n-d}s_{1i}$\newline
Найдем $S^{-1}\vec{a}=\matrixl
\begin{array}{ccc}
s'_{11} & ... & s'_{d1}\\
... &  & ...\\
s'_{1d} & ... & s'_{dd}\\
\end{array}
\matrixr
\matrixl
\begin{array}{c}
a_d\\
a_{d-1}\\
...\\
a_1
\end{array}
\matrixr$
($s'_{ij}$~--- элементы матрицы $S^{-1}$), $i$-й элемент в этом столбце равен $(S^{-1}\vec{a})_i=\sum\limits_{j=1}^da_{d-j+1}s'_{ji}$\newline
Получаем $a_n=\sum\limits_{i=1}^d \lambda_i^{n-d}s_{1i}\sum\limits_{j=1}^da_{d-j+1}s'_{ji}\overset{?}{=}\sum\limits_{i=1}^dk_i\lambda_i^n$. Это равенство верно: в случае $\lambda_i=0$ можно взять любое $k_i$ (например, $k_i=0$), иначе~--- $k_i=\lambda_i^{-d}s_{1i}\sum\limits_{j=1}^da_{d-j+1}s'_{ji}$ $\blacksquare$
\subsection*{(каноническое) Задача 6}
$T(n)=7T(\frac{n}{2})+f(n)$, $f(n)=O(n^2)$. Дерево рекурсии:\newline
\begin{tabular}{lr}
\begin{minipage}{0.47\textwidth}
\begin{tikzpicture}[scale=0.75,transform shape,level/.style={sibling distance = 5cm/#1, level distance = 1.5cm}]
\node [circle,draw] (z){$n^2$}
  child {node [circle,draw] (a) {$\frac{n^2}{2^2}$}
    child {node [circle,draw] (b) {$\frac{n^2}{2^4}$}
      child {node {$\vdots$}
        child {node [circle,draw] (d) {$T(1)$}}
        child {node [circle,draw] (e) {$T(1)$}}
      } 
      child {node {$\vdots$}}
    }
    child {node [circle,draw] (g) {$\frac{n^2}{2^4}$}
      child {node {$\vdots$}}
      child {node {$\vdots$}}
    }
  }
  child {node [circle,draw] (j) {$\frac{n^2}{2^2}$}
    child {node [circle,draw] (k) {$\frac{n^2}{2^4}$}
      child {node {$\vdots$}}
      child {node {$\vdots$}}
    }
  child {node [circle,draw] (l) {$\frac{n^2}{2^4}$}
    child {node {$\vdots$}}
    child {node (c){$\vdots$}
      child {node [circle,draw] (o) {$T(1)$}}
      child {node [circle,draw] (p) {$T(1)$}
          child [grow=right] {node (q) {$7^hT(1)$} edge from parent[draw=none]
          child [grow=up, level distance=0.7cm] {node (r) {$\vdots$} edge from parent[draw=none]
          child [grow=up, level distance=0.7cm] {node (r) {$7^k\frac{n^2}{2^{2k}}$} edge from parent[draw=none]
            child [grow=up, level distance=1cm] {node (r) {$\vdots$} edge from parent[draw=none]
              child [grow=up, level distance=0.6cm] {node (s) {$7^2\frac{n^2}{2^4}$} edge from parent[draw=none]
                child [grow=up, level distance=1.5cm] {node (t) {$7\frac{n^2}{2^2}$} edge from parent[draw=none]
                  child [grow=up, level distance=1.5cm] {node (u) {$n^2$} edge from parent[draw=none]}
                }
              }
              }
              }
            }
          }
        }
    }
  }
};
\end{tikzpicture}
\end{minipage} &
\begin{minipage}{0.46\textwidth}
Высота дерева $h=\log_2 n$.\newline
$T(n)=\sum\limits_{k=0}^{h-1}7^kf(\frac{n}{2^k})+7^hT(1)\boxed{\leqslant}$.\newline Из определения $O$ $\exists C>0\,\exists n_0\colon\forall n\geqslant n_0\, f(n)\leqslant Cn^2$, откуда первая сумма
$\sum\limits_{k=0}^{h-1}7^kf(\frac{n^2}{2^{2k}})\leqslant Cn^2\sum\limits_{k=0}^{h-1}(\frac{7}{4})^k=Cn^2\frac{(7/4)^{h-1}-1}{7/4-1}=C_1n^2((7/4)^{\log_2 n}-C_2)=C_1n^2n^{\log_2\frac{7}{4}}-C_3n^2=C_1n^{\log_2 7}-C_3n^2$. Второе слагаемое $7^hT(1)=7^{\log_2 n}T(1)=Cn^{\log_2 7}$\newline
Поэтому $T(n)\leqslant n^{\log_2 7}-C_5n^2$
\end{minipage}\\
\end{tabular}\newline
Оценка снизу $T(n)\geqslant 7^hT(1)=O(n^{\log_2 7})$, откуда\newline
Ответ: $\boxed{T(n)=\Theta(n^{\log_2 7})}$
\subsection*{(каноническое) Задача 7}
Вход: точки $\{x_i,y_i\}_{i=1}^n$.\newline
Алгоритм: считаем массив расстояний $r_i\eqdef \sqrt{x_i^2+y_i^2}$ {\em (можно $r_i^2$)}. Ищем медиану $r_m$ в массиве за $O(n)$\newline
Ответ: $r_{m}$ {\em ($r_{m+1}$?)}.
\subsection*{(каноническое) Задача 9}
Пусть $\Sigma=\{\underbrace{0}_{\sigma_0},\underbrace{1}_{\sigma_1},\underbrace{2}_{\sigma_2}\}$, $\Sigma^*\supset G=\{w\big|\exists n\colon w=w_1...w_n,\,\underbrace{\forall i\in\overline{1,n-1}\hookrightarrow |w_i-w_{i+1}|}_{(*)}\leqslant 1\}$. Пусть $g_n=|\{w\in L\big||w|=n\}|$~--- количество слов длины $n$ в языке $G$. Определим $g^i_n=|\{w\in G\big||w|=n,\,w_n=\sigma_i\}|$~--- количество слов длины $n$ из $G$, оканчивающихся на $i$-й символ. Поскольку каждое слово оканчивается на один из символов $\sigma_i$, получаем $g_n=g_n^0+g_n^1+g_n^2$.
\begin{enumerate}
\item \label{9c_rec} Найдем рекуррентное соотношение для последовательностей $g_n^i$. Получим слово $w\in G$ длины $n+1$: $w=w_1...w_nw_{n+1}$. Поскольку слово из языка, для него верно $(*)$. Но это условие верно и для подслова $w_1...w_n$. Рассмотрим последний символ слова $w$~--- $w_{n+1}$:\begin{enumerate}
\item \label{9c_rec0} $w_{n+1}=0$. Но тогда предпоследний символ слова $w$~--- $w_n$ может быть $0$ либо $1$ для выполнения $(*)$. Слово $w_1...w_n$ может быть получено $g_n^0$ и $g_n^1$ способами соответственно. Поэтому количество способов получить $w$ в этом случае $g_{n+1}^0=g_n^0+g_n^1$ $(**)$.
\item \label{9c_rec1} $w_{n+1}=1$. Тогда $w_n\in\{0,1,2\}$, и $g_{n+1}^1=g_n^0+g_n^1+g_n^2$.
\item \label{9c_rec2} $w_{n+1}=2$. Тогда $w_n\in\{1,2\}$, и $g_{n+1}^2=g_n^1+g_n^2$.
\end{enumerate}
\item Определим вектор ${\mathbb{R}}^3\ni \vec{g_n}=\matrixl
\begin{array}{c}
g_n^0\\
g_n^1\\
g_n^2
\end{array}
\matrixr$. Определим матрицу $A\eqdef\matrixl
\begin{array}{ccc}
1 & 1 & 0\\
1 & 1 & 1\\
0 & 1 & 1
\end{array}
\matrixr$.\newline
Снова рассмотрим соотношения $
\begin{cases}
\ref{9c_rec0}\\
\ref{9c_rec1}\\
\ref{9c_rec2}
\end{cases}\Leftrightarrow
\begin{cases}
g_{n+1}^0=g_n^0+g_n^1\\
g_{n+1}^1=g_n^0+g_n^1+g_n^2\\
g_{n+1}^2=g_n^1+g_n^2
\end{cases}.$\newline
Заметим, что в матричном виде они записываются как $\vec{g_{n+1}}=A\vec{g_n}$ $(***)$
\item Найдем $g_1^0=g_1^1=g_1^2=1$, так как слово из одного символа удовлетворяет $(*)$. Определим $\vec{\xi}\eqdef\matrixl
\begin{array}{c}
1\\
1\\
1
\end{array}
\matrixr$.\newline
Тогда, применяя $(***)$ {\em (доказывается тривиально по индукции)} получаем $\vec{g_n}=A^{n-1}\vec{\xi}$
\item $g_n=g_n^0+g_n^1+g_n^2$. Но это равно $g_n=(\vec{\xi},A^{n-1}\vec{\xi})=\vec{\xi}^TA^{n-1}\vec{\xi}$
\item Найдем ОНБ, в котором $A$ имеет диагональный вид\begin{enumerate}
\item Характеристический многочлен $\det(A-\lambda E)=\left|
\begin{array}{ccc}
(1-\lambda) & 1 & 0\\
1 & (1-\lambda) & 1\\
0 & 1 & (1-\lambda)
\end{array}
\right|=(1-\lambda)^3-2(1-\lambda)=(1-\lambda)\cdot(1+\lambda^2-2\lambda-2)=(1-\lambda)\cdot(\lambda^2-2\lambda-1)$. Корни характеристического уравнения $\lambda=1$ и $\lambda\in\frac{2\pm\sqrt{4\cdot 2}}{2}=1\pm\sqrt{2}$. Далее ищем собственные векторы.
\item ($\lambda=\lambda_1=1$). $A-1\cdot E=\matrixl
\begin{array}{ccc}
0 & 1 & 0\\
1 & 0 & 1\\
0 & 1 & 0
\end{array}
\matrixr\sim
\matrixl\begin{array}{ccc}
1 & 0 & 1\\
0 & 1 & 0\\
\end{array}
\matrixr$, откуда $\vec{h}_1^0=\matrixl
\begin{array}{ccc}
-1 & 0 & 1
\end{array}
\matrixr^T$, $\vec{h}_1=\matrixl
\begin{array}{c}
\frac{-1}{\sqrt{2}}\\
0\\
\frac{1}{\sqrt{2}}
\end{array}
\matrixr$
\item $(\lambda=\lambda_2=1+\sqrt{2})$. $A-(1+\sqrt{2})\cdot E=\matrixl
\begin{array}{ccc}
-\sqrt{2} & 1 & 0\\
1 & -\sqrt{2} & 1\\
0 & 1 & -\sqrt{2}
\end{array}\matrixr\sim
\matrixl
\begin{array}{ccc}
1 & -\sqrt{2} & 1\\
-\sqrt{2} & 1 & 0\\
0 & 1 & -\sqrt{2}
\end{array}\matrixr\sim
\matrixl
\begin{array}{ccc}
1 & -\sqrt{2} & 1\\
0 & 1 & -\sqrt{2}
\end{array}\matrixr\sim\newline\sim
\matrixl
\begin{array}{ccc}
1 & -\sqrt{2} & 1\\
0 & 1 & -\sqrt{2}
\end{array}\matrixr\sim
\matrixl
\begin{array}{ccc}
1 & 0 & -1\\
0 & 1 & -\sqrt{2}
\end{array}\matrixr$, откуда $\vec{h}_2^0=\matrixl
\begin{array}{ccc}
1 & \sqrt{2} & 1
\end{array}
\matrixr^T$, $\vec{h}_2=\matrixl
\begin{array}{c}
1/2\\
1/\sqrt{2}\\
1/2
\end{array}
\matrixr\perp\vec{h}_1$
\item $(\lambda=\lambda_3=1-\sqrt{2})$. $A-(1-\sqrt{2})\cdot E=\matrixl
\begin{array}{ccc}
\sqrt{2} & 1 & 0\\
1 & \sqrt{2} & 1\\
0 & 1 & \sqrt{2}
\end{array}
\matrixr\sim\matrixl
\begin{array}{ccc}
1 & \sqrt{2} & 1\\
\sqrt{2} & 1 & 0\\
0 & 1 & \sqrt{2}
\end{array}
\matrixr\sim
\matrixl
\begin{array}{ccc}
1 & \sqrt{2} & 1\\
0 & 1 & \sqrt{2}
\end{array}
\matrixr\sim
\matrixl
\begin{array}{ccc}
1 & 0 & -1\\
0 & 1 & \sqrt{2}
\end{array}
\matrixr
$, откуда $\vec{h}_3^0=\matrixl
\begin{array}{ccc}
1 & -\sqrt{2} & 1
\end{array}
\matrixr$, $\vec{h}_3=\matrixl
\begin{array}{c}
1/2\\
-1/\sqrt{2}\\
1/2
\end{array}
\matrixr\perp \vec{h}_1,\,\vec{h}_2$
\end{enumerate}
Получаем $S\eqdef\matrixl
\begin{array}{ccc}
\frac{-1}{\sqrt{2}} & 1/2 & 1/2\\
0 & 1/\sqrt{2} & -1/\sqrt{2}\\
\frac{1}{\sqrt{2}} & 1/2 & 1/2
\end{array}
\matrixr$~--- ортогональная матрица перехода к базису из собственных векторов.\newline
Тогда $A'=S^{-1}AS\Rightarrow A=SA'S^{-1}\equiv SA'S^T$, Но $A'=\mbox{diag}(\lambda_1,\lambda_2,\lambda_3)=\matrixl
\begin{array}{ccc}
1 & 0 & 0\\
0 & (1+\sqrt{2}) & 0\\
0 & 0 & (1-\sqrt{2})\\
\end{array}
\matrixr$, поэтому\newline
$A'^n=\mbox{diag}(\lambda_1^n,\lambda_2^n,\lambda_3^n)$
\item $A^n=\underbrace{SA'\cancelto{E}{S^T\cdot S}A'S^T\cdot...\cdot SA'\cancelto{E}{S^T\cdot S}A'S^T}_n=SA'^nS^T=S\mbox{diag}(\lambda_1^n,\lambda_2^n,\lambda_3^n)S^T$
\item \label{9c_gn} Вернемся к $g_n=\vec{\xi}^TA^{n-1}\vec{\xi}=\vec{\xi}^TS\mbox{diag}(\lambda_1^{n-1},\lambda_2^{n-1},\lambda_3^{n-1})S^T\vec{\xi}=\boxed{\frac{1}{2}\left[(1+\sqrt{2})^{n+1}+(1-\sqrt{2})^{n+1}\right]}$
\item Попробуем найти рекуррентное соотношение следующим образом. Предположим, что последовательность $\{g_n\}_{n=1}^\infty$~--- ЛРП порядка $d$, причем все корни характеристического многочлена ее матрицы вещественные и различные. Тогда (Задача 1) $\exists k_1,...,k_d\colon g_n=k_1\lambda_1^n+...+k_2\lambda_d^n$. Сравнивая с выражением выше, получаем $d=2$, т.е. ищем рекуррентное соотношение вида $g_n=c_1g_{n-1}+c_2g_{n-2}$. Подставляя выражение $\ref{9c_gn}$ для $g_n$, получаем $(1+\sqrt{2})^{n+1}+(1-\sqrt{2})^{n+1}=c_1(1+\sqrt{2})^{n}+c_1(1-\sqrt{2})^{n}+c_2(1+\sqrt{2})^{n-1}+c_2(1-\sqrt{2})^{n-1}\Leftrightarrow (1+\sqrt{2})^{n-1}(3+2\sqrt{2}-c_1(1+\sqrt{2})-c_2)+(1-\sqrt{2})^{n-1}(3-2\sqrt{2}-c_1(1-\sqrt{2})-c_2)=0$, что будет выполнено при любых $n$ при $\begin{cases}
(1+\sqrt{2})c_1+c_2=3+2\sqrt{2}\\
(1-\sqrt{2})c_1+c_2=3-2\sqrt{2}
\end{cases}\Leftrightarrow
\begin{cases}
c_1=2\\
c_2=1
\end{cases}$,\newline т.е. $\boxed{g_n=2g_{n-1}+g_{n-2}}$
\item Найдем $g_{2014}=9816936009995503230901557247246041662063072822494753312759700362719597435946538528221300925\newline
671858801599363935274622877500162506956619048900408718181041413222318236818715345484376136537862497272785\newline
247720491012219807232607980494871964788980842814109033161842422339596260323417836542815901642749689573589\newline
070088974641306848102517213985023530762354797649521475871449969940200866323482540594978486708923597366885\newline
750142187523485222503097287926012700695073990739801458896041837993605326294700244522632962855241858966782\newline
631798710557997423351374248485616450622394012426366144662745043995902048923883147167702198223719419200759\newline
471729710067440801808039863672079281506822373369234466827616569206575038689737028383771817685667299606446\newline
92272395910326789357589123767900512319408352202559$
\end{enumerate}
\end{document}